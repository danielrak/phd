% Options for packages loaded elsewhere
\PassOptionsToPackage{unicode}{hyperref}
\PassOptionsToPackage{hyphens}{url}
\PassOptionsToPackage{dvipsnames,svgnames,x11names}{xcolor}
%
\documentclass[
]{book}
\usepackage{amsmath,amssymb}
\usepackage{lmodern}
\usepackage{iftex}
\ifPDFTeX
  \usepackage[T1]{fontenc}
  \usepackage[utf8]{inputenc}
  \usepackage{textcomp} % provide euro and other symbols
\else % if luatex or xetex
  \usepackage{unicode-math}
  \defaultfontfeatures{Scale=MatchLowercase}
  \defaultfontfeatures[\rmfamily]{Ligatures=TeX,Scale=1}
\fi
% Use upquote if available, for straight quotes in verbatim environments
\IfFileExists{upquote.sty}{\usepackage{upquote}}{}
\IfFileExists{microtype.sty}{% use microtype if available
  \usepackage[]{microtype}
  \UseMicrotypeSet[protrusion]{basicmath} % disable protrusion for tt fonts
}{}
\makeatletter
\@ifundefined{KOMAClassName}{% if non-KOMA class
  \IfFileExists{parskip.sty}{%
    \usepackage{parskip}
  }{% else
    \setlength{\parindent}{0pt}
    \setlength{\parskip}{6pt plus 2pt minus 1pt}}
}{% if KOMA class
  \KOMAoptions{parskip=half}}
\makeatother
\usepackage{xcolor}
\usepackage[top = 2.5cm, right = 2.5cm, bottom = 2.5cm, left = 2.5cm]{geometry}
\usepackage{longtable,booktabs,array}
\usepackage{calc} % for calculating minipage widths
% Correct order of tables after \paragraph or \subparagraph
\usepackage{etoolbox}
\makeatletter
\patchcmd\longtable{\par}{\if@noskipsec\mbox{}\fi\par}{}{}
\makeatother
% Allow footnotes in longtable head/foot
\IfFileExists{footnotehyper.sty}{\usepackage{footnotehyper}}{\usepackage{footnote}}
\makesavenoteenv{longtable}
\usepackage{graphicx}
\makeatletter
\def\maxwidth{\ifdim\Gin@nat@width>\linewidth\linewidth\else\Gin@nat@width\fi}
\def\maxheight{\ifdim\Gin@nat@height>\textheight\textheight\else\Gin@nat@height\fi}
\makeatother
% Scale images if necessary, so that they will not overflow the page
% margins by default, and it is still possible to overwrite the defaults
% using explicit options in \includegraphics[width, height, ...]{}
\setkeys{Gin}{width=\maxwidth,height=\maxheight,keepaspectratio}
% Set default figure placement to htbp
\makeatletter
\def\fps@figure{htbp}
\makeatother
\setlength{\emergencystretch}{3em} % prevent overfull lines
\providecommand{\tightlist}{%
  \setlength{\itemsep}{0pt}\setlength{\parskip}{0pt}}
\setcounter{secnumdepth}{5}
\newlength{\cslhangindent}
\setlength{\cslhangindent}{1.5em}
\newlength{\csllabelwidth}
\setlength{\csllabelwidth}{3em}
\newlength{\cslentryspacingunit} % times entry-spacing
\setlength{\cslentryspacingunit}{\parskip}
\newenvironment{CSLReferences}[2] % #1 hanging-ident, #2 entry spacing
 {% don't indent paragraphs
  \setlength{\parindent}{0pt}
  % turn on hanging indent if param 1 is 1
  \ifodd #1
  \let\oldpar\par
  \def\par{\hangindent=\cslhangindent\oldpar}
  \fi
  % set entry spacing
  \setlength{\parskip}{#2\cslentryspacingunit}
 }%
 {}
\usepackage{calc}
\newcommand{\CSLBlock}[1]{#1\hfill\break}
\newcommand{\CSLLeftMargin}[1]{\parbox[t]{\csllabelwidth}{#1}}
\newcommand{\CSLRightInline}[1]{\parbox[t]{\linewidth - \csllabelwidth}{#1}\break}
\newcommand{\CSLIndent}[1]{\hspace{\cslhangindent}#1}
\ifLuaTeX
\usepackage[bidi=basic]{babel}
\else
\usepackage[bidi=default]{babel}
\fi
\babelprovide[main,import]{french}
% get rid of language-specific shorthands (see #6817):
\let\LanguageShortHands\languageshorthands
\def\languageshorthands#1{}
% Version propre. Voir divers fichiers au cours de la thèse pour les détails.
\usepackage{mdsymbol, lscape, pdfpages, makecell, booktabs, multirow, enumitem, float, longtable, tikz, caption, biblatex, lscape, setspace, upgreek}  
\usepackage[fontsize=12pt]{scrextend}  
\usepackage[utf8]{inputenc}
\AtEndPreamble{\hypersetup{hypertexnames=false}}
%\onehalfspacing
\setlist[itemize,1]{label=$\bullet$} 
\newcommand{\blandscape}{\begin{landscape}}
\newcommand{\elandscape}{\end{landscape}}
\newcommand\latexcode[1]{#1}
\captionsetup[table]{name = \textsc{Tableau}}
\DeclareUnicodeCharacter{1D52}{$^o$} 


\maxdeadcycles=200

\raggedbottom
\usepackage[bottom]{footmisc}
%\index{Paragraph}paragraph




\usepackage{booktabs}
\usepackage{longtable}
\usepackage{array}
\usepackage{multirow}
\usepackage{wrapfig}
\usepackage{float}
\usepackage{colortbl}
\usepackage{pdflscape}
\usepackage{tabu}
\usepackage{threeparttable}
\usepackage{threeparttablex}
\usepackage[normalem]{ulem}
\usepackage{makecell}
\usepackage{xcolor}
\ifLuaTeX
  \usepackage{selnolig}  % disable illegal ligatures
\fi
\IfFileExists{bookmark.sty}{\usepackage{bookmark}}{\usepackage{hyperref}}
\IfFileExists{xurl.sty}{\usepackage{xurl}}{} % add URL line breaks if available
\urlstyle{same} % disable monospaced font for URLs
\hypersetup{
  pdflang={fr},
  colorlinks=true,
  linkcolor={Blue},
  filecolor={Maroon},
  citecolor={Blue},
  urlcolor={Blue},
  pdfcreator={LaTeX via pandoc}}

\author{}
\date{\vspace{-2.5em}}

\begin{document}

\newpage
\fontsize{12}{16}\selectfont
\pagestyle{plain}

% Version propre.

\begin{titlepage}
\begin{center}

  \includegraphics[width = 3cm, height = 5cm]{logo_UR.jpg} \hfill 
  \includegraphics[width = 3cm, height = 5cm]{logo_CEMOI.png} \hfill
  \includegraphics[width = 3cm, height = 5cm]{logo_RR.jpg} \hfill
  \includegraphics[width = 3cm, height = 5cm]{logo_UE.jpg}

	\vfill

{
\small
Université de La Réunion \\
École Doctorale Sciences Humaines et Sociales ED541 \\
Centre d'Économie et de Management de l'Océan Indien
}


\vfill

 {
 \Large\bfseries 
 \rule{\linewidth}{1pt} 
 Essais sur des déterminants de performances scolaires et universitaires à La Réunion 
 \rule{\linewidth}{1pt}
 }
 
	\vfill
	
{
\renewcommand*{\thefootnote}{\fnsymbol{footnote}}
\textbf{Thèse de doctorat}\footnote[2]{
\small
Cette thèse a reçu le soutien financier de la Région Réunion et de l'Union Européenne - Fonds européen de développement régional (FEDER), dans le cadre du Programme Opérationnel européen INTERREG V - 2014-2020.
}
\renewcommand*{\thefootnote}{\arabic{footnote}}
} 

	\vfill

{dirigée par}

  \vfill

{
\renewcommand*{\thefootnote}{\fnsymbol{footnote}}
\textbf{Yves Croissant\footnote[1]{
\small 
Professeur à l'Université de La Réunion.
}}}

  \vfill

{présentée par}

	\vfill
	
{\textbf{Daniel Rakotomalala}}
	
	\vfill

{le}

  \vfill

{\textbf{12 décembre 2022}}	

  \vfill
	
{pour l'obtention du grade de} 

	\vfill
	
{\textsc{\textbf{{Docteur en sciences économiques}}}} 

	\vfill
	
{devant le jury composé de : }

	\vfill
  
{
\raggedright  

\begin{flalign*}
&\text{Habiba Djebbari} &&\text{Professeure à l'\textit{Aix-Marseille School of Economics} (Rapporteure)} & \text{ }& \\
&\text{Liliane Bonnal} &&\text{Professeure à l'Université de Poitiers (Examinatrice)} & \text{ }& \\
&\text{Yannick L'Horty} &&\text{Professeur à l'Université Gustave Eiffel (Rapporteur)} & \text{ }& \\
&\text{Nicolas Moreau} &&\text{Professeur à l'Université de La Réunion (Président)} & \text{ }& \\
&\text{Jeannot Ramiaramanana} &&\text{Professeur à l'Université Catholique de Madagascar (Co-directeur)} & \text{ }& 
\end{flalign*}

}



\end{center}
\end{titlepage}

\frontmatter

\hfill  
\begin{minipage}{0.45\textwidth}
\vspace{3cm} 
L'université n'entend donner aucune approbation ni improbation aux opinions émises dans les thèses : ces opinions doivent être considérées comme propres à leurs auteurs.
\end{minipage}

\hypertarget{remerciements}{%
\chapter*{Remerciements}\label{remerciements}}
\addcontentsline{toc}{chapter}{Remerciements}

Merci premièrement à mon directeur de thèse, Yves Croissant, pour avoir accepté de diriger ma thèse. Outre toutes vos indispensables aides techniques, toutes nos réunions remotivantes et enrichissantes, il vous a fallu mobiliser patience, disponibilité et mots d'encouragements à mon égard. Merci également aux membres de mon comité de suivi de thèse, qui sont Liliane Bonnal, Alexis Parmentier et Nicolas Moreau, pour les conseils et les encouragements que vous m'avez prodigués. Merci tout particulièrement à Nicolas Moreau pour toutes vos relectures et discussions approfondies de mes travaux et pour avoir permis au troisième chapitre de ma thèse d'exister. Mes remerciements vont aussi au Rectorat de La Réunion et particulièrement à Fabrice Payet et Jean-Eric Parvedy. Si j'ai pu obtenir et comprendre les données que j'exploite dans ma thèse, c'est grâce à vous.

\quad Je suis très honoré et très reconnaissant envers Habiba Djebbari et Yannick L'Horty pour avoir accepté de rapporter mon travail. Je remercie pareillement les autres membres de mon jury de thèse : Liliane Bonnal, Jeannot Ramiaramanana et Nicolas Moreau. Les échanges que nous effectuerons me seront sans doute intéressantes.

\quad Je n'oublie pas tous les collègues du CEMOI ainsi que les membres du personnel administratif. Vous m'aurez permis de me former à la recherche dans un cadre particulièrement bienveillant. Je me dois de remercier particulièrement Anna Genave. On aura partagé le bureau, une amitié et du café aussi. À un de ces jours. Une pensée aux doctorants de ma promotion et à ceux des promotions suivantes. Avec toutes les heures que j'ai passées dans les locaux, le laboratoire était déjà peut-être un deuxième CHEZ-MOI.

\quad À tous mes amis, que je ne saurais citer exhaustivement mais qui se reconnaîtront, où qu'ils soient, merci. Vous avez contribué à ce que je sors grandi de ce parcours. Une pensée à des proches, dont des membres de famille, que j'ai perdus en chemin et que je ne retrouverai pas.

\quad Enfin, je ne saurais jamais assez remercier toute ma famille, particulièrement mes parents, mes deux petites sœurs et ma famille proche en métropole, pour leur présence et leur soutien constants. Je peux affirmer que je n'y serais pas arrivé sans vous.

\hypertarget{abstract}{%
\chapter*{Abstract}\label{abstract}}
\addcontentsline{toc}{chapter}{Abstract}

This thesis is a work on microeconometrics applied to economics of education that analyses some determinants of academic performances in Reunion Island. Compared to the population of metropolitan France, the population of Reunion Island suffers from a context that may hinders human capital accumulation. In the first chapter, we analyse the causal effect of age at test on academic performances in primary and secondary education. We find a strong and significant positive effect at the end of primary school that fades out over time. We additionally show that being younger relatively to classmates boost test scores. In the second chapter, we study peer effects in education. To be more precise, for pupils at the end of lower secondary school, we measure the effect of peers' (defined as classmates) past academic performances and peers' contemporary academic behaviours on pupil's academic performances. We find that the weakest pupils cannot benefit from any peer effect at all. However, the strongest peers have a positive effect on other pupils. The positive and strong effects of peers' academic behaviours that we additionaly figure out suggest that it may be interesting to implement interventions to a fraction of students in the classroom, expecting positive spillovers to their classmates, especially in mathematics. The last chapter focuses on the impact of an online platform that seeks to keep graduate students in touch with mathematics between the end of secondary school and the beginning of university, on first-year university mathematics performances. We observe that the platform was barely used and its use did not impact mathematics skills of students. However, we strongly suspect that the intervention influenced behaviours of the most motivated students in mathematics, in a way suggesting that better implementation parameters of the platform (a longer duration of use, for instance) could result in positive effects.

\textbf{Keywords: } determinants of academic performances, effects of age, peer effects, interventions in mathematics, test scores, instrumental variables.

\hypertarget{ruxe9sumuxe9}{%
\chapter*{Résumé}\label{ruxe9sumuxe9}}
\addcontentsline{toc}{chapter}{Résumé}

Cette thèse est un travail de microéconométrie appliquée à l'économie de l'éducation qui s'intéresse à certains déterminants des performances scolaires et universitaires à La Réunion. Par rapport à la population de la France métropolitaine, celle de La Réunion est marquée par un contexte plus défavorable à l'accumulation de capital humain. Dans le premier chapitre, nous nous intéressons au lien causal entre l'âge au moment des examens et les performances scolaires. Nous trouvons que ce lien est positif et important et s'atténue au fil du temps. Nous mettons en plus en évidence qu'être plus jeune dans la classe implique de meilleures performances scolaires. Dans le deuxième chapitre, nous étudions les effets de pairs en éducation. Plus précisément, nous examinons l'effet du niveau scolaire et des comportements des camarades de classe en fin de collège sur la performance en fin de collège. Il en ressort essentiellement que les élèves les plus faibles scolairement ne bénéficient pas des effets de pairs alors que les autres d'élèves, bénéficient quant à eux de la présence des pairs les plus forts. Les effets positifs et importants du comportement des pairs sur les performances que nous trouvons en plus suggèrent qu'il peut être intéressant de proposer des interventions uniquement à une partie des élèves de la classe pour en faire bénéficier le reste, particulièrement en mathématiques. Le dernier chapitre s'intéresse à l'efficacité d'une plateforme numérique en ligne visant à maintenir le contact avec les mathématiques entre la fin du lycée et le début des études universitaires pour des étudiants néo-bacheliers en première année d'université. Nous constatons que la plateforme a été très peu utilisée par les étudiants et qu'elle n'a eu aucun impact sur les performances aux évaluations de mathématiques en fin de semestre. Toutefois, des effets sur le comportement des plus motivés nous suggèrent que de meilleures configurations de proposition de la plateforme, notamment une durée plus longue de mise à disposition aux étudiants, pourraient générer des effets positifs.

\textbf{Mots-clés : } déterminants des performances scolaires et universitaires, effets de l'âge, effets de pairs, interventions en mathématiques, notes, variables instrumentales.

\hypertarget{acronymes-et-abruxe9viations}{%
\chapter*{Acronymes et abréviations}\label{acronymes-et-abruxe9viations}}
\addcontentsline{toc}{chapter}{Acronymes et abréviations}

ABS : Absolu\\
ABSREL : Absolu et relatif\\
AES : Administration Économique et Sociale\\
APOGEE : Application Pour l'Organisation et la Gestion des Enseignements et des Étudiants\\
BTS : Brevet de Technicien Supérieur\\
CE1 : Cours Élémentaire 1\textsuperscript{ère} année\\
CE2 : Cours Élémentaire 2\textsuperscript{ème} année\\
CM1 : Cours Moyen 1\textsuperscript{ère} année\\
CM2 : Cours Moyen 2\textsuperscript{ème} année\\
CP : Cours Préparatoire\\
CSP : Catégorie Socio-Professionnelle\\
DEPP : Direction de l'Évaluation, de la Prospective et de la Performance\\
DIMA : Dispositif d'Initiation aux Métiers en Alternance\\
DNB : Diplôme National de Brevet\\
DROM : Département et Région d'Outre-Mer\\
Eco-Ge : Économie et Gestion\\
ES : Économique et Social\\
EE : Enquête Emploi\\
EP : Éducation Prioritaire\\
FCH : Fonction de Contrôle avec Hétérogénéité\\
FRD : \emph{Fuzzy Regression Discontinuity}\\
GT : Général et Technologique\\
HEP : Hors Éducation Prioritaire\\
ICILS : \emph{International Computer and Information Literacy Study}\\
IDEOM : Institut d'Émission des Départements d'Outre-Mer\\
INE : Identifiant National de l'Élève\\
INSEE : Institut National de la Statistique et des Études Économiques\\
JDC : Journée Défense Citoyenneté\\
KS : Kolmogorov-Smirnov\\
L : Littéraire\\
MCO : Moindres Carrés Ordinaires\\
NAEP : \emph{National Assessment of Educational Progress}\\
NLSY : \emph{National Longitudinal Survey of Youth Mother-Child}\\
OCDE : Organisation de Coopération et de Développement Économiques\\
PIB : Produit Intérieur Brut\\
PIRLS : \emph{Progress in International Reading Literacy Study}\\
PISA : \emph{Programme for International Student Assessment}/Programme International pour le Suivi des Acquis des élèves\\
Pro : Professionnel\\
REL : Relatif\\
S : Scientifique\\
S2TMD : Sciences et Techniques du Théâtre, de la Musique et de la Danse\\
SEGPA : Section d'Enseignement Général et Professionnel Adapté\\
ST2S : Sciences et Technologies de la Santé et du Social\\
STAR : \emph{Student-Teacher Achievement Ratio}\\
STAV : Sciences et Technologies de l'Agronomie et du Vivant\\
STD2A : Sciences et Technologies du Design et des Arts Appliqués\\
STHR : Sciences et Technologies de l'Hôtellerie et de la Restauration\\
STL : Sciences et Technologies de Laboratoire\\
STMG : Sciences et Technologies du Management et de la Gestion\\
TD : Travaux Dirigés\\
TEMA : \emph{Tests for Early Mathematics Ability}\\
TIMSS : \emph{Trends in Mathematics and Science Study}\\
UE : Unité d'Enseignement\\
UET : Unité d'Écart-Type\\
ULIS : Unités Localisées pour l'Inclusion Scolaire\\
VI : Variable Instrumentale\\
VI-EXCL : Variable Instrumentale - Test indirect de la restriction d'Exclusion (\protect\hyperlink{ref-GOU:MAU:07}{Goux \& Maurin, 2007})

\tableofcontents
\addcontentsline{toc}{chapter}{\listfigurename}
\addcontentsline{toc}{chapter}{\listtablename}
\listoftables
\listoffigures

\mainmatter

\hypertarget{introduction-guxe9nuxe9rale}{%
\chapter*{Introduction générale}\label{introduction-guxe9nuxe9rale}}
\addcontentsline{toc}{chapter}{Introduction générale}

Cette thèse propose trois essais empiriques sur les déterminants des performances académiques dans les trois niveaux d'éducation (primaire, secondaire et supérieure) du système français, sur données réunionnaises.

\hypertarget{le-capital-humain-son-importance-et-le-ruxf4le-des-scores-uxe0-des-uxe9valuations-dans-sa-mesure}{%
\section*{Le capital humain : son importance et le rôle des scores à des évaluations dans sa mesure}\label{le-capital-humain-son-importance-et-le-ruxf4le-des-scores-uxe0-des-uxe9valuations-dans-sa-mesure}}
\addcontentsline{toc}{section}{Le capital humain : son importance et le rôle des scores à des évaluations dans sa mesure}

Le capital humain d'un individu désigne l'ensemble de ses compétences et connaissances, innées ou acquises, qui sont mobilisables pour de la production (\protect\hyperlink{ref-KUC:11}{Kucharčíková, 2011}, par exemple).\\
Gary Becker popularise le lien théorique entre le niveau de capital humain d'un pays et sa croissance économique (voir \protect\hyperlink{ref-BEC:09}{Becker, 2009}, par exemple). Le point de départ de cette théorie est le constat d'une augmentation du produit intérieur brut (PIB) aux États-Unis qui ne peut pas être complètement expliquée par l'augmentation de la taille de la population productive (\protect\hyperlink{ref-WEI:15}{Weiss, 2015}), suggérant qu'une certaine ``qualité de la productivité'' est à l'origine de cette part inexpliquée. Afin de vérifier cette hypothèse empiriquement, les économistes ont principalement mesuré le niveau de capital humain des pays par la durée de scolarité moyenne de leurs populations et ont mis cette mesure en relation avec leur PIB (\protect\hyperlink{ref-WOE:03}{Wößmann, 2003}). Le résultat macroéconomique généralement obtenu est que la durée de scolarité a un effet causal positif sur la croissance (\protect\hyperlink{ref-SIA:REE:03}{Sianesi \& Reenen, 2003}).

\quad Mesurer le niveau de capital humain par la durée de scolarité pose cependant différents problèmes. Notamment, cette mesure ne procure aucune information sur la qualité de la scolarité en question. Les résultats aux évaluations cognitives internationales, constituent vraisemblablement une des meilleures mesures du niveau de capital humain mais ne sont disponibles que sur une période récente. Avec le développement des méthodes d'estimation des liens de causalité, une branche récente de la littérature confirme empiriquement l'existence d'un effet causal positif et significatif du niveau de capital humain, mesuré par les résultats aux évaluations internationales, sur la croissance (\protect\hyperlink{ref-HAN:WOE:15}{Hanushek \& Woessmann, 2015} ; \protect\hyperlink{ref-HAN:WOE:20}{Hanushek \& Woessmann, 2020}). Cette littérature met également en évidence que l'effet causal en question est d'autant plus fort que le contexte institutionnel du pays est de meilleure qualité, même en prenant en compte l'effet direct de la qualité institutionnelle sur la croissance (\protect\hyperlink{ref-HAN:WOE:20}{Hanushek \& Woessmann, 2020, p. 178}). Ce résultat supplémentaire est robuste à différentes mesures de ladite qualité institutionnelle qui peuvent être l'ouverture au commerce international (\protect\hyperlink{ref-TAH:eal:14}{Tahir et al., 2014}) ou l'allocation des compétences (\protect\hyperlink{ref-MUR:eal:91}{Murphy et al., 1991}), entre autres.

Sur le plan microéconomique, il est admis que la durée de scolarité affecte positivement le salaire (\protect\hyperlink{ref-CAR:01}{Card, 2001} ; \protect\hyperlink{ref-HEC:eal:06}{Heckman et al., 2006}). Récemment, une méta-analyse sur 139 pays de 1950 à 2014 de Psacharopoulos \& Patrinos (\protect\hyperlink{ref-PSA:PAT:18}{2018}) fait ressortir qu'un an de scolarité en plus implique en moyenne une augmentation de salaire de 9\%, toutes choses égales par ailleurs.
Le lien de causalité entre les résultats aux évaluations et le salaire n'est quant à lui établi qu'au sein des pays ayant une croissance rapide (\protect\hyperlink{ref-HAN:eal:15}{Hanushek et al., 2015} ; \protect\hyperlink{ref-HAN:eal:17}{Hanushek et al., 2017} ; \protect\hyperlink{ref-WAT:20}{Watts, 2020}).\\
Investir en capital humain est également intéressant pour l'individu et pour la société sur d'autres dimensions. Cela pourrait apporter un meilleur état de santé\footnote{Cela ne semble toutefois pas admis dans le sens où seul l'effet de l'éducation (la mesure imparfaite du niveau de capital humain) sur l'état de santé a été démontré (\protect\hyperlink{ref-SIL:09}{Silles, 2009}), notamment par l'intermédiaire de l'amélioration des comportements favorables à la santé (\protect\hyperlink{ref-BRU:eal:15}{Brunello et al., 2016}). Le problème principal de cette branche de la littérature est que la corrélation positive entre l'éducation et la santé peut traduire aussi bien l'effet de l'éducation sur la santé ou l'effet dans le sens inverse.}, ainsi que des externalités positives sur la société telles que la baisse de la criminalité (\protect\hyperlink{ref-LOC:04}{Lochner, 2004} ; \protect\hyperlink{ref-LOC:20}{Lochner, 2020}) ou l'amélioration de certains comportements civiques (\protect\hyperlink{ref-DEE:20}{Dee, 2020}). Dans le cas de l'effet sur le crime, une des explications possibles est que l'éducation augmente le niveau de capital humain et que cette dernière augmentation réduit la probabilité d'exercice d'activités criminelles futures (\protect\hyperlink{ref-LOC:04}{Lochner, 2004}). De la même manière, l'effet sur le comportement civique peut théoriquement passer par l'acquisition de connaissances et de capacités d'analyses des sujets liés à la civilité.

\hypertarget{la-fonction-de-production-uxe9ducative-et-les-politiques-publiques-uxe9ducatives}{%
\section*{La fonction de production éducative et les politiques publiques éducatives}\label{la-fonction-de-production-uxe9ducative-et-les-politiques-publiques-uxe9ducatives}}
\addcontentsline{toc}{section}{La fonction de production éducative et les politiques publiques éducatives}

Puisqu'il est impossible de contrôler explicitement le niveau de capital humain, il est naturel de s'interroger sur les déterminants de celui-ci.
Le concept clef correspondant est la fonction de production d'éducation. Il s'agit d'une fonction générique qui combine, pour un individu donné, des ``entrées'' spécifiques à cet individu, à sa famille, à son éducation formelle ou à ses camarades\footnote{Camarades au sens large, tels que les amis, les camarades de classe ou les voisins. Le terme ``pairs'' est le plus souvent utilisé par les chercheurs.} pour obtenir une ``sortie'' de type performances éducatives. Cette sortie peut largement désigner le niveau de diplôme atteint, le redoublement de classe, la durée de scolarité ou les notes à des évaluations bien déterminées, entre autres. Elle correspond simplement au niveau de capital humain et les entrées correspondent aux déterminants que nous évoquons.
Nous renverrons principalement à Britton \& Vignoles (\protect\hyperlink{ref-BRI:VIG:17}{2017}) pour une récente et enrichissante revue du concept de la fonction de production éducative. Depuis au moins la fin des années 70 (\protect\hyperlink{ref-BRI:eal:79}{Bridge et al., 1979}), il est d'usage de classifier les déterminants comme susmentionné (individu, famille, éducation formelle et pairs).

\quad Les caractéristiques spécifiques de l'individu susceptibles de jouer directement ou non sur ses performances éducatives sont, sans prétendre à l'exhaustivité, le sexe, l'âge, les potentiels cognitifs et non cognitifs de naissance, l'état de santé initial, l'ethnie et la nationalité. Les caractéristiques familiales renvoient principalement aux parents. Ce sont les facteurs héréditaires (cognitifs, non cognitifs et liés à l'état de santé), l'éducation des parents, et les ressources que ces derniers peuvent investir dans l'éducation de leur enfant. Les déterminants liés à l'éducation formelle de l'individu englobent les caractéristiques de l'établissement scolaire, celles des enseignants et les ressources scolaires. Les premières peuvent concerner le statut (public ou privé), le degré de sélectivité ou le degré de compétitivité. Les secondes désignent essentiellement la qualité des enseignants (année d'expérience, qualification, efficacité pour l'apprentissage ou la pédagogie). Les dernières correspondent aux ressources financières des établissements, à la taille des classes ou au salaire des enseignants. Les caractéristiques des pairs renvoient simplement d'abord à une définition de ce qui est considéré comme pairs (camarades de classe, amis ou voisins, par exemple) puis à une agrégation des caractéristiques spécifiques aux individus considérés comme des pairs. Par exemple, pour des pairs définis au niveau de la classe, les composantes de la fonction de production éducative peuvent être la proportion de filles parmi les camarades de classe ou l'âge moyen de ces derniers. Sacerdote (\protect\hyperlink{ref-SAC:11}{2011}), Sacerdote (\protect\hyperlink{ref-SAC:14}{2014}) et Monso et al. (\protect\hyperlink{ref-MON:eal:19}{2019}) présentent de récentes revues des enjeux, méthodes et résultats sur le rôle des pairs dans la détermination des performances éducatives.

\quad Bien que des contre-arguments existent, il est généralement admis que la construction et la régulation du système scolaire d'un pays doit principalement revenir à l'État et non au marché. Plank \& Davis (\protect\hyperlink{ref-PLA:DAV:20}{2020}) fournissent l'un des plus récents états du débat. En résumé, l'éducation peut être considérée comme un bien privé mais dont l'offre doit être gérée par l'État à cause des imperfections du marchés telles que les externalités positives (dont certaines ont été citées plus haut), l'asymétrie d'information sur la qualité de l'éducation en défaveur des consommateurs que sont les parents et l'incertitude de la réalisation des bénéfices escomptés du point de vue des parents\footnote{Recevoir une certaine quantité d'éducation ne garantit pas absolument un certain niveau de revenu, par exemple.}. Il est alors de la responsabilité de l'État de proposer des mesures pour optimiser le niveau de capital humain des membres de sa société. Les politiques publiques éducatives constituent l'outil conceptuel pour ce faire. Elles consistent à agir sur certains termes de la fonction de production éducative, c'est-à-dire sur certains déterminants des performances.\\
Certains déterminants peuvent être plus influencés par les politiques publiques que d'autres. Clairement, il est difficile, voire impossible pour les décideurs publics d'agir sur les caractéristiques de naissance mentionnées plus haut tandis que des mesures publiques visant certaines caractéristiques des familiales sont concevables. Les politiques publiques qui correspondent aux caractéristiques des pairs sont particulièrement connus pour leur très faible coût puisqu'elles consistent essentiellement à des mesures de réarrangement des individus à travers les groupes afin d'optimiser les performances compte tenu du rôle de ces caractéristiques (\protect\hyperlink{ref-PAL:20}{Paloyo, 2020}).

\hypertarget{des-couxfbts-consuxe9quents-et-des-ruxe9sultats-insuffisants-pour-le-cas-franuxe7ais}{%
\section*{Des coûts conséquents et des résultats insuffisants pour le cas français}\label{des-couxfbts-consuxe9quents-et-des-ruxe9sultats-insuffisants-pour-le-cas-franuxe7ais}}
\addcontentsline{toc}{section}{Des coûts conséquents et des résultats insuffisants pour le cas français}

Le système éducatif représente un coût pour les pays. L'importance susmentionnée du niveau de capital humain sur la croissance des pays justifie le degré d'effort que ces derniers effectuent en éducation. Les pays développés y trouvent particulièrement leur intérêt pour rester dans la compétition économique mondiale (\protect\hyperlink{ref-PLA:DAV:20}{Plank \& Davis, 2020, p. 446}). En guise d'illustration, les pays de l'OCDE (Organisation de Coopération et de Développement Économiques) consacrent, à l'éducation, en moyenne 6.1\% de leur PIB en 1997 (\protect\hyperlink{ref-OCDR:00}{OCDE, 2000, p. 49}), 5.7\% en 2007 (\protect\hyperlink{ref-OCDR:10}{OCDE, 2010, p. 218}) et 4.9\% en 2017 (\protect\hyperlink{ref-OCDR:20}{OCDE, 2020, p. 290})\footnote{Cette diminution vient du fait que le montant des dépenses d'éducation a globalement augmenté mais pas autant que le PIB (\protect\hyperlink{ref-OCDR:20}{OCDE, 2020, p. 290}).}.

\quad La France attire l'attention puisqu'avec un PIB par tête qui se trouve dans la moyenne des pays de l'OCDE (Panel A de la Figure \ref{fig:pibdepeducgraph})\footnote{Confirmé par Ragoucy (\protect\hyperlink{ref-RAG:08}{2008}), p.~46, par exemple.}, elle se situe au-dessus de la moyenne en termes de part de PIB consacré à l'éducation (Panel B de la Figure \ref{fig:pibdepeducgraph})\footnote{Propos confirmés par Ragoucy (\protect\hyperlink{ref-RAG:08}{2008}) et Brière \& Rudolf (\protect\hyperlink{ref-BRI:RUD:11}{2011}).}. Cette comparaison internationale reste vraie à travers le temps et en considérant la dépense annuelle par élève/étudiant comme indicateur de l'effort national en faveur de l'éducation (Panel C de la Figure \ref{fig:pibdepeducgraph}, également confirmé par Dalous et al. (\protect\hyperlink{ref-DAL:eal:11}{2011})).

Malgré ces efforts, les élèves de la France ont généralement des résultats en dessous de la moyenne de l'OCDE aux évaluations internationales portant sur diverses compétences. Par exemple, Colmant \& Le Cam (\protect\hyperlink{ref-COL:LEC:17}{2017}) montrent que pour les performances en compréhension de l'écrit issues de l'évaluation PIRLS 2016 (\emph{Progress in International Reading Literacy}), les élèves en quatrième année d'éducation obligatoire de la France obtiennent un score nettement inférieur la moyenne des pays de l'OCDE. Ces auteurs montrent en plus que les élèves français sont surreprésentés parmi les plus faibles à cette évaluation et 6\% n'atteignent pas le niveau le plus élémentaire\footnote{Des niveaux définis par les responsables de PIRLS selon des seuils de note, à l'instar des mentions aux examens scolaires officiels.} contre 4\% seulement en Europe. Les mêmes faits stylisés sont rapportés par Colmant \& Le Cam (\protect\hyperlink{ref-COL:LEC:20}{2020}) en ce qui concerne les résultats en mathématiques et en sciences des élèves en quatrième année d'éducation obligatoire ayant participé aux éditions 2019 des évaluations TIMSS (\emph{Trends in Mathematics and Science Study}).

\quad En plus de cette forme d'inefficacité coût-résultats, la France est connue pour présenter une association plus forte par rapport à la plupart des autres pays de l'OCDE entre le niveau social des parents et les performances scolaires de leurs enfants. Par exemple, à l'aide d'un indicateur de statut socio-économique et culturel construit à partir des données récoltées lors des évaluations PISA (\emph{Program for International Student Assessment}) sur le niveau d'éducation des parents, leur profession et l'accès du foyer à la culture et à des ressources matérielles (\protect\hyperlink{ref-OEC:13}{OECD, 2013, p. 136}), Chabanon et al. (\protect\hyperlink{ref-CHA:eal:19}{2019, p. 3}) notent qu'aux éditions 2018 des évaluations PISA\footnote{La différence avec les autres évaluations mentionnées plus haut est que les participants de PISA sont des élèves de l'OCDE de 15 ans représentatifs de leurs pays, indépendamment de leur niveau scolaire.}, un point d'écart de cet indicateur en France est associé à l'écart le plus élevé aux résultats aux épreuves d'écrit parmi tous les pays de l'OCDE. Bernigole et al. (\protect\hyperlink{ref-BER:eal:19}{2019, p. 2}) font un constat similaire aux résultats en culture mathématique, culture scientifique et vie de l'élève lors de la même édition de PISA. Ce fait stylisé est également retrouvé par Le Cam \& Pac (\protect\hyperlink{ref-LEC:PAC:19}{2019}) dans une moindre mesure\footnote{La France a participé pour la première fois aux éditions 2018 de cette évaluation et les pays participants ne concernent pas spécifiquement que les pays de l'OCDE et de l'Union européenne.} avec une autre évaluation internationale (ICILS - \emph{International Computer and Information Literacy Study}) destinée aux élèves en huitième année d'éducation obligatoire et qui porte sur la littératie numérique et la pensée informatique.

En plus de confirmer l'existence des inégalités aux résultats d'évaluations internationales en fonction du statut socio-économique et culturel, l'étude de Le Mener et al. (\protect\hyperlink{ref-LEM:eal:17}{2017}) montre que selon les données PISA entre 2000 et 2012, ces inégalités sont relativement récentes et ont augmenté ces 20 dernières années en France plus que dans les autres pays de l'OCDE (p.~2). De plus, ces auteurs attribuent en partie cette augmentation à des facteurs liés au système scolaire. Dans le même ordre d'idée, Cayouette-Remblière et al. (\protect\hyperlink{ref-CAY:19}{2019}) montrent que cette forme d'inégalité augmente lors du collège, la partie non sélective\footnote{C'est-à-dire sans différenciation de parcours scolaire en fonction des performances académiques.} de l'éducation secondaire.

\begin{figure}[H]

{\centering \includegraphics[width=1\linewidth]{000_files/figure-latex/pibdepeducgraph-1} 

}

\caption{La France dans l'OCDE : PIB par habitant, part du PIB affectée à l'éducation et dépense d'éducation par élève (hors éducation de la petite enfance)}\label{fig:pibdepeducgraph}
\end{figure}

\hypertarget{la-ruxe9union-un-drom-uxe0-fort-potentiel-damuxe9liorations}{%
\section*{La Réunion, un DROM à fort potentiel d'améliorations}\label{la-ruxe9union-un-drom-uxe0-fort-potentiel-damuxe9liorations}}
\addcontentsline{toc}{section}{La Réunion, un DROM à fort potentiel d'améliorations}

Les moyennes nationales pour la France masquent d'importantes différences territoriales, notamment entre la France métropolitaine et les Départements et Régions d'Outre-Mer (DROM), et en particulier La Réunion.

Sur le plan démographique par exemple, à l'exception notable de la Guadeloupe et de la Martinique, les DROM sont constitués de populations plus jeunes que celles de l'hexagone et avec des taux de croissance plus élevés (\protect\hyperlink{ref-BRE:eal:21}{Breton et al., 2021}, par exemple). Plus particulièrement, l'âge médian calculé en 2020 pour La Réunion est de 37.1 ans, presque 4 ans de moins qu'en France métropolitaine (\protect\hyperlink{ref-BRE:eal:21}{Breton et al., 2021, p. 29}). Toutefois, la population réunionnaise est en train de vieillir petit à petit (\protect\hyperlink{ref-IDE:20}{IDEOM, 2020, p. 31}).

Le contexte culturel est également considérablement différent dans les DROM et entre chaque DROM. En ce qui concerne La Réunion, elle est entre autres caractérisée par une grande proportion de la population pour laquelle la langue française n'est pas la langue maternelle\footnote{Monteil (\protect\hyperlink{ref-MON:10}{2010}) ; Cour des comptes (\protect\hyperlink{ref-COU:20}{2020, p. 74}) ; Daleau-Gauvin (\protect\hyperlink{ref-DAL:21}{2021, p. 15}) ou Valat (\protect\hyperlink{ref-VAL:21}{2021, p. 138}).}, en parallèle avec sa pluriethnicité, du fait de son histoire (\protect\hyperlink{ref-CAR:12}{Carron, 2012, p. 15}).\\
Du côté social, La Réunion est également marquée par plus de familles monoparentales (\protect\hyperlink{ref-INS:21}{INSEE, 2021, p. 107}; \protect\hyperlink{ref-MAR:BRE:15}{Marie \& Breton, 2015, p. 59}) et des ménages de plus grande taille (\protect\hyperlink{ref-AJI:15}{Ajir, 2015})\footnote{La taille des ménages diminue à travers le temps mais cette diminution se retrouve également en France métropolitaine.}.

Des différences existent également en ce qui concerne le système éducatif. Les établissements scolaires sont moins bien répartis dans les DROM qu'en métropole (\protect\hyperlink{ref-VAL:21}{Valat, 2021, p. 118}). En particulier, La Réunion fait partie des académies dont certaines communes pour lesquelles le collège le plus proche à vol d'oiseau se situe à plus de 20 kilomètres (\protect\hyperlink{ref-CAR:18}{Caro, 2018, p. 49}). Cela suggère un coût plus élevé pour se rendre à l'établissement scolaire, en termes de temps de transport ou de recours à l'internat. Les établissements possèdent moins de matériels informatiques, surtout dans les écoles primaires avec moins de 10 ordinateurs pour 100 élèves alors que pour la grande majorité des académies métropolitaines, cet indicateur est deux fois plus élevé (\protect\hyperlink{ref-DEP:21GEO}{DEPP, 2021a, p. 49}). La Réunion est également marquée par une grande proportion d'établissements scolaires en éducation prioritaire (\protect\hyperlink{ref-COU:20}{Cour des comptes, 2020, p. 16}), suggérant des coûts élevés. Récemment, il a été évalué que 40\% du budget des outre-mer pour l'éducation est attribué à La Réunion. Le coût d'enseignement par élève est plus élevé qu'en métropole, notamment du fait de la sur-rémunérations des enseignants (\protect\hyperlink{ref-COU:20}{Cour des comptes, 2020, p. 92}).

\quad Ces moyens ne vont pas de pairs avec des résultats scolaires satisfaisants. En guise d'illustration, Si Moussa (\protect\hyperlink{ref-SIM:15}{2015}) rapporte que 42\% des élèves réunionnais ont obtenu la moyenne lors des évaluations du CM2 de l'année scolaire 2007-2008 alors que cette proportion est de 65\% au niveau national. Ce constat n'est qu'un exemple parmi d'autres illustrant la difficulté relative des élèves réunionnais. Toujours selon le rapport de la Cour des comptes (\protect\hyperlink{ref-COU:20}{2020}), les élèves réunionnais connaissent plus de difficultés par rapport à la métropole bien avant le CM2 (p.~26) et après (p.~27). Ce type de différence est très probablement structurel, les taux d'illettrisme des jeunes (17 à 25 ans) de nationalité française constatés lors des Journées Défense et Citoyenneté (JDC) de 2009 à 2015 étant plus de 10 points de pourcentage supérieurs à La Réunion par rapport à la France métropolitaine\footnote{Par exemple, ce taux est de 4.5\% en métropole en 2009 contre 15\% à La Réunion. Avec une tendance à la baisse dans les deux champs, ils sont de 3.6\% en métropole contre 14.8\% à La Réunion (\url{http://www.anlci.gouv.fr/Illettrisme/Les-chiffres/Niveau-regional/Journees-Defense-Citoyennete-en-region} puis consulter ``Chiffres par région'').}.\\
Face à ses résultats problématiques qui reflètent probablement des problèmes d'apprentissage, les élèves réunionnais sont soumis au risque de décrochage scolaire. Baktavatsalou et al. (\protect\hyperlink{ref-BAK:eal:19}{2019, p. 2}) montrent par exemple sur des données de 2016 que 29\% des réunionnais de 15 à 29 ans sortent du système scolaire sans diplôme qualifiant, ce qui est deux fois plus élevé qu'en métropole. Le décrochage est vraisemblablement un phénomène en baisse mais reste élevé vu que sur des données de 2017, l\textquotesingle{}IDEOM (\protect\hyperlink{ref-IDE:20}{2020, p. 125}) rapporte que ce taux pour les 15 à 24 ans est de 22.1\%.

\quad En partie à cause des difficultés ci-dessus, La Réunion se retrouve marquée par un contexte social difficile, illustré entre autres par Boudesseul et al. (\protect\hyperlink{ref-BOU:eal:16}{2016, p. 149}) qui montrent qu'elle est le DROM avec le plus faible taux de scolarisation des 15 à 24 ans et le taux de chômage des 15 à 64 ans le plus élevé\footnote{Le problème du chômage à La Réunion persiste, comme le montre l\textquotesingle{}INSEE (\protect\hyperlink{ref-INS:21}{2021, p. 149}) par exemple.}. La pauvreté y est plus marquée qu'en France métropolitaine. Par exemple, l\textquotesingle{}INSEE (\protect\hyperlink{ref-INS:21}{2021, p. 192}) rapporte que 39\% des réunionnais vivent sous le seuil de pauvreté en 2018 contre 15.1\% au niveau national. Le rapport de l\textquotesingle{}IDEOM (\protect\hyperlink{ref-IDE:20}{2020, p. 16}) appuie ces propos.

\quad Tous ces éléments indiquent que la population réunionnaise présente \emph{a priori} des caractéristiques défavorables à l'accumulation de capital humain, est accueilli par un système éducatif relativement moins efficace, et sort du système éducatif dans une situation difficile (sans diplôme, au chômage et touchée par la pauvreté).

\hypertarget{pruxe9sentation-de-la-thuxe8se}{%
\section*{Présentation de la thèse}\label{pruxe9sentation-de-la-thuxe8se}}
\addcontentsline{toc}{section}{Présentation de la thèse}

\hypertarget{les-questions-de-recherche}{%
\subsection*{Les questions de recherche}\label{les-questions-de-recherche}}
\addcontentsline{toc}{subsection}{Les questions de recherche}

Cette thèse analyse trois spécifications de la fonction de production éducative pour La Réunion. Dans toutes les spécifications, le niveau de capital humain est mesuré par des notes à des évaluations.

\quad Avant la rentrée scolaire de 2019-2020, les parents sur le territoire français doivent envoyer leur enfant à l'école dès ses 6 ans mais avec possibilité de décaler ou d'avancer l'entrée à l'école. Cette décision est importante pour les parents puisqu'ils souhaitent généralement que leur enfant ait la meilleure performance scolaire possible. Les parents font de plus partie de ceux qui estiment le mieux si l'enfant est prêt pour l'école ou non, vu sa maturité physique, intellectuelle et émotionnelle. Envoyer son enfant à l'école conduit également à une réduction des coûts de garde d'enfant. En outre, dans l'hypothèse où la scolarité de l'enfant est sans embûche, l'envoyer au plus tôt à l'école permet une insertion plus précoce sur le marché du travail, ce qui augmente le revenu tout au long de sa vie.\\
De leur côté, les enseignants devront adapter leur pédagogie à l'âge des élèves s'il y a une différence d'apprentissage entre les élèves d'âges différents. \\
Pour leur part, les décideurs publics doivent par exemple décider des règles liées à l'âge d'entrée à l'école, à la durée de l'éducation obligatoire, à la composition d'âge de la classe et aux dates des évaluations. Ils doivent en plus prendre en compte les coûts qui y sont liés.

Pour ces raisons, il est intéressant d'analyser une première fonction de production d'éducation qui spécifie le lien entre l'âge des élèves et leurs performances scolaires.

\quad Les parents sont également soucieux de l'environnement scolaire dans lequel leur enfant évolue. Ils peuvent notamment craindre que trop d'interactions avec d'autres élèves ayant de mauvais résultats scolaires ou issus de familles défavorisées ne nuise à l'apprentissage et donc aux résultats scolaires de leur enfant. Selon leurs moyens, les parents peuvent alors choisir l'école de son enfant (donc leur lieu de résidence si nécessaire) en fonction de cette considération. \\
Les enseignants quant à eux doivent également adapter leurs enseignements en fonction du niveau scolaire de leur classe (moyenne, composition).\\
Les décideurs, de leur côté, doivent se demander comment affecter les élèves de différents niveaux scolaires et différentes origines sociales dans différentes classes dans le but d'en tirer le maximum de performances scolaires, entre autres.

Une deuxième spécification de la fonction de production d'éducation qui mérite une analyse approfondie concerne alors les relations entre les caractéristiques/comportements de ceux qui interagissent avec l'élève dans le cadre de sa scolarité et les performances scolaires de ce dernier.

\quad Passé les 16 ans, les élèves du système éducatif français peuvent décider de poursuivre leurs études ou non. Pour ceux qui décident de le faire et qui intègrent l'université, l'échec en première année constitue un obstacle considérable. Les causes sont multiples. Par exemple, il y a moins de cadrage au niveau national sur le contenu et les modalités des enseignements et des évaluations (programmes scolaires, examens nationaux, etc.). Aussi, contrairement au secondaire, il n'existe pas d'objectif national tel que la réussite au baccalauréat pour tous dans le supérieur. Les individus deviennent alors beaucoup plus dépendants de leurs moyens propres pour la réussite des études supérieures (capacités innées et acquises, motivation, environnement familial, etc.). À cela s'ajoute un grand écart du niveau des enseignements dans le supérieur par rapport au secondaire, notamment en mathématiques (\protect\hyperlink{ref-GUE:VAN:22}{Gueudet \& Vandebrouck, 2022}). Rajouté à une tendance globale de baisse du niveau des élèves en mathématiques, il est raisonnable d'avancer qu'actuellement, les individus qui se présentent aux portes de l'université n'ont pas le niveau suffisant en mathématiques. Cela constitue potentiellement un facteur important d'échec en première année, particulièrement dans un contexte où les mathématiques sont nécessaires à l'assimilation de savoirs et de compétences pour les enseignements proposés par l'université en question. Les filières liées à l'économie, en particulier, sont concernées.

Les constats de la section précédente sur La Réunion suggèrent que cette transition secondaire-supérieur est relativement plus dure pour les élèves réunionnais qui doivent déjà faire face à d'autres difficultés importantes. Il n'est pas évident que les parents soient particulièrement conscients de cette transition et la responsabilité des enseignants du secondaire sur la scolarité de l'élève s'achève au moment où celui-ci obtient son bac. S'il faut atténuer les difficultés des élèves dues à cette transition, cela relève du rôle des décideurs publics et des acteurs de la formation supérieure.

Sur ces motivations, la troisième et dernière spécification de la fonction de production d'éducation analysée dans cette thèse relève de l'efficacité d'une intervention visant à maintenir le contact entre les mathématiques et des diplômés du secondaire de La Réunion inscrits à l'université.

\quad Les questions de recherche auxquelles nous tenterons de répondre sont alors les suivantes. (i) Quel est l'effet causal de l'âge au moment des évaluations sur les performances éducatives des élèves réunionnais en fin de primaire, et persiste-t-il jusqu'à la fin du collège ? (ii) Quel est l'effet du niveau scolaire et du comportement des camarades de classe sur les résultats scolaires des réunionnais ? (iii) Une plateforme de révision de mathématiques du lycée administrée aux néo-bacheliers de l'Université de La Réunion est-elle efficace ?

\hypertarget{duxe9marches-et-contributions}{%
\subsection*{Démarches et contributions}\label{duxe9marches-et-contributions}}
\addcontentsline{toc}{subsection}{Démarches et contributions}

Les tentatives de réponse aux trois questions de recherche ci-dessus se structurent en trois chapitres.

\quad Le premier chapitre tient à répondre à la première question de recherche sur données réunionnaises. Nous mobilisons principalement deux bases de données exhaustives sur les résultats aux évaluations nationales en fin de primaire et en fin de collège de La Réunion. Pour chaque élève scolarisé dans un établissement scolaire public ou privé sous-contrat, nous disposons des résultats en français, en mathématiques aux évaluations dits de Cours Moyen 2\textsuperscript{ème} année (CM2)\footnote{La cinquième année dans l'éducation obligatoire avant 2019.} pour les quatre années scolaires allant de 2008-2009 à 2011-2012. Pour trois de ces cohortes d'élève en fin de primaire (2009-2010 à 2011-2012) et pour la grande majorité de ces derniers, nous disposons de leurs résultats aux examens officiels marquant la fin de la scolarité au collège, les épreuves du Diplôme National de Brevet (DNB), également en français et en mathématiques. Dans les deux bases, l'âge au moment de ces évaluations est calculé pour chaque élève grâce à sa date de naissance exacte qui est disponible et aux dates connues des évaluations. D'autres informations sur chaque élève, sa classe et son école sont utilisées dans le but de mesurer toutes choses égales par ailleurs le lien entre l'âge au moment des évaluations et les résultats. Ce travail est à notre connaissance le premier sur le sujet pour La Réunion \footnote{Les deux seuls travaux français similaires sont Grenet (\protect\hyperlink{ref-GRE:09}{2009}) et Fleury (\protect\hyperlink{ref-FLE:12}{2012}).}.

La prise en compte des caractéristiques observables disponibles des élèves ne permet pas de dégager l'effet causal de l'âge à cause de deux types de phénomènes de sélection : le décalage ou l'avancement de l'entrée à l'école et le redoublement ou le saut de classe. Les chercheurs proposent en général une approche par variables instrumentales en ne considérant que la variation de l'âge due à la variation de la date de naissance dans leurs estimations. Cette approche présente deux limites qui concernent deux hypothèses sur lesquelles reposent l'estimation par variables instrumentales : l'exogénéité de l'instrument et la monotonie. Nous justifions avec les données que la première critique n'est pas justifiée pour La Réunion et nous surmontons la seconde critique en utilisant une méthode d'estimation alternative, dite approche par fonction de contrôle (\protect\hyperlink{ref-WOO:15}{Wooldridge, 2015}). Pour le primaire, nous sommes en capacité d'estimer en plus un modèle de régression sur une discontinuité. \\
Nous exploitons en plus deux éléments pour tenter de séparer différents effets spécifiques de l'âge clairement définis dans la littérature et qui ont des implications différentes en termes de politiques publiques. D'une part, les élèves de 2012 ont passé les évaluations de CM2 en étant environ 4 mois plus âgés par rapport à leurs homologues des années scolaires précédents pour les mêmes valeurs moyennes d'âge d'entrée. D'autre part, en fin de primaire comme en fin de collège, nous sommes capables d'identifier les élèves de la même classe.

Nos résultats pour le primaire démontrent des effets positifs et forts de l'âge au moment des examens. Nous avons des premiers éléments qui semblent indiquer que ces effets sont surtout dus à l'âge d'entrée. De plus, nous ne détectons pas d'effet spécifique à la maturité elle-même, ce qui vient renforcer le dernier propos. Les effets estimés pour le collège sont en somme atténués mais restent significatifs et considérables. Nous trouvons de plus un effet négatif de l'âge relatif, signifiant qu'être plus jeune dans la classe constitue en soi un avantage sur les résultats\footnote{Ou symétriquement qu'être plus âgé dans la classe constitue en soi un désavantage en termes de résultats.}.

\quad Le deuxième chapitre traite de la deuxième question de recherche. Pour ce faire, nous effectuons une liaison entre les trois bases de données au DNB ci-dessus pour récupérer la note au CM2 des candidats au DNB concernés, ce qui permet d'obtenir pour chacun d'eux la note moyenne au CM2 de ses camarades classe de 3\textsuperscript{ème}. Les camarades de classe ici sont dénommés des pairs. Cette moyenne sert d'une première mesure du niveau scolaire des pairs. En effectuant les mêmes calculs pour les informations individuelles disponibles dans la base DNB (part de filles parmi les pairs, par exemple), nous obtenons des mesures des autres caractéristiques des pairs. Ces informations nous permettent \emph{a priori} d'estimer l'effet du niveau scolaire des pairs sur les performances en fin de collège.

Il existe cependant des problèmes connus de sélection des élèves à travers les établissements et à travers les classes, à établissement donné, ce qui peut biaiser les résultats. Nous sommes confrontés spécifiquement à un problème supplémentaire de données manquantes puisque nous ne pouvons pas récupérer la note au CM2 de tous les candidats au DNB du fait de la limite des 4 années scolaires au CM2 et des phénomènes de redoublement et de saut de classe entre le CM2 (y compris) et la 3\textsuperscript{ème} (y compris). Il est alors impossible de calculer la vraie note moyenne au CM2 des pairs. Nos données nous permettent de surmonter le problème de sélection à travers les établissements en utilisant des effets fixes d'établissement dans les estimations. Quant à la sélection à travers les classes, nous restreignons l'échantillon d'estimation une première fois en enlevant des classes qui regroupent explicitement des élèves dans certaines classe, de manière très corrélée avec leur note au CM2. Sans ces classes, nous restreignons l'échantillon une seconde fois en enlevant quelques établissements dans lesquels des tests statistiques appropriés détectent une probable sélection des élèves dans les classes en fonction de leur note au CM2, d'une manière ou d'une autre. Nous utilisons le modèle de Sojourner (\protect\hyperlink{ref-SOJ:13}{2013}) pour traiter le problème de données manquantes de CM2, un modèle particulièrement adapté à notre contexte. Il consiste à prendre en compte dans les estimations la proportion, parmi les pairs, d'individus pour lesquels la note au CM2 des pairs est disponible. Nous considérons des spécifications plus flexibles du niveau scolaire des pairs autre que la note moyenne au CM2.\\
Aidé d'un modèle développé par Lee (\protect\hyperlink{ref-LEE:07}{2007}) et d'une méthode d'estimation récemment utilisée par Boucher et al. (\protect\hyperlink{ref-BOU:eal:14}{2014}), nous tentons d'estimer l'effet du comportement des pairs (du CM2 ou du DNB, respectivement) sur les performances scolaires au CM2 ou DNB. La mesure du comportement des pairs renvoie cette fois à la note moyenne au CM2 ou au DNB, respectivement de ces derniers\footnote{Uniquement la note moyenne, contrairement au cas ci-dessus dans lequel nous pouvons considérer d'autres spécifications.}. Cette seconde approche pour traiter la deuxième question de recherche possède au moins deux avantages, à savoir la possibilité d'estimer le modèle pour le primaire et pour le collège et de comparer les résultats et de traiter directement toute forme de sélection des élèves à travers les établissements ou les classes vu que des effets fixes de classe peuvent être utilisées. Le paramètre qui fait varier la mesure du comportement des pairs et qui permet ainsi l'identification est la taille de classe qui ne peut être pratiquement fixe pour toutes les classes.

Nous montrons essentiellement un lien entre le niveau scolaire des pairs du collège et leurs performances, les camarades les plus forts ont un impact positif sur les élèves de tous les niveaux scolaires, à l'exception des plus faibles. Aussi, nous suspectons fortement un lien entre le comportement des pairs (au primaire ou collège) et leurs résultats, un lien plus fort au collège qu'au primaire. Dans les deux approches, les effets en mathématiques apparaissent plus élevés que ceux en français.

\quad Dans le troisième et dernier chapitre de cette thèse qui traite de la dernière question de recherche, nous avons développé un protocole d'encouragement pour les néo-bacheliers souhaitant s'inscrire en première année d'Économie-Gestion ou d'Administration Économique et Sociale à l'Université de La Réunion pour l'année universitaire 2020-2021 et intéressés par la plateforme. La plateforme propose un compte individuel à travers lequel l'utilisateur peut bénéficier de vidéos de révision et de feuilles d'exercices corrigés. Un ordinateur avec une connexion internet est nécessaire. À une moitié uniquement des participants, explicitement choisis au hasard, de courtes incitations par courrier électronique ont été envoyées. L'expérimentation a duré environ 4 semaines. Ces 4 semaines représentent alors la durée pendant laquelle les étudiants avaient accès à la plateforme et pendant laquelle la moitié incitée a reçu les incitations. Nous lions trois bases de données. La première contient de nombreuses informations sur les participants, à savoir leurs caractéristiques socio-économiques, des informations sur leurs établissements d'origine, leurs profils et performances au bac de 2020. La deuxième source de données est la plateforme elle-même. Nous y collectons différentes mesures simples de l'utilisation de la plateforme. La troisième source de données est une base spécifique aux universités françaises à partir de laquelle nous extrayons principalement les résultats du premier semestre de l'année 2020-2021 en mathématiques, pour chaque étudiant.

Comme dans la majorité des interventions, la décision des étudiants d'utiliser la plateforme n'est pas aléatoire. Cela implique un biais de sélection classique de variable omise. Le protocole d'encouragement est expressément construit pour surmonter cette difficulté en utilisant l'incitation comme variable instrumentale de la mesure d'utilisation de la plateforme considérée.

Nous trouvons que les étudiants ont très peu utilisé la plateforme. Toutefois, il y a bien une différence significative entre la quantité d'utilisation de ceux qui ont été incités et celle de ceux qui ne l'ont pas été. Pour le peu d'étudiants ayant utilisé plateforme grâce à l'incitation, nos résultats suggèrent qu'ils s'en sont servis uniquement pour tenter d'augmenter leurs notes sans réel effet sur leurs connaissances et compétences en mathématiques. Nous arrivons à caractériser cette sous-population et justifions qu'il s'agit de personnes uniquement motivées mais pas spécialement plus forts \emph{a priori} en mathématiques.

\hypertarget{age}{%
\chapter{Les effets de l'âge sur les performances éducatives en fin de primaire et en fin de collège à La Réunion : estimations par fonction de contrôle et par régressions sur une discontinuité}\label{age}}

\chaptermark{Âge aux examens et performances éducatives}

\newpage

\hypertarget{ageintro}{%
\section{Introduction}\label{ageintro}}

La plupart des pays fixent un âge minimum d'entrée à l'école via une date seuil. En France, les parents doivent inscrire leur enfant en primaire si ce dernier fête ses 6 ans avant le 31 décembre de l'année civile de la date de rentrée scolaire (\protect\hyperlink{ref-BED:DHU:06}{Bedard \& Dhuey, 2006}, Annexe 1). Combinée à la distribution de dates de naissance pour une année donnée, une telle règle, si elle est strictement suivie, implique que les élèves nés en janvier sont un an plus âgés que ceux nés en décembre. Ces élèves sont exposés quasiment aux mêmes enseignements. Lorsqu'ils sont testés au bout d'une certaine durée d'études, il est connu qu'à un âge précoce, ceux nés en janvier présentent de meilleures performances comparés à leurs camarades nés en décembre (\protect\hyperlink{ref-DAT:06}{Datar, 2006}).\\
Ces différentiels pourraient entraîner des différences à plus long terme s'ils entraînent des différentiels de parcours scolaire et d'opportunités professionnelles en défaveur des plus jeunes.

\quad Au moins trois types d'effets distincts peuvent expliquer ces différences (\protect\hyperlink{ref-CRA:eal:07}{Crawford et al., 2007}). Le premier est un effet d'âge d'entrée qui suppose que les plus âgés ont de meilleurs résultats car ils sont entrés à l'école en étant plus âgés, soit plus préparés, et donc plus aptes à assimiler les enseignements prévus. Le deuxième est un effet d'âge absolu qui attribue la différence de résultats à une différence de maturité au moment des évaluations\footnote{On peut penser que les élèves plus âgés ont de meilleures aptitudes car ils ont un cerveau plus développé, par exemple.}. Il est également appelé effet de l'âge aux examens. Le troisième est un effet d'âge relatif \(-\) par rapport à ses camarades de classe \(-\) qui est lié au fait qu'être parmi les plus âgés de la classe peut procurer un avantage via la confiance en soi, par exemple (\protect\hyperlink{ref-CAS:08}{Cascio et al., 2008}).\\
Mesurer séparément ces effets est important dans la mesure où ils ont des implications différentes en termes de politiques publiques. Il s'agit d'un exercice difficile dans les pays où une unique date seuil est utilisée car l'âge d'entrée et l'âge aux examens varient parallèlement\footnote{L'âge aux examens étant la somme de l'âge d'entrée et de la durée d'études.}. L'âge d'entrée et l'âge aux examens varient aussi chacun parallèlement avec l'âge relatif si les élèves nés à certaines périodes de l'année ne sont pas systématiquement regroupés dans certaines classes.\\
Autrement dit, s'il y a une différence de performances entre des élèves nés en janvier et d'autres nés en décembre, il est difficile de savoir si c'est parce que les premiers sont entrés un an plus tard, sont plus matures au moment des évaluations, ou sont plus âgés que leurs camarades de classe.

\quad En France, mesurer séparément ces différents effets est également rendu difficile en soi à cause de trois facteurs institutionnels. Le premier est que la règle de la date de seuil n'est pas strictement suivie puisque certains parents peuvent décaler ou avancer l'entrée de leur enfant en primaire en fonction de caractéristiques qui peuvent influer sur ses résultats scolaires et qui sont inobservées par le chercheur. Typiquement, les parents décalent ou avancent l'inscription de leur enfant car ils jugent que ce dernier n'est pas prêt pour l'école même s'il fête ses 6 ans avant la fin de l'année civile, ou est prêt même s'il a moins de 6 ans, respectivement. Ceux qui sont entrés tardivement (respectivement en avance) ont des caractéristiques inobservées plutôt défavorables (respectivement plutôt favorables) à la réussite scolaire. Le deuxième facteur est la pratique du redoublement et du saut de classe puisque le redoublement (respectivement le saut de classe) d'un enfant est un signal de présence de caractéristiques inobservés plutôt défavorables (respectivement plutôt favorables) à la réussite scolaire (\protect\hyperlink{ref-ALE:eal:13}{Alet et al., 2013}). Le dernier facteur est le possible regroupement des élèves en fonction de leur âge puisque cela serait plus pratique pour les enseignants, par exemple.

\quad Ce chapitre mobilise des données exhaustives pour mesurer le lien causal entre l'âge aux examens et les performances en fin de primaire et en fin de collège à La Réunion. Pour chaque année scolaire, nous exploitons la disponibilité de la date de naissance exacte afin d'extraire la variation de l'âge aux examens uniquement due à la variation du jour de naissance dans l'année. Nous montrons que cette variation peut être considérée comme exogène dans notre cas puisqu'elle est la même à travers les différentes strates sociales des parents. Contrairement à la majorité des articles usant de stratégies d'identification similaires, nous n'estimons pas l'effet causal d'intérêt par variable instrumentale classique puisque l'hypothèse de monotonie (\protect\hyperlink{ref-IMB:ANG:94}{Imbens \& Angrist, 1994}) qui permet de donner un sens précis au paramètre estimé correspondant n'est pas vérifiée dans notre étude (\protect\hyperlink{ref-BAR:LAN:09}{Barua \& Lang, 2009} ; \protect\hyperlink{ref-FIO:STE:21}{Fiorini \& Stevens, 2021}). Nous mobilisons plutôt une approche par fonction de contrôle (\protect\hyperlink{ref-WOO:15}{Wooldridge, 2015}) qui permet d'estimer un effet du type ``effet moyen de traitement'' qui ne repose pas sur l'hypothèse de monotonie. Les données à disposition pour le primaire nous permettent également de mesurer les effets de l'âge aux examens grâce à un modèle de régression sur une discontinuité qui présente l'avantage de pouvoir isoler d'éventuels effets directs de la date de naissance et de combler l'absence d'études sur le lien entre l'âge et les résultats scolaires à La Réunion et leur rareté en France. Avoir à disposition les résultats en fin de collège est également un avantage car cela permet d'analyser les effets de l'âge à moyen terme. Des effets de l'âge aux examens en primaire qui persistent ou s'aggravent au fil du temps devraient renforcer la préoccupation des décideurs publics sur le sujet. Puisqu'il est possible d'identifier les élèves d'une même classe, aussi bien en primaire qu'au collège, nous exploitons la variation de la position d'âge de la classe pour un même âge aux examens afin d'isoler l'effet de l'âge relatif (voir par exemple \protect\hyperlink{ref-CAS:SCH:16}{Cascio \& Schanzenbach, 2016}). La crédibilité d'un tel exercice repose sur l'absence de regroupement systématique des élèves nés à certaines périodes de l'année dans certaines classes. Notre contexte et nos données vont dans le sens de cette hypothèse. Nous tentons enfin de mesurer l'effet de l'âge absolu au primaire \emph{per se} en exploitant le fait que les évaluations du primaire ont été décalées de 4 mois pour une année scolaire donnée. Ce décalage est en effet vraisemblablement exogène et procure 4 mois de plus aux élèves concernés par rapport aux élèves des autres années.

\quad Les différentes stratégies d'identification ci-dessus conduisent aux résultats suivants. Sans tenter de séparer les effets de l'âge aux examens de celui de l'âge relatif, les estimations avec l'approche par fonction de contrôle montrent qu'un an de plus au moment des examens donne un avantage d'environ 0.3 unité d'écart-type (UET dans le reste du texte) de note en fin de primaire. Il s'agit d'un effet important mais cumulé avec celui de l'âge d'entrée. L'effet est plus fort chez les filles et chez les enfants issus de familles très favorisées. L'estimation des modèles de régressions sur une discontinuité conduisent à des coefficients de plus grande ampleur (entre 0.5 et 0.7 UET) mais robustes et également plus forts chez les filles. En fin de collège, les effets diminuent en ampleur mais restent significatifs. Les hétérogénéités des effets en fonction du sexe et de la catégorie sociale ne sont plus détectées.

Lorsque nous tentons d'isoler l'effet de l'âge relatif, nous trouvons que celui-ci est négatif aussi bien au primaire qu'au collège. Cela signifie qu'à âge absolu égal, être parmi les plus jeunes de la classe constitue un avantage. Cela peut suggérer que les enseignants portent plus d'attention aux plus jeunes de la classe ou bien que les plus âgés de la classe influencent positivement leurs plus jeunes camarades.

Enfin, être plus mature en valeur absolu de 4 mois en plus ne semble pas avoir d'effet significatif sur les notes en fin de primaire.

\quad Les sections suivantes sont organisées comme suit. La Section \ref{agelitt} établit l'état du débat sur les liens entre l'âge et les performances scolaires. La Section \ref{ageinst} décrit le contexte institutionnel du système éducatif de La Réunion. La Section \ref{agedata} discute des données mobilisées pour atteindre l'objectif du chapitre. La Section \ref{agemethods} développe les différentes stratégies d'identification utilisées. La Section \ref{ageres} expose et interprète les résultats et la Section \ref{ageconcl} conclue le chapitre notamment en proposant des préconisations de politiques publiques.

\hypertarget{agelitt}{%
\section{Revue de littérature}\label{agelitt}}

À quel âge doit-on faire entrer son enfant à l'école ? Il s'agit d'une préoccupation majeure des parents. D'une part, inscrire son enfant au plus tôt à l'école minimise les coûts de garde d'enfant associés aux années supplémentaires hors de l'école, en plus de maximiser le revenu tout au long de la vie via une insertion professionnelle précoce. Certains résultats de recherche semblent de plus indiquer que les plus jeunes sont plus réceptifs à l'éducation (\protect\hyperlink{ref-MAY:KNU:97}{Mayer \& Knutson, 1997}). D'autre part, inscrire son enfant à l'école sans que ce dernier ait atteint un niveau de maturité émotionnelle et intellectuelle suffisant peut considérablement nuire à son apprentissage et par conséquent à son parcours scolaire et à la situation professionnelle qui en découle.\\
Cette question est également une préoccupation des pouvoirs publics dans la mesure où de larges différences en performances scolaires dues au seul facteur d'âge d'entrée peuvent se transformer en larges différences en termes de situations professionnelles. En effet, les enfants ne choisissent pas leur date de naissance alors que cette dernière implique, avec la mise en place d'une unique date seuil d'entrée à l'école, que les plus jeunes d'une cohorte scolaire en début d'éducation sont 1 an plus jeunes que leurs camarades les plus âgés.

\quad Ces considérations ont induit une littérature importante concernant les effets de l'âge d'entrée à l'école sur différentes mesures en éducation et sur l'avenir professionnel. Stipek (\protect\hyperlink{ref-STI:02}{2002}) propose une revue des études menées sur le sujet entre 1980 et les débuts des années 2000 tandis que Peña (\protect\hyperlink{ref-PEN:17}{2017}) recense des articles plus récents. Par exemple, Langer et al. (\protect\hyperlink{ref-LAN:eal:84}{1984}) mobilisent des données de NAEP de 1979 (\emph{National Assessment of Educational Progress}), une évaluation nationale américaine sur les mathématiques, les sciences et la lecture pour les élèves de 9, 13 et 17 ans, afin d'étudier la corrélation entre l'âge d'entrée et les performances issues de cette évaluation. Similairement, Mayer \& Knutson (\protect\hyperlink{ref-MAY:KNU:97}{1997}) utilisent des données du NLSY (\emph{National Longitudinal Survey of Youth Mother-Child}) de 1986 à 1992, une enquête longitudinale américaine récoltant les résultats d'élèves américains issus de différents tests de mathématiques, de vocabulaire et de lecture, pour établir le lien entre l'âge d'entrée à l'école et ces résultats. Cet article s'intéresse également à des mesures non cognitives. Les études sur d'autres pays semblent rares durant cette période. Après le début des années 2000, des recherches sur les pays européens commencent à émerger avec Leuven et al. (\protect\hyperlink{ref-LEU:eal:04}{2004}) pour les Pays-Bas, Strøm (\protect\hyperlink{ref-STO:04}{2004}) pour la Norvège, Puhani \& Weber (\protect\hyperlink{ref-PUH:WEB:05}{2005}) pour l'Allemagne et Fredriksson \& Ockert (\protect\hyperlink{ref-FRE:OCK:05}{2005}) pour la Suède. Le premier exploite des données d'enquête suivant les élèves de primaire du Pays-Bas sur quatre années scolaires\footnote{L'éducation primaire néerlandaise est constituée de 8 années scolaires, les données de Leuven et al. (\protect\hyperlink{ref-LEU:eal:04}{2004}) couvrent les 2\textsuperscript{ème}, 4\textsuperscript{ème}, 6\textsuperscript{ème} et 8\textsuperscript{ème} années.} afin de mesurer l'effet de la durée d'études sur les résultats en arithmétiques et en langue\footnote{Il est possible d'obtenir une variation de la durée d'études au sein d'une même année scolaire grâce à la règle d'entrée au Pays-Bas conditionnée sur l'âge même de l'enfant au lieu d'une unique date seuil : un enfant peut commencer le primaire le lendemain de son 4\textsuperscript{ème} anniversaire et y est obligé avant ses 5 ans.}. Quant à Strøm (\protect\hyperlink{ref-STO:04}{2004}), les résultats des tests PISA (Programme International pour le Suivi des Acquis des élèves) de l'an 2000 sur les élèves norvégiens de 15 à 16 ans sont utilisés pour analyser les effets de l'âge d'entrée nets des effets de la durée d'études\footnote{En exploitant le fait qu'en Norvège, les parents n'ont pas la liberté d'avancer ou de décaler l'inscription de leur enfant et que le redoublement ou le saut de classe n'existe pratiquement pas.}. Une partie de l'étude de Puhani \& Weber (\protect\hyperlink{ref-PUH:WEB:05}{2005}) sur l'Allemagne analyse la relation entre l'âge d'entrée et les notes en lecture d'élèves allemands issus du PIRLS (\emph{Progress in International Reading Literacy Study}). Ces données sont constituées d'échantillons de 4\textsuperscript{ème} année de primaire. Enfin, Fredriksson \& Ockert (\protect\hyperlink{ref-FRE:OCK:05}{2005}) utilisent des données exhaustives de la population suédoise née entre 1935 et 1984 afin d'analyser l'effet de l'âge d'entrée à l'école sur les performances en fin d'études obligatoires (à 16 ans), sur le nombre d'années d'éducation et les sur salaires.\\
À notre connaissance, notre étude est la première effectuée à La Réunion et la troisième en France, après Grenet (\protect\hyperlink{ref-GRE:09}{2009}) et Fleury (\protect\hyperlink{ref-FLE:12}{2012}). Il est particulièrement important d'informer les pouvoirs publics locaux des effets de l'âge sur les performances scolaires, dans la mesure où il est connu que les effets sont pour la plupart plus forts chez les individus défavorisés (voir ci-dessous) et que La Réunion présente un contexte économique et social défavorisé comparé à d'autres régions françaises.

\quad La plupart des articles utilisent le salaire comme variable expliquée pour savoir si les effets des différentiels d'âge issus de divers systèmes éducatifs restent visibles à long terme. Par exemple, Grenet (\protect\hyperlink{ref-GRE:10}{2010}), qui utilise les données issues de l'Enquête Emploi (EE) couvrant chaque année un échantillon de ménages français, compare toutes choses égales par ailleurs les salaires horaires des personnes nées en janvier et celles nées en décembre de la même année. Pour la Norvège, Black et al. (\protect\hyperlink{ref-BLA:eal:11}{2011}) utilisent des données d'éducation et des données fiscales de la population, ce qui leur permet d'établir le lien causal entre l'âge d'entrée et le revenu à 24 et 35 ans. Des recherches américaines ou sur le continent asiatique des effets de l'âge d'entrée sur le salaire existent également\footnote{voir Dobkin \& Ferreira (\protect\hyperlink{ref-DOB:FER:10}{2010}) pour les États-Unis, Nam (\protect\hyperlink{ref-NAM:14}{2014}) pour la Corée du sud ou encore Kawaguchi (\protect\hyperlink{ref-KAW:11}{2011}) pour le Japon.}.

\quad Il est important de comprendre que lorsque les outputs d'intérêt sont mesurés quand les individus concernés ne sont plus en cours d'étude (salaires ou entrées sur le marché du travail, par exemple), l'effet mesuré est net de l'effet de maturité (effet de l'âge absolu, expliquée dans la Section \ref{ageintro}). Cela est possible car l'âge d'entrée peut varier indépendamment de la maturité absolue (\protect\hyperlink{ref-BLA:eal:11}{Black et al., 2011} ; \protect\hyperlink{ref-FRE:OCK:05}{Fredriksson \& Ockert, 2005}).

\quad Lorsque les effets sont mesurés en utilisant les notes au bout d'un ou plusieurs niveaux d'éducations, selon les données à disposition, une différence d'âge d'entrée correspond exactement à la même différence en âge absolu (âge aux examens des niveaux considérés). Dans cette configuration précise, il est difficile de mesurer séparément les effets de l'âge d'entrée de ceux de l'âge absolu et la majorité des articles concernés mesurent systématiquement un effet mixte (\protect\hyperlink{ref-BED:DHU:06}{Bedard \& Dhuey, 2006} ; \protect\hyperlink{ref-GRE:10}{Grenet, 2010} ; \protect\hyperlink{ref-PUH:WEB:05}{Puhani \& Weber, 2005}). Quelques articles ont pu explicitement séparer les effets de l'âge absolu pour en faire la comparaison avec l'âge d'entrée. Le premier à notre connaissance est Crawford et al. (\protect\hyperlink{ref-CRA:eal:07}{2007}) qui exploitent le fait qu'en Angleterre, le gouvernement central laisse aux autorités locales la liberté de fixer leur dates seuils d'entrée à l'école afin de dépasser la colinéarité parfaite entre l'âge d'entrée et l'âge absolu. Un autre travail sur données anglaise (\protect\hyperlink{ref-CRA:eal:14}{Crawford et al., 2014}) effectue deux estimations sur deux types de notes : le premier est une note mesurée au même moment pour des élèves d'âges différents et le second est une note mesurée pour des élèves du même âge à des moments différents. Sous l'hypothèse que les deux évaluations sont comparables, le premier type d'estimation permet de capter d'éventuels effets de maturité tandis que ce n'est pas le cas pour le second puisque les élèves sont évalués au même âge. Black et al. (\protect\hyperlink{ref-BLA:eal:11}{2011}) et Carlsson et al. (\protect\hyperlink{ref-CAR:eal:15}{2015}), quant à eux, utilisent comme variable expliquée des tests de QI administrés à des moments différents pour des individus étant entrés à l'école à des moments différents.\\
Quelques articles apportent leur contribution en tentant d'isoler cette fois l'effet de la position dans la distribution d'âge de la classe (âge relatif). Cascio \& Schanzenbach (\protect\hyperlink{ref-CAS:SCH:16}{2016}) exploitent les données issues du projet STAR (\emph{Student-Teacher Achievement Ratio}) aux États-Unis qui présente l'intérêt que les élèves sont assignés aléatoirement aux classes, ce qui permet d'obtenir deux variations indépendantes : celle de l'âge d'entrée et celle de l'âge des camarades de classe. Elder \& Lubotsky (\protect\hyperlink{ref-ELD:LUB:09}{2009}) et Peña (\protect\hyperlink{ref-PEN:17}{2017}) estiment des équations similaires à celles de Cascio \& Schanzenbach (\protect\hyperlink{ref-CAS:SCH:16}{2016}).\\
L'étude présente, n'observant que l'âge au moment des examens pour des niveaux d'éducation bien précis, ne pourra pas explicitement séparer les effets de l'âge aux examens de ceux de l'âge d'entrée. Toutefois, nous pouvons identifier les élèves d'une même classe et discutons de l'hypothèse que la date de naissance des élèves n'a pas d'influence sur la constitution des classes. Cela nous permet d'isoler d'éventuels effets d'âge relatif à La Réunion. À notre connaissance, notre étude est la première à tenter cette mesure en France.

\quad Outre la difficile séparation de leurs effets, l'âge d'entrée, l'âge aux examens ou l'âge relatif peuvent être endogènes pour des raisons bien identifiées dans la littérature.\\
Dans plusieurs pays, malgré une règle d'âge d'entrée établie, les parents peuvent décaler ou avancer l'entrée de leur enfant. De tels choix sont très probablement liés à des caractéristiques de leur enfant pouvant aussi affecter la réussite scolaire puisque les parents qui jugent leur enfant pas assez mature sur le plan intellectuel et émotionnel ont tendance à attendre un an de plus avant d'envoyer leur enfant à l'école (\protect\hyperlink{ref-GRE:10}{Grenet, 2010}, par exemple). De la même manière, les parents qui trouvent que leur enfant est prêt pour l'école peuvent l'y envoyer (\protect\hyperlink{ref-LEU:eal:04}{Leuven et al., 2004}, par exemple). Dans ces pays, une simple différence de performances entre les plus âgés à l'entrée et les plus jeunes peut être attribuée à l'effet de l'âge d'entrée ou à la différence \emph{a priori} entre ceux qui ont décalé/avancé leur inscription et ceux qui ne l'ont pas fait.\\
En faisant abstraction du décalage ou de l'avance d'entrée à l'école, l'âge aux examens reste potentiellement endogène si l'on mesure les performances après une ou plusieurs années qui suivent l'entrée à l'école à cause de la pratique du redoublement et du saut de classe : le redoublement étant un signal de difficultés, les élèves qui redoublent, et de fait sont plus âgés aux examens auraient très probablement eu de moins bons résultats s'ils n'avaient pas redoublé.\\
Enfin, si certaines écoles regroupent systématiquement des élèves selon leurs âge, l'âge relatif est endogène puisque les camarades de classe d'un élève âgé ne sont pas comparables à ceux d'un élève jeune (\protect\hyperlink{ref-BED:DHU:06}{Bedard \& Dhuey, 2006}).

\quad La stratégie principale pour traiter cette endogénéité est d'utiliser la date de naissance dans l'année\footnote{Bedard \& Dhuey (\protect\hyperlink{ref-BED:DHU:06}{2006}) pour les pays de l'OCDE ; Grenet (\protect\hyperlink{ref-GRE:09}{2009}) pour la France ; Puhani \& Weber (\protect\hyperlink{ref-PUH:WEB:05}{2005}) pour l'Allemagne et Elder \& Lubotsky (\protect\hyperlink{ref-ELD:LUB:09}{2009}) ou Cascio \& Schanzenbach (\protect\hyperlink{ref-CAS:SCH:16}{2016}) pour les États-Unis utilisent des instruments de même nature.} comme instrument de l'âge réel. Cette stratégie est soumise à deux critiques principales. La première est le doute sur l'exogénéité de la date de naissance, principalement motivé par Buckles \& Hungerman (\protect\hyperlink{ref-BUC:HUN:13}{2013}) qui mettent en évidence sur données américaines des liens systématiques entre les caractéristiques familiales et le mois de naissance, indépendamment du système éducatif\footnote{Par exemple, les mères des enfants nés en début d'année civile ont plus tendance à être dans des catégories sociales modestes par rapport aux mères des enfants nés plus tard dans l'année.}. Malgré cette critique, les recherches ultérieures continuent d'utiliser la date de naissance comme instrument, en montrant qu'elle peut être considérée comme exogène. En guise d'illustration, Crawford et al. (\protect\hyperlink{ref-CRA:eal:14}{2014}) ne trouvent pas de différence systématique entre les élèves d'Angleterre nés juste avant la date seuil et ceux nés juste après. Une autre possibilité est de vérifier la robustesse des résultats à la prise en compte des caractéristiques familiales (en tant que variables de contrôle ou effets fixes) dans les équations (\protect\hyperlink{ref-BLA:eal:11}{Black et al., 2011} ; \protect\hyperlink{ref-ELD:LUB:09}{Elder \& Lubotsky, 2009}).\\
La seconde critique est que cet instrument ne vérifie pas l'hypothèse de monotonie (\protect\hyperlink{ref-IMB:ANG:94}{Imbens \& Angrist, 1994}) lorsque la variable explicative potentiellement endogène est l'âge réel. Cela rend le coefficient estimé quasiment impossible à interpréter (\protect\hyperlink{ref-ALI:12TRUE}{Aliprantis, 2012} ; \protect\hyperlink{ref-FIO:STE:21}{Fiorini \& Stevens, 2021}). La monotonie requiert qu'être né plus tard dans l'année implique (au sens contrefactuel du terme), pour tout individu, d'être plus jeune. Il est connu que cette condition ne peut pas être vérifiée pour tout le monde dans la mesure où être né plus tard dans l'année augmente empiriquement la probabilité que les parents décident de décaler l'inscription ou la probabilité de redoubler. Cette critique a été explicitement formulée par Barua \& Lang (\protect\hyperlink{ref-BAR:LAN:09}{2009}) et formalisée par Fiorini \& Stevens (\protect\hyperlink{ref-FIO:STE:21}{2021}). Par rapport à cette critique, Black et al. (\protect\hyperlink{ref-BLA:eal:11}{2011}) considèrent que le coefficient estimé par variables instrumentales est une bonne approximation de l'effet moyen de traitement. Attar \& Cohen-Zada (\protect\hyperlink{ref-ATT:COH:18}{2018}), quant à eux, trouvent un cadre empirique dans lequel cette hypothèse n'est pas violée.\\
Une manière très peu exploitée de contourner le problème d'identification liée à la monotonie est de faire appel à l'approche par fonction de contrôle (\protect\hyperlink{ref-GAR:84}{Garen, 1984}). En imposant des relations paramétriques entre les inobservables déterminants de l'âge et de la performance scolaire, il est possible d'estimer un effet moyen de traitement sans recourir à l'hypothèse de monotonie (\protect\hyperlink{ref-WOO:15}{Wooldridge, 2015}). À notre connaissance, Hámori \& Köllő (\protect\hyperlink{ref-HAM:KOL:12}{2012}) sont les premiers et les seuls, jusqu'à présent, à avoir utilisé cette approche pour estimer les effets de l'âge d'entrée sur les notes en 4\textsuperscript{ème} et 8\textsuperscript{ème} année de primaire. Cette rareté est probablement en partie due à une compréhension encore relativement limitée des implications de l'hypothèse de monotonie dans les études sur les effets de l'âge en éducation (\protect\hyperlink{ref-FIO:STE:21}{Fiorini \& Stevens, 2021}).

\quad Une troisième critique sur cette stratégie est qu'elle ne permet pas d'isoler d'éventuels effets directs de la date de naissance puisque cette dernière est utilisée comme instrument (\protect\hyperlink{ref-SMI:09}{Smith, 2009}). Ces effets peuvent être importants, comme le suggèrent les résultats de Cascio \& Lewis (\protect\hyperlink{ref-CAS:LEW:06}{2006}) ou de Attar \& Cohen-Zada (\protect\hyperlink{ref-ATT:COH:18}{2018}).

\quad Généralement, lorsque des cohortes de naissance sont à disposition, il est aussi possible de faire appel aux régressions sur une discontinuité pour estimer les effets de l'âge en éducation (\protect\hyperlink{ref-CRA:eal:14}{Crawford et al., 2014} ; \protect\hyperlink{ref-KAI:17}{Kaila, 2017} ; \protect\hyperlink{ref-MAT:eal:16}{Matta et al., 2016}, par exemple). Typiquement, les performances des enfants nés juste après la date seuil sont comparés avec celles des enfants nés juste avant, profitant du fait que ces deux groupes sont très similaires et diffèrent uniquement par leur âge d'entrée (ou aux examens). Par rapport à la stratégie par variable instrumentale ci-dessus, les modèles de régressions sur une discontinuité ont l'avantage d'inclure explicitement des variables de date de naissance parmi les variables explicatives.\\
Les stratégies basées sur la régression sur une discontinuité sont elles aussi soumises à certaines critiques. De manière générale, les résultats présentent des problèmes de validité externe vu que les paramètres estimés ne valent que pour les individus nés au voisinage de la date seuil.\\
Basé sur le même type de raisonnement qu'avec Buckles \& Hungerman (\protect\hyperlink{ref-BUC:HUN:13}{2013}), certains parents pourraient volontairement faire naître leur enfant après le seuil plutôt qu'avant pour des raisons liées à la réussite scolaire. Typiquement, si la règle d'entrée est strictement suivie, un enfant né juste après le seuil fera partie des plus âgés de sa promotion tandis qu'un enfant né juste avant fera partie des plus jeunes. Les parents aisés ayant conscience du désavantage subi par l'enfant né juste avant le seuil pourraient préférer viser explicitement une naissance juste après le seuil (\protect\hyperlink{ref-KIM:21}{Kim, 2021} ; \protect\hyperlink{ref-SHI:15}{Shigeoka, 2015}).\\
Les individus nés après le seuil n'ont pas systématiquement un an de plus à l'entrée ou aux examens toujours via les décalages/avances d'entrée et les redoublements/sauts de classe. Dans ce cas, les modèles de \emph{Fuzzy Regression Discontinuity}, dans lesquels l'âge est instrumenté par le fait d'être né après ou avant le seuil sont utilisés. Hahn et al. (\protect\hyperlink{ref-HAH:eal:01}{2001}) montrent que l'hypothèse de monotonie est également requise pour pouvoir correctement interpréter le paramètre estimé et Fiorini \& Stevens (\protect\hyperlink{ref-FIO:STE:21}{2021}) montrent que celle-ci n'est généralement pas vérifiée dans les études sur les effets d'âge en éducation.

\quad Par rapport à ces considérations méthodologiques, nous exploitons également la date de naissance dans l'année comme instrument mais nous estimons les effets de l'âge par fonction de contrôle. Nous présentons des arguments empiriques en faveur de la validité de notre instrument. Nous présentons aussi systématiquement les résultats obtenus par variables instrumentales usuelles pour visualiser les éventuels biais liés à la violation de l'hypothèse de monotonie.\\
Nous utilisons également des modèles de régressions sur une discontinuité similaire à ceux de Smith (\protect\hyperlink{ref-SMI:09}{2009}) pour prendre en compte d'éventuels effets directs de la date de naissance.

\quad Les études susmentionnées trouvent quasiment toutes un effet positif et important de l'âge d'entrée sur les performances scolaires pendant les premières années d'éducation lorsque la date de naissance est utilisée comme instrument. Par exemple, de tous les pays de l'OCDE concernés par l'article de Bedard \& Dhuey (\protect\hyperlink{ref-BED:DHU:06}{2006}), un an de plus à l'entrée procure un avantage de plus ou moins 0.3 UET en notes de mathématiques et de science. Aux États-Unis, Datar (\protect\hyperlink{ref-DAT:06}{2006}), Elder \& Lubotsky (\protect\hyperlink{ref-ELD:LUB:09}{2009}) et Cascio \& Schanzenbach (\protect\hyperlink{ref-CAS:SCH:16}{2016}) trouvent un effet positif considérable d'être un an plus âgé à l'entrée à l'école\footnote{Une exception notable est Mayer \& Knutson (\protect\hyperlink{ref-MAY:KNU:97}{1997}) qui montrent que les élèves étant entrés plus jeunes ont de meilleures notes. Il est toutefois bien compris aujourd'hui qu'une simple comparaison est biaisée par le fait que les plus jeunes entrants ont vraisemblablement des caractéristiques plus favorables à la réussite par rapport à leurs camarades.}. En France, Grenet (\protect\hyperlink{ref-GRE:09}{2009}) montre qu'être né en janvier plutôt qu'en décembre procure un avantage de 0.3 UET en 4\textsuperscript{ème} année de primaire.\\
Crawford et al. (\protect\hyperlink{ref-CRA:eal:07}{2007}), Crawford et al. (\protect\hyperlink{ref-CRA:eal:14}{2014}) (Angleterre) ainsi que Black et al. (\protect\hyperlink{ref-BLA:eal:11}{2011}) (Norvège) démontrent plus particulièrement que l'âge aux examens est le facteur principal de cette différence. À l'inverse, les résultats de Elder \& Lubotsky (\protect\hyperlink{ref-ELD:LUB:09}{2009}) affirment qu'aux États-Unis, c'est l'âge d'entrée (plus précisément, l'investissement effectué par les parents avant et hors de l'école) qui importe par rapport à la maturité au moment de l'examen. Ce point semble alors être un élément de désaccord de la littérature mais il dépend très probablement du système éducatif considéré.\\
En revanche, l'âge relatif a un effet négatif sur les notes (\protect\hyperlink{ref-CAS:SCH:16}{Cascio \& Schanzenbach, 2016} (États-Unis); \protect\hyperlink{ref-ELD:LUB:09}{Elder \& Lubotsky, 2009} (États-Unis); \protect\hyperlink{ref-PEN:17}{Peña, 2017} (Mexique); \protect\hyperlink{ref-SAN:STO:05}{Sandgren \& Strøm, 2005} (Norvège); \protect\hyperlink{ref-LEU:RON:11}{Leuven \& Rønning, 2016} (Norvège)). Autrement dit, être plus jeune dans la classe est avantageux pour un élève. Les raisons précises de ce constat ne semblent toutefois pas encore claires, à notre connaissance (ajustement du comportement de l'enseignant en fonction de la distribution d'âge de la classe ou effets de pairs).

\quad En ce qui concerne les notes, les résultats obtenus avec les modèles de régressions sur une discontinuité sont qualitativement les mêmes (\protect\hyperlink{ref-MAT:eal:16}{Matta et al., 2016} (Brésil); \protect\hyperlink{ref-CRA:eal:14}{Crawford et al., 2014} (Angleterre); \protect\hyperlink{ref-KAI:17}{Kaila, 2017} (Finlande)).

\quad De manière générale, il est montré que ces effets diminuent avec le temps à moins que le système éducatif ne pratique des différenciations de parcours très tôt dans le cycle d'éducation. Cela est clairement confirmé pour Israël par Attar \& Cohen-Zada (\protect\hyperlink{ref-ATT:COH:18}{2018}) qui trouvent que les effets en mathématiques doublent quasiment lorsqu'ils sont mesurés ultérieurement, parce que des parcours différenciés de mathématiques sont mises en place dès l'âge de 7 ans.

\quad De manière plus nuancée, Hámori \& Köllő (\protect\hyperlink{ref-HAM:KOL:12}{2012}) trouvent des coefficients inférieurs mais toujours significatifs lorsque l'approche par fonction de contrôle est utilisée. Il faudrait alors être plus prudent dans l'interprétation des résultats obtenus par variables instrumentales classiques. Plus d'application de l'approche par fonction de contrôle est alors également souhaitable pour de futures recherches.

\hypertarget{ageinst}{%
\section{Contexte institutionnel}\label{ageinst}}

À La Réunion comme dans toute région française, l'instruction est obligatoire dès 6 ans sur la période étudiée\footnote{À partir de la rentrée 2019, elle est obligatoire dès 3 ans (\url{https://www.education.gouv.fr/la-loi-pour-une-ecole-de-la-confiance-5474}).}. Cet âge correspond à l'âge normal d'entrée en primaire. Pour s'inscrire en première année de primaire, les élèves doivent avoir 6 ans au plus tard le 31 décembre de l'année civile de la rentrée. Cependant, certains parents inscrivent leur enfant avec une année de retard ou d'avance. Les parents retardent l'inscription de leur enfant soit parce qu'ils considèrent que ce dernier n'est pas prêt pour l'école soit parce qu'ils espèrent de meilleurs résultats d'un enfant plus âgé. Ils avancent l'inscription de leur enfant s'ils considèrent que ce dernier a atteint une maturité intellectuelle et émotionnelle suffisante pour commencer l'école.

\quad L'école primaire comporte 5 années scolaires : le Cours Préparatoire (CP), les Cours Élémentaire 1\textsuperscript{ère} année et 2\textsuperscript{ème} année (CE1 et CE2), et les Cours Moyen 1\textsuperscript{ère} année et 2\textsuperscript{ème} année (CM1 et CM2). Exceptionnellement, un élève peut redoubler une année de primaire si ses acquis sont jugés insuffisants. Au niveau national, autour de la période d'étude, les taux de redoublement en CP, CE1, CE2, CM1 et CM2 sont respectivement de l'ordre de 3.5\%, 4\%, 1.5\%, 1\% et 1.5\% (\protect\hyperlink{ref-DEP:21}{DEPP, 2021b, p. 64}). Bien qu'ils concernent des périodes plus anciennes, les taux de redoublement présentés dans Chevillon \& Parain (\protect\hyperlink{ref-CHE:PAR:94}{1994}) pour La Réunion sont systématiquement plus élevés que ceux de la métropole\footnote{Par ailleurs, en fin de primaire, la part d'élèves étant plus âgé que s'ils étaient entrés en primaire à 6 ans et n'avaient connu aucun redoublement ni saut de classe est supérieure d'environ 4 points de pourcentage à La Réunion par rapport au niveau national (\protect\hyperlink{ref-ACA:12}{Académie de La Réunion, 2012}).}.

\quad L'année scolaire de CM2 débute vers la fin du mois d'août et se termine vers le début du mois de juillet. Les élèves inscrits à 6 ans en primaire et n'ayant connu aucun redoublement ni saut de classe fêtent leur 10\textsuperscript{ème} anniversaire\footnote{Pour faire court, nous dirons désormais qu'ils ont \emph{normalement} 10 ans.} dans l'année civile qui contient le mois d'août de la rentrée de CM2.

\quad Dans certaines écoles, les élèves de CM2 et de CM1 partagent la même salle de classe : ce sont les classes multiniveaux.

\quad Généralement, vers la fin du mois de janvier, les élèves passent une évaluation nationale des acquis. Il est constitué de 100 \emph{items}. Chaque item correspond à un exercice ou une question bien précis(e). Un élève obtient 1 point s'il a bien répondu et 0 sinon. Il n'y a pas de point intermédiaire. La note d'un élève à cette évaluation est un nombre entier allant de 1 à 100 et représente le pourcentage de bonnes réponses.
Parmi les 100 items, 60 correspondent à des exercices de français et 40 à des exercices de mathématiques. De plus, les 60 et 40 items peuvent, chacun de leur côté, être subdivisés en 5 sous-composantes d'items évaluant des capacités très précises. Le nombre d'items de mathématiques dans les sous-composantes change au fil des années. Ce n'est pas le cas des items de français. Plus précisément, les 60 items de français sont regroupés en 10, 15, 15, 10 et 10 items d'écriture, grammaire, lecture, orthographe et vocabulaire respectivement. Les 40 items de mathématiques sont regroupés, selon l'année considérée, en 12, 13 ou 15 items de calcul ; 7 ou 5 items de géométrie ; 6, 7 ou 8 items de grandeurs-et-mesures ; 6, 7 ou 8 items de nombre ; et 6 ou 7 items d'organisation-et-gestion-des-données (voir la partie inférieure du Tableau \ref{tab:agestatsevol}).

\quad Après le CM2, l'élève entre au collège pour 4 années, de la 6\textsuperscript{ème} à la 3\textsuperscript{ème}. Normalement, les élèves y entrent à 11 ans et en sortent à 15 ans. Il arrive que certains d'entre eux redoublent. Autour de la période d'étude, le taux de redoublement au niveau national est de 3.3\%, 1.9\%, 3\% et 4.3\% en 6\textsuperscript{ème}, 5\textsuperscript{ème}, 4\textsuperscript{ème} et 3\textsuperscript{ème}, respectivement. Ces chiffres sont légèrement inférieurs à La Réunion. Ils valent respectivement 2.6\%, 1.4\%, 2.6\% et 2.2\%.

\quad Au collège, il n'y a pas de classe multiniveau.
De manière générale, les collégiens suivent le même programme mais des dispositifs spéciaux peuvent accueillir les élèves présentant de grandes difficultés scolaires. Des options ayant tendance à attirer les élèves les plus performants existent également. Nous désignons par \emph{sections} ces dispositifs ou options spécifiques au niveau des élèves. Plus précisément, la section désavantagée regroupe des élèves en difficulté, la section avantagée regroupe des élèves particulièrement bons et la section normale regroupe le reste. Le Chapitre \ref{pe} en discute plus en détails.

\quad La 3\textsuperscript{ème} est l'année scolaire normale pour obtenir le Diplôme National de Brevet (DNB). Ce dernier est composé d'une part de contrôles continus pour lesquels les élèves sont évalués tout au long de l'année scolaire. D'autre part, vers la fin du mois de juin, les élèves passent les épreuves finales. Elles sont constituées des épreuves orales et des épreuves écrites communes à tous les élèves. Nous considérons uniquement les notes aux épreuves écrites dans nos estimations puisque les contrôles continus et épreuves orales sont spécifiques aux enseignants. Ils ne sont donc pas comparables pour tous les élèves. Les matières concernées par les épreuves écrites qui nous intéressent principalement sont le français et les mathématiques\footnote{Les autres matières sont l'histoire-et-géographie, la physique-chimie, les sciences de la vie et de la terre et les sciences technologiques. Les épreuves écrites de français sont partiellement constituées d'épreuves de dictée et de rédaction. Dans les données, en plus des notes en français et en mathématiques aux épreuves écrites, nous avons à notre disposition les notes en dictée et en rédaction. Les notes sur les autres matières des épreuves écrites ne sont pas disponibles.}.

\hypertarget{agedata}{%
\section{Données}\label{agedata}}

Nous mobilisons principalement deux types de fichiers administratifs bien distincts : un ensemble de fichiers de résultats aux évaluations de CM2 et un autre aux épreuves du DNB. Ils permettent de mesurer les effets de l'âge en fin de primaire et en fin de collège, respectivement. Un troisième type de fichier, les fichiers dits ``constat'', est utilisé pour récupérer les informations sur la catégorie sociale des parents, sur l'école ou sur la classe des candidats aux examens de CM2 ou au DNB. Les détails et discussions sur les fichiers constats se trouvent en Annexe \ref{ageconstats}.\\
Le caractère exhaustif des informations sur les résultats scolaires rend nos résultats principaux généralisables pour La Réunion. La nature en coupe des données constitue une de leur limite puisqu'elle ne permet pas de connaître la progression des élèves à travers le temps.

\quad Les fichiers de CM2 sont des données en coupe couvrant les 4 années scolaires successives de 2008-2009 à 2011-2012. Pour simplifier, nous désignons l'année scolaire \((t-1)\text{-}t\) par \(t\)\footnote{Par exemple, l'année scolaire 2008-2009 est désignée par l'année scolaire 2009.}. Pour chaque année scolaire, nous retrouvons tous les candidats aux évaluations de CM2 inscrits dans les écoles publiques et privées sous-contrat de La Réunion. Le nombre d'écoles augmentent de 2009 à 2011 pour baisser en 2012, comme le montrent respectivement la première, la deuxième et la dernière ligne du Tableau \ref{tab:agestatsevol}. Cela est dû à l'ouverture, à la fermeture ou à la fusion des écoles au fil des années d'un côté et à la fermeture des classes de CM2 (hors classes multiniveaux) de l'autre. Il en est de même pour le nombre moyen d'élèves de CM2 par école et le nombre d'élèves. Ce dernier passe d'environ 12000 en 2009 à presque 15000 en 2011 pour redescendre à moins de 14000 en 2012.\\
Le \emph{baby boom} des années 2000, discutée dans la Section \ref{agemethodsrd}, constitue une autre explication de ces évolutions d'effectif.

\quad Au niveau individuel, nous disposons de la note totale, en français et en mathématiques. Les notes en sous-composantes de français (écriture, lecture, grammaire, orthographe, vocabulaire) et de mathématiques (calcul, géométrie, grandeurs-et-mesures, nombre, organisation-et-gestion-de-données) sont également disponibles. Il n'y a pas de valeurs manquantes dans cette variable. Il s'agit de la variable dépendante. Comme illustrée par la première colonne du Tableau \ref{tab:agestats}, la moyenne de note totale sur les 4 années scolaires est de 50. Elle se décline en une moyenne de 32 en français et 19 en mathématiques. Derrière ces chiffres se cachent une hétérogénéité selon les années scolaires puisque la note totale augmente en moyenne d'année en année sauf entre 2009 et 2010 (deuxième groupe de lignes du Tableau \ref{tab:agestatsevol}. Cette hétérogénéité est due à celle de la note en français qui passe de 19 points à 16 points entre 2009 et 2010 et de 16 points à 21 points entre 2010 et 2012\footnote{L'observation des distributions des notes totales, en français et en mathématiques par année scolaire nous conduit à la même conclusion.}.

\quad Nous connaissons également la date de naissance exacte de tous les élèves. En conjonction avec les dates des évaluations de chaque année scolaire, nous pouvons construire l'âge au moment des évaluations de CM2 (que nous désignerons indifféremment par âge aux examens). Les examens se déroulent sur une période de 4 jours. Pour chaque année scolaire, nous considérons un jour ``au milieu'' pour désigner le jour des évaluations. Les évaluations ont eu lieu le 04 février 2009, les 20 janvier 2010 et 2011 et le 25 mai 2012. Les élèves ont en moyenne un peu moins de 11 ans sur les 4 années scolaires sauf en 2012 du fait du décalage de sa date d'évaluation.\\
La connaissance de l'année de naissance et de la règle d'entrée en CP nous permet de déduire l'année scolaire de CM2 dans laquelle un élève devrait normalement se trouver. Nous savons alors s'il est à l'heure, en retard ou avance par rapport à cela. Nous désignons cette dernière information par position. Notons bien que nous ne savons pas si un élève est en retard parce que ses parents ont décalé son inscription ou parce qu'il a redoublé. Les élèves de CM2 sont nés entre 1996 (retrouvés en 2009, ayant deux ans de retard) et 2003 (retrouvés en 2012, ayant deux ans d'avance)\footnote{Nous avons exclu les élèves ayant plus de deux ans de retard ou d'avance. Ces situations sont invraisemblables et de fréquence négligeable.}. 81\% parmi eux sont à l'heure, 17\% en retard et 2\% en avance.

\quad Nous savons si l'élève est une fille ou un garçon. Il y a 50\% de filles et 50\% de garçons.

\quad Plus important, nous disposons d'une variable renseignant la catégorie socio-professionnelle de l'un des parents ou d'un tuteur de l'enfant. Cette variable a été récupérée dans les fichiers constats via l'Identifiant National de l'Élève (INE). Une description détaillée de la récupération de cette variable, accompagnée d'une discussion de sa qualité est proposée dans l'Annexe \ref{ageconstats}. Cette variable n'est pas disponible pour 2009 et compte un peu moins de 16\% de valeurs manquantes pour les trois années suivantes. Au lieu d'une version détaillée disponible originalement, nous retenons la version en 4 postes (défavorisée, moyenne, favorisée et très favorisée) de la Direction de l'Évaluation, de la Prospection et de la Performance (DEPP) qui est plus intuitive et préférée dans les études empiriques en éducation en France (\protect\hyperlink{ref-MET:eal:17}{Métayer et al., 2017}). Les correspondances entre la version détaillée et la version en 4 postes se trouvent dans le Tableau \ref{tab:agecorrespcsp} de l'Annexe \ref{agecorrespcsp}. De plus, nous disposons d'une modalité ``Autre'' qui regroupe les catégories non renseignées ou sans objet. Il est important de les distinguer des valeurs manquantes issues de la récupération de la CSP car les deux catégories peuvent différer considérablement en termes de déterminant des performances scolaires\footnote{Par exemple, dans la catégorie ``Autre'', il y a 55\% de filles tandis que ce chiffre est de 48\% parmi les valeurs manquantes issues de la récupération de la CSP (non reporté).}. La Réunion compte 56\% d'élèves issus de parents défavorisés. Les élèves de CSP moyenne représentent 24\% de la population. Les très favorisés sont plus nombreux (11\%) que les favorisés (7\%).

\quad Nous disposons également de l'identifiant de la classe avec lequel il est possible de calculer la taille de la classe à l'exception des classes multiniveaux. Pour ces dernières, l'identifiant permet juste de compter le nombre d'élèves de CM2. Nous ne pouvons pas identifier les classes multiniveaux et désignons simplement par taille de classe le nombre de CM2 dans la classe, que cette dernière soit multiniveau ou non. En moyenne, la taille de classe de CM2 est de 23 élèves.

\quad Enfin, nous sommes en possession de l'identifiant de l'école. Il permet de récupérer au sein des fichiers constats deux variables dichotomiques indiquant respectivement si l'école est publique ou privée (statut de l'école) et si elle est en éducation prioritaire ou non. Le système est majoritairement constitué d'écoles publiques (93\%). Le statut d'éducation prioritaire peut changer d'une année à l'autre. Nous repérons globalement 354 écoles dans nos données. Parmi ces dernières, 19 (5\%) ont acquis le statut d'éducation prioritaire et 36 (10\%) l'ont perdue. Sur notre période d'étude, 150 écoles (42\%) sont constamment restées hors éducation prioritaires et 149 (42\%) sont constamment restées en éducation prioritaire. D'ailleurs, le Tableau \ref{tab:agestatsevol} montre que 51\% et 52\% des écoles sont en éducation prioritaire en 2009 et 2010 tandis que ces chiffres baissent à 48\% en 2011 et 2012\footnote{Cependant, un test de \(\chi^2\) indique que cette différence n'est pas significative.}.\\
En tout, 37\% des écoles ont une classe de CM2, 37\% en ont 2, 22\% en ont 3 et seulement 5\% ont 4 classes ou plus.

\quad Les fichiers de DNB concernent les candidats au DNB pour les trois années scolaires de 2014 à 2016\footnote{Ces candidats sont pratiquement tous en 3\textsuperscript{ème}. Le nombre d'élèves provenant d'autres niveaux (4\textsuperscript{ème} ou 2\textsuperscript{nde}) est négligeable.}. Ces trois années scolaires correspondent respectivement aux années dans lesquelles les élèves de CM2 de 2010, 2011 et 2012 devraient être retrouvés s'ils n'ont connu aucun redoublement ou saut de classe entre le CM2 (y compris) et le collège (y compris), et s'ils n'ont pas quitté le système éducatif réunionnais. Les collèges sont moins nombreux mais accueillent plus d'élèves. Nous comptons 82 collèges à La Réunion qui scolarisent environ 170 élèves de 3\textsuperscript{ème} par établissement. Parallèlement aux nombres d'élèves de CM2, le nombre d'élèves de 3\textsuperscript{ème} augmente entre 2014 et 2015 et baisse, dans une moindre mesure, en 2016. Cela s'explique toujours par l'ouverture, la fermeture ou la fusion des collèges ; ou par la fermeture des classes de 3\textsuperscript{ème} ou le \emph{baby boom} des années 2000.

\quad Nous exploitons principalement les notes globales aux épreuves écrites ainsi que les notes en français et en mathématiques à ces épreuves. Ces variables comptent environ 3\% de valeurs manquantes\footnote{Ces valeurs manquantes ne sont vraisemblablement pas aléatoires puisqu'elles regroupent significativement plus de garçons, plus d'élèves défavorisés et plus d'élèves en régime scolaire externe. Ces variables que nous citons sont présentées ci-dessous. Nous pensons que les biais liés à cette sélection sont relativement faibles, au vu de la faible proportion des valeurs manquantes.}. Ce sont les seules variables avec des valeurs manquantes dans les fichiers de DNB. Ce sont des notes sur 20 et elles ne sont pas des nombres entiers comme les notes au CM2. La moyenne de la note globale est d'environ 10 sur 20. Cette moyenne est de 11 en français et 9 en mathématiques. En s'intéressant à l'hétérogénéité de ces moyennes par année scolaire (Tableau \ref{tab:agestatsevol}), l'année 2015 apparaît comme exceptionnelle puisque dans celle-ci, les moyennes globales, en français et en mathématiques sont équivalentes tandis que la moyenne en mathématiques est nettement inférieure à celle en français pour les autres années.

\quad Comme dans les fichiers de CM2, la date de naissance exacte est renseignée dans les fichiers de DNB et nous permet de calculer, sachant les dates des épreuves, l'âge au moment des examens de DNB. Les candidats au DNB ont un peu plus de 15 ans en moyenne. Cet âge moyen est le même chaque année car les épreuves ont lieu aux mêmes périodes de l'année, à quelques jours près.\\
La part des élèves à l'heure en 3\textsuperscript{ème} est de 79\%. Ce chiffre est inférieur à celui de CM2 (81\%) puisque le redoublement est pratiqué entre le CM2 (y compris) et la 3\textsuperscript{ème} (y compris).

\quad Nous disposons également d'une variable dichotomique de sexe. Les filles et les garçons sont toujours également répartis.

\quad La variable de CSP est présente dans les fichiers de DNB. Nous retenons ici encore la version en 4 postes pour les mêmes raisons évoquées que dans le cas du CM2. Les proportions sont les mêmes qu'au CM2 : 56\% de défavorisés, 24\% d'élèves de CSP moyenne, 8\% et 12\% de favorisés et très favorisés respectivement.

\quad Outre l'absence de valeurs manquantes dans les variables autres que les notes, un avantage des bases de DNB est qu'elles contiennent une information sur le régime scolaire. Il s'agit des modalités d'entrée et de sortie de l'élève au collège, notamment pendant le repas du midi. Bien que nous disposions de détails plus fins, nous exploitons une version groupée de la variable. Cela permet d'avoir assez d'observations sans perdre l'intuition. Un élève en régime de demi-pensionnaire prend son déjeuner à la cantine puis quitte l'établissement lorsque la journée des cours est terminée. Un élève en régime interne prend son déjeuner et son dîner au sein de l'établissement. Un élève en régime externe ne prend aucun repas au sein de l'établissement.\\
Il est connu que cette variable est liée au revenu des parents car le régime de demi-pensionnaire ou interne implique de payer les repas. En l'occurrence, la proportion d'élèves très favorisés parmi ceux en régime de demi-pensionnaire est de 16\% tandis qu'elle n'est que de 5\% parmi ceux en régime externe.

\quad Nous récupérons l'identifiant de la classe de 3\textsuperscript{ème} depuis les fichiers constats (voir Annexe \ref{ageconstats}). Comme il n'y a pas de classes multiniveaux en 3\textsuperscript{ème}, cette variable permet de calculer la taille de classe de 3\textsuperscript{ème}. Elle est de 24 élèves en moyenne. Cette moyenne est la même pour toutes les années scolaires.

\quad L'identifiant de l'école a été également récupéré depuis les fichiers constats sans qu'aucune valeur manquante ne soit à mentionner au bout. Les caractéristiques des collèges (statut et éducation prioritaire) sont fondamentalement les mêmes que les écoles (CM2) : 93\% des établissements sont publics et la moitié environ sont en éducation prioritaire. Le statut d'éducation prioritaire n'est toujours pas le même au cours des années scolaires. Sur les 82 collèges détectées chaque année, 10 (12\%) acquièrent le statut d'éducation prioritaire et 5 (6\%) le perdent. 32 (39\%) et 35 (43\%) sont constamment hors éducation prioritaire et en éducation prioritaire.

\newpage

\begin{landscape}\begingroup\fontsize{8}{10}\selectfont

\begin{ThreePartTable}
\begin{TableNotes}
\item \textit{Sources :} Fichiers CM2 (2009 à 2012), Fichiers DNB (2014 à 2016), calculs de l'auteur.
\item \textit{Notes :} La première ligne renseigne des effectifs, la seconde une moyenne et la troisième des proportions. Les 3 lignes suivantes montrent des moyennes. Au CM2, la note totale est un nombre entier entre 1 et 100, la note en français entre 1 et 60 et la note en mathématiques entre 1 et 40. Au DNB, ce sont des notes sur 20. CM2 : Cours Moyen 2\textsuperscript{ème} année. DNB : Diplôme National du Brevet.
\end{TableNotes}
\begin{longtable}[t]{llllllll}
\caption{\label{tab:agestatsevol}Évolution de quelques chiffres sur l'éducation à La Réunion}\\
\toprule
\multicolumn{1}{c}{} & \multicolumn{4}{c}{CM2} & \multicolumn{3}{c}{DNB} \\
\cmidrule(l{3pt}r{3pt}){2-5} \cmidrule(l{3pt}r{3pt}){6-8}
  & 2009 & 2010 & 2011 & 2012 & 2014 & 2015 & 2016\\
\midrule
\endfirsthead
\caption[]{\label{tab:agestatsevol}Évolution de quelques chiffres sur l'éducation à La Réunion (suite)}\\
\toprule
\multicolumn{1}{c}{} & \multicolumn{4}{c}{CM2} & \multicolumn{3}{c}{DNB} \\
\cmidrule(l{3pt}r{3pt}){2-5} \cmidrule(l{3pt}r{3pt}){6-8}
  & 2009 & 2010 & 2011 & 2012 & 2014 & 2015 & 2016\\
\midrule
\endhead

\endfoot
\bottomrule
\insertTableNotes
\endlastfoot
\addlinespace[0.3em]
\multicolumn{8}{l}{\textbf{ }}\\
\hspace{1em}Établissements & 328 & 344 & 349 & 331 & 82 & 82 & 82\\
\hspace{1em}Élèves par établissement & 37.53 & 39.62 & 42.11 & 41.4 & 167.62 & 178.34 & 173.62\\
\hspace{1em}Écoles en éducation prioritaire & 0.51 & 0.52 & 0.48 & 0.48 & 0.49 & 0.49 & 0.55\\
\addlinespace[0.3em]
\multicolumn{8}{l}{\textbf{ }}\\
\hspace{1em}Note totale & 51.79 & 47.41 & 50.7 & 53.19 & 9.96 & 10.42 & 10.45\\
\hspace{1em}Note en français & 32.74 & 31.38 & 31.41 & 31.95 & 11.67 & 10.59 & 11.27\\
\hspace{1em}Note en mathématiques & 19.05 & 16.04 & 19.29 & 21.24 & 8.07 & 10.29 & 9.56\\
\addlinespace[0.3em]
\multicolumn{8}{l}{\textbf{ }}\\
\hspace{1em}Âge aux examens & 10.78 & 10.72 & 10.7 & 11.04 & 15.19 & 15.16 & 15.15\\
\addlinespace[0.3em]
\multicolumn{8}{l}{\textbf{Nombre d'items en}}\\
\hspace{1em}Écriture & 10 & 10 & 10 & 10 & - & - & -\\
\hspace{1em}Grammaire & 15 & 15 & 15 & 15 & - & - & -\\
\hspace{1em}Lecture & 15 & 15 & 15 & 15 & - & - & -\\
\hspace{1em}Orthographe & 10 & 10 & 10 & 10 & - & - & -\\
\hspace{1em}Vocabulaire & 10 & 10 & 10 & 10 & - & - & -\\
\hspace{1em}Calcul & 12 & 12 & 13 & 15 & - & - & -\\
\hspace{1em}Géométrie & 7 & 7 & 7 & 5 & - & - & -\\
\hspace{1em}Grandeurs et mesures & 7 & 7 & 6 & 8 & - & - & -\\
\hspace{1em}Nombres & 8 & 8 & 7 & 6 & - & - & -\\
\hspace{1em}Organisation et gestion de données & 6 & 6 & 7 & 6 & - & - & -\\
\hspace{1em} &  &  &  &  &  &  \vphantom{1} & \\
 &  &  &  &  &  &  & \\
Observations & 12311 & 13628 & 14698 & 13704 & 13745 & 14624 & 14237\\*
\end{longtable}
\end{ThreePartTable}
\endgroup{}
\end{landscape}

\newpage  
\begingroup\fontsize{8}{10}\selectfont

\begin{ThreePartTable}
\begin{TableNotes}
\item \textit{Sources :} Fichiers CM2 (2009 à 2012), Fichiers DNB (2014 à 2016), calculs de l'auteur.
\item \textit{Notes :} Moyennes et proportions. Écart-types entre parenthèses. La note totale au CM2  est un nombre entier entre 1 et 100, la note en français entre 1 et 60 et la note en mathématiques entre 1 et 40. Les notes au DNB sont des notes sur 20. Les proportions de CSP au CM2 et au DNB sont calculées hors valeurs manquantes. On décompte 16\% de valeurs manquantes au CM2 et aucune au DNB. Les fichiers de CM2 ne contiennent pas d'informations sur le régime scolaire. La deuxième colonne concerne les élèves nés en 1999 et 2000. Elle représente les données de départ pour les estimations de régression sur une discontinuité. CM2 : Cours Moyen 2\textsuperscript{ème} année. DNB : Diplôme National du Brevet. CSP : Catégorie Socio-Professionnelle.
\end{TableNotes}
\begin{longtable}[t]{llll}
\caption{\label{tab:agestats}Statistiques descriptives}\\
\toprule
  & CM2 & \makecell[l]{CM2, nés en \\ 1999 et 2000} & DNB\\
\midrule
\endfirsthead
\caption[]{\label{tab:agestats}Statistiques descriptives (suite)}\\
\toprule
  & CM2 & \makecell[l]{CM2, nés en \\ 1999 et 2000} & DNB\\
\midrule
\endhead

\endfoot
\bottomrule
\insertTableNotes
\endlastfoot
\addlinespace[0.3em]
\multicolumn{4}{l}{\textbf{ }}\\
\hspace{1em}Note totale & 50.75 (21.75) & 49.58 (21.48) & 10.28 (3.64)\\
\hspace{1em}Note en français & 31.84 (13.51) & 31.49 (13.53) & 11.16 (3.89)\\
\hspace{1em}Note en mathématiques & 18.91 (9.42) & 18.09 (9.19) & 9.33 (4.43)\\
\addlinespace[0.3em]
\multicolumn{4}{l}{\textbf{ }}\\
\hspace{1em}Âge aux examens & 10.81 (0.5) & 10.72 (0.51) & 15.16 (0.51)\\
\addlinespace[0.3em]
\multicolumn{4}{l}{\textbf{Position}}\\
\hspace{1em}En retard & 0.17 & 0.16 & 0.19\\
\hspace{1em}À l'heure & 0.81 & 0.82 & 0.79\\
\hspace{1em}En avance & 0.02 & 0.02 & 0.02\\
\addlinespace[0.3em]
\multicolumn{4}{l}{\textbf{Sexe}}\\
\hspace{1em}Filles & 0.5 & 0.5 & 0.5\\
\hspace{1em}Garçons & 0.5 & 0.5 & 0.5\\
\addlinespace[0.3em]
\multicolumn{4}{l}{\textbf{CSP}}\\
\hspace{1em}Défavorisés & 0.56 & 0.56 & 0.56\\
\hspace{1em}Moyens & 0.24 & 0.24 & 0.24\\
\hspace{1em}Favorisés & 0.07 & 0.07 & 0.08\\
\hspace{1em}Très favorisés & 0.11 & 0.11 & 0.12\\
\hspace{1em}Autres & 0.01 & 0.01 & 0.01\\
\addlinespace[0.3em]
\multicolumn{4}{l}{\textbf{Régime scolaire}}\\
\hspace{1em}Demi-pensionnaires & - & - & 0.6\\
\hspace{1em}Internes & - & - & 0\\
\hspace{1em}Externes & - & - & 0.4\\
 &  &  & \\
Observations & 54341 & 27893 & 42606\\*
\end{longtable}
\end{ThreePartTable}
\endgroup{}

\newpage

\hypertarget{agemethods}{%
\section{Méthodologie}\label{agemethods}}

\hypertarget{agemethodscfh}{%
\subsection{Approche par fonction de contrôle}\label{agemethodscfh}}

Nous souhaitons mesurer l'effet de l'âge au moment des examens sur les performances en fin de primaire et de collège à La Réunion. Notre équation d'intérêt est la suivante ~:

\begin{equation}
\label{eq:ageols}
y_{i} = \alpha_0 + \alpha_1 a_{i} + x'_{i} \alpha_2 + \epsilon_{i},
\end{equation}

dans laquelle \(y_{i}\) désigne une note (totale, en français ou ses composantes, et en mathématiques ou ses composantes) aux évaluations de CM2 de l'individu \(i\) et \(a_i\) son âge au moment des examens\footnote{La nature en coupe de nos données implique que nous observons généralement un élève une fois. Les très rares cas où il est observé plusieurs fois correspondent aux redoublements de CM2. Ainsi, pour alléger les notations, nous n'utilisons des indices de classe, école et année que si cela est nécessaire.}. Comme \(a_i\) est construite avec les dates de naissance exactes, elle bénéficie d'un maximum de variabilité. Le vecteur \(x_{i}\) inclut les variables de contrôle suivantes : le sexe, la catégorie sociale et l'année scolaire\footnote{Contrôler par l'année scolaire permettrait de capturer l'effet des facteurs inobservés spécifiques à l'année (spécificité des examens, par exemple). De plus, cela atténue le biais lié à l'inflation de notes observée dans les données (voir Section \ref{agedata}).}. Le terme d'erreur \(\epsilon_{i}\) représente l'ensemble des facteurs inobservés qui déterminent les résultats aux évaluations nationales. Les facultés cognitives et non cognitives (que nous désignerons désormais par \emph{facultés}) constituent un exemple typique de ces caractéristiques.

\quad Le paramètre d'intérêt est \(\alpha_1\). Il désigne l'effet d'avoir une année en plus au moment des examens sur la note. Rappelons toutefois que dans notre contexte, il contient également l'effet de l'âge d'entrée et de la durée d'études (et, dans une moindre mesure, celui de l'âge relatif, comme discuté dans la Section \ref{ageintro}) puisque ces trois quantités varient parallèlement, pour un individu donné : l'âge aux examens est la somme de l'âge d'entrée et la durée d'études.

\quad Les différences d'âge entre élèves au moment des examens ne proviennent pas uniquement des différences dans les mois de naissance. Elles proviennent également du fait que les élèves dans la même année scolaire ne sont pas nés la même année, impliquant la présence des élèves en retard et en avance.

\quad Si un élève est en retard, c'est soit parce que ses parents ont décalé son inscription, soit parce qu'il a redoublé avant ou au CM2. Dans le premier cas, les parents auraient constaté que leur enfant n'est pas encore prêt à aller à l'école ou bien auraient voulu attendre que leur enfant soit relativement plus âgé que ses futurs camarades pour en tirer un avantage (\protect\hyperlink{ref-DAT:06}{Datar, 2006}). Dans le second cas où l'élève a redoublé, cela veut dire que ses résultats scolaires antérieurs ont été jugés comme insuffisants.\\
Le retard après redoublement ou retard d'entrée par manque de maturité induit une corrélation négative entre \(a_i\) et \(\epsilon_i\) tandis que le cas du retard par stratégie parentale induit une corrélation positive entre \(a_i\) et \(\epsilon_i\).\\

\quad Un élève est en avance au CM2 parce qu'il est entré en avance ou a sauté une ou plusieurs classes. Dans le premier cas, les parents auraient jugé que leur enfant est suffisamment mature pour entrer à l'école. Le second cas traduit le fait que l'enfant possède un capital humain et un potentiel d'accumulation de capital humain suffisant pour se permettre le saut de classe.\\
Tous les cas d'avance induisent une corrélation positive entre \(a_i\) et \(\epsilon_i\).

\quad Les données à disposition ne permettent pas de déterminer empiriquement lequel des cas de retard ci-dessus prévaut. Grenet (\protect\hyperlink{ref-GRE:09}{2009}) indique que pour la France entière, les redoublants sont les plus nombreux. Nous faisons l'hypothèse que cela est également vrai pour La Réunion. Cette hypothèse nous semble crédible, le système éducatif étant le même. La part des élèves en avance est moindre (2\%). Cela implique que l'estimation par les moindres carrés ordinaires de l'équation \eqref{eq:ageols} conduit à un paramètre sous-estimé.

\quad Pour résoudre ce problème, nous utilisons une stratégie basée sur les variables instrumentales (voir par exemple \protect\hyperlink{ref-BED:DHU:06}{Bedard \& Dhuey, 2006}).

\quad Pour développer notre stratégie d'identification, notons un fait que nous n'avons pas encore évoqué jusqu'ici : la date de naissance\footnote{Sous quelle forme que ce soit : jour ou mois, par exemple.} \emph{dans l'année} (c'est-à-dire indépendamment de l'année de naissance) est liée à la probabilité d'être en retard ou en avance. Cela est illustré par la Figure \ref{fig:agemoisposition} qui montre, pour chaque mois de naissance\footnote{Nous choisissons de prendre le mois de naissance au lieu du jour de naissance pour avoir un nombre assez élevé d'observations dans chaque modalité.}, la proportion d'individus en retard et en avance. Nous constatons clairement qu'être né de plus en plus tard (respectivement plus tôt) dans l'année augmente la probabilité d'être en retard (respectivement en avance). Ce lien entre la date de naissance dans l'année et le fait d'être en retard ou en avance peut être aisément expliqué. En effet, être né plus tard dans l'année civile équivaut à être relativement plus jeune, c'est-à-dire moins prêt à entrer à l'école ou encore être moins mature. Cela augmente alors la probabilité que les parents décalent l'entrée à l'école (qui jugent leur enfant trop jeune) ou de redoubler. Un raisonnement symétrique s'applique aux élèves en avance.

\begin{figure}[H]

{\centering \includegraphics[width=1\linewidth]{000_files/figure-latex/agemoisposition-1} 

}

\caption{Proportion d'élèves en retard et en avance par mois de naissance}\label{fig:agemoisposition}
\end{figure}

Cela nous amène à utiliser la date de naissance dans l'année comme instrument de l'âge aux examens.\\
Concrètement, nous ajoutons au modèle l'équation de première étape suivante :

\begin{equation}
\label{eq:agepe}
a_{i} = \delta_0 + \delta_1 z_{i} + x'_{i} \delta_2 + \nu_{i}, 
\end{equation}

où la variable \(z_{i}\) désigne la transformation linéaire du jour de naissance suivante :

\[
z_{i} = \frac{365 + I(t = 2012) - d_{i}}{365.25}, 
\]

telle que \(d_{i}\) représente le jour de naissance dans l'année de l'individu\footnote{\(d_i = 1\) pour un élève né le premier janvier, \(d_i = 2\) pour celui né le 02 Janvier, etc..} et \(I()\) la fonction indicatrice. L'écriture du numérateur permet de prendre en compte le fait que 2012 était une année bissextile. Ce mode de calcul fait ressortir la distance entre le jour de naissance de l'élève et le dernier jour de l'année (numérateur), exprimée en années (dénominateur)\footnote{Exprimer l'instrument en années est pratique pour interpréter le coefficient estimé de première étape puisque la variable dépendante de la première étape, l'âge aux examens, est exprimé en années.}. Elle prend des valeurs comprises entre 0 et 1. Dans la littérature des effets de l'âge, \(z_{i}\) est appelée \emph{âge relatif théorique} (\protect\hyperlink{ref-GRE:09}{Grenet, 2009}, par exemple)\footnote{\emph{Relatif} par rapport au plus jeune de l'année scolaire qui est né le dernier jour de l'année. \emph{Théorique} dans le sens où l'élève serait entré à l'heure à l'école et n'a connu aucun redoublement ou saut de classe.}. Cette variable équivaut à la variable de jour de naissance dans l'année. La transformation linéaire rend plus pratique l'interprétation des résultats d'estimation de la première étape. Le terme d'erreur \(\nu_{i}\) regroupe les facteurs inobservés déterminants de l'âge aux examens (parallèlement de l'âge d'entrée et de la durée d'études jusqu'au CM2). Ils peuvent contenir, par exemple, en plus des facultés, la maturité sous différents angles (\protect\hyperlink{ref-HAM:KOL:12}{Hámori \& Köllő, 2012}) qui déterminent le choix des parents de retarder ou d'avancer l'entrée à l'école de leur enfant ; ou la décision de redoublement ou de saut de classe de la part des responsables d'éducation.

\quad En suivant Imbens \& Angrist (\protect\hyperlink{ref-IMB:ANG:94}{1994}), Angrist \& Imbens (\protect\hyperlink{ref-ANG:IMB:95}{1995}) et Angrist et al. (\protect\hyperlink{ref-ANG:eal:96}{1996}), deux conditions doivent être remplies pour que notre instrument soit valide.

\quad Premièrement, \(z_i\) doit pouvoir être considérée comme exogène. Cela implique l'absence de corrélation entre la date de naissance et les facultés (déterminants inobservables des notes). Cette hypothèse peut être testée en analysant si certaines catégories sociales de parents tendent à faire naître leurs enfants à certaines périodes de l'année, et si oui, dans quelle mesure cela nuit à l'identification. Buckles \& Hungerman (\protect\hyperlink{ref-BUC:HUN:13}{2013}) et Grenet (\protect\hyperlink{ref-GRE:09}{2009}) montrent qu'il existe un lien entre la catégorie sociale des parents et la date de naissance. Nous nous demandons s'il en est de même à La Réunion. À La Réunion, la distribution des naissances n'est clairement pas uniforme, comme nous pouvons le voir sur la Figure \ref{fig:agenais}. Un test de \(\chi^2\) rejette d'ailleurs l'hypothèse d'uniformité de cette distribution.

\begin{figure}[H]

{\centering \includegraphics[width=1\linewidth]{000_files/figure-latex/agenais-1} 

}

\caption{Distribution des valeurs de l'âge relatif (en mois de naissance) à La Réunion}\label{fig:agenais}
\end{figure}

Cependant, la distribution du mois de naissance ne semble pas liée à la CSP des parents à La Réunion, comme nous pouvons le constater sur la Figure \ref{fig:agemoispcs}. L'année 2009 ne comportant pas de variable de CSP, cette figure est construite sur les données de 2010 à 2012. Les mois de naissance semblent distribués de la même manière pour toutes les catégories (y compris les valeurs manquantes), à l'exception de la catégorie ``Autres''. Au sein de cette catégorie, les élèves nés en septembre et en janvier sont sous-représentés. Toutefois, cela n'est très probablement dû qu'au très faible nombre d'observations dans la catégorie ``Autres'' (290, soit 0.7\% du total en comptant les valeurs manquantes). Ainsi, visuellement, nous ne détectons \emph{a priori} pas de lien entre la date de naissance et la catégorie sociale des parents à La Réunion.

En plus de ce diagnostic visuel, nous effectuons un test de \(\chi^2\) dont l'hypothèse nulle est l'indépendance entre le mois de naissance et la catégorie sociale.\footnote{Nous empilons toujours les 3 années 2010, 2011 et 2012 pour effectuer ce test et nous utilisons les valeurs manquantes comme modalité à part entière.} La statistique de test correspondante est de 71.2 (probabilité critique de 0.07). Cette conclusion demeure que l'on effectue le test par année et en excluant les valeurs ``Autres'' et manquantes de la catégorie sociale (voir Annexe \ref{agechisqsupp})\footnote{De plus, même en détectant un tel lien, le conditionnement sur la catégorie sociale atténuerait d'éventuels différences en facultés des enfants nés à différentes périodes de l'année. Intuitivement, même s'il s'avérait que les parents aisés visent le premier trimestre par exemple, il ne devrait pas y avoir de différences en facultés entre les aisés du premier trimestre et ceux des trimestres restants.}.~

\begin{figure}[H]

{\centering \includegraphics[width=1\linewidth]{000_files/figure-latex/agemoispcs-1} 

}

\caption{Distribution des mois de naissance par catégorie sociale à La Réunion}\label{fig:agemoispcs}
\end{figure}

L'exogénéité de l'instrument signifie également que \(z_i\) ne doit avoir d'effet sur \(y_i\) qu'à travers \(a_i\). Cette condition est appelée plus communément la restriction d'exclusion. Formulé autrement, la date de naissance ne devrait pas avoir d'effet direct sur la note (\protect\hyperlink{ref-DAT:06}{Datar, 2006}). Cette hypothèse n'est pas directement testable en l'absence d'autres instruments. Néanmoins, nous pouvons évaluer sa crédibilité en divisant la variation de \(z_i\) en plusieurs parties et en utilisant une partie de sa variation comme instrument et une autre partie comme variables de contrôle de l'équation structurelle \eqref{eq:ageols}. Concrètement, nous prenons l'indicatrice de naissance au mois de janvier comme instrument de l'âge aux examens et utilisons les autres indicatrices comme variables explicatives (voir \protect\hyperlink{ref-GOU:MAU:07}{Goux \& Maurin, 2007} qui utilisent une technique similaire). Les résultats sont indiqués dans la colonne (4) du Tableau \ref{tab:agemodels}. Nous constatons que l'effet de l'âge mesuré par le modèle est similaire aux résultats principaux. De plus, les coefficients devant les indicatrices de mois de naissance sont conjointement non significatifs. Bien que cela ne prouve pas l'absence d'effet direct de la forme principale de notre instrument sur la note, ce résultat nous fait aller dans ce sens\footnote{Utiliser des versions alternatives de bimestre ou trimestre de naissance (régressions non reportées) ne change pas cette conclusion.}.

\quad Deuxièmement, \(z_i\) doit être corrélée avec \(a_i\). Cette condition est quant à elle directement vérifiable avec l'estimation de la première étape (équation \ref{eq:agepe}). Le résultat est clair : être né en janvier plutôt qu'en décembre augmente la moyenne d'âge aux examens de 0.8 année (colonne 1 du Tableau \ref{tab:agemodels}). Ce coefficient est élevé (presque égale à 1) et très significatif. Cette forte relation entre l'âge relatif théorique et l'âge aux examens s'explique par la grande proportion d'élèves à l'heure dans le système (80\%).

\quad Enfin, il convient de donner un sens clair du coefficient estimé \(\hat{\alpha_1}\). Sous l'hypothèse d'homogénéité de l'effet de l'âge, il est connu que nous estimons un paramètre du type effet moyen de traitement. En effet, nous pouvons définir l'effet de l'âge pour un individu en particulier :

\begin{equation}
\label{eq:ageeffethetero}
\alpha_{1i} = \alpha_1 + e_{1i}
\end{equation}

Cette équation traduit le fait que l'effet de l'âge pour un individu est une variable aléatoire composée d'un effet moyen \(\alpha_1 = E(\alpha_{1i})\) et d'un terme aléatoire propre à l'individu \(e_{1i}\). Ainsi, l'équation structurelle (équation \ref{eq:ageols}) suppose implicitement \(\alpha_{1i} = \alpha_1, \forall \ {i}\). Cette hypothèse est difficilement crédible : l'effet de l'âge peut dépendre des facultés. Et certains parents peuvent choisir l'âge d'entrée (et donc dans une certaine mesure l'âge aux examens) de leur enfant en fonction de cette information. Dans cette configuration où l'effet de l'âge est hétérogène, nous savons que le paramètre \(\alpha_1\) n'a pas de sens bien défini (\protect\hyperlink{ref-ANG:IMB:95}{Angrist \& Imbens, 1995}).\\
\strut \\
Pour surmonter ce problème, Imbens \& Angrist (\protect\hyperlink{ref-IMB:ANG:94}{1994}) et Angrist \& Imbens (\protect\hyperlink{ref-ANG:IMB:95}{1995}) montrent que sous une hypothèse moins forte qu'est la monotonie, \(\alpha_1\) est l'effet de l'âge pour les individus dont la date de naissance affecte (au sens contrefactuel du terme) l'âge aux examens. Cette hypothèse implique que le signe de l'effet de \(z_i\) sur \(a_i\) (au sens contrefactuel du terme) doit être le même pour tous les individus. Ramenée à notre étude, l'hypothèse de monotonie implique qu'être né plus tard dans l'année (respectivement plus tôt) induise tous les individus à passer les examens plus jeunes (respectivement plus âgés). Or,~dans le contexte présent, nous savons qu'être né plus tard (respectivement plus tôt) augmente la probabilité d'entrer à l'école tardivement ou de redoubler (respectivement entrer à l'école en avance). Cela veut dire qu'être né plus tard (respectivement plus tôt) dans l'année induit certains élèves à être plus âgés (respectivement plus jeunes) aux examens. Cette dernière proposition est directement contraire à la condition de monotonie. La violation de l'hypothèse de monotonie dans la majorité des études sur l'effet de l'âge est bien documentée dans la littérature récente sur les effets de l'âge. Elle a par exemple été signalée par Barua \& Lang (\protect\hyperlink{ref-BAR:LAN:09}{2009}), discutée dans Black et al. (\protect\hyperlink{ref-BLA:eal:11}{2011}) et plus récemment explicitée dans Fiorini \& Stevens (\protect\hyperlink{ref-FIO:STE:21}{2021}).

\quad L'hypothèse de monotonie et d'homogénéité de l'effet de l'âge ne sont pas crédibles : nous ne pouvons pas correctement identifier la nature du paramètre \(\alpha_1\) par variables instrumentales traditionnelles. Pour surmonter ce problème, nous présentons le modèle avec une approche par fonction de contrôle, qui ne requiert pas l'hypothèse de monotonie, en suivant Hámori \& Köllő (\protect\hyperlink{ref-HAM:KOL:12}{2012}).
L'équation structurelle \eqref{eq:ageols} peut se réécrire en prenant directement en compte l'effet hétérogène de l'âge :

\begin{equation}
\label{eq:ageolshetero}
y_{i} = \alpha_0 + \alpha_{1i} a_{i} + x'_{i} \alpha_2 + \epsilon_{i}.
\end{equation}

Dans l'équation \eqref{eq:ageolshetero}, nous substituons \(\alpha_{1i}\) par son expression dans l'équation \eqref{eq:ageeffethetero} :

\begin{equation}
\label{eq:ageolsheterodev}
y_{i} = \alpha_0 + \alpha_1 a_{i} + x'_{i} \alpha_2 + \epsilon_{i} + e_{1i} a_{i}
\end{equation}

À partir de l'équation \eqref{eq:ageolsheterodev}, nous pouvons retrouver l'expression de \(E(y_i \mid a_i, x_i)\), en suivant Garen (\protect\hyperlink{ref-GAR:84}{1984, p. 1205}) :

\begin{equation}
\label{eq:ageolsheterodev2}
E(y_i \mid a_i, x_i) = \alpha_0 + \alpha_1 a_i + x_i' \alpha_2 + E(\epsilon_i +e_{1i} a_i \mid a_i, x_i),
\end{equation}

avec

\begin{equation}
\label{eq:ageolserror}
\begin{aligned}
& E(\epsilon_i +e_{1i} a_i \mid a_i, x_i) &=& 
E(\epsilon_i +e_{1i} a_i \mid a_i = \delta_0 + \delta_1 z_i + x_i'\delta_2 + \nu_i, x_i) \\
& &=& E(\epsilon_i + e_{1i} a_i \mid \nu_i, z_i, x_i) \\
& &=& E(\epsilon_i \mid \nu_i, z_i, x_i) \\
& && + E(e_{1i} a_i \mid \nu_i, z_i, x_i)
\end{aligned}
\end{equation}

Sous l'hypothèse d'exogénéité de \(z_i\) et considérant que \(x_i\) est exogène par construction, \(E(\epsilon_i \mid \nu_i, z_i, x_i) = E(\epsilon_i \mid \nu_i)\) et \(E(e_{1i} a_i \mid \nu_i, z_i, x_i) = E(e_{1i} a_i \mid \nu_i)\). Sachant cela et en s'aidant de l'équation \eqref{eq:ageolserror}, nous pouvons réécrire l'équation \eqref{eq:ageolsheterodev2} comme suit :

\begin{equation}
\label{eq:ageolsheterodev3}
E(y_i \mid a_i, x_i) = \alpha_0 + \alpha_1 a_i + x_i' \alpha_2 + E(\epsilon_i \mid \nu_i) + E(e_{1i} a_i \mid \nu_i) 
\end{equation}

À partir de l'équation \eqref{eq:ageolsheterodev3}, l'approche par fonction de contrôle prenant en compte l'hétérogénéité de l'effet de l'âge nécessite deux hypothèses qui concernent l'écriture des deux derniers termes. La première est la relation linéaire entre l'erreur de l'équation structurelle \(\epsilon_i\) et l'erreur de première étape \(\nu_i\). La seconde est la relation linéaire entre l'aléa de l'effet de l'âge \(e_{1i}\) et l'erreur de première étape \(\nu_i\). Cela nous permettra d'arriver à une réécriture de l'équation structurelle telle que les paramètres estimés par les moindres carrés ordinaires sont convergents (voir \protect\hyperlink{ref-WOO:15}{Wooldridge, 2015}, par exemple). Formellement, ces deux hypothèses peuvent s'écrire comme suit :

\begin{equation} 
\label{eq:agehypcf1}
E(\epsilon_i \mid \nu_i) = c_1 \nu_i
\end{equation}

et

\begin{equation}
\label{eq:agehypcf2}
E(e_{1i} \mid \nu_i) = c_2 \nu_i
\end{equation}

En intégrant les expressions des équations \eqref{eq:agehypcf1} et \eqref{eq:agehypcf2} dans l'équation \eqref{eq:ageolsheterodev3}, nous obtenons :

\begin{equation}
\label{eq:ageolscfhesp}
E(y_i \mid a_i, x_i) = \alpha_0 + \alpha_1 a_i + x_i' \alpha_2 + c_1 \nu_i + c_2 \nu_i a_i
\end{equation}

Dans l'équation \eqref{eq:ageolscfhesp}, les valeurs de \(\nu_i\) sont inconnues. Toutefois, grâce à la première étape (équation \ref{eq:agepe}), nous remarquons que \(\nu_i = a_i - \delta_0 - \delta_1 z_i\). Puisque les paramètres estimés issus de la première étape sont non biaisés et convergents (sous les deux hypothèses de validité de \(z_i\) que nous avons analysées plus haut), \(\hat{\nu_i} = a_i - \hat{\delta_0} - \hat{\delta_1} z_i\) est un estimateur sans biais et convergent de \(\nu_i\). L'idée est donc de remplacer \(\nu_i\) par \(\hat{\nu_i}\) dans l'équation \eqref{eq:ageolscfhesp}. Concrètement,

\begin{equation}
\label{eq:agecfhesp2}
\begin{aligned}
& E(y_i \mid a_i, x_i) &=& \alpha_0 + \alpha_1 a_i + x_i' \alpha_2 + c_1 \nu_i + c_2 \nu_i a_i
 + (c_1 \hat{\nu_i} - c_1 \hat{\nu_i})
 + (c_2 \hat{\nu_i} a_i - c_2 \hat{\nu_i} a_i) \\
& &=& \alpha_0 + \alpha_1 a_i + x_i' \alpha_2 + c_1 \hat{\nu_i} + c_2 \hat{\nu_i} a_i + c_1(\nu_i - \hat{\nu_i}) + c_2 (\nu_i - \hat{\nu_i}) a_i
\end{aligned}
\end{equation}

Nous obtenons alors l'équation estimable de l'approche par fonction de contrôle suivante qui prend en compte l'hétérogénéité de l'effet de l'âge :

\begin{equation}
\label{eq:agecfh}
y_i = \alpha_0 + \alpha_1 a_i + x_i' \alpha_2 + c_1 \hat{\nu_i} + c_2 \hat{\nu_i} a_i + s_i,
\end{equation}

avec

\[
s_i = c_1 (\nu_i - \hat{\nu_i}) + c_2 (\nu_i - \hat{\nu_i}) a_i + r_i,
\]
\(r_i\) étant un terme d'erreur par construction non corrélé avec les variables explicatives observables de l'équation \eqref{eq:agecfh}. L'estimateur par moindres carrés ordinaires de l'équation \eqref{eq:agecfh} est convergent\footnote{Le terme d'erreur \(s_i\) tend vers \(r_i\) en échantillon large car \(\hat{\nu}_i\) tend vers \(\nu_i\).}.

\quad L'équation \eqref{eq:agecfh} traduit l'intuition selon laquelle les facultés (\(\hat{\nu_i}\)) et son hétérogénéité en fonction de l'âge (\(\hat{\nu_i}a_i\)) sont explicitement pris en compte dans l'équation structurelle.

\quad Puisque \(\hat{\nu}_i\) est issu d'une première étape et est utilisé comme régresseur, les procédures d'inférence standard ne sont pas valides. Nous suivons alors Wooldridge (\protect\hyperlink{ref-WOO:15}{2015}) et calculons nos écart-types par wild bootstrap\footnote{Davidson \& Flachaire (\protect\hyperlink{ref-DAV:FLA:08}{2008}).} pour prendre en compte l'hétéroscédasticité de forme inconnue du terme d'erreur. La procédure exacte est décrite dans l'Annexe \ref{agewbootdesc}.

\hypertarget{agemethodsrd}{%
\subsection{Régression sur une discontinuité}\label{agemethodsrd}}

Puisque nos données contiennent suffisamment de cohortes de naissance, elles nous permettent d'identifier les effets de l'âge par régression sur une discontinuité. Ce type de régression permet de prendre en compte d'éventuels effets directs de la date de naissance. Elle permet également de confirmer nos résultats et comble la rareté des études sur les effets de l'âge en éducation menées sur le territoire français.

\quad Selon la règle d'entrée à l'école, les élèves nés au 31 décembre 1999 doivent normalement passer les examens de CM2 de 2010 tandis que ceux nés au 01 janvier 2000 doivent normalement les passer en 2011, soit en étant un an plus âgés (donc physiologiquement et intellectuellement plus matures)\footnote{Nous prenons particulièrement les années 1999 et 2000 dans cet exemple pour coller aux données et au modèle mobilisés pour les estimations.}. Sous l'hypothèse que les parents ne manipulent pas précisément la date de naissance de leur enfant pour des raisons liées à la réussite scolaire, nous pouvons supposer que c'est le hasard qui a déterminé que les uns soient nés un jour avant les autres. Dit autrement, c'est comme si leur âge aux examens leur avait été attribué aléatoirement. Une différence de performance aux examens entre ces deux individus peut alors être interprétée comme un effet causal de l'âge aux examens. L'idée de la régression sur une discontinuité revient à comparer les notes des élèves ayant un an de plus aux examens car nés en 2000 avec celles des élèves un an plus jeunes aux examens car nés en 1999.

\quad Concrètement, nous partons des élèves nés en 1999 et 2000. Nous justifions le choix de ces deux années par le fait que ce sont les deux seules années pour lesquelles nous récupérons toutes les positions (à l'heure, en retard et en avance), comme nous pouvons le voir dans le Tableau \ref{tab:agerdnaiscohpos} qui montre, par année de naissance et par cohorte, la présence ou non des élèves dans les trois positions dans les données. Les lignes avec les années de naissance 1999 et 2000 sont les seules complètes. Par exemple, nous n'avons pas dans nos données ceux nés en 2001 et plus (respectivement avant 1998) et qui sont en retard (respectivement en avance) car ils seraient en 2013 et au-delà (respectivement avant 2008), des périodes non couvertes par nos données.

\begingroup\fontsize{8}{10}\selectfont

\begin{ThreePartTable}
\begin{TableNotes}
\item \textit{Sources :} Fichiers CM2 (2009 à 2012), calculs de l'auteur.
\item \textit{Notes :} Renseigne la disponibilité des positions (en retard, à l'heure et en avance) selon l'année de naissance et l'année scolaire de CM2. Vide : individus non disponibles dans les données à disposition.
\end{TableNotes}
\begin{longtable}[t]{lllll}
\caption{\label{tab:agerdnaiscohpos}Années de naissance, année scolaire et position}\\
\toprule
\multicolumn{1}{c}{} & \multicolumn{4}{c}{Année scolaire} \\
\cmidrule(l{3pt}r{3pt}){2-5}
Année de naissance & 2009 & 2010 & 2011 & 2012\\
\midrule
\endfirsthead
\caption[]{\label{tab:agerdnaiscohpos}Années de naissance, année scolaire et position (suite)}\\
\toprule
Année de naissance & 2009 & 2010 & 2011 & 2012\\
\midrule
\endhead

\endfoot
\bottomrule
\insertTableNotes
\endlastfoot
1996 & En retard & - & - & -\\
1997 & En retard & En retard & - & -\\
1998 & À l'heure & En retard & En retard & -\\
1999 & En avance & À l'heure & En retard & En retard\\
2000 & En avance & En avance & À l'heure & En retard\\
2001 & - & En avance & En avance & À l'heure\\
2002 & - & - & En avance & En avance\\
2003 & - & - & - & En avance\\*
\end{longtable}
\end{ThreePartTable}
\endgroup{}

\quad Les caractéristiques descriptives de ceux nés en 1999 et 2000 (deuxième colonne du Tableau \ref{tab:agestats}) sont essentiellement les mêmes que celles de la totalité des individus à disposition (première colonne du Tableau \ref{tab:agestats}).

\hfill\break
La Figure \ref{fig:agerdgraph} illustre les moyennes d'âge (panel du haut) et de notes (panel du bas) par intervalles de 6 jours de date de naissance (excepté le premier, le plus à gauche de l'axe des abscisses, qui compte 5 jours)\footnote{Le choix de la largeur de l'intervalle n'est pas important tant qu'il y a assez d'observations dans chaque intervalle. Nous avons choisi la largeur de 6 jours pour être cohérent avec la présentation du modèle qui vient après : définir la date du 01 janvier 2000 exactement comme seuil. Ainsi, avec cette manière précise de construire les intervalles sur les dates de naissance de nos données (nés en 1999 et 2000), la borne gauche fermée du premier intervalle qui contient l'année de naissance 2000 est le 01 janvier 2000.}. La ligne verticale est tracée sur l'intervalle du 01\textsuperscript{er} au 06 janvier 2000. Son intérêt est de montrer le saut clair dans les moyennes d'âge et de la note en français par rapport à l'intervalle précédent dont la borne droite est le 31 décembre 1999.

\begin{figure}[H]

{\centering \includegraphics[width=1\linewidth]{000_files/figure-latex/agerdgraph-1} 

}

\caption{Moyennes d'âge aux examens et de note en français des élèves nés en 1999 et 2000, par intervalles de 6 jours de la date de naissance}\label{fig:agerdgraph}
\end{figure}

\quad Nous nous demandons dans quelle mesure le saut sur la moyenne d'âge aux examens est la cause du saut sur la moyenne de la note en français. Pour répondre à cette question, nous proposons l'équation structurelle correspondant à la régression sur une discontinuité ci-après :

\begin{equation}
\label{eq:ageolsrd}
y_i = \alpha_0 + \alpha_1 a_i + \sum_{k = 1}^p \pi_{(1)k} dist_i^k + \sum_{k = 1}^p \pi_{(1)p+k} old_i dist_i^k + x'_i \alpha_2 + \epsilon_i.
\end{equation}

Pour alléger les notations, nous utilisons essentiellement les mêmes que dans la Section \ref{agemethodscfh}. L'année scolaire n'est pas incluse dans \(x_i\) car elle représenterait ici essentiellement la même information que l'âge aux examens.\\
La variable \(dist_i\) est la distance en jours entre la date de naissance et le seuil (01\textsuperscript{er} janvier 2000). Elle est différente pour deux individus nés le même jour mais pas la même année et n'est donc pas à confondre avec la variable instrumentale de la Section \ref{agemethodscfh}. Elle est centrée au 01 janvier 2000. En guise d'illustration, \(dist_i = 0, - 1, - 2, 1\) et \(2\) pour des individus nés le 01 janvier 2000, 02 janvier 2000, 31 décembre 1999 et 30 décembre 1999 respectivement. La même logique s'applique pour toutes les dates de naissance de 1999 et 2000. La dichotomique \(old_i\) indique si l'individu est né en 2000 ou en 1999. Plus formellement,

\[
old_i = 
\begin{cases} 
1 \ si \ dist_i \geq 0 \\
0 \ si \ dist_i < 0
\end{cases}
\]

Le terme \(\sum_{k = 1}^p \pi_{(1)k} dist_i^k + \sum_{k = 1}^p \pi_{(1)p+k} old_i dist_i^k\) est un polynôme d'ordre \(p\) qui capture la tendance continue entre la variable dépendante et la date de naissance.

Le terme d'interaction \(old_i dist_i\) reflète le fait que nous autorisons la relation continue à se comporter différemment de part et d'autre du seuil.
Notre paramètre d'intérêt est toujours \(\alpha_1\). Il représente l'effet causal d'avoir un an de plus aux examens sur la note.

\quad Nous nous limitons à \(p = \{1, 2\}\) (\protect\hyperlink{ref-GEL:IMB:19}{Gelman \& Imbens, 2019}) et pour toute estimation, nous regardons systématiquement les résultats pour chacun de ces deux degrés de polynôme.

\quad Nous commençons par effectuer nos estimations en utilisant des fenêtres de 30 jours, c'est-à-dire en utilisant uniquement les individus nés 30 jours avant et 30 jours après le seuil. Une fenêtre de plus en plus étroite rend de plus en plus similaires en inobservables les enfants nés d'une part et d'autre du 01 janvier 2000, au détriment de la précision de nos estimations. Nous discutons de la sensibilité de nos résultats en fonction de la largeur de la fenêtre.

Les élèves nés en 1999 ou en 2000 ne passent pas tous leur examen en 2010 ou en 2011, respectivement. Les élèves en retard le passent après et les élèves en avance le passent avant. Les raisons de ces décalages sont essentiellement les mêmes que celles exposées dans la Section \ref{agemethodscfh}, c'est-à-dire liées aux facultés. Cela nous ramène à l'idée d'instrumenter l'âge \(a_i\) par l'année de naissance \(old_i\). Nous intégrons alors dans notre modèle l'équation de première étape suivante :

\begin{equation}
\label{eq:ageperd}
a_i = \delta_0 + \delta_1 old_i + \sum_{k = 1}^p \pi_{(2)k} dist_i^k + \sum_{k = 1}^p \pi_{(2)p+k} old_i dist_i^k + x' _i \delta_2 + \nu_i.
\end{equation}

Nous trouvons dans le cadre d'un \emph{fuzzy regression design} (\protect\hyperlink{ref-IMB:LEM:08}{Imbens \& Lemieux, 2008}). Une différence par rapport au modèle théorique est que notre variable endogène \(a_i\) est une variable continue au lieu d'une variable binaire (passer ou pas ses examens en 2011). Si nous notons \(w_i\) une telle variable, elle vaudrait 1 si l'élève passe ses examens en 2011 et 0 sinon. Cette valeur 0 n'a pas de sens clair car elle englobe les élèves qui passent les examens en 2010 mais aussi en 2009 (nés en 1999 et en avance) et 2012 (nés en 2000 avec deux ans de retard) comme le Tableau \ref{tab:agerdnaiscohpos} le montre. Autrement dit, en prenant \(w_i\) qui collerait mieux aux modèles standards de régressions sur une discontinuité, nous comparerons l'effet de passer les examens en 2011 plutôt qu'une autre. Cet effet n'équivaut pas à l'effet d'avoir un an de plus aux examens.~
Smith (\protect\hyperlink{ref-SMI:09}{2009}) utilise similairement une spécification continue de l'âge à la seule différence que ses données contiennent plusieurs seuils de discontinuité, l'auteur possédant plus de deux cohortes d'années de naissance pour lesquelles il récupère toutes les positions possibles.

\quad Pour que \(old_i\) soit un instrument valide, il ne faudrait pas que les parents de La Réunion manipulent la date de naissance de leur enfant de manière à le faire naître avant ou après le seuil pour des raisons liées à la réussite scolaire. Cette hypothèse est directement le parallèle de l'hypothèse d'exogénéité de l'instrument \(z_i\) de la Section \ref{agemethodscfh}. En effet, les parents les plus avantagés pourraient être capables de choisir stratégiquement la date de conception pour viser un intervalle de mois de naissance de leur enfant. Ils pourraient de plus avoir accès aux pratiques médicales permettant de choisir de faire naître leur enfant d'un côté de l'autre du seuil. Nous analysons cette possibilité en testant si la densité de la date de naissance présente un saut significatif au niveau du 01 janvier 2000 (\protect\hyperlink{ref-MCC:08}{McCrary, 2008}). L'hypothèse nulle est l'absence de discontinuité. Nous appliquons sans nous restreindre à une fenêtre. Notre période d'étude présente une spécificité puisqu'il y a eu un \emph{baby boom} en 2000, dans laquelle nous comptons 4.2\% naissances en plus par rapport à 1999. Cette brusque augmentation entraîne un rejet de l'hypothèse nulle du test de McCrary (\protect\hyperlink{ref-MCC:08}{2008}) (probabilité critique de 0.03), dans sa version originale. Ce rejet peut faire craindre \emph{a priori} une présence de manipulation de la date de naissance de la part des parents. Cependant, ce saut n'est vraisemblablement pas lié à des stratégies liées à la réussite scolaire puisqu'elle a été documentée pour toute la France et dans plusieurs pays (\protect\hyperlink{ref-DOI:01}{Doisneau, 2001}). Nous apportons alors une légère modification au test : nous calculons les fréquences de base qui servent à estimer la densité, non pas sur la totalité des observations (version originale), mais sur chaque côté du seuil. La probabilité critique dans la version corrigée est de 0.09 et ne nous permet pas de rejeter l'hypothèse nulle au seuil de 5\%. Les résultats des deux versions du test sont montrés par la Figure \ref{fig:agerdmctestgraph}.

\begin{figure}[H]

{\centering \includegraphics[width=1\linewidth]{000_files/figure-latex/agerdmctestgraph-1} 

}

\caption{Tests de discontinuité de McCrary (2008) sur la densité de la date de naissance}\label{fig:agerdmctestgraph}
\end{figure}

\quad Nous avons alors des raisons de penser que l'hypothèse d'absence de manipulation précise de la date de naissance à La Réunion est crédible.

\quad Le prochain test consiste à regarder si les observables sont similaires d'un côté et d'autre du seuil. Cela crédibiliserait plus l'hypothèse selon laquelle les inobservables le seraient aussi. Pour ce faire, nous estimons la forme réduite du modèle en utilisant des pseudo-variables dépendantes : les variables de contrôle. Concrètement, pour une pseudo-variable dépendante \(y^0_i\), nous estimons :

\begin{equation}
\label{eq:agerdbcheck}
y^0_i = \rho_0 + \rho_1 old_i + \sum_{k = 1}^p \pi_{(3)k} dist_i^k + \sum_{k = 1}^p \pi_{(3)p+k} old_i dist_i^k + x_i ' \rho_2 + \eta_i
\end{equation}

Les \(y^0_i\) possibles sont les indicatrices suivantes : garçon, catégorie sociale moyenne, favorisée et très favorisée. Nous estimons l'équation \eqref{eq:agerdbcheck} par les moindres carrés ordinaires dans une fenêtre de 30 jours et pour \(p = \{1, 2\}\). À chaque valeur estimée de \(\rho_1\), nous effectuons un test de significativité. Ainsi, si la distribution des inobservables est continue d'une part et d'autre du seuil, nous nous attendons à obtenir des \(\hat{\rho}_1\) tous non significatifs. Les résultats sont illustrés dans le Tableau \ref{tab:agebchecks} et sont clairs : nous ne trouvons aucun coefficient significatif à 5\%.

\begingroup\fontsize{8}{10}\selectfont

\begin{ThreePartTable}
\begin{TableNotes}
\item \textit{Sources :} Fichiers CM2 (2009 à 2012) - élèves nés en 1999 et 2000, calculs de l'auteur.
\item \textit{Notes :} Tests pour détecter d'éventuels discontinuité des observables au niveau du seuil. Simples régressions linéaires avec variables dépendantes binaires. Une colonne correspond à une régression. On note $p$ le degré de polynôme de $dist$ et $old \times dist$ utilisé. Les contrôles utilisés sont le sexe et la CSP lorsque cela est possible. Fenêtre de 30 jours.
\item CSP : Catégorie Socio-Professionnelle.
\item Significativité : 10\% * 5\% ** 1\% ***.
\end{TableNotes}
\begin{longtable}[t]{lllllllll}
\caption{\label{tab:agebchecks}Tests d'équilibre des observables d'un côté à l'autre du seuil : résultats de régressions avec des pseudo-variables dépendantes}\\
\toprule
\multicolumn{1}{c}{} & \multicolumn{8}{c}{Variable dépendante : } \\
\cmidrule(l{3pt}r{3pt}){2-9}
\multicolumn{1}{c}{} & \multicolumn{2}{c}{Sexe - Garçon} & \multicolumn{2}{c}{CSP - Moyenne} & \multicolumn{2}{c}{CSP - Favorisée} & \multicolumn{2}{c}{CSP - Très favorisée} \\
\cmidrule(l{3pt}r{3pt}){2-3} \cmidrule(l{3pt}r{3pt}){4-5} \cmidrule(l{3pt}r{3pt}){6-7} \cmidrule(l{3pt}r{3pt}){8-9}
 & $p=1$ & $p=2$ & $p=1$ & $p=2$ & $p=1$ & $p=2$ & $p=1$ & $p=2$\\
\midrule
\endfirsthead
\caption[]{\label{tab:agebchecks}Tests d'équilibre des observables d'un côté à l'autre du seuil : résultats de régressions avec des pseudo-variables dépendantes (suite)}\\
\toprule
\multicolumn{1}{c}{} & \multicolumn{8}{c}{Variable dépendante : } \\
\cmidrule(l{3pt}r{3pt}){2-9}
\multicolumn{1}{c}{} & \multicolumn{2}{c}{Sexe - Garçon} & \multicolumn{2}{c}{CSP - Moyenne} & \multicolumn{2}{c}{CSP - Favorisée} & \multicolumn{2}{c}{CSP - Très favorisée} \\
\cmidrule(l{3pt}r{3pt}){2-3} \cmidrule(l{3pt}r{3pt}){4-5} \cmidrule(l{3pt}r{3pt}){6-7} \cmidrule(l{3pt}r{3pt}){8-9}
 & $p=1$ & $p=2$ & $p=1$ & $p=2$ & $p=1$ & $p=2$ & $p=1$ & $p=2$\\
\midrule
\endhead

\endfoot
\bottomrule
\insertTableNotes
\endlastfoot
old & $-$0.027 & $-$0.014 & $-$0.046 & $-$0.098$^{*}$ & 0.02 & 0.006 & 0.003 & 0.027\\
 & (0.042) & (0.064) & (0.034) & (0.051) & (0.018) & (0.028) & (0.024) & (0.035)\\
dist & 0.002 & 0.002 & 0.001 & 0.006 & $-$0.001 & $-$0.002 & 0 & $-$0.003\\
 & (0.002) & (0.007) & (0.001) & (0.006) & (0.001) & (0.003) & (0.001) & (0.004)\\
dist$^2$ & - & 0 & - & 0 & - & 0 & - & 0\\
 & - & (0) & - & (0) & - & (0) & - & (0)\\
old $\times$ dist & $-$0.001 & $-$0.003 & 0 & 0 & 0.001 & 0.005 & $-$0.001 & 0.002\\
 & (0.002) & (0.01) & (0.002) & (0.008) & (0.001) & (0.004) & (0.001) & (0.005)\\
old $\times$ dist$^2$ & - & 0 & - & 0 & - & 0 & - & 0\\
 & - & (0) & - & (0) & - & (0) & - & (0)\\
 &  &  &  &  &  &  &  & \\
Contrôles & Oui & Oui & Oui & Oui & Oui & Oui & Oui & Oui\\
Observations & 2271 & 2271 & 2271 & 2271 & 2271 & 2271 & 2271 & 2271\\
R$^2$ ajusté & -0.001 & -0.002 & -0.001 & -0.001 & -0.001 & -0.001 & -0.001 & -0.002\\*
\end{longtable}
\end{ThreePartTable}
\endgroup{}

\quad Le dernier test de validité consiste à vérifier qu'il n'existe pas d'autres discontinuités sur la note à d'autres dates de naissance que le seuil. Si tel était le cas, il faudrait prendre en compte ces multiples discontinuités et notre spécification actuelle ne serait pas valide.~
Concrètement, nous suivons toujours Imbens \& Lemieux (\protect\hyperlink{ref-IMB:LEM:08}{2008}) et estimons l'équation \eqref{eq:agerdbcheck} avec les changements ci-après. La variable dépendante est désormais la note totale, contrairement à celles du test précédent où nous retenions les observables (sexe et catégorie sociale). Nous n'utilisons pas de fenêtre. Pour chaque côté du seuil, nous déterminons la médiane de \(dist_i\). La médiane à gauche et à droite du seuil correspond respectivement au 30 Juin 1999 et au 04 Juillet 2000. Ensuite, pour chaque côté du seuil, nous estimons l'équation \eqref{eq:agerdbcheck} en remplaçant \(old_i\) par l'indicatrice si l'individu est né après la médiane ou pas. L'idée est d'utiliser des seuils placebos au niveau desquels il ne devrait pas y avoir de discontinuité de la note en fonction de la date de naissance.~
L'intérêt principal de diviser l'échantillon en deux (à gauche et à droite du seuil) est d'exclure du test le seuil d'origine au niveau duquel nous savons qu'il y a une discontinuité. Les résultats montrent que les coefficients devant les indicatrices de médiane ne sont en aucun cas significatifs à 5\% (Tableau \ref{tab:agecutchecks}) : il n'y a vraisemblablement pas de discontinuité de la note à d'autres dates de naissance que le 01\textsuperscript{er} Janvier 2000\footnote{Par ailleurs, cela se vérifie visuellement en regardant le panel du bas de la Figure \ref{fig:agerdgraph}.}.

\begingroup\fontsize{8}{10}\selectfont

\begin{ThreePartTable}
\begin{TableNotes}
\item \textit{Sources :} Fichiers CM2 (2009 à 2012) - élèves nés en 1999 et 2000, calculs de l'auteur.
\item \textit{Notes :} Tests pour détecter d'éventuels discontinuités ailleurs qu'au niveau du seuil. Une colonne correspond à une régression. On note $p$ le degré de polynôme de $dist$ et $old \times dist$ utilisé. Les contrôles utilisés sont le sexe et la CSP. Fenêtre de 30 jours.
\item CSP : Catégorie Socio-Professionnelle.
\item Significativité : 10\% * 5\% ** 1\% ***.
\end{TableNotes}
\begin{longtable}[t]{lllll}
\caption{\label{tab:agecutchecks}Tests de présence d'autres discontinuités : régressions séparées d'un côté et de l'autre du seuil}\\
\toprule
\multicolumn{1}{c}{} & \multicolumn{4}{c}{Variable dépendante : Note en français} \\
\cmidrule(l{3pt}r{3pt}){2-5}
\multicolumn{1}{c}{} & \multicolumn{2}{c}{Nés en 1999} & \multicolumn{2}{c}{Nés en 2000} \\
\cmidrule(l{3pt}r{3pt}){2-3} \cmidrule(l{3pt}r{3pt}){4-5}
 & $p=1$ & $p=2$ & $p=1$ & $p=2$\\
\midrule
\endfirsthead
\caption[]{\label{tab:agecutchecks}Tests de présence d'autres discontinuités : régressions séparées d'un côté et de l'autre du seuil (suite)}\\
\toprule
\multicolumn{1}{c}{} & \multicolumn{4}{c}{Variable dépendante : Note en français} \\
\cmidrule(l{3pt}r{3pt}){2-5}
\multicolumn{1}{c}{} & \multicolumn{2}{c}{Nés en 1999} & \multicolumn{2}{c}{Nés en 2000} \\
\cmidrule(l{3pt}r{3pt}){2-3} \cmidrule(l{3pt}r{3pt}){4-5}
 & $p=1$ & $p=2$ & $p=1$ & $p=2$\\
\midrule
\endhead

\endfoot
\bottomrule
\insertTableNotes
\endlastfoot
À droite de la médiane (1999) & 0.059$^{*}$ & 0.121 & - & -\\
 & (0.032) & (0.087) & - & -\\
À droite de la médiane (2000) & - & - & $-$0.046 & 0.035\\
 & - & - & (0.032) & (0.087)\\
dist & $-$0.001$^{***}$ & $-$0.002$^{**}$ & 0$^{***}$ & $-$0.001$^{*}$\\
 & (0) & (0.001) & (0) & (0.001)\\
dist$^2$ & - & 0 & - & 0\\
 & - & (0) & - & (0)\\
À droite de la médiane (1999) $\times$ dist$^2$ & - & 0 & - & -\\
 & - & (0) & - & -\\
À droite de la médiane (2000) $\times$ dist$^2$ & - & - & - & 0\\
 & - & - & - & (0)\\
 &  &  &  & \\
Contrôles & Oui & Oui & Oui & Oui\\
Observations & 13654 & 13654 & 14239 & 14239\\
R$^2$ ajusté & 0.099 & 0.099 & 0.128 & 0.128\\*
\end{longtable}
\end{ThreePartTable}
\endgroup{}

\quad L'hypothèse de monotonie est également nécessaire (\protect\hyperlink{ref-HAH:eal:01}{Hahn et al., 2001}). Elle implique qu'être né après (respectivement avant) le seuil devrait induire les individus à être un an plus âgés (respectivement moins âgés) au moment des examens. Or, être né après (respectivement avant) le seuil augmente la probabilité d'être en avance (respectivement en retard), soit un an moins âgé (respectivement plus âgé) au moment des examens. Cette dernière proposition est contraire à la condition de monotonie. Nous estimons alors l'équation \eqref{eq:ageolsrd} par fonction de contrôle en utilisant l'équation \eqref{eq:ageperd} comme première étape. Nous mobilisons exactement le même raisonnement que dans la Section \ref{agemethodscfh}, pour ce faire.
Néanmoins, suivant Black et al. (\protect\hyperlink{ref-BLA:eal:11}{2011}), les résultats obtenus avec les variables instrumentales usuelles du FRD devraient fournir une approximation relativement pertinente des effets de l'âge du type effets moyens de traitement étant donné la grande proportion qu'occupe les élèves à l'heure\footnote{Des résultats non reportés lorsque les équations de FRD sont estimées par variables instrumentales sont pratiquement identiques à ceux estimés par fonction de contrôle.}.

\quad Au total, le contexte et les données semblent adaptés à l'application du modèle de régression sur une discontinuité pour calculer les effets de l'âge aux examens sur les performances scolaires. Il est toutefois connu que le modèle de régression sur une discontinuité identifie un effet de traitement local, dans le sens où il s'agit de l'effet au voisinage du seuil.

\hypertarget{ageres}{%
\section{Résultats et discussions}\label{ageres}}

\hypertarget{agemodelsres}{%
\subsection{L'effet de l'âge aux examens : résultats de l'approche par fonction de contrôle}\label{agemodelsres}}

Le Tableau \ref{tab:agemodels} illustre nos résultats principaux. Les coefficients estimés en première étape (équation \ref{eq:agepe}) apparaissent dans la colonne (1). Nous avons déjà mis en évidence dans la Section \ref{agemethodscfh} que nous n'avons pas un problème d'instrument faible (coefficient de 0.8 sur l'instrument). Ce propos est renforcé par une statistique de Fisher de 2540 largement supérieur à 10 (\protect\hyperlink{ref-STA:STO:97}{Staiger \& Stock, 1997}).

\quad Les colonnes (2) à (5) présentent respectivement les résultats d'estimation par les moindres carrés ordinaires\footnote{Elles sont présentées afin d'illustrer le biais de sous-estimation des effets de l'âge aux examens discuté précédemment.}, les doubles moindres carrés, les doubles moindres carrés avec test d'effet direct d'une partie de la variation de l'instrument, et finalement par la fonction de contrôle (équation \ref{eq:agecfh}).

\quad Selon l'estimation ``naïve'' par les moindres carrés ordinaires, un an de plus au moment des examens procure 0.5 unité d'écart-type de désavantage aux évaluations de CM2. Nous savons toutefois que ce résultat est biaisé surtout par la présence d'élèves en retard parmi les plus âgés et dans une moindre mesure celle des élèves en avance parmi les plus jeunes.

\quad Dès que l'âge relatif théorique est utilisé comme instrument de l'âge aux examens (colonne 3), nous constatons effectivement un coefficient estimé de 0.33 unité d'écart-type très significatif. Ce changement de signe par rapport à la colonne (2) confirme bien le signe attendu du biais dû aux élèves en retard. Ce biais est négatif et est de l'ordre de - 0.23.\\
Un changement d'une telle ampleur entre le coefficient des moindres carrés ordinaires et celui des doubles moindres carré peut être étonnant. Plusieurs éléments nous rassurent. Premièrement, il s'agit d'un résultat communément observé dans plusieurs pays dans lesquels le retard d'entrée et/ou le redoublement ne sont pas négligeables. Nous pensons que la meilleure référence pour appuyer ce dernier propos est le Tableau III de Bedard \& Dhuey (\protect\hyperlink{ref-BED:DHU:06}{2006}). En effet, l'étude couvre plusieurs pays de l'OCDE et utilisent des tests de même nature dispensés à travers ces pays. Les taux d'élèves en retard documentés par ces auteurs dans leur Annexe 1 montrent que la France et le Portugal font partie des pays de l'OCDE avec le plus d'élèves en retard. Être en retard est de plus un signal fort de facultés relativement faibles (\protect\hyperlink{ref-ALE:eal:13}{Alet et al., 2013}). Tous ces éléments rationalisent la différence substantielle entre un coefficient d'effet d'âge aux examens biaisé à cause de l'omission des facultés (colonne 2) et un autre qui prend en compte ces dernières (colonne 3).\\
Deuxièmement et plus concrètement, les coefficients associés aux variables de contrôle sont comparables en signe et en ampleur entre les deux modèles. Par exemple, l'effet d'être un garçon plutôt qu'une fille est compris entre - 0.23 et - 0.3. De la même manière, être un enfant issu de parents très favorisés procure un avantage entre 0.8 et 1 unité d'écart-type. Seul le signe du coefficient associé à l'âge aux examens change entre les colonnes (2) et (3).

\newpage
\begingroup\fontsize{7}{9}\selectfont

\begin{ThreePartTable}
\begin{TableNotes}
\item \textit{Sources :} Fichiers CM2 (2009 à 2012), calculs de l'auteur.
\item \textit{Notes :} Une colonne correspond à une régression. La note est normalisée sur l'année scolaire. Écart-types entre parenthèses. Les écart-types dans la colonne (5) sont calculés par wild bootstrap avec 1001 réplications. La variable $\hat{\nu}$ est le résidu de la première étape. Les contrôles utilisés sont le sexe, la CSP et l'année scolaire.
\item MCO : Moindres Carrés Ordinaires, VI : Variable Instrumentale. VI-EXCL : Variable Instrumentale avec test indirect de la restriction d'Exclusion. FCH : Fonction de Contrôle avec prise en compte de l'Hétérogénéité de l'effet de l'âge. CSP : Catégorie Socio-Professionnelle.
\item Significativité : 10\% * 5\% ** 1\% ***.
\end{TableNotes}
\begin{longtable}[t]{llllll}
\caption{\label{tab:agemodels}Tests et résultats principaux sur les effets de l'âge aux examens}\\
\toprule
\multicolumn{1}{c}{} & \multicolumn{5}{c}{Variable dépendante : } \\
\cmidrule(l{3pt}r{3pt}){2-6}
\multicolumn{1}{c}{} & \multicolumn{1}{c}{Âge aux examens} & \multicolumn{4}{c}{Note totale} \\
\cmidrule(l{3pt}r{3pt}){2-2} \cmidrule(l{3pt}r{3pt}){3-6}
 & \makecell{\makecell{Première \\ étape} \\ (1) } & \makecell{\makecell{MCO \\ \ } \\ (2) } & \makecell{\makecell{VI \\ \ } \\ (3) } & \makecell{\makecell{VI-EXCL \\ \ } \\ (4) } & \makecell{\makecell{Fonction de \\ contrôle} \\ (5) }\\
\midrule
\endfirsthead
\caption[]{\label{tab:agemodels}Tests et résultats principaux sur les effets de l'âge aux examens (suite)}\\
\toprule
\multicolumn{1}{c}{} & \multicolumn{5}{c}{Variable dépendante : } \\
\cmidrule(l{3pt}r{3pt}){2-6}
\multicolumn{1}{c}{} & \multicolumn{1}{c}{Âge aux examens} & \multicolumn{4}{c}{Note totale} \\
\cmidrule(l{3pt}r{3pt}){2-2} \cmidrule(l{3pt}r{3pt}){3-6}
 & \makecell{\makecell{Première \\ étape} \\ (1) } & \makecell{\makecell{MCO \\ \ } \\ (2) } & \makecell{\makecell{VI \\ \ } \\ (3) } & \makecell{\makecell{VI-EXCL \\ \ } \\ (4) } & \makecell{\makecell{Fonction de \\ contrôle} \\ (5) }\\
\midrule
\endhead

\endfoot
\bottomrule
\insertTableNotes
\endlastfoot
Âge aux examens & - & $-$0.558$^{***}$ & 0.328$^{***}$ & 0.366$^{***}$ & 0.326$^{***}$\\
 & - & (0.008) & (0.019) & (0.029) & (0.016)\\
Âge relatif théorique & 0.809$^{***}$ & - & - & - & -\\
 & (0.006) & - & - & - & -\\
$\hat{\nu}$ & - & - & - & - & $-$0.529$^{***}$\\
 & - & - & - & - & (0.146)\\
Âge aux examens $\times$ $\hat{\nu}$ & - & - & - & - & $-$0.056$^{***}$\\
 & - & - & - & - & (0.013)\\
Sexe - Homme & 0.075$^{***}$ & $-$0.229$^{***}$ & $-$0.295$^{***}$ & $-$0.297$^{***}$ & $-$0.292$^{***}$\\
 & (0.004) & (0.008) & (0.009) & (0.009) & (0.008)\\
CSP - Moyenne & $-$0.115$^{***}$ & 0.311$^{***}$ & 0.417$^{***}$ & 0.421$^{***}$ & 0.413$^{***}$\\
 & (0.005) & (0.011) & (0.012) & (0.013) & (0.011)\\
CSP - Favorisée & $-$0.166$^{***}$ & 0.479$^{***}$ & 0.637$^{***}$ & 0.644$^{***}$ & 0.632$^{***}$\\
 & (0.006) & (0.018) & (0.02) & (0.02) & (0.018)\\
CSP - Très favorisée & $-$0.239$^{***}$ & 0.797$^{***}$ & 1.015$^{***}$ & 1.024$^{***}$ & 1.01$^{***}$\\
 & (0.006) & (0.015) & (0.016) & (0.017) & (0.015)\\
CSP - Autre & 0.042$^{*}$ & 0.096$^{*}$ & 0.062 & 0.062 & 0.063\\
 & (0.025) & (0.058) & (0.06) & (0.061) & (0.057)\\
CSP - Manquante & 0.07$^{***}$ & $-$0.092$^{***}$ & $-$0.146$^{***}$ & $-$0.149$^{***}$ & $-$0.141$^{***}$\\
 & (0.007) & (0.014) & (0.016) & (0.017) & (0.014)\\
Année - 2010 & 0.055$^{***}$ & $-$0.275$^{***}$ & $-$0.322$^{***}$ & $-$0.324$^{***}$ & $-$0.317$^{***}$\\
 & (0.008) & (0.016) & (0.019) & (0.019) & (0.015)\\
Année - 2011 & 0.038$^{***}$ & $-$0.288$^{***}$ & $-$0.315$^{***}$ & $-$0.316$^{***}$ & $-$0.311$^{***}$\\
 & (0.008) & (0.016) & (0.019) & (0.019) & (0.016)\\
Année - 2012 & 0.375$^{***}$ & $-$0.111$^{***}$ & $-$0.44$^{***}$ & $-$0.455$^{***}$ & $-$0.436$^{***}$\\
 & (0.008) & (0.017) & (0.02) & (0.022) & (0.017)\\
Mois de naissance - Février & - & - & - & $-$0.053$^{**}$ & -\\
 & - & - & - & (0.021) & -\\
Mois de naissance - Mars & - & - & - & $-$0.038$^{*}$ & -\\
 & - & - & - & (0.02) & -\\
Mois de naissance - Avril & - & - & - & $-$0.007 & -\\
 & - & - & - & (0.019) & -\\
Mois de naissance - Mai & - & - & - & $-$0.03 & -\\
 & - & - & - & (0.018) & -\\
Mois de naissance - Juin & - & - & - & 0.004 & -\\
 & - & - & - & (0.018) & -\\
Mois de naissance - Juillet & - & - & - & $-$0.014 & -\\
 & - & - & - & (0.018) & -\\
Mois de naissance - Août & - & - & - & $-$0.024 & -\\
 & - & - & - & (0.018) & -\\
Mois de naissance - Septembre & - & - & - & $-$0.012 & -\\
 & - & - & - & (0.019) & -\\
Mois de naissance - Octobre & - & - & - & $-$0.012 & -\\
 & - & - & - & (0.02) & -\\
Mois de naissance - Novembre & - & - & - & 0.013 & -\\
 & - & - & - & (0.021) & -\\
 &  &  &  &  & \\
Observations & 54341 & 54341 & 54341 & 54341 & 54341\\
R$^2$ ajusté & 0.318 & 0.167 & - & - & 0.223\\*
\end{longtable}
\end{ThreePartTable}
\endgroup{}
\newpage

\quad La colonne (4) a été discutée précédemment et sert à discuter de la présence d'effets directs de la date de naissance sur la réussite aux examens. Au vu de la similarité des coefficients estimés de cette colonne (surtout celui associé à l'âge aux examens) avec ceux de la colonne précédente\footnote{Un \(z\)-test d'égalité des coefficients associés à l'âge ne permet pas de rejeter l'hypothèse nulle (probabilité critique de 0.28).} ; et de l'absence de significativité jointe des coefficients associés aux indicatrices de mois de naissance, nous avons toutes les raisons de penser que d'éventuels effets directs de saison de naissance ne remettent pas en cause nos conclusions (\protect\hyperlink{ref-GOU:MAU:07}{Goux \& Maurin, 2007}).

\quad Les résultats de l'estimation par fonction de contrôle se trouvent en colonne (5). L'effet estimé correspond dès lors à une estimation de l'effet de l'âge pour un individu tiré aléatoirement de la population (du type effet moyen de traitement). Le coefficient est de 0.326 : un an de plus aux examens procure en moyenne un avantage d'environ 0.33 unité d'écart-type aux évaluations de CM2\footnote{Outre, de manière rassurante, les coefficients associés aux variables de contrôle restent très cohérents avec les résultats des modèles précédents (colonnes 2 à 4).}. Ce coefficient est pratiquement égal à celui estimé par simple variable instrumentale (colonne 3). Toutefois, il est possible que ce soit juste une coïncidence.

\quad Les résultats d'estimation par fonction de contrôle fournissent également un test direct d'endogénéité de l'âge aux examens (voir \protect\hyperlink{ref-WOO:15}{Wooldridge, 2015}, par exemple). Il correspond à un test de significativité jointe des coefficients associés à \(\hat{\nu}\) et à son interaction avec l'âge aux examens. La statistique de test de Fisher est égale à 1948.04 (probabilité critique pratiquement nulle). Les paramètres estimés par les coefficients de - 0.53 et - 0.06 associés respectivement à \(\hat{\nu}\) et son interaction avec l'âge aux examens sont significativement différents de 0.

\quad Nous montrons dans l'Annexe \ref{ageheterofac} que les signes négatifs des coefficients associés à \(\hat{\nu}\) et à son interaction avec l'âge aux examens indiquent que l'effet de l'âge aux examens à La Réunion augmente avec les facultés.

\quad Jusque-là, la variable sur laquelle nous avons analysé les effets d'âge est la note totale aux évaluations de CM2. Rappelons qu'il s'agit une note composite constituée à 60\% de la note en français et à 40\% de la note en mathématiques. Nous analysons les effets de l'âge séparément sur les résultats en français et en mathématiques.

\quad À cette fin, le Tableau \ref{tab:agemodelsfm} propose les résultats des mêmes exercices que dans les colonnes (4) (doubles moindres carrés) et (5) (fonction de contrôle) du Tableau \ref{tab:agemodels} mais en prenant comme variables dépendantes les notes en français (colonnes 1 et 2) et en mathématiques (colonnes 3 et 4).\\
Nous constatons que l'effet n'est pas modifié de manière significative : il est toujours à environ 0.3 et des \(z\)-tests sur les coefficients associés à l'âge, deux à deux (colonne 1 avec colonne 3 et colonne 2 avec colonne 4) ne détectent pas de différence entre ces coefficients. Il en est de même pour les coefficients associés à \(\hat{\nu}\) et son interaction avec l'âge aux examens.\footnote{De la même manière que dans les résultats principaux du Tableau \ref{tab:agemodels}, nous observons de manière robuste une proximité entre les coefficients estimés par doubles moindres carrés et par fonction de contrôle.}\\
Ainsi, l'effet de l'âge ne semble pas être différent sur les résultats en français et en mathématiques.

\begingroup\fontsize{8}{10}\selectfont

\begin{ThreePartTable}
\begin{TableNotes}
\item \textit{Sources :} Fichiers CM2 (2009 à 2012), calculs de l'auteur.
\item \textit{Notes :} Une colonne correspond à une régression. Les notes sont normalisées sur l'année scolaire. Écart-types entre parenthèses. Les écart-types des coefficients estimés par fonction de contrôle sont calculées par wild bootstrap avec 1001 réplications. La variable $\hat{\nu}$ est le résidu de la première étape. Les contrôles utilisés sont le sexe, la CSP et l'année scolaire.
\item VI : Variable Instrumentale. CSP : Catégorie Socio-Professionnelle.
\item Significativité : 10\% * 5\% ** 1\% ***.
\end{TableNotes}
\begin{longtable}[t]{lllll}
\caption{\label{tab:agemodelsfm}Effets de l'âge aux examens sur les notes en français et en mathématiques}\\
\toprule
\multicolumn{1}{c}{} & \multicolumn{4}{c}{Variable dépendante : } \\
\cmidrule(l{3pt}r{3pt}){2-5}
\multicolumn{1}{c}{} & \multicolumn{2}{c}{Note en français} & \multicolumn{2}{c}{Note en mathématiques} \\
\cmidrule(l{3pt}r{3pt}){2-3} \cmidrule(l{3pt}r{3pt}){4-5}
 & \makecell{VI \\ (1) } & \makecell{FCH \\ (2) } & \makecell{VI \\ (3) } & \makecell{FCH \\ (4) }\\
\midrule
\endfirsthead
\caption[]{\label{tab:agemodelsfm}Effets de l'âge aux examens sur les notes en français et en mathématiques (suite)}\\
\toprule
\multicolumn{1}{c}{} & \multicolumn{4}{c}{Variable dépendante : } \\
\cmidrule(l{3pt}r{3pt}){2-5}
\multicolumn{1}{c}{} & \multicolumn{2}{c}{Note en français} & \multicolumn{2}{c}{Note en mathématiques} \\
\cmidrule(l{3pt}r{3pt}){2-3} \cmidrule(l{3pt}r{3pt}){4-5}
 & \makecell{VI \\ (1) } & \makecell{FCH \\ (2) } & \makecell{VI \\ (3) } & \makecell{FCH \\ (4) }\\
\midrule
\endhead

\endfoot
\bottomrule
\insertTableNotes
\endlastfoot
Âge aux examens & 0.302$^{***}$ & 0.299$^{***}$ & 0.327$^{***}$ & 0.326$^{***}$\\
 & (0.018) & (0.016) & (0.019) & (0.017)\\
$\hat{\nu}$ & - & $-$0.421$^{***}$ & - & $-$0.628$^{***}$\\
 & - & (0.148) & - & (0.142)\\
Âge aux examens $\times$ $\hat{\nu}$ & - & $-$0.065$^{***}$ & - & $-$0.037$^{***}$\\
 & - & (0.013) & - & (0.013)\\
 &  &  &  & \\
Contrôles & Oui & Oui & Oui & Oui\\
Observations & 54341 & 54341 & 54341 & 54341\\
R$^2$ ajusté & 0.012 & 0.24 & - & 0.163\\*
\end{longtable}
\end{ThreePartTable}
\endgroup{}

\quad Dans la même logique, la disponibilité des sous-composantes de français et mathématiques (voir Section \ref{agedata}) nous permettent de pousser l'analyse à des niveaux plus fins. Pour le français, nous pouvons analyser les effets de l'âge aux examens sur la note en écriture, en grammaire, en lecture, en orthographe et en grammaire. Pour les mathématiques, nous pouvons mesurer les effets de l'âge aux examens sur la note en calcul, en géométrie, en grandeurs-et-mesures, en nombre et en organisation-et-gestion-des-données.
Les Tableaux \ref{tab:agemodelsssitemsfrench} et \ref{tab:agemodelsssitemsmaths} en Annexe \ref{agemodelsssitems} qui montrent les résultats des estimations sur les sous-composantes de français et de mathématiques, respectivement, nous enseignent que les effets de l'âge aux examens sur ces sous-composantes ne sont pas non plus fondamentalement différents.

\quad Ainsi, l'effet de l'âge est positif et, en moyenne, semble être le même indépendamment de la nature de la note et, par extension, des capacités cognitives et non cognitives qui en seraient la source. Cela rejoint finalement l'intuition : il n'y a \emph{a priori} pas de raison pour que le développement physiologique des enfants à travers l'âge favorise en moyenne des capacités bien précises. De telles différences détectées dans les études précédentes (\protect\hyperlink{ref-DAT:06}{Datar, 2006}, par exemple) sont probablement causées par la difficulté relative entre les matières, bien qu'il soit difficile d'avoir une mesure de difficulté d'une matière par rapport à une autre. C'est probablement pour cette raison que les articles précédents ne donnent pas d'interprétation entre la différence entre les effets selon la nature de la note en tant que variable dépendante (\protect\hyperlink{ref-SMI:09}{Smith, 2009} ; \protect\hyperlink{ref-GRE:09}{Grenet, 2009} ; \protect\hyperlink{ref-ATT:COH:18}{Attar \& Cohen-Zada, 2018}, par exemple).

\hypertarget{agemodelsheterores}{%
\subsection{De l'hétérogénéité selon les observables}\label{agemodelsheterores}}

À La Réunion, 17\% des élèves de CM2 sont en retard. Cette proportion est beaucoup plus grande chez les garçons (20\%) que chez les filles (13\%). Ce constat semble être un fait général à travers le monde et une raison possible est que les garçons mûrissent moins rapidement que les filles, ce qui a pour conséquence que leurs parents décident plus souvent de décaler l'inscription de leur fils (\protect\hyperlink{ref-ATT:COH:18}{Attar \& Cohen-Zada, 2018}, par exemple).\\
De manière similaire, les proportions de positions sont très différentes d'une catégorie sociale à une autre. Par exemple, la part d'élèves en retard chez les défavorisés est de 19\% tandis qu'elle n'est que de 2\% chez les très favorisés. Ici, ce constat pourrait être expliqué par le fait que les parents très favorisés fournissent un cadre permettant à leur enfant de maturer beaucoup plus rapidement que son homologue enfant de parents défavorisés.

\quad Ces deux groupes de constats et d'interprétations nous suggèrent que les effets de l'âge sont supérieurs respectivement chez les filles et au fur et à mesure que les parents appartiennent à une catégorie sociale supérieure. Si cela est le cas, cela peut débuter la réflexion sur des mesures particulières ciblées vers ces sous-populations pour atténuer l'aggravation des désavantages causés par les différentielles d'âge aux examens, toutes choses égales par ailleurs, via le sexe et la catégorie sociale.

\quad Notre modèle nous permet de déceler d'éventuels effets hétérogènes selon ces observables. Pour ce faire, pour chacun des variables sexe et catégorie sociale, nous effectuons une régression dans laquelle nous rajoutons respectivement à l'équation \eqref{eq:agecfh} les régresseurs suivants\footnote{Wooldridge (\protect\hyperlink{ref-WOO:15}{2015}), équation (29).} :\\
\[a_i \ Garçon_i , \ \hat{\nu}_i \ Garçon_i \ \text{et} \  a_i \ \hat{\nu}_i \ Garçon_i\] ~;
et
\[a_i \ \begin{pmatrix} Moyenne_i \\ Favorisée_i \\ TrèsFavorisée_i \\ Autre_i \end{pmatrix}', \  \hat{\nu}_i \ \begin{pmatrix} Moyenne_i \\ Favorisée_i \\ TrèsFavorisée_i \\ Autre_i \end{pmatrix}' \ \text{et} \ a_i \ \hat{\nu}_i \begin{pmatrix} Moyenne_i \\ Favorisée_i \\ TrèsFavorisée_i \\ Autre_i \end{pmatrix}'.\]

\(Garçon_i\) est une indicatrice qui vaut 1 si l'élève est un garçon. La même logique s'applique pour les régresseurs concernant la catégorie sociale.\\
Pour les modèles estimés correspondants, nous n'interprétons pas les coefficients associés à \(\hat{\nu_i}\) et à son interaction avec l'âge aux examens comme nous l'avons fait dans la Section \ref{agemodelsres}, par souci de simplicité.

\quad Les résultats principaux avec effets hétérogènes selon le sexe sont présentés dans le Tableau \ref{tab:agemodelssexe}\footnote{Pour les colonnes des doubles moindres carrés, les régresseurs incluant les interactions de l'âge avec d'autres variables exogènes sont instrumentés par l'interaction de l'âge relatif théorique avec ces mêmes variables.}. Les effets de l'âge estimés par doubles moindres carrés et par fonction de contrôle tournent respectivement autour de 0.3 et 0.36. De manière intéressante, les doubles moindres carrés ne permettent pas de détecter l'effet hétérogène (coefficients non significatifs autour de 0.1) tandis que l'estimation par fonction de contrôle semble montrer de manière robuste (à travers les trois variables dépendantes) que l'effet est plus faible chez les garçons. Cela est illustré par la ligne \emph{Âge aux examens} \(\times\) \emph{Sexe - Garçon}.
Les résultats sur les notes en sous-composantes de français et de mathématiques sont reportés dans l'Annexe \ref{agemodelssexessitems} et aboutissent à des conclusions similaires.

\begingroup\fontsize{8}{10}\selectfont

\begin{ThreePartTable}
\begin{TableNotes}
\item \textit{Sources :} Fichiers CM2 (2009 à 2012), calculs de l'auteur.
\item \textit{Notes :} Une colonne correspond à une régression. Les notes sont normalisées sur l'année scolaire. Écart-types entre parenthèses. Les écart-types des coefficients estimés par fonction de contrôle sont calculées par wild bootstrap avec 1001 réplications. La variable $\hat{\nu}$ est le résidu de la première étape. Les contrôles utilisés sont le sexe, la CSP et l'année scolaire.
\item VI : Variable Instrumentale. FCH : Fonction de Contrôle prenant en compte l'Hétérogénéité de l'effet de l'âge. CSP : Catégorie Socio-Professionnelle.
\item Significativité : 10\% * 5\% ** 1\% ***.
\end{TableNotes}
\begin{longtable}[t]{lllllll}
\caption{\label{tab:agemodelssexe}Estimations des effets hétérogènes de l'âge aux examens selon le sexe}\\
\toprule
\multicolumn{1}{c}{} & \multicolumn{6}{c}{Variable dépendante : } \\
\cmidrule(l{3pt}r{3pt}){2-7}
\multicolumn{1}{c}{} & \multicolumn{2}{c}{Note totale} & \multicolumn{2}{c}{Note en français} & \multicolumn{2}{c}{Note en mathématiques} \\
\cmidrule(l{3pt}r{3pt}){2-3} \cmidrule(l{3pt}r{3pt}){4-5} \cmidrule(l{3pt}r{3pt}){6-7}
 & \makecell{VI \\ (1) } & \makecell{FCH \\ (2) } & \makecell{VI \\ (3) } & \makecell{FCH \\ (4) } & \makecell{VI \\ (5) } & \makecell{FCH \\ (6) }\\
\midrule
\endfirsthead
\caption[]{\label{tab:agemodelssexe}Estimations des effets hétérogènes de l'âge aux examens selon le sexe (suite)}\\
\toprule
\multicolumn{1}{c}{} & \multicolumn{6}{c}{Variable dépendante : } \\
\cmidrule(l{3pt}r{3pt}){2-7}
\multicolumn{1}{c}{} & \multicolumn{2}{c}{Note totale} & \multicolumn{2}{c}{Note en français} & \multicolumn{2}{c}{Note en mathématiques} \\
\cmidrule(l{3pt}r{3pt}){2-3} \cmidrule(l{3pt}r{3pt}){4-5} \cmidrule(l{3pt}r{3pt}){6-7}
 & \makecell{VI \\ (1) } & \makecell{FCH \\ (2) } & \makecell{VI \\ (3) } & \makecell{FCH \\ (4) } & \makecell{VI \\ (5) } & \makecell{FCH \\ (6) }\\
\midrule
\endhead

\endfoot
\bottomrule
\insertTableNotes
\endlastfoot
Âge aux examens & 0.323$^{***}$ & 0.358$^{***}$ & 0.3$^{***}$ & 0.324$^{***}$ & 0.319$^{***}$ & 0.363$^{***}$\\
 & (0.025) & (0.021) & (0.025) & (0.021) & (0.026) & (0.023)\\
Âge aux examens $\times$ Sexe - Homme & 0.01 & $-$0.065$^{**}$ & 0.006 & $-$0.05$^{*}$ & 0.016 & $-$0.076$^{***}$\\
 & (0.037) & (0.027) & (0.037) & (0.026) & (0.037) & (0.029)\\
 &  &  &  &  &  & \\
Contrôles & Oui & Oui & Oui & Oui & Oui & Oui\\
Observations & 54341 & 54341 & 54341 & 54341 & 54341 & 54341\\
R$^2$ ajusté & - & 0.223 & - & 0.24 & - & 0.163\\*
\end{longtable}
\end{ThreePartTable}
\endgroup{}

\quad Quant aux estimations des effets de l'âge aux examens différenciés selon la catégorie sociale, le Tableau \ref{tab:agemodelspcsg2} montre que pour l'individu de référence (\emph{CSP - Défavorisée}), un an de plus aux évaluations procure environ 0.3 unité d'écart-type d'avantage sur la note.\\
Sauf en mathématiques, les deux modèles (VI et FCH) nous apprennent que l'effet de l'âge est plus fort pour un élève de catégorie très favorisée.

\quad Les effets hétérogènes de l'âge aux examens selon la catégorie sociale sur les notes en sous-composantes de français et de mathématiques sont analysés et sont illustrés dans l'Annexe \ref{agemodelspcsg2ssitems}. Les estimations par fonction de contrôle nous montrent que l'effet de l'âge est plus fort chez les très favorisés sur la note en écriture, en lecture et en vocabulaire (Tableau \ref{tab:agemodelspcsg2ssitemsfrench}). Ce n'est pas le cas en grammaire et en orthographe. Une raison plutôt intuitive pourrait expliquer ce constat : l'écriture, la lecture et le vocabulaire sont des domaines plutôt vastes tandis que la grammaire et l'orthographe sont plus techniques. Si nous supposons que l'effet de l'âge est plus fort chez les très favorisés parce que ces derniers sont significativement plus en contact avec de l'information littéraire, la nature de cette information relève plus probablement du général que de la technique\footnote{Ils ont par exemple plus de livres et journaux chez eux, ils ont en moyenne plus accès à internet, etc..}. Plus intuitivement, les défavorisés et les très favorisés seraient en moyenne exposés à la même quantité de livres de grammaire ou d'orthographe purs mais les très favorisés auraient plus de matériels d'écriture, lisent des livres plus généraux et ont donc une gamme de vocabulaire plus large (\protect\hyperlink{ref-CHA:eal:21}{De Chaisemartin et al., 2021}, par exemple).
En ce qui concerne les effets différenciés sur les notes en sous-composantes de mathématiques (Tableau \ref{tab:agemodelspcsg2ssitemsmaths}), nous n'observons d'effet plus fort chez les très favorisés (par fonction de contrôle) que sur la note de grandeurs-et-mesures. Cela est cohérent avec l'absence d'effet en plus chez les très favorisés en mathématiques, démontrée précédemment par le Tableau \ref{tab:agemodelspcsg2}. Ce résultat est également cohérent avec notre proposition d'explication sur les sous-items de français car les mathématiques sont plutôt techniques.

\begingroup\fontsize{8}{10}\selectfont

\begin{ThreePartTable}
\begin{TableNotes}
\item \textit{Sources :} Fichiers CM2 (2009 à 2012), calculs de l'auteur.
\item \textit{Notes :} Une colonne correspond à une régression. Les notes sont normalisées sur l'année scolaire. Écart-types entre parenthèses. Les écart-types des coefficients estimés par fonction de contrôle sont calculées par wild bootstrap avec 1001 réplications. La variable $\hat{\nu}$ est le résidu de la première étape. Les contrôles utilisés sont le sexe, la CSP et l'année scolaire.
\item VI : Variable Instrumentale. FCH : Fonction de Contrôle prenant en compte l'Hétérogénéité de l'effet de l'âge. CSP : Catégorie Socio-Professionnelle.
\item Significativité : 10\% * 5\% ** 1\% ***.
\end{TableNotes}
\begin{longtable}[t]{lllllll}
\caption{\label{tab:agemodelspcsg2}Estimations des effets hétérogènes de l'âge aux examens selon la catégorie sociale}\\
\toprule
\multicolumn{1}{c}{} & \multicolumn{6}{c}{Variable dépendante : } \\
\cmidrule(l{3pt}r{3pt}){2-7}
\multicolumn{1}{c}{} & \multicolumn{2}{c}{Note totale} & \multicolumn{2}{c}{Note en français} & \multicolumn{2}{c}{Note en mathématiques} \\
\cmidrule(l{3pt}r{3pt}){2-3} \cmidrule(l{3pt}r{3pt}){4-5} \cmidrule(l{3pt}r{3pt}){6-7}
 & \makecell{VI \\ (1) } & \makecell{FCH \\ (2) } & \makecell{VI \\ (3) } & \makecell{FCH \\ (4) } & \makecell{VI \\ (5) } & \makecell{FCH \\ (6) }\\
\midrule
\endfirsthead
\caption[]{\label{tab:agemodelspcsg2}Estimations des effets hétérogènes de l'âge aux examens selon la catégorie sociale (suite)}\\
\toprule
\multicolumn{1}{c}{} & \multicolumn{6}{c}{Variable dépendante : } \\
\cmidrule(l{3pt}r{3pt}){2-7}
\multicolumn{1}{c}{} & \multicolumn{2}{c}{Note totale} & \multicolumn{2}{c}{Note en français} & \multicolumn{2}{c}{Note en mathématiques} \\
\cmidrule(l{3pt}r{3pt}){2-3} \cmidrule(l{3pt}r{3pt}){4-5} \cmidrule(l{3pt}r{3pt}){6-7}
 & \makecell{VI \\ (1) } & \makecell{FCH \\ (2) } & \makecell{VI \\ (3) } & \makecell{FCH \\ (4) } & \makecell{VI \\ (5) } & \makecell{FCH \\ (6) }\\
\midrule
\endhead

\endfoot
\bottomrule
\insertTableNotes
\endlastfoot
Âge aux examens & 0.269$^{***}$ & 0.288$^{***}$ & 0.245$^{***}$ & 0.255$^{***}$ & 0.274$^{***}$ & 0.304$^{***}$\\
 & (0.029) & (0.024) & (0.029) & (0.023) & (0.03) & (0.025)\\
Âge aux examens $\times$ CSP - Moyenne & 0.093$^{*}$ & 0.076$^{**}$ & 0.092$^{*}$ & 0.087$^{**}$ & 0.083 & 0.051\\
 & (0.051) & (0.039) & (0.05) & (0.037) & (0.052) & (0.042)\\
Âge aux examens $\times$ CSP - Favorisée & $-$0.072 & 0.032 & $-$0.075 & 0.048 & $-$0.059 & 0.005\\
 & (0.076) & (0.059) & (0.074) & (0.057) & (0.08) & (0.065)\\
Âge aux examens $\times$ CSP - Très favorisée & 0.144$^{**}$ & 0.097$^{**}$ & 0.132$^{**}$ & 0.121$^{***}$ & 0.145$^{**}$ & 0.05\\
 & (0.067) & (0.047) & (0.065) & (0.046) & (0.072) & (0.051)\\
Âge aux examens $\times$ CSP - Autre & 0.117 & 0.27 & 0.218 & 0.328 & $-$0.047 & 0.154\\
 & (0.225) & (0.226) & (0.231) & (0.231) & (0.214) & (0.215)\\
Âge aux examens $\times$ CSP - Manquante & 0.106$^{**}$ & - & 0.105$^{**}$ & - & 0.094$^{**}$ & -\\
 & (0.046) & - & (0.046) & - & (0.046) & -\\
 &  &  &  &  &  & \\
Contrôles & Oui & Oui & Oui & Oui & Oui & Oui\\
Observations & 54341 & 54341 & 54341 & 54341 & 54341 & 54341\\
R$^2$ ajusté & - & 0.226 & - & 0.243 & - & 0.166\\*
\end{longtable}
\end{ThreePartTable}
\endgroup{}

\quad En somme, les effets de l'âge apparaissent plus forts chez les filles et chez les enfants très favorisés. Même si nous pouvons associer le premier résultat avec une interprétation cohérente (les filles mûrissent plus vite que les garçons), un effet plus fort chez les filles n'est pas systématiquement retrouvé dans la littérature\footnote{Par exemple, pour la France, à bas âge, Grenet (\protect\hyperlink{ref-GRE:09}{2009}) ne documente pas d'effet significativement différent selon le sexe. Il en est de même dans Puhani \& Weber (\protect\hyperlink{ref-PUH:WEB:05}{2005}) pour l'Allemagne ou encore Datar (\protect\hyperlink{ref-DAT:06}{2006}) pour les États-Unis. Dans certains articles, l'effet est retrouvé comme étant plus fort chez les garçons. C'est le cas par exemple au Brésil avec Matta et al. (\protect\hyperlink{ref-MAT:eal:16}{2016}) pour le Brésil, McEwan \& Shapiro (\protect\hyperlink{ref-MCE:SHA:08}{2008}) en Chili ou encore Cascio \& Schanzenbach (\protect\hyperlink{ref-CAS:SCH:16}{2016}) aux États-Unis.}.\\
L'effet plus prononcé chez les très favorisé trouvé pour La Réunion, quant à lui, va à l'encontre de ce qui ressort de la littérature : les effets de l'âge sous ses diverses formes (âge d'entrée, âge aux examens, par exemple), apparaissent en général plus fort chez les individus plutôt ``défavorisés''. Cette classification de ``défavorisée'' peut être définie de manières différentes en fonction des données disponibles. L'éducation des parents (\protect\hyperlink{ref-MAT:eal:16}{Matta et al., 2016} ; \protect\hyperlink{ref-ATT:COH:18}{Attar \& Cohen-Zada, 2018}, par exemple) ou la tranche de revenu de la famille (\protect\hyperlink{ref-DAT:06}{Datar, 2006} ; \protect\hyperlink{ref-SMI:09}{Smith, 2009}) en sont deux exemples.

\quad Pourquoi trouvons-nous des effets plus forts chez les individus issus de parents très favorisés ? Une explication possible est l'efficacité de la remédiation ciblée en faveur des élèves défavorisés. En effet, une des raisons pour laquelle un effet plus fort est détecté chez les ``défavorisés'' serait que les familles correspondantes ont plus de mal à accompagner leur enfant dans son développement à travers le temps. Cela a pour conséquence de creuser l'écart de performances dû à un même écart d'âge chez les défavorisés par rapport aux familles favorisées (\protect\hyperlink{ref-HAM:KOL:12}{Hámori \& Köllő, 2012}). Or, la France est caractérisée par l'existence des écoles en éducation prioritaire qui se voient allouer des moyens supplémentaires (\protect\hyperlink{ref-ALE:eal:13}{Alet et al., 2013}) car elles concentrent une part considérable d'élèves défavorisés. Empiriquement, à La Réunion, les écoles en éducation prioritaire comptent 64\% d'élèves défavorisés alors que cette proportion n'est que de 48\% dans les écoles hors éducation prioritaire. Si ce type de mesure réduit les écarts de performance dus aux écarts d'âge aux examens, nous nous attendons à trouver un effet plus faible chez les élèves défavorisés dans les écoles en éducation prioritaire comparés aux défavorisés dans les écoles hors éducation prioritaire.

\quad Nous proposons alors le Tableau \ref{tab:agemodelspcsg2reseaubinsep} qui montre les résultats des estimations par fonction de contrôle, séparément en fonction du statut d'éducation prioritaire. Les colonnes (1), (3) et (5) correspondent aux écoles hors éducation prioritaire tandis que les colonnes (2), (4) et (6) correspondent aux écoles en éducation prioritaire.\\
Ce tableau ne fait pas ressortir de manière robuste le résultat attendu vu que sur la note totale, l'effet chez les défavorisés n'est pas significativement différent selon le statut d'éducation prioritaire\footnote{En comparant deux à deux dans l'ordre les colonnes de la première ligne du tableau.}.
Le même constat est fait lorsque les notes en sous-composantes sont utilisées comme variables dépendantes (Annexe \ref{agemodelssexereseaubinsep}).
La remédiation ne semble alors pas expliquer l'effet plus fort observé chez les très favorisés à La Réunion.

\quad Un autre phénomène peut alternativement expliquer la supériorité des effets de l'âge chez les très favorisés. Il est probable que les élèves en retard chez les très favorisés sont en majorité, cette fois, d'élèves qui sont entrés en retard pour des raisons stratégiques (être plus mature à l'entrée et ainsi obtenir de meilleurs résultats). La stratégie aurait alors été très efficace pour les enfants des parents très favorisés (\protect\hyperlink{ref-CAS:SCH:16}{Cascio \& Schanzenbach, 2016}). En effet, un enfant issu de famille aisée qui attend un an de plus avant d'entrer à l'école cumulerait plus de capital humain grâce à son environnement familial et les moyens mis à sa disposition par rapport à son homologue issu de famille défavorisée. Les familles très favorisées seraient également indifférentes aux coûts d'attente d'une année de plus avant l'inscription à l'école (crèches ou services domestiques, par exemple), ce qui n'est pas le cas des familles défavorisées. Vu sous un autre angle, les effets supérieurs mesurés chez les enfants très favorisés seraient dus à leurs âges d'entrée. Nos données ne disposent pas des variations nécessaires pour tester une telle hypothèse.\\
Cette explication rejoint celle de Elder \& Lubotsky (\protect\hyperlink{ref-ELD:LUB:09}{2009}). Selon ces auteurs, si on trouve de tels effets, c'est que c'est l'investissement avant le primaire qui importe par rapport à la maturité au moment des examens dans les premières années du primaire.

\quad Remarquons enfin qu'il existe une cohérence entre l'effet croissant avec les facultés (Section \ref{agemodelsres}) et l'effet plus fort chez les filles et les très favorisés.

\begingroup\fontsize{8}{10}\selectfont

\begin{ThreePartTable}
\begin{TableNotes}
\item \textit{Sources :} Fichiers CM2 (2009 à 2012), calculs de l'auteur.
\item \textit{Notes :} Une colonne correspond à une régression. Les notes sont normalisées sur l'année scolaire. Écart-types entre parenthèses. Les écart-types des coefficients estimés par fonction de contrôle sont calculées par wild bootstrap avec 1001 réplications. La variable $\hat{\nu}$ est le résidu de la première étape. Les contrôles utilisés sont le sexe, la CSP et l'année scolaire.
\item FCH : Fonction de Contrôle prenant en compte l'Hétérogénéité de l'effet de l'âge. EP : Éducation Prioritaire. CSP : Catégorie Socio-Professionnelle.
\item Significativité : 10\% * 5\% ** 1\% ***.
\end{TableNotes}
\begin{longtable}[t]{lllllll}
\caption{\label{tab:agemodelspcsg2reseaubinsep}Estimations des effets hétérogènes de l'âge selon la catégorie sociale, par statut d'éducation prioritaire}\\
\toprule
\multicolumn{1}{c}{} & \multicolumn{6}{c}{Variable dépendante : } \\
\cmidrule(l{3pt}r{3pt}){2-7}
\multicolumn{1}{c}{} & \multicolumn{2}{c}{Note totale} & \multicolumn{2}{c}{Note en français} & \multicolumn{2}{c}{Note en mathématiques} \\
\cmidrule(l{3pt}r{3pt}){2-3} \cmidrule(l{3pt}r{3pt}){4-5} \cmidrule(l{3pt}r{3pt}){6-7}
 & \makecell{FCH, Hors EP \\ (1) } & \makecell{FCH, EP \\ (2) } & \makecell{FCH, Hors EP \\ (3) } & \makecell{FCH, EP \\ (4) } & \makecell{FCH, Hors EP \\ (5) } & \makecell{FCH, EP \\ (6) }\\
\midrule
\endfirsthead
\caption[]{\label{tab:agemodelspcsg2reseaubinsep}Estimations des effets hétérogènes de l'âge selon la catégorie sociale, par statut d'éducation prioritaire (suite)}\\
\toprule
\multicolumn{1}{c}{} & \multicolumn{6}{c}{Variable dépendante : } \\
\cmidrule(l{3pt}r{3pt}){2-7}
\multicolumn{1}{c}{} & \multicolumn{2}{c}{Note totale} & \multicolumn{2}{c}{Note en français} & \multicolumn{2}{c}{Note en mathématiques} \\
\cmidrule(l{3pt}r{3pt}){2-3} \cmidrule(l{3pt}r{3pt}){4-5} \cmidrule(l{3pt}r{3pt}){6-7}
 & \makecell{FCH, Hors EP \\ (1) } & \makecell{FCH, EP \\ (2) } & \makecell{FCH, Hors EP \\ (3) } & \makecell{FCH, EP \\ (4) } & \makecell{FCH, Hors EP \\ (5) } & \makecell{FCH, EP \\ (6) }\\
\midrule
\endhead

\endfoot
\bottomrule
\insertTableNotes
\endlastfoot
Âge aux examens & 0.301$^{***}$ & 0.29$^{***}$ & 0.252$^{***}$ & 0.269$^{***}$ & 0.339$^{***}$ & 0.289$^{***}$\\
 & (0.036) & (0.033) & (0.035) & (0.033) & (0.037) & (0.034)\\
Âge aux examens $\times$ CSP - Moyenne & 0.12$^{**}$ & 0.015 & 0.143$^{***}$ & 0.014 & 0.07 & 0.017\\
 & (0.054) & (0.055) & (0.054) & (0.053) & (0.056) & (0.06)\\
Âge aux examens $\times$ CSP - Favorisée & $-$0.019 & 0.138 & 0.018 & 0.124 & $-$0.07 & 0.143\\
 & (0.075) & (0.109) & (0.073) & (0.104) & (0.081) & (0.118)\\
Âge aux examens $\times$ CSP - Très favorisée & 0.097 & 0.103 & 0.146$^{**}$ & 0.079 & 0.013 & 0.128\\
 & (0.062) & (0.096) & (0.061) & (0.093) & (0.067) & (0.102)\\
Âge aux examens $\times$ CSP - Autre & 0.21 & 0.323 & 0.195 & 0.459 & 0.206 & 0.087\\
 & (0.29) & (0.345) & (0.288) & (0.35) & (0.294) & (0.326)\\
Âge aux examens $\times$ CSP - Manquante & 0.071 & 0.017 & 0.098$^{**}$ & 0.015 & 0.021 & 0.013\\
 & (0.05) & (0.048) & (0.049) & (0.048) & (0.052) & (0.05)\\
 &  &  &  &  &  & \\
Contrôles & Oui & Oui & Oui & Oui & Oui & Oui\\
Observations & 28506 & 25835 & 28506 & 25835 & 28506 & 25835\\
R$^2$ ajusté & 0.219 & 0.218 & 0.236 & 0.237 & 0.162 & 0.156\\*
\end{longtable}
\end{ThreePartTable}
\endgroup{}

\hypertarget{agefrdcfhmodelsres}{%
\subsection{Résultats des estimations par régressions sur une discontinuité}\label{agefrdcfhmodelsres}}

Les résultats principaux liés aux estimations par régression sur une discontinuité sont donnés par le Tableau \ref{tab:agefrdcfhmodels}.

\quad La première étape (colonnes 1 et 2) montre que nous n'avons pas un problème d'instrument faible : être né en 2000 plutôt qu'en 1999 augmente en moyenne l'âge aux examens de 0.74 année.

\quad Les résultats du \emph{fuzzy regression discontinuity} (estimé par fonction de contrôle, voir Section \ref{agemethodsrd}) en utilisant une fenêtre de 30 jours nous montrent que l'effet d'avoir un an de plus aux examens varie entre environ 0.5 et environ 0.7 unité d'écart-type. Les ampleurs sont nettement supérieures à celles trouvées dans les estimations sur données en coupe précédentes. Cela s'explique par la fenêtre de 30 jours utilisée en régression sur une discontinuité puisque les individus concernés ont des différentiels d'âge largement supérieurs à ceux des individus dans les estimations en coupe. Ces ampleurs rejoignent celles trouvées par Matta et al. (\protect\hyperlink{ref-MAT:eal:16}{2016}) (Brésil) et Crawford et al. (\protect\hyperlink{ref-CRA:eal:14}{2014}) (Angleterre). Smith (\protect\hyperlink{ref-SMI:09}{2009, p. 23}) trouvent également que les coefficients estimés par régressions sur une discontinuités sont supérieurs à ceux estimés sur les données en coupe\footnote{Mais pour des élèves en 10\textsuperscript{ème} année d'éducation du premier cycle.}. Les résultats ne diffèrent pas significativement selon le degré de polynôme considéré lorsque nous regardons par les coefficients associés à \(dist_i\) et \(dist_i^ 2\) qui sont non significatifs et de très faible ampleur. Ce constat peut également être interprété par la vraisemblable absence d'effets directs de la date de naissance absolue (en considérant l'année). Cela rejoint l'absence d'effet direct de la date de naissance dans l'année discutée dans la Section précédente \ref{agemodelsres} (colonne 4 du Tableau \ref{tab:agemodels}).

\quad Les effets de l'âge sur la note en mathématiques sont légèrement supérieurs à ceux sur la note en français mais pas de manière significative lorsque des \emph{z-tests} sont effectués.
Excepté en écriture où ils sont relativement moins prononcés, les effets mesurés ne diffèrent pas non plus à travers les sous-composantes de français entre elles comme le montre le Tableau \ref{tab:agefrdcfhmodelsssitemsfrench} de l'Annexe \ref{agefrdcfhmodelsssitems}. Le même constat peut être fait pour les effets sur les sous-composantes de mathématiques entre elles (Tableau \ref{tab:agefrdcfhmodelsssitemsmaths} de l'Annexe \ref{agefrdcfhmodelsssitems}). L'ampleur des coefficients dans le cas des sous-composantes de mathématiques apparaissent supérieure par rapport aux sous-composantes de français.

\newpage
\begingroup\fontsize{8}{10}\selectfont

\begin{ThreePartTable}
\begin{TableNotes}
\item \textit{Sources :} Fichiers CM2 (2009 à 2012) - nés en 1999 et 2000, calculs de l'auteur.
\item \textit{Notes :} Une colonne correspond à une régression. On note $p$ le degré de polynôme de $dist$ et $old \times dist$ utilisé. Les notes sont normalisées sur l'année scolaire. Écart-types entre parenthèses. Les contrôles utilisés sont le sexe, la CSP et l'année scolaire.
\item FRD : \textit{Fuzzy Regression Discontinuity.} CSP : Catégorie Socio-Professionnelle.
\item Significativité : 10\% * 5\% ** 1\% ***.
\end{TableNotes}
\begin{longtable}[t]{lllllllll}
\caption{\label{tab:agefrdcfhmodels}Estimations des effets de l'âge aux examens par régressions sur une discontinuité}\\
\toprule
\multicolumn{1}{c}{} & \multicolumn{8}{c}{Variable dépendante : } \\
\cmidrule(l{3pt}r{3pt}){2-9}
\multicolumn{1}{c}{} & \multicolumn{2}{c}{Âge aux examens} & \multicolumn{2}{c}{Note totale} & \multicolumn{2}{c}{Note en français} & \multicolumn{2}{c}{Note en mathématiques} \\
\cmidrule(l{3pt}r{3pt}){2-3} \cmidrule(l{3pt}r{3pt}){4-5} \cmidrule(l{3pt}r{3pt}){6-7} \cmidrule(l{3pt}r{3pt}){8-9}
 & \makecell{\makecell{Première étape \\ p = 1} \\ (1) } & \makecell{\makecell{Première étape \\ p = 2} \\ (2) } & \makecell{\makecell{FRD \\ p = 1} \\ (3) } & \makecell{\makecell{FRD \\ p = 2} \\ (4) } & \makecell{\makecell{FRD \\ p = 1} \\ (5) } & \makecell{\makecell{FRD \\ p = 2} \\ (6) } & \makecell{\makecell{FRD\\ p = 1} \\ (7) } & \makecell{\makecell{FRD\\ p = 2} \\ (8) }\\
\midrule
\endfirsthead
\caption[]{\label{tab:agefrdcfhmodels}Estimations des effets de l'âge aux examens par régressions sur une discontinuité (suite)}\\
\toprule
 & \makecell{\makecell{Première étape \\ p = 1} \\ (1) } & \makecell{\makecell{Première étape \\ p = 2} \\ (2) } & \makecell{\makecell{FRD \\ p = 1} \\ (3) } & \makecell{\makecell{FRD \\ p = 2} \\ (4) } & \makecell{\makecell{FRD \\ p = 1} \\ (5) } & \makecell{\makecell{FRD \\ p = 2} \\ (6) } & \makecell{\makecell{FRD\\ p = 1} \\ (7) } & \makecell{\makecell{FRD\\ p = 2} \\ (8) }\\
\midrule
\endhead

\endfoot
\bottomrule
\insertTableNotes
\endlastfoot
Âge aux examens & - & - & 0.647$^{***}$ & 0.691$^{***}$ & 0.552$^{***}$ & 0.612$^{***}$ & 0.714$^{***}$ & 0.73$^{***}$\\
 & - & - & (0.095) & (0.136) & (0.094) & (0.137) & (0.099) & (0.141)\\
old & 0.741$^{***}$ & 0.742$^{***}$ & - & - & - & - & - & -\\
 & (0.04) & (0.061) & - & - & - & - & - & -\\
dist & $-$0.004$^{**}$ & $-$0.001 & $-$0.003 & $-$0.025$^{**}$ & $-$0.002 & $-$0.025$^{**}$ & $-$0.003 & $-$0.022$^{*}$\\
 & (0.002) & (0.007) & (0.003) & (0.011) & (0.003) & (0.012) & (0.003) & (0.012)\\
old $\times$ dist & 0.003 & $-$0.003 & $-$0.001 & 0.04$^{**}$ & $-$0.002 & 0.038$^{**}$ & $-$0.001 & 0.037$^{**}$\\
 & (0.002) & (0.009) & (0.004) & (0.016) & (0.004) & (0.016) & (0.004) & (0.016)\\
dist$^2$ & - & 0 & - & $-$0.001$^{**}$ & - & $-$0.001$^{**}$ & - & $-$0.001\\
 & - & (0) & - & (0) & - & (0) & - & (0)\\
old $\times$ dist$^2$ & - & 0 & - & 0 & - & 0 & - & 0\\
 & - & (0) & - & (0.001) & - & (0.001) & - & (0.001)\\
 &  &  &  &  &  &  &  & \\
Contrôles & Oui & Oui & Oui & Oui & Oui & Oui & Oui & Oui\\
Observations & 2271 & 2271 & 2271 & 2271 & 2271 & 2271 & 2271 & 2271\\
R$^2$ ajusté & 0.356 & 0.356 & 0.24 & 0.241 & 0.249 & 0.249 & 0.191 & 0.191\\*
\end{longtable}
\end{ThreePartTable}
\endgroup{}

\quad N'ayant pas de justification formelle du choix spécifique de la fenêtre de 30 jours, nous présentons les résultats obtenus en fonction de la fenêtre retenue dans la Figure \ref{fig:agefrdcfhmodelsh}. Concrètement, pour chaque fenêtre allant de 5 à 30 jours, par pas de 1 jour, nous réestimons le modèle de \emph{fuzzy regression discontinuity}. Les fenêtres utilisées sont représentées sur l'axe horizontal de la figure et la hauteur des points représente les effets estimés de l'âge aux examens. Les nombres d'observations correspondant à chaque fenêtre sont représentés à l'aide d'un diagramme en couleurs. Les lignes en pointillées mettent en évidence les résultats avec 30 jours de fenêtre. Les lignes non horizontales autour des points représentent les intervalles de confiance à 95\% construits à partir d'écart-types calculés par wild bootstrap (voir Section \ref{agemethodscfh} et Annexe \ref{agewbootdesc}). Les variables dépendantes changent d'une ligne de graphiques à l'autre et le degré de polynôme d'une colonne de graphique à l'autre.
De manière générale, nos résultats sont robustes par rapport à la fenêtre vu que les points ne s'éloignent de la ligne horizontale en pointillée que pour des cas de fenêtres très petites\footnote{Il est également attendu d'avoir des intervalles de confiances de plus en plus larges au fur et à mesure que les fenêtres sont petites car de moins en moins d'observations sont utilisées dans les régressions correspondantes.}.

\quad Ce premier ensemble de résultats va globalement dans le sens de nos résultats obtenus par fonction de contrôle (Section \ref{agemodelsres}).

\begin{figure}[H]

{\centering \includegraphics[width=1\linewidth]{000_files/figure-latex/agefrdcfhmodelsh-1} 

}

\caption{Sensibilité des effets estimés de l'âge aux examens par régressions sur une discontinuité selon la fenêtre utilisée (de 5 à 30 jours)}\label{fig:agefrdcfhmodelsh}
\end{figure}

\quad L'ampleur des coefficients devant l'interaction entre l'âge aux examens et l'indicatrice du fait d'être un garçon du Tableau \ref{tab:agefrdcfhmodelssexe} suggèrent que les effets de l'âge sont plus forts chez les filles. Ces ampleurs sont également supérieures à leurs homologues obtenues par fonction de contrôle. Les coefficients concernés sont toutefois mesurés avec très peu de précision. Cela peut être dû à la taille de l'échantillon plus réduite (environ 2000 contre environ 50000 dans l'approche par fonction de contrôle). Mais cette dernière est de l'ordre de 2000 individus et cette justification est alors à considérer avec prudence. L'autre explication serait alors qu'il n'y a pas d'effet de l'âge supplémentaire chez les filles pour les individus au voisinage du seuil. Le fait que les coefficients sont relativement importants et stables (environ entre - 0.13 UET et - 0.16) constitue tout de même un point faible de cet argument.\\
Les résultats sur les sous-items de français montrés par le Tableau \ref{tab:agefrdcfhmodelssexessitemsfrench} de l'Annexe \ref{agefrdcfhmodelssexessitems} nous permettent de confirmer des effets supérieurs chez les filles surtout sur les notes en grammaire et en lecture. Ces résultats s'accordent avec leurs homologues sur données en coupe. Nous retrouvons moins de cette cohérence en ce qui concerne les sous-items de mathématiques (Tableau \ref{tab:agefrdcfhmodelssexessitemsfrench} de l'Annexe \ref{agefrdcfhmodelssexessitems}).

\begingroup\fontsize{8}{10}\selectfont

\begin{ThreePartTable}
\begin{TableNotes}
\item \textit{Sources :} Fichiers CM2 (2009 à 2012) - nés en 1999 et 2000, calculs de l'auteur.
\item \textit{Notes :} Une colonne correspond à une régression. On note $p$ le degré de polynôme de $dist$ et $old \times dist$ utilisé. Les notes sont normalisées sur l'année scolaire. Écart-types entre parenthèses. Fenêtre de 30 jours. Les contrôles utilisés sont le sexe, la CSP et l'année scolaire.
\item FRD : \textit{Fuzzy Regression Discontinuity.} CSP : Catégorie Socio-Professionnelle.
\item Significativité : 10\% * 5\% ** 1\% ***.
\end{TableNotes}
\begin{longtable}[t]{lllllll}
\caption{\label{tab:agefrdcfhmodelssexe}Estimations des effets hétérogènes de l'âge selon le sexe par régressions sur une discontinuité}\\
\toprule
\multicolumn{1}{c}{} & \multicolumn{6}{c}{Variable dépendante : } \\
\cmidrule(l{3pt}r{3pt}){2-7}
\multicolumn{1}{c}{} & \multicolumn{2}{c}{Note totale} & \multicolumn{2}{c}{Note totale en français} & \multicolumn{2}{c}{Note en mathématiques} \\
\cmidrule(l{3pt}r{3pt}){2-3} \cmidrule(l{3pt}r{3pt}){4-5} \cmidrule(l{3pt}r{3pt}){6-7}
 & \makecell{FRD, p = 1 \\ (1) } & \makecell{FRD, p = 2 \\ (2) } & \makecell{FRD, p = 1 \\ (3) } & \makecell{FRD, p = 2 \\ (4) } & \makecell{FRD, p = 1 \\ (5) } & \makecell{FRD, p = 2 \\ (6) }\\
\midrule
\endfirsthead
\caption[]{\label{tab:agefrdcfhmodelssexe}Estimations des effets hétérogènes de l'âge selon le sexe par régressions sur une discontinuité (suite)}\\
\toprule
 & \makecell{FRD, p = 1 \\ (1) } & \makecell{FRD, p = 2 \\ (2) } & \makecell{FRD, p = 1 \\ (3) } & \makecell{FRD, p = 2 \\ (4) } & \makecell{FRD, p = 1 \\ (5) } & \makecell{FRD, p = 2 \\ (6) }\\
\midrule
\endhead

\endfoot
\bottomrule
\insertTableNotes
\endlastfoot
Âge aux examens & 0.725$^{***}$ & 0.775$^{***}$ & 0.635$^{***}$ & 0.699$^{***}$ & 0.782$^{***}$ & 0.804$^{***}$\\
 & (0.109) & (0.148) & (0.109) & (0.149) & (0.113) & (0.152)\\
Âge aux examens $\times$ Sexe - Homme & $-$0.147 & $-$0.151 & $-$0.156 & $-$0.161 & $-$0.126 & $-$0.13\\
 & (0.106) & (0.105) & (0.107) & (0.106) & (0.108) & (0.108)\\
dist & $-$0.003 & $-$0.026$^{**}$ & $-$0.003 & $-$0.026$^{**}$ & $-$0.004 & $-$0.023$^{**}$\\
 & (0.003) & (0.011) & (0.003) & (0.012) & (0.003) & (0.012)\\
old $\times$ dist & $-$0.001 & 0.041$^{***}$ & $-$0.001 & 0.039$^{**}$ & 0 & 0.038$^{**}$\\
 & (0.004) & (0.016) & (0.004) & (0.016) & (0.004) & (0.016)\\
dist$^2$ & - & $-$0.001$^{**}$ & - & $-$0.001$^{**}$ & - & $-$0.001\\
 & - & (0) & - & (0) & - & (0)\\
old $\times$ dist$^2$ & - & 0 & - & 0 & - & 0\\
 & - & (0) & - & (0.001) & - & (0.001)\\
 &  &  &  &  &  & \\
Contrôles & Oui & Oui & Oui & Oui & Oui & Oui\\
Observations & 2271 & 2271 & 2271 & 2271 & 2271 & 2271\\
R$^2$ ajusté & 0.241 & 0.242 & 0.249 & 0.25 & 0.192 & 0.192\\*
\end{longtable}
\end{ThreePartTable}
\endgroup{}

\quad Nous montrons également les résultats en fonction de la fenêtre choisie. La Figure \ref{fig:agefrdcfhmodelssexeh} est construite de la même manière que la figure précédente à la seule différence que l'ampleur et la significativité des coefficients associés à l'interaction entre l'âge aux examens et l'indicatrice du sexe masculin sont en plus mises en évidence avec les points en forme de triangles.\\
Nous observons ici aussi généralement que nos résultats sont robustes et cohérents avec ceux obtenus avec l'approche par fonction de contrôle. Au vu de leurs ampleurs, les effets supplémentaires pour les garçons ne sont très probablement non significatifs qu'à cause de la taille de l'échantillon.

\begin{figure}[H]

{\centering \includegraphics[width=1\linewidth]{000_files/figure-latex/agefrdcfhmodelssexeh-1} 

}

\caption{Sensibilité des effets estimés de l'âge aux examens par régressions sur une discontinuité selon la fenêtre utilisée (de 5 à 30 jours), hétérogènes selon le sexe}\label{fig:agefrdcfhmodelssexeh}
\end{figure}

\quad Puisque nous ne disposons pas des CSP pour les élèves de 2009, c'est-à-dire pour les élèves nés en 1999 et en avance, nous nous abstenons de montrer les effets de l'âge selon la catégorie sociale estimés par régressions sur une discontinuité puisqu'ils souffrent d'un biais de sélection important.

\quad Les résultats obtenus par régressions sur une discontinuité supportent nos conclusions principales.

\hypertarget{agemodelsmtres}{%
\subsection{Les effets à moyen terme de l'âge aux examens}\label{agemodelsmtres}}

Nous avons montré qu'être âgé d'un an de plus aux évaluations de CM2 procure environ 0.3 unité d'écart-type d'avantage pour les élèves de même année scolaire et 0.6 unité d'écart-type ou plus pour les élèves nés au voisinage du 01\textsuperscript{er} janvier 2000. Ces effets sont importants. Néanmoins, nous les avons mesurés en fin de primaire, c'est-à-dire lorsque les élèves sont encore très jeunes. Au vu de la vitesse de croissance d'un enfant par rapport à celui d'un humain plus âgé, il est plutôt intuitif de penser qu'une différence d'âge entre deux enfants de bas âge joue plus de rôle dans la performance aux examens par rapport à une même différence entre deux adolescents, par exemple. Si le fonctionnement du système scolaire atténue, voire même élimine les différences de résultats causées par des différences d'âge à moyen ou à long terme, les effets de l'âge ne devraient alors pas être une préoccupation majeure des décideurs publics. Par contre, si les différences de performances survenues très tôt (causées par l'âge aux examens) influent considérablement sur le parcours scolaire ultérieur qui, lui est un déterminant des performances futures, alors il est important de penser à réduire cette ``injustice''\footnote{Les différences de résultats purement causées par des différences d'âge aux examens peuvent être vues comme une injustice car l'enfant ne choisit pas sa date de naissance.} le plus tôt possible. De plus, il est connu que l'enfance est la meilleure période pour investir en capital humain (\protect\hyperlink{ref-CUN:HEC:07}{Cunha \& Heckman, 2007}).

\quad Les mécanismes du système français qui font que les effets de l'âge aux examens plus tôt dans la scolarité d'un enfant peuvent être maintenus, voire amplifiés, sont nombreux. Les sections du collège (voir Section \ref{ageinst}) constituent un premier exemple si ces dernières ``scellent'' en quelque sorte le sort des élèves qui s'y retrouvent (\protect\hyperlink{ref-PAL:11}{Palheta, 2011}). La pratique du redoublement au collège peut également maintenir ou aggraver les effets de l'âge au CM2 si les moins performants au CM2 ont tendance à redoubler au collège et si le redoublement influe lui-même sur les performances futures via la démotivation de l'élève par exemple\footnote{Alet et al. (\protect\hyperlink{ref-ALE:eal:13}{2013}) fournissent quelques éléments par lesquels l'effet du redoublement peut être positif ou négatif. Le consensus n'est toutefois pas encore établi, à notre connaissance.}.

\quad Nos données de DNB peuvent nous permettent d'estimer pratiquement les mêmes modèles que ceux au CM2 pour analyser les effets à moyen terme de l'âge aux examens. Concrètement, nous prenons comme variables dépendantes principales la note totale, la note en français et la note en mathématiques au DNB. La note totale ici n'est pas uniquement composée de la note en français et en mathématiques. Nous considérons les notes aux écrits d'histoire-et-géographie, de dictée et de rédaction comme variables dépendantes secondaires pour nous assurer de la robustesse de nos résultats. La variable explicative d'intérêt est l'âge au moment des épreuves finales de DNB. Nous utilisons le régime scolaire comme variable de contrôle supplémentaire.

Les résultats sont présentés par le Tableau \ref{tab:agemodelsmt}. Le coefficient associé à l'âge relatif théorique de la colonne (1) indique une fois de plus que l'instrument utilisé est fort. Il est normal de constater que le coefficient de première étape (0.7 UET) est légèrement plus faible que celui au CM2, vu les possibles redoublements et sauts de classe entre le CM2 et la 3\textsuperscript{ème}, ce qui rend en quelque sorte la date de naissance moins prédictive de l'âge aux épreuves du DNB.\\
Les effets de l'âge sur les résultats aux examens du DNB sont compris entre environ 0.13 et 0.18 unité d'écart-type. Ces amplitudes représentent moins de la moitié des effets mesurés en fin de primaire.\\
Ces résultats suggèrent que les effets de l'âge aux examens diminuent à travers le temps mais restent significatifs jusqu'à la fin du collège. Cette proposition est en accord avec la littérature qui couvre les pays ne pratiquant pas de spécialisation précoce en éducation (\protect\hyperlink{ref-BED:DHU:06}{Bedard \& Dhuey, 2006} ; \protect\hyperlink{ref-GRE:09}{Grenet, 2009}).
Notons que les coefficients associés aux indicatrices de sexe et de catégorie sociale restent cohérents avec ceux présentés dans la section précédente.

\quad Les effets ne sont pas significativement différents en fonction de la variable dépendante, comme le montre le Tableau \ref{tab:agemodelsmtssmoy} de l'Annexe \ref{agemodelsmtssmoy}.

\begingroup\fontsize{8}{10}\selectfont

\begin{ThreePartTable}
\begin{TableNotes}
\item \textit{Sources :} Fichiers DNB (2014 à 2016), calculs de l'auteur.
\item \textit{Notes :} Une colonne correspond à une régression. Les notes sont normalisées sur l'année scolaire. Écart-types entre parenthèses. Les écart-types des coefficients estimés par fonction de contrôle sont calculées par wild bootstrap avec 1001 réplications. La variable $\hat{\nu}$ est le résidu de la première étape. Les contrôles utilisés sont le sexe, la CSP, le régime scolaire et l'année scolaire.
\item VI : Variable Instrumentale. FCH : Fonction de Contrôle prenant en compte l'Hétérogénéité de l'effet de l'âge. CSP : Catégorie Socio-Professionnelle.
\item Significativité : 10\% * 5\% ** 1\% ***.
\end{TableNotes}
\begin{longtable}[t]{llllllll}
\caption{\label{tab:agemodelsmt}Estimations des effets à moyen terme de l'âge aux examens}\\
\toprule
\multicolumn{1}{c}{} & \multicolumn{1}{c}{} & \multicolumn{6}{c}{Variable dépendante : Note} \\
\cmidrule(l{3pt}r{3pt}){3-8}
\multicolumn{1}{c}{} & \multicolumn{1}{c}{Âge aux examens} & \multicolumn{2}{c}{Totale} & \multicolumn{2}{c}{En français (écrits)} & \multicolumn{2}{c}{En mathématiques (écrits)} \\
\cmidrule(l{3pt}r{3pt}){2-2} \cmidrule(l{3pt}r{3pt}){3-4} \cmidrule(l{3pt}r{3pt}){5-6} \cmidrule(l{3pt}r{3pt}){7-8}
 & \makecell{\makecell{Première \\ étape} \\ (1) } & \makecell{\makecell{VI \\ \ } \\ (2) } & \makecell{\makecell{FCH \\ \ } \\ (3) } & \makecell{\makecell{VI \\ \ } \\ (4) } & \makecell{\makecell{FCH \\ \ } \\ (5) } & \makecell{\makecell{VI \\ \ } \\ (6) } & \makecell{\makecell{FCH \\ \ } \\ (7) }\\
\midrule
\endfirsthead
\caption[]{\label{tab:agemodelsmt}Estimations des effets à moyen terme de l'âge aux examens (suite)}\\
\toprule
\multicolumn{1}{c}{} & \multicolumn{1}{c}{} & \multicolumn{6}{c}{Variable dépendante : Note} \\
\cmidrule(l{3pt}r{3pt}){3-8}
\multicolumn{1}{c}{} & \multicolumn{1}{c}{Âge aux examens} & \multicolumn{2}{c}{Totale} & \multicolumn{2}{c}{En français (écrits)} & \multicolumn{2}{c}{En mathématiques (écrits)} \\
\cmidrule(l{3pt}r{3pt}){2-2} \cmidrule(l{3pt}r{3pt}){3-4} \cmidrule(l{3pt}r{3pt}){5-6} \cmidrule(l{3pt}r{3pt}){7-8}
 & \makecell{\makecell{Première \\ étape} \\ (1) } & \makecell{\makecell{VI \\ \ } \\ (2) } & \makecell{\makecell{FCH \\ \ } \\ (3) } & \makecell{\makecell{VI \\ \ } \\ (4) } & \makecell{\makecell{FCH \\ \ } \\ (5) } & \makecell{\makecell{VI \\ \ } \\ (6) } & \makecell{\makecell{FCH \\ \ } \\ (7) }\\
\midrule
\endhead

\endfoot
\bottomrule
\insertTableNotes
\endlastfoot
Âge aux examens & - & 0.179$^{***}$ & 0.185$^{***}$ & 0.177$^{***}$ & 0.179$^{***}$ & 0.135$^{***}$ & 0.144$^{***}$\\
 & - & (0.02) & (0.019) & (0.02) & (0.019) & (0.021) & (0.02)\\
$\hat{\nu}$ & - & - & $-$1.697$^{***}$ & - & $-$0.991$^{***}$ & - & $-$1.964$^{***}$\\
 & - & - & (0.201) & - & (0.205) & - & (0.212)\\
Âge aux examens $\times$ $\hat{\nu}$ & - & - & 0.057$^{***}$ & - & 0.015 & - & 0.082$^{***}$\\
 & - & - & (0.013) & - & (0.013) & - & (0.013)\\
Âge relatif théorique & 0.783$^{***}$ & - & - & - & - & - & -\\
 & (0.008) & - & - & - & - & - & -\\
Sexe - Homme & 0.089$^{***}$ & $-$0.273$^{***}$ & $-$0.279$^{***}$ & $-$0.472$^{***}$ & $-$0.476$^{***}$ & $-$0.112$^{***}$ & $-$0.121$^{***}$\\
 & (0.004) & (0.009) & (0.008) & (0.009) & (0.009) & (0.01) & (0.009)\\
CSP - Moyenne & $-$0.12$^{***}$ & 0.407$^{***}$ & 0.416$^{***}$ & 0.377$^{***}$ & 0.383$^{***}$ & 0.331$^{***}$ & 0.343$^{***}$\\
 & (0.005) & (0.012) & (0.011) & (0.012) & (0.01) & (0.012) & (0.011)\\
CSP - Favorisée & $-$0.16$^{***}$ & 0.636$^{***}$ & 0.65$^{***}$ & 0.575$^{***}$ & 0.585$^{***}$ & 0.529$^{***}$ & 0.545$^{***}$\\
 & (0.007) & (0.019) & (0.018) & (0.018) & (0.017) & (0.019) & (0.018)\\
CSP - Très favorisée & $-$0.261$^{***}$ & 1.12$^{***}$ & 1.134$^{***}$ & 0.965$^{***}$ & 0.975$^{***}$ & 0.989$^{***}$ & 1.005$^{***}$\\
 & (0.006) & (0.016) & (0.015) & (0.015) & (0.014) & (0.017) & (0.016)\\
CSP - Autres & 0.218$^{***}$ & $-$0.166$^{***}$ & $-$0.177$^{***}$ & $-$0.121$^{**}$ & $-$0.128$^{**}$ & $-$0.105$^{*}$ & $-$0.125$^{**}$\\
 & (0.034) & (0.06) & (0.051) & (0.058) & (0.051) & (0.058) & (0.051)\\
Régime scolaire - Interne & $-$0.102$^{**}$ & $-$0.011 & $-$0.006 & $-$0.028 & $-$0.022 & $-$0.162$^{*}$ & $-$0.159$^{*}$\\
 & (0.05) & (0.093) & (0.086) & (0.094) & (0.083) & (0.094) & (0.089)\\
Régime scolaire - Externe & 0.072$^{***}$ & $-$0.261$^{***}$ & $-$0.267$^{***}$ & $-$0.243$^{***}$ & $-$0.247$^{***}$ & $-$0.237$^{***}$ & $-$0.245$^{***}$\\
 & (0.005) & (0.01) & (0.009) & (0.01) & (0.009) & (0.01) & (0.01)\\
Cohorte - 2015 & $-$0.019$^{***}$ & $-$0.004 & $-$0.002 & $-$0.003 & $-$0.001 & $-$0.004 & $-$0.002\\
 & (0.005) & (0.011) & (0.01) & (0.011) & (0.01) & (0.012) & (0.011)\\
Cohorte - 2016 & $-$0.039$^{***}$ & $-$0.001 & 0.004 & $-$0.001 & 0.003 & $-$0.002 & 0.004\\
 & (0.005) & (0.011) & (0.011) & (0.011) & (0.01) & (0.012) & (0.011)\\
 &  &  &  &  &  &  & \\
Contrôles & Oui & Oui & Oui & Oui & Oui & Oui & Oui\\
Observations & 42606 & 41567 & 41567 & 41497 & 41497 & 41343 & 41343\\
R$^2$ ajusté & 0.251 & - & 0.251 & - & 0.238 & - & 0.186\\*
\end{longtable}
\end{ThreePartTable}
\endgroup{}
\newpage

\quad Le Tableau \ref{tab:agemodelsmtsexemod} nous enseigne qu'il n'y a plus d'hétérogénéité de ces effets selon le sexe puisque les coefficients associés à l'interaction entre l'âge aux examens et l'indicatrice de sexe masculin ne sont pas significatifs et sont de faible ampleur.\\
Le Tableau \ref{tab:agemodelsmtsexemodssmoy} de l'Annexe \ref{agemodelsmtssmoysexemod} qui tente de capturer les éventuels effets hétérogènes de l'âge au DNB sur les notes en histoire-et-géographie, en dictée et en rédaction fournit le même enseignement.
En reprenant notre interprétation dans la Section \ref{agemodelsheterores}, il semble que si les filles mûrissent plus vite que les garçons en primaire, soit ce n'est plus le cas au collège, soit la différence de vitesse de maturité n'affecte plus les performances en fin de collège.

\begingroup\fontsize{8}{10}\selectfont

\begin{ThreePartTable}
\begin{TableNotes}
\item \textit{Sources :} Fichiers DNB (2014 à 2016), calculs de l'auteur.
\item \textit{Notes :} Une colonne correspond à une régression. Les notes sont normalisées sur l'année scolaire. Écart-types entre parenthèses. Les écart-types des coefficients estimés par fonction de contrôle sont calculées par wild bootstrap avec 1001 réplications. La variable $\hat{\nu}$ est le résidu de la première étape. Les contrôles utilisés sont le sexe, la CSP, le régime scolaire et l'année scolaire.
\item VI : Variable Instrumentale. FCH : Fonction de Contrôle prenant en compte l'Hétérogénéité de l'effet de l'âge. CSP : Catégorie Socio-Professionnelle.
\item Significativité : 10\% * 5\% ** 1\% ***.
\end{TableNotes}
\begin{longtable}[t]{lllllll}
\caption{\label{tab:agemodelsmtsexemod}Estimations des effets de l'âge aux examens hétérogènes de l'âge selon le sexe, à moyen terme}\\
\toprule
\multicolumn{1}{c}{} & \multicolumn{6}{c}{Variable dépendante : Note} \\
\cmidrule(l{3pt}r{3pt}){2-7}
\multicolumn{1}{c}{} & \multicolumn{2}{c}{Totale} & \multicolumn{2}{c}{En français (écrits)} & \multicolumn{2}{c}{En mathématiques (écrits)} \\
\cmidrule(l{3pt}r{3pt}){2-3} \cmidrule(l{3pt}r{3pt}){4-5} \cmidrule(l{3pt}r{3pt}){6-7}
 & \makecell{\makecell{VI \\ \ } \\ (1) } & \makecell{\makecell{FCH \\ \ } \\ (2) } & \makecell{\makecell{VI \\ \ } \\ (3) } & \makecell{\makecell{FCH \\ \ } \\ (4) } & \makecell{\makecell{VI \\ \ } \\ (5) } & \makecell{\makecell{FCH \\ \ } \\ (6) }\\
\midrule
\endfirsthead
\caption[]{\label{tab:agemodelsmtsexemod}Estimations des effets de l'âge aux examens hétérogènes de l'âge selon le sexe, à moyen terme (suite)}\\
\toprule
\multicolumn{1}{c}{} & \multicolumn{6}{c}{Variable dépendante : Note} \\
\cmidrule(l{3pt}r{3pt}){2-7}
\multicolumn{1}{c}{} & \multicolumn{2}{c}{Totale} & \multicolumn{2}{c}{En français (écrits)} & \multicolumn{2}{c}{En mathématiques (écrits)} \\
\cmidrule(l{3pt}r{3pt}){2-3} \cmidrule(l{3pt}r{3pt}){4-5} \cmidrule(l{3pt}r{3pt}){6-7}
 & \makecell{\makecell{VI \\ \ } \\ (1) } & \makecell{\makecell{FCH \\ \ } \\ (2) } & \makecell{\makecell{VI \\ \ } \\ (3) } & \makecell{\makecell{FCH \\ \ } \\ (4) } & \makecell{\makecell{VI \\ \ } \\ (5) } & \makecell{\makecell{FCH \\ \ } \\ (6) }\\
\midrule
\endhead

\endfoot
\bottomrule
\insertTableNotes
\endlastfoot
Âge aux examens & 0.174$^{***}$ & 0.18$^{***}$ & 0.175$^{***}$ & 0.177$^{***}$ & 0.131$^{***}$ & 0.15$^{***}$\\
 & (0.027) & (0.026) & (0.028) & (0.025) & (0.027) & (0.025)\\
Âge aux examens $\times$ Sexe - Homme & 0.01 & 0.01 & 0.004 & 0.005 & 0.009 & $-$0.012\\
 & (0.04) & (0.034) & (0.04) & (0.033) & (0.041) & (0.035)\\
 &  &  &  &  &  & \\
Contrôles & Oui & Oui & Oui & Oui & Oui & Oui\\
Observations & 41567 & 41567 & 41497 & 41497 & 41343 & 41343\\
R$^2$ ajusté & - & 0.251 & - & 0.238 & - & 0.186\\*
\end{longtable}
\end{ThreePartTable}
\endgroup{}

\quad Quant aux résultats d'estimation des effets de l'âge aux épreuves de DNB, hétérogènes selon la catégorie sociale, le Tableau \ref{tab:agemodelsmtpcsregmod} montre qu'il n'y a clairement pas d'effets plus forts chez les très favorisés en français\footnote{Coefficients associés à l'interaction entre l'âge aux examens et l'indicatrice de CSP très favorisée de faible ampleur et non significatifs.}. L'effet mesuré sur la note en mathématiques est d'ampleur non négligeable mais estimé avec peu de précision. L'effet sur la note totale par fonction de contrôle est à peine significatif au seuil de 5\% et n'est significatif qu'au seuil de 10\% lorsqu'il est estimé par variables instrumentales traditionnelles.\\
Nous pouvons alors considérer que l'effet à moyen terme de l'âge aux examens sur les performances en fin de collège ne diffèrent pas en fonction de la catégorie sociale.\\
En réitérant notre interprétation de la Section \ref{agemodelsheterores} qui explique la supériorité des effets chez les très favorisés par la prévalence des décalages stratégiques d'inscription de la part des parents, nos résultats à moyen terme suggèrent que les avantages considérables de décaler l'inscription de son enfant ne sont plus visibles à la fin du collège. De même, l'explication de Elder \& Lubotsky (\protect\hyperlink{ref-ELD:LUB:09}{2009}) sur l'importance des investissements pré-primaires est cohérente avec ce constat.

\begingroup\fontsize{8}{10}\selectfont

\begin{ThreePartTable}
\begin{TableNotes}
\item \textit{Sources :} Fichiers DNB (2014 à 2016), calculs de l'auteur.
\item \textit{Notes :} Une colonne correspond à une régression. Les notes sont normalisées sur l'année scolaire. Écart-types entre parenthèses. Les écart-types des coefficients estimés par fonction de contrôle sont calculées par wild bootstrap avec 1001 réplications. La variable $\hat{\nu}$ est le résidu de la première étape. Les contrôles utilisés sont le sexe, la CSP, le régime scolaire et l'année scolaire.
\item VI : Variable Instrumentale. FCH : Fonction de Contrôle prenant en compte l'Hétérogénéité de l'effet de l'âge. CSP : Catégorie Socio-Professionnelle.
\item Significativité : 10\% * 5\% ** 1\% ***.
\end{TableNotes}
\begin{longtable}[t]{lllllll}
\caption{\label{tab:agemodelsmtpcsregmod}Estimations des effets de l'âge aux examens hétérogènes de l'âge selon la catégorie sociale, à moyen terme}\\
\toprule
\multicolumn{1}{c}{} & \multicolumn{6}{c}{Variable dépendante : Note} \\
\cmidrule(l{3pt}r{3pt}){2-7}
\multicolumn{1}{c}{} & \multicolumn{2}{c}{Totale} & \multicolumn{2}{c}{En français (écrits)} & \multicolumn{2}{c}{En mathématiques (écrits)} \\
\cmidrule(l{3pt}r{3pt}){2-3} \cmidrule(l{3pt}r{3pt}){4-5} \cmidrule(l{3pt}r{3pt}){6-7}
 & \makecell{\makecell{VI \\ \ } \\ (1) } & \makecell{\makecell{FCH \\ \ } \\ (2) } & \makecell{\makecell{VI \\ \ } \\ (3) } & \makecell{\makecell{FCH \\ \ } \\ (4) } & \makecell{\makecell{VI \\ \ } \\ (5) } & \makecell{\makecell{FCH \\ \ } \\ (6) }\\
\midrule
\endfirsthead
\caption[]{\label{tab:agemodelsmtpcsregmod}Estimations des effets de l'âge aux examens hétérogènes de l'âge selon la catégorie sociale, à moyen terme (suite)}\\
\toprule
\multicolumn{1}{c}{} & \multicolumn{6}{c}{Variable dépendante : Note} \\
\cmidrule(l{3pt}r{3pt}){2-7}
\multicolumn{1}{c}{} & \multicolumn{2}{c}{Totale} & \multicolumn{2}{c}{En français (écrits)} & \multicolumn{2}{c}{En mathématiques (écrits)} \\
\cmidrule(l{3pt}r{3pt}){2-3} \cmidrule(l{3pt}r{3pt}){4-5} \cmidrule(l{3pt}r{3pt}){6-7}
 & \makecell{\makecell{VI \\ \ } \\ (1) } & \makecell{\makecell{FCH \\ \ } \\ (2) } & \makecell{\makecell{VI \\ \ } \\ (3) } & \makecell{\makecell{FCH \\ \ } \\ (4) } & \makecell{\makecell{VI \\ \ } \\ (5) } & \makecell{\makecell{FCH \\ \ } \\ (6) }\\
\midrule
\endhead

\endfoot
\bottomrule
\insertTableNotes
\endlastfoot
Âge aux examens & 0.159$^{***}$ & 0.168$^{***}$ & 0.169$^{***}$ & 0.168$^{***}$ & 0.121$^{***}$ & 0.135$^{***}$\\
 & (0.028) & (0.025) & (0.029) & (0.026) & (0.028) & (0.025)\\
Âge aux examens $\times$ CSP - Moyenne & 0.062 & 0.089$^{**}$ & 0.045 & 0.087$^{**}$ & 0.028 & 0.033\\
 & (0.049) & (0.044) & (0.049) & (0.044) & (0.05) & (0.047)\\
Âge aux examens $\times$ CSP - Favorisée & $-$0.07 & $-$0.035 & $-$0.062 & $-$0.026 & $-$0.045 & $-$0.002\\
 & (0.074) & (0.071) & (0.073) & (0.073) & (0.076) & (0.075)\\
Âge aux examens $\times$ CSP - Très favorisée & 0.112$^{*}$ & 0.114$^{**}$ & 0.038 & 0.047 & 0.117 & 0.118$^{*}$\\
 & (0.067) & (0.057) & (0.062) & (0.054) & (0.072) & (0.063)\\
Âge aux examens $\times$ CSP - Autres & $-$0.357 & $-$0.192 & $-$0.343 & $-$0.233 & $-$0.233 & $-$0.029\\
 & (0.222) & (0.231) & (0.224) & (0.237) & (0.216) & (0.223)\\
 &  &  &  &  &  & \\
Contrôles & Oui & Oui & Oui & Oui & Oui & Oui\\
Observations & 41567 & 41567 & 41497 & 41497 & 41343 & 41343\\
R$^2$ ajusté & - & 0.253 & - & 0.239 & - & 0.189\\*
\end{longtable}
\end{ThreePartTable}
\endgroup{}

\hypertarget{agemodelssuppres}{%
\subsection{\texorpdfstring{Extensions : isolation des effets de l'âge aux examens \emph{per se} et de l'âge relatif dans la classe}{Extensions : isolation des effets de l'âge aux examens per se et de l'âge relatif dans la classe}}\label{agemodelssuppres}}

Dans cette section, nous tentons d'isoler les effets de l'âge aux examens au CM2 en soi et de l'âge relatif. Ce sont des tentatives et extensions puisque la première n'exploite qu'un évènement très particulier survenu à La Réunion et nous n'utilisons que l'estimateur par double moindres carrés dans la seconde.

\quad D'une part, rappelons que les évaluations de CM2 de 2012 se sont déroulées 4 mois plus tard que d'habitude pour des raisons exogènes. À même âge d'entrée, les élèves de 2012 sont alors âgés de 4 mois de plus que ceux des années scolaires précédentes. Nous exploitons ce phénomène pour tenter de séparer l'effet de l'âge aux examens net de celui de l'âge d'entrée. Pour ce faire, nous rajoutons le régresseur \(a_i \times I(t = 2012)\) dans l'équation structurelle \eqref{eq:ageols}. Le coefficient associé est donc l'effet supplémentaire d'être spécifiquement âgé 4 mois de plus aux examens. Nous utilisons l'approche par fonction de contrôle.

\quad D'autre part, l'identifiant de la classe est à notre disposition au CM2 comme en 3\textsuperscript{ème}. Nous pouvons alors tenter de séparer l'effet de l'âge relatif de la classe de l'effet de l'âge aux examens. Intuitivement, nous nous demandons dans quelle mesure être le plus âgé parmi les camarades de classe (les \emph{pairs}) affecte la performance scolaire, indépendamment de la valeur absolue de l'âge de l'élève\footnote{Cet effet est par définition le même que l'élève fasse partie des plus jeunes parmi des adolescents ou parmi des enfants de bas âge.}. Pratiquement, nous suivons Cascio \& Schanzenbach (\protect\hyperlink{ref-CAS:SCH:16}{2016}) (États-Unis) ou encore Peña (\protect\hyperlink{ref-PEN:17}{2017}) (Mexique) en rajoutant dans l'équation structurelle \eqref{eq:ageols} le régresseur supplémentaire \((a_i - \bar{a}_{-i})\) où \(\bar{a}_{-i}\) représente la moyenne de l'âge aux examens chez les pairs de l'individu \(i\)\footnote{Les pairs sont constitués des élèves de la classe à l'exception de l'individu.}. Nous instrumentons \(\bar{a}_{-i}\) par \(\bar{z}_{-i}\)\footnote{Pour comprendre cet instrument, il peut être utile d'imaginer sa version en mois de naissance, par exemple. Dans ce cas, cet instrument serait les proportions de pairs nés en janvier, février, etc..}. En plus des conditions de validité de \(z_i\), une condition supplémentaire est l'absence de lien entre les classes et \(z_{-i}\). En guise d'illustration, il ne faudrait pas que certaines classes regroupent une grande part d'élèves nés en janvier et d'autres une grande part d'élèves nés en décembre. D'après les informations fournies par des responsables de l'Académie de La Réunion, les classes ne devraient pas être formées en fonction de la date de naissance. Pour appuyer ce propos empiriquement, nous suivons Cascio \& Schanzenbach (\protect\hyperlink{ref-CAS:SCH:16}{2016}) et régressons chaque variable de contrôle en fonction de l'instrument de l'âge relatif : \((z_i - \bar{z}_{-i})\) ainsi que des variables de contrôles restantes. Les résultats sont présentés dans le Tableau \ref{tab:ageexoinstagerel}. Au CM2 (panel A), on ne détecte aucune corrélation entre l'instrument de l'âge relatif et les observables, toutes choses égales par ailleurs et au sein d'un établissement. Au CM2 comme en 3\textsuperscript{ème}, nous ne détectons aucune corrélation robuste entre les observables et le jour de naissance dans l'année des pairs.

\quad Les tests de présence d'éventuels effets directs de \(\bar{z}_{-i}\) sont globalement négatifs (non reportés).

Les conditions de validité de notre stratégie d'identification de l'effet de l'âge relatif semblent alors être remplies.

Une limite de cette seconde extension est que nous estimons les modèles uniquement par variables instrumentales. En effet, deux variables potentiellement endogènes sont à instrumenter : l'âge absolu et l'âge relatif. Nous laissons aux futures recherches l'utilisation de l'approche par fonction de contrôle pour ce type d'estimation.

\newpage
\begingroup\fontsize{8}{10}\selectfont

\begin{ThreePartTable}
\begin{TableNotes}
\item \textit{Sources :} Fichiers CM2 (2010 à 2012), Fichiers DNB (2014 à 2016), calculs de l'auteur.
\item \textit{Notes :} Un panel-colonne correspond à une régression par moindres carrés ordinaires. L'instrument de l'âge relatif est décrit dans le texte. Effets fixes d'établissement-année puisque c'est à ce niveau que les classes sont attribuées. Écart-types robustes entre parenthèses.
\item CSP : Catégorie Socio-Professionnelle. CM2 : Cours Moyen 2\textsuperscript{ème} année.
\item Significativité : 10\% * 5\% ** 1\% ****.
\end{TableNotes}
\begin{longtable}[t]{lllllllll}
\caption{\label{tab:ageexoinstagerel}Crédibilité de l'hypothèse d'indépendance entre la date de naissance et la classe assignée à l'élève}\\
\toprule
\multicolumn{1}{c}{} & \multicolumn{8}{c}{Variable dépendante : } \\
\cmidrule(l{3pt}r{3pt}){2-9}
\multicolumn{1}{c}{} & \multicolumn{1}{c}{Sexe - } & \multicolumn{5}{c}{CSP - } & \multicolumn{2}{c}{Régime scolaire - } \\
\cmidrule(l{3pt}r{3pt}){2-2} \cmidrule(l{3pt}r{3pt}){3-7} \cmidrule(l{3pt}r{3pt}){8-9}
 & \makecell{Garçon \\ (1) } & \makecell{Moyenne \\ (2) } & \makecell{Favorisée \\ (3) } & \makecell{Très favorisée \\ (4) } & \makecell{Autres \\ (5) } & \makecell{Manquante \\ (6) } & \makecell{Interne \\ (7) } & \makecell{Externe \\ (8) }\\
\midrule
\endfirsthead
\caption[]{\label{tab:ageexoinstagerel}Crédibilité de l'hypothèse d'indépendance entre la date de naissance et la classe assignée à l'élève (suite)}\\
\toprule
 & \makecell{Garçon \\ (1) } & \makecell{Moyenne \\ (2) } & \makecell{Favorisée \\ (3) } & \makecell{Très favorisée \\ (4) } & \makecell{Autres \\ (5) } & \makecell{Manquante \\ (6) } & \makecell{Interne \\ (7) } & \makecell{Externe \\ (8) }\\
\midrule
\endhead

\endfoot
\bottomrule
\insertTableNotes
\endlastfoot
\addlinespace[.5cm]
\multicolumn{9}{l}{\textbf{Panel A : CM2}}\\
\hline
\hspace{1em}Âge relatif théorique & $-$0.033 & $-$0.05 & $-$0.013 & 0.005 & 0.004 & 0.016 & - & -\\
\hspace{1em} & (0.039) & (0.042) & (0.027) & (0.03) & (0.008) & (0.037) & - & -\\
\hspace{1em}Instrument de l'âge relatif & 0.036 & 0.049 & 0.005 & $-$0.008 & $-$0.004 & $-$0.028 & - & -\\
\hspace{1em} & (0.038) & (0.04) & (0.026) & (0.029) & (0.008) & (0.036) & - & -\\
\hspace{1em} &  &  &  &  &  &  & - & -\\
\hspace{1em}Contrôles & Oui & Oui & Oui & Oui & Oui & Oui & - & -\\
\hspace{1em}Observations & 42015 & 42015 & 42015 & 42015 & 42015 & 42015 & - & -\\
\hspace{1em}R$^2$ ajusté & -0.024 & -0.025 & -0.025 & -0.025 & -0.025 & -0.024 & - & -\\
\addlinespace[.5cm]
\multicolumn{9}{l}{\textbf{Panel B : 3ème}}\\
\hline
\hspace{1em}Âge relatif théorique & 0.026 & $-$0.022 & 0.007 & 0.036 & $-$0.013 & - & 0.044$^{**}$ & 0.033\\
\hspace{1em} & (0.046) & (0.041) & (0.023) & (0.038) & (0.011) & - & (0.021) & (0.049)\\
\hspace{1em}Instrument de l'âge relatif & $-$0.03 & 0.018 & $-$0.018 & $-$0.044 & 0.013 & - & $-$0.04$^{**}$ & $-$0.024\\
\hspace{1em} & (0.046) & (0.04) & (0.022) & (0.037) & (0.011) & - & (0.019) & (0.047)\\
\hspace{1em} &  &  &  &  &  & - &  & \\
\hspace{1em}Contrôles & Oui & Oui & Oui & Oui & Oui & - & Oui & Oui\\
\hspace{1em}Observations & 42606 & 42606 & 42606 & 42606 & 42606 & - & 42606 & 42606\\
\hspace{1em}R$^2$ ajusté & -0.005 & -0.003 & -0.002 & 0.004 & -0.005 & - & -0.002 & 0.02\\*
\end{longtable}
\end{ThreePartTable}
\endgroup{}

\quad Le Tableau \ref{tab:agemodelsrel} montre les résultats d'estimation de trois types de modèles lorsque la note totale (colonnes 1 à 3), la note en français (colonnes 4 à 6) et la note en mathématiques (colonnes 7 à 9) sont retenues comme variables dépendantes. Ces trois modèles sont tels que nous tentons uniquement d'isoler l'effet de l'âge absolu dans le premier (``ABS'', colonnes 1, 5 et 7, issus d'estimations par fonction de contrôle), l'effet d'âge relatif dans le deuxième (``REL'', colonnes, 2, 5 et 8, issus d'estimations par variables instrumentales) et les deux à la fois dans le dernier qui sert de test de robustesse (``ABSREL'', colonnes 3, 6 et 9, issus d'estimations par variables instrumentales).

Avec le modèle ABS, nous ne détectons aucun effet d'avoir passé les évaluations de 2012 4 mois plus tard, comme cela est indiqué par le coefficient devant l'interaction entre l'âge aux examens et l'indicatrice de cohorte de 2012 qui est non significatif, de faible ampleur et même négatif.

Le modèle REL nous montre que l'effet de l'âge relatif est négatif et d'ampleur considérable (- 0.33 pour la note totale, - 0.21 pour la note en français et - 0.46 pour la note en mathématiques). En plus, les coefficients devant l'âge aux examens avec des ampleurs qui sont deux fois plus élevées et très significatifs (0.66 pour la note totale, 0.5 pour la note en français et 0.78 pour la note en mathématiques). Cette fois, les effets en mathématiques ressortent significativement supérieurs à ceux en français (probabilité critique de 0.01 issu d'un test d'égalité des coefficients associés à l'âge aux examens des colonnes 5 et 8). De la même manière, l'effet de l'âge relatif en mathématiques est supérieur à celui en français (probabilité critique de 0.02).

Les résultats d'estimation du modèle ABSREL sont qualitativement les mêmes que ceux issus des deux premiers.

\quad Le fait que l'âge absolu en soi ne soit pas significatif va en partie à l'encontre des résultats de la littérature puisque Crawford et al. (\protect\hyperlink{ref-CRA:eal:07}{2007}) et Black et al. (\protect\hyperlink{ref-BLA:eal:11}{2011}) trouvent que l'âge absolu est plus important que l'effet de l'âge d'entrée en Angleterre et en Norvège, respectivement. Une possible raison de l'absence d'effet dans notre cas est que les enfants absolument plus âgés (même âge d'entrée que les autres enfants) ne le sont que de 4 mois. Ces 4 mois peuvent ne pas générer assez de variation pour que l'on arrive à détecter un effet isolé de l'âge absolu, s'il existe. L'autre explication est qu'il n'y pas d'effet d'âge absolu au CM2 à La Réunion. Nous préférons cette seconde explication pour deux raisons. Premièrement, les coefficients devant l'interaction entre l'âge aux examens et l'indicatrice de cohorte de 2012 sont négatifs et non significatifs (colonnes 1, 3, 4, 6, 7 et 9 du Tableau \ref{tab:agemodelsrel}) alors que nous nous attendrions plutôt à ce qu'ils soient positifs et significatifs, d'après les résultats de la littérature. Deuxièmement, nos résultats précédents sur la supériorité des effets de l'âge chez les très favorisés est cohérent avec un modèle théorique dans lequel l'âge d'entrée est important mais pas l'âge absolu en soi (\protect\hyperlink{ref-ELD:LUB:09}{Elder \& Lubotsky, 2009}).

\quad Être relativement plus jeune dans sa classe semble donc être considérablement bénéfique aux élèves de CM2. Le même type de résultat est retrouvé dans Cascio \& Schanzenbach (\protect\hyperlink{ref-CAS:SCH:16}{2016}) ou encore Peña (\protect\hyperlink{ref-PEN:17}{2017}). Nous ne savons cependant pas si cet avantage en faveur des plus jeunes de la classe est dû au fait que les enseignants porteraient plus d'attention à ces élèves ou au fait que les élèves plus âgés ont un effet positif direct sur les élèves plus jeunes (via des effets de motivation ou d'entraide, par exemple).

\newpage
\begingroup\fontsize{7}{9}\selectfont

\begin{ThreePartTable}
\begin{TableNotes}
\item \textit{Sources :} Fichiers CM2 (2009 à 2012), calculs de l'auteur.
\item \textit{Notes :} Une colonne correspond à une régression. Les notes sont normalisées sur l'année scolaire. Écart-types entre parenthèses. Ils sont calculés par wild bootstrap avec 1001 réplications. Les contrôles utilisés sont le sexe, la CSP et l'année scolaire.
\item ABS : modèle qui tente de mesurer séparément l'effet de l'âge absolu. REL : modèle qui tente de mesurer séparément l'effet de l'âge relatif. ABSREL : croisement des spécifications ABS et REL. CSP : Catégorie Socio-Professionnelle.
\item Significativité : 10\% * 5\% ** 1\% ***.
\end{TableNotes}
\begin{longtable}[t]{llllllllll}
\caption{\label{tab:agemodelsrel}Estimations des effets séparés de l'âge absolu et de l'âge relatif}\\
\toprule
\multicolumn{1}{c}{} & \multicolumn{9}{c}{Variable dépendante : Note} \\
\cmidrule(l{3pt}r{3pt}){2-10}
\multicolumn{1}{c}{} & \multicolumn{3}{c}{Totale} & \multicolumn{3}{c}{En français} & \multicolumn{3}{c}{En mathématiques} \\
\cmidrule(l{3pt}r{3pt}){2-4} \cmidrule(l{3pt}r{3pt}){5-7} \cmidrule(l{3pt}r{3pt}){8-10}
 & \makecell{ABS \\ (1) } & \makecell{REL \\ (2) } & \makecell{ABSREL \\ (3) } & \makecell{ABS \\ (4) } & \makecell{REL \\ (5) } & \makecell{ABSREL \\ (6) } & \makecell{ABS \\ (7) } & \makecell{REL \\ (8) } & \makecell{ABSREL \\ (9) }\\
\midrule
\endfirsthead
\caption[]{\label{tab:agemodelsrel}Estimations des effets séparés de l'âge absolu et de l'âge relatif (suite)}\\
\toprule
\multicolumn{1}{c}{} & \multicolumn{9}{c}{Variable dépendante : Note} \\
\cmidrule(l{3pt}r{3pt}){2-10}
\multicolumn{1}{c}{} & \multicolumn{3}{c}{Totale} & \multicolumn{3}{c}{En français} & \multicolumn{3}{c}{En mathématiques} \\
\cmidrule(l{3pt}r{3pt}){2-4} \cmidrule(l{3pt}r{3pt}){5-7} \cmidrule(l{3pt}r{3pt}){8-10}
 & \makecell{ABS \\ (1) } & \makecell{REL \\ (2) } & \makecell{ABSREL \\ (3) } & \makecell{ABS \\ (4) } & \makecell{REL \\ (5) } & \makecell{ABSREL \\ (6) } & \makecell{ABS \\ (7) } & \makecell{REL \\ (8) } & \makecell{ABSREL \\ (9) }\\
\midrule
\endhead

\endfoot
\bottomrule
\insertTableNotes
\endlastfoot
Âge aux examens & 0.352$^{***}$ & 0.657$^{***}$ & 0.691$^{***}$ & 0.318$^{***}$ & 0.514$^{***}$ & 0.543$^{***}$ & 0.355$^{***}$ & 0.788$^{***}$ & 0.824$^{***}$\\
 & (0.034) & (0.079) & (0.087) & (0.034) & (0.078) & (0.086) & (0.035) & (0.079) & (0.088)\\
Âge relatif (classe) & - & $-$0.33$^{***}$ & $-$0.329$^{***}$ & - & $-$0.212$^{***}$ & $-$0.211$^{***}$ & - & $-$0.463$^{***}$ & $-$0.462$^{***}$\\
 & - & (0.077) & (0.077) & - & (0.076) & (0.076) & - & (0.077) & (0.078)\\
Âge aux examens $\times$ Année - 2010 & 0.03 & - & $-$0.028 & 0.031 & - & $-$0.029 & 0.035 & - & $-$0.023\\
 & (0.046) & - & (0.056) & (0.045) & - & (0.055) & (0.048) & - & (0.056)\\
Âge aux examens $\times$ Année - 2011 & $-$0.084$^{*}$ & - & $-$0.053 & $-$0.069 & - & $-$0.048 & $-$0.093$^{**}$ & - & $-$0.049\\
 & (0.045) & - & (0.055) & (0.045) & - & (0.054) & (0.046) & - & (0.055)\\
Âge aux examens $\times$ Année - 2012 & $-$0.041 & - & $-$0.056 & $-$0.028 & - & $-$0.039 & $-$0.049 & - & $-$0.071\\
 & (0.047) & - & (0.056) & (0.046) & - & (0.055) & (0.048) & - & (0.056)\\
 &  &  &  &  &  &  &  &  & \\
Contrôles & Oui & Oui & Oui & Oui & Oui & Oui & Oui & Oui & Oui\\
Observations & 54341 & 54319 & 54319 & 54341 & 54319 & 54319 & 54341 & 54319 & 54319\\
R$^2$ ajusté & 0.223 & - & - & 0.24 & - & - & 0.164 & - & -\\*
\end{longtable}
\end{ThreePartTable}
\endgroup{}

\quad Lorsque les interactions nécessaires sont intégrées pour analyser les effets de l'âge relatif séparément selon le genre (Tableau \ref{tab:agemodelsrelsexe}), les estimations issues du modèle ABS (colonnes 1, 4 et 7) ne nous renseignent pas davantage sur les effets de l'âge aux examens en soi.

Les effets de l'âge aux examens, lorsqu'ils sont isolés de ceux de l'âge relatif (modèle REL, colonnes 2, 5 et 8), apparaissent cette fois plus fort chez les garçons. En même temps, l'amplitude de l'effet négatif de l'âge relatif est plus fort chez les garçons. Autrement dit, les garçons sont plus avantagés d'être de plus en plus jeune dans la classe et en même temps sont plus avantagés par un même écart d'âge aux examens par rapport aux filles.\\
\quad Le modèle ABSREL (colonnes 3, 6 et 9) nous ramène aux mêmes conclusions mais nous apprend en plus qu'être plus âgé aux examens en 2012 est désavantageux pour les garçons, au vu des coefficients associés à l'interaction entre l'âge aux examens, indicatrice de cohorte de 2012 et l'indicatrice de sexe masculin.

\quad Les effets en mathématiques montrés dans ce tableau sont également significativement supérieurs aux effets en français.

\quad Les effets supérieurs chez les filles trouvés dans les modèles sans prise en compte de l'âge relatif peuvent alors s'expliquer en partie par le fait que les garçons bénéficient plus d'être plus âgés en soi mais être parmi les plus âgés de la classe nuit plus à ces derniers qu'aux filles.

\begin{landscape}\begingroup\fontsize{8}{10}\selectfont

\begin{ThreePartTable}
\begin{TableNotes}
\item \textit{Sources :} Fichiers CM2 (2009 à 2012), calculs de l'auteur.
\item \textit{Notes :} Une colonne correspond à une régression. Les notes sont normalisées sur l'année scolaire. Écart-types entre parenthèses. Ils sont calculés par wild bootstrap avec 1001 réplications. Les contrôles utilisés sont le sexe, la CSP et l'année scolaire.
\item ABS : modèle qui tente de mesurer séparément l'effet de l'âge absolu. REL : modèle qui tente de mesurer séparément l'effet de l'âge relatif. ABSREL : croisement des spécifications ABS et REL. CSP : Catégorie Socio-Professionnelle.
\item Significativité : 10\% * 5\% ** 1\% ***.
\end{TableNotes}
\begin{longtable}[t]{llllllllll}
\caption{\label{tab:agemodelsrelsexe}Estimations des effets séparés de l'âge absolu et de l'âge relatif, hétérogènes selon le sexe}\\
\toprule
\multicolumn{1}{c}{} & \multicolumn{9}{c}{Variable dépendante : Note} \\
\cmidrule(l{3pt}r{3pt}){2-10}
\multicolumn{1}{c}{} & \multicolumn{3}{c}{Totale} & \multicolumn{3}{c}{En français} & \multicolumn{3}{c}{En mathématiques} \\
\cmidrule(l{3pt}r{3pt}){2-4} \cmidrule(l{3pt}r{3pt}){5-7} \cmidrule(l{3pt}r{3pt}){8-10}
 & \makecell{ABS \\ (1) } & \makecell{REL \\ (2) } & \makecell{ABSREL \\ (3) } & \makecell{ABS \\ (4) } & \makecell{REL \\ (5) } & \makecell{ABSREL \\ (6) } & \makecell{ABS \\ (7) } & \makecell{REL \\ (8) } & \makecell{ABSREL \\ (9) }\\
\midrule
\endfirsthead
\caption[]{\label{tab:agemodelsrelsexe}Estimations des effets séparés de l'âge absolu et de l'âge relatif, hétérogènes selon le sexe (suite)}\\
\toprule
\multicolumn{1}{c}{} & \multicolumn{9}{c}{Variable dépendante : Note} \\
\cmidrule(l{3pt}r{3pt}){2-10}
\multicolumn{1}{c}{} & \multicolumn{3}{c}{Totale} & \multicolumn{3}{c}{En français} & \multicolumn{3}{c}{En mathématiques} \\
\cmidrule(l{3pt}r{3pt}){2-4} \cmidrule(l{3pt}r{3pt}){5-7} \cmidrule(l{3pt}r{3pt}){8-10}
 & \makecell{ABS \\ (1) } & \makecell{REL \\ (2) } & \makecell{ABSREL \\ (3) } & \makecell{ABS \\ (4) } & \makecell{REL \\ (5) } & \makecell{ABSREL \\ (6) } & \makecell{ABS \\ (7) } & \makecell{REL \\ (8) } & \makecell{ABSREL \\ (9) }\\
\midrule
\endhead

\endfoot
\bottomrule
\insertTableNotes
\endlastfoot
Âge aux examens & 0.32$^{***}$ & 0.507$^{***}$ & 0.532$^{***}$ & 0.278$^{***}$ & 0.371$^{***}$ & 0.393$^{***}$ & 0.339$^{***}$ & 0.643$^{***}$ & 0.67$^{***}$\\
 & (0.046) & (0.106) & (0.115) & (0.046) & (0.105) & (0.114) & (0.048) & (0.108) & (0.117)\\
Âge aux examens $\times$ Sexe - Garçon & 0.058 & 0.295$^{**}$ & 0.315$^{**}$ & 0.078 & 0.281$^{*}$ & 0.297$^{*}$ & 0.023 & 0.286$^{*}$ & 0.307$^{*}$\\
 & (0.07) & (0.15) & (0.157) & (0.069) & (0.148) & (0.156) & (0.073) & (0.151) & (0.158)\\
Âge relatif (classe) & - & $-$0.184$^{*}$ & $-$0.176 & - & $-$0.071 & $-$0.065 & - & $-$0.325$^{***}$ & $-$0.316$^{***}$\\
 & - & (0.105) & (0.107) & - & (0.103) & (0.106) & - & (0.106) & (0.109)\\
Âge relatif (classe) $\times$ Sexe - Garçon & - & $-$0.287$^{*}$ & $-$0.301$^{*}$ & - & $-$0.278$^{*}$ & $-$0.288$^{*}$ & - & $-$0.271$^{*}$ & $-$0.287$^{*}$\\
 & - & (0.147) & (0.154) & - & (0.146) & (0.152) & - & (0.148) & (0.155)\\
Âge aux examens $\times$ Cohorte - 2010 & 0.126$^{**}$ & - & $-$0.027 & 0.139$^{**}$ & - & $-$0.027 & 0.097 & - & $-$0.022\\
 & (0.062) & - & (0.056) & (0.062) & - & (0.055) & (0.066) & - & (0.056)\\
Âge aux examens $\times$ Cohorte - 2011 & $-$0.006 & - & $-$0.052 & 0.024 & - & $-$0.047 & $-$0.046 & - & $-$0.048\\
 & (0.061) & - & (0.055) & (0.062) & - & (0.054) & (0.062) & - & (0.055)\\
Âge aux examens $\times$ Cohorte - 2012 & 0.001 & - & $-$0.052 & 0.011 & - & $-$0.035 & $-$0.01 & - & $-$0.066\\
 & (0.064) & - & (0.056) & (0.063) & - & (0.055) & (0.066) & - & (0.056)\\
Âge aux examens $\times$ Cohorte - 2010 $\times$ Sexe - Garçon & $-$0.191$^{**}$ & - & $-$0.003 & $-$0.219$^{**}$ & - & $-$0.003 & $-$0.127 & - & $-$0.001\\
 & (0.093) & - & (0.003) & (0.092) & - & (0.003) & (0.098) & - & (0.003)\\
Âge aux examens $\times$ Cohorte - 2011 $\times$ Sexe - Garçon & $-$0.151$^{*}$ & - & $-$0.002 & $-$0.183$^{**}$ & - & $-$0.003 & $-$0.089 & - & $-$0.001\\
 & (0.092) & - & (0.003) & (0.091) & - & (0.003) & (0.094) & - & (0.003)\\
Âge aux examens $\times$ Cohorte - 2012 $\times$ Sexe - Garçon & $-$0.061 & - & $-$0.012$^{***}$ & $-$0.068 & - & $-$0.01$^{**}$ & $-$0.045 & - & $-$0.013$^{***}$\\
 & (0.094) & - & (0.005) & (0.093) & - & (0.004) & (0.097) & - & (0.005)\\
 &  &  &  &  &  &  &  &  & \\
Contrôles & Oui & Oui & Oui & Oui & Oui & Oui & Oui & Oui & Oui\\
Observations & 54341 & 54319 & 54319 & 54341 & 54319 & 54319 & 54341 & 54319 & 54319\\
R$^2$ ajusté & 0.223 & - & - & 0.24 & - & - & 0.164 & - & -\\*
\end{longtable}
\end{ThreePartTable}
\endgroup{}
\end{landscape}

\quad En analysant les éventuelles hétérogénéités des effets de l'âge absolu et de l'âge relatif en fonction de la catégorie sociale (Tableau \ref{tab:agemodelsrelpcs}), nous observons dans le modèle ABS (colonnes 1, 4 et 7) que passer les examens 4 mois plus tard entraîne une diminution significative de la note pour les élèves défavorisés.

\quad Le modèle REL (colonnes 2, 5 et 8) montre d'une part que l'effet de l'âge aux examens net de celui de l'âge relatif est plus le plus fort chez les enfants défavorisés au vu de l'ampleur des coefficients négatifs (bien que non précisément estimés pour les catégories plus élevées) devant l' interaction entre l'âge aux examens et les indicatrices de catégories sociales. Il est également nettement supérieur en mathématiques (0.76 UET contre 0.42 UET en français).

\quad L'effet de l'âge relatif est, quant à lui, non significatif et de plus faible ampleur en français et très important en mathématiques. Le coefficient correspondant n'est significatif que pour les défavorisés. Cela suggère que les défavorisés sont les plus sensibles à d'éventuels effets négatifs de la présence de camarades plus jeunes (via des perturbations ou de la monopolisation de l'attention de l'enseignant). Cette interprétation est renforcée par les valeurs positives des coefficients correspondant aux interactions entre l'âge relatif et la catégorie sociale, bien que ces derniers soient estimés de manière très imprécise. Ces valeurs positives signifient en effet de moindres désavantages pour les classes sociales plus élevées.

\quad Ces constats permettent en partie d'expliquer les effets d'âge supérieurs chez les élèves très favorisés trouvés précédemment puisque les avantages qu'obtiennent les défavorisés à être plus âgés \emph{per se} sont contrecarrés par les désavantages à être les plus âgés de la classe. Or, ces désavantages sont limités chez les élèves très favorisés.

\begin{landscape}\begingroup\fontsize{8}{10}\selectfont

\begin{ThreePartTable}
\begin{TableNotes}
\item \textit{Sources :} Fichiers CM2 (2009 à 2012), calculs de l'auteur.
\item \textit{Notes :} Une colonne correspond à une régression. Les notes sont normalisées sur l'année scolaire. Écart-types entre parenthèses. Ils sont calculés par wild bootstrap avec 1001 réplications. Les contrôles utilisés sont le sexe, la CSP et l'année scolaire.
\item ABS : modèle qui tente de mesurer séparément l'effet de l'âge absolu. REL : modèle qui tente de mesurer séparément l'effet de l'âge relatif. ABSREL : croisement des spécifications ABS et REL. CSP : Catégorie Socio-Professionnelle.
\item Significativité : 10\% * 5\% ** 1\% ***.
\end{TableNotes}
\begin{longtable}[t]{llllllllll}
\caption{\label{tab:agemodelsrelpcs}Estimations des effets séparés de l'âge absolu et de l'âge relatif, hétérogènes selon la catégorie sociale}\\
\toprule
\multicolumn{1}{c}{} & \multicolumn{9}{c}{Variable dépendante : Note} \\
\cmidrule(l{3pt}r{3pt}){2-10}
\multicolumn{1}{c}{} & \multicolumn{3}{c}{Totale} & \multicolumn{3}{c}{En français} & \multicolumn{3}{c}{En mathématiques} \\
\cmidrule(l{3pt}r{3pt}){2-4} \cmidrule(l{3pt}r{3pt}){5-7} \cmidrule(l{3pt}r{3pt}){8-10}
 & \makecell{ABS \\ (1) } & \makecell{REL \\ (2) } & \makecell{ABSREL \\ (3) } & \makecell{ABS \\ (4) } & \makecell{REL \\ (5) } & \makecell{ABSREL \\ (6) } & \makecell{ABS \\ (7) } & \makecell{REL \\ (8) } & \makecell{ABSREL \\ (9) }\\
\midrule
\endfirsthead
\caption[]{\label{tab:agemodelsrelpcs}Estimations des effets séparés de l'âge absolu et de l'âge relatif, hétérogènes selon la catégorie sociale (suite)}\\
\toprule
\multicolumn{1}{c}{} & \multicolumn{9}{c}{Variable dépendante : Note} \\
\cmidrule(l{3pt}r{3pt}){2-10}
\multicolumn{1}{c}{} & \multicolumn{3}{c}{Totale} & \multicolumn{3}{c}{En français} & \multicolumn{3}{c}{En mathématiques} \\
\cmidrule(l{3pt}r{3pt}){2-4} \cmidrule(l{3pt}r{3pt}){5-7} \cmidrule(l{3pt}r{3pt}){8-10}
 & \makecell{ABS \\ (1) } & \makecell{REL \\ (2) } & \makecell{ABSREL \\ (3) } & \makecell{ABS \\ (4) } & \makecell{REL \\ (5) } & \makecell{ABSREL \\ (6) } & \makecell{ABS \\ (7) } & \makecell{REL \\ (8) } & \makecell{ABSREL \\ (9) }\\
\midrule
\endhead

\endfoot
\bottomrule
\insertTableNotes
\endlastfoot
Âge aux examens & 0.338$^{***}$ & 0.582$^{***}$ & 0.577$^{***}$ & 0.312$^{***}$ & 0.419$^{***}$ & 0.408$^{***}$ & 0.337$^{***}$ & 0.757$^{***}$ & 0.76$^{***}$\\
 & (0.046) & (0.121) & (0.129) & (0.046) & (0.121) & (0.129) & (0.048) & (0.123) & (0.131)\\
Âge aux examens $\times$ CSP - Moyenne & 0.066 & $-$0.143 & $-$0.124 & 0.038 & $-$0.164 & $-$0.148 & 0.099 & $-$0.1 & $-$0.077\\
 & (0.087) & (0.21) & (0.22) & (0.084) & (0.208) & (0.218) & (0.092) & (0.215) & (0.225)\\
Âge aux examens $\times$ CSP - Favorisée & $-$0.089 & $-$0.45 & $-$0.439 & $-$0.136 & $-$0.408 & $-$0.408 & $-$0.007 & $-$0.464 & $-$0.436\\
 & (0.146) & (0.345) & (0.371) & (0.138) & (0.338) & (0.364) & (0.161) & (0.357) & (0.385)\\
Âge aux examens $\times$ CSP - Très favorisée & 0.216$^{**}$ & $-$0.1 & $-$0.066 & 0.18$^{*}$ & $-$0.034 & $-$0.008 & 0.244$^{**}$ & $-$0.188 & $-$0.144\\
 & (0.106) & (0.268) & (0.295) & (0.102) & (0.261) & (0.287) & (0.118) & (0.287) & (0.316)\\
Âge aux examens $\times$ CSP - Autre & 0.339 & $-$0.159 & $-$0.227 & 0.36 & 0.065 & 0.046 & 0.265 & $-$0.491 & $-$0.622\\
 & (0.329) & (0.966) & (1.139) & (0.312) & (0.962) & (1.132) & (0.373) & (0.981) & (1.162)\\
Âge aux examens $\times$ CSP - Manquante & 0.055 & 0.339$^{*}$ & 0.233 & 0.077 & 0.382$^{**}$ & 0.324 & 0.016 & 0.226 & 0.068\\
 & (0.099) & (0.186) & (0.3) & (0.1) & (0.185) & (0.299) & (0.097) & (0.185) & (0.291)\\
Âge relatif (classe) & - & $-$0.315$^{***}$ & $-$0.296$^{**}$ & - & $-$0.176 & $-$0.16 & - & $-$0.486$^{***}$ & $-$0.466$^{***}$\\
 & - & (0.119) & (0.122) & - & (0.119) & (0.122) & - & (0.121) & (0.124)\\
Âge relatif (classe) $\times$ CSP - Moyenne & - & 0.239 & 0.215 & - & 0.258 & 0.236 & - & 0.188 & 0.161\\
 & - & (0.205) & (0.212) & - & (0.203) & (0.211) & - & (0.21) & (0.217)\\
Âge relatif (classe) $\times$ CSP - Favorisée & - & 0.385 & 0.362 & - & 0.341 & 0.328 & - & 0.411 & 0.374\\
 & - & (0.349) & (0.367) & - & (0.342) & (0.36) & - & (0.363) & (0.382)\\
Âge relatif (classe) $\times$ CSP - Très favorisée & - & 0.244 & 0.208 & - & 0.165 & 0.137 & - & 0.334 & 0.289\\
 & - & (0.265) & (0.289) & - & (0.257) & (0.28) & - & (0.284) & (0.31)\\
Âge relatif (classe) $\times$ CSP - Autre & - & 0.28 & 0.351 & - & 0.155 & 0.187 & - & 0.449 & 0.571\\
 & - & (0.94) & (1.095) & - & (0.928) & (1.078) & - & (0.964) & (1.127)\\
Âge relatif (classe) $\times$ CSP - Manquante & - & $-$0.234 & $-$0.105 & - & $-$0.279 & $-$0.188 & - & $-$0.131 & 0.033\\
 & - & (0.182) & (0.288) & - & (0.181) & (0.287) & - & (0.181) & (0.279)\\
Âge aux examens $\times$ Cohorte - 2011 & $-$0.099 & - & $-$0.02 & $-$0.1 & - & $-$0.014 & $-$0.089 & - & $-$0.023\\
 & (0.061) & - & (0.051) & (0.061) & - & (0.05) & (0.064) & - & (0.051)\\
Âge aux examens $\times$ Cohorte - 2012 & $-$0.132$^{**}$ & - & $-$0.021 & $-$0.128$^{**}$ & - & $-$0.002 & $-$0.121$^{*}$ & - & $-$0.045\\
 & (0.062) & - & (0.052) & (0.063) & - & (0.051) & (0.065) & - & (0.052)\\
Âge aux examens $\times$ Cohorte - 2011 $\times$ CSP - Moyenne & 0.061 & - & 0.008$^{**}$ & 0.092 & - & 0.008$^{***}$ & 0.008 & - & 0.007$^{**}$\\
 & (0.114) & - & (0.003) & (0.111) & - & (0.003) & (0.122) & - & (0.003)\\
Âge aux examens $\times$ Cohorte - 2012 $\times$ CSP - Moyenne & 0.165 & - & 0.008 & 0.211$^{*}$ & - & 0.01 & 0.079 & - & 0.003\\
 & (0.117) & - & (0.007) & (0.115) & - & (0.007) & (0.124) & - & (0.007)\\
Âge aux examens $\times$ Cohorte - 2011 $\times$ CSP - Favorisée & 0.241 & - & 0.011$^{**}$ & 0.27 & - & 0.009$^{*}$ & 0.17 & - & 0.012$^{**}$\\
 & (0.187) & - & (0.005) & (0.179) & - & (0.005) & (0.206) & - & (0.006)\\
Âge aux examens $\times$ Cohorte - 2012 $\times$ CSP - Favorisée & 0.145 & - & 0.016 & 0.211 & - & 0.017 & 0.03 & - & 0.012\\
 & (0.188) & - & (0.012) & (0.18) & - & (0.012) & (0.206) & - & (0.012)\\
Âge aux examens $\times$ Cohorte - 2011 $\times$ CSP - Très favorisée & $-$0.139 & - & 0.012$^{***}$ & $-$0.114 & - & 0.011$^{***}$ & $-$0.159 & - & 0.013$^{***}$\\
 & (0.143) & - & (0.004) & (0.138) & - & (0.004) & (0.158) & - & (0.005)\\
Âge aux examens $\times$ Cohorte - 2012 $\times$ CSP - Très favorisée & 0.082 & - & 0.005 & 0.113 & - & 0.006 & 0.027 & - & 0.002\\
 & (0.145) & - & (0.01) & (0.14) & - & (0.009) & (0.162) & - & (0.01)\\
Âge aux examens $\times$ Cohorte - 2011 $\times$ CSP - Autre & 0.248 & - & $-$0.013 & 0.3 & - & $-$0.015 & 0.157 & - & $-$0.009\\
 & (0.586) & - & (0.015) & (0.591) & - & (0.015) & (0.587) & - & (0.015)\\
Âge aux examens $\times$ Cohorte - 2012 $\times$ CSP - Autre & $-$0.037 & - & $-$0.001 & 0.232 & - & $-$0.01 & $-$0.433 & - & 0.013\\
 & (0.616) & - & (0.036) & (0.639) & - & (0.036) & (0.6) & - & (0.037)\\
Âge aux examens $\times$ Cohorte - 2011 $\times$ CSP - Manquante & 0.013 & - & 0.014$^{***}$ & $-$0.019 & - & 0.014$^{***}$ & 0.059 & - & 0.013$^{***}$\\
 & (0.134) & - & (0.004) & (0.136) & - & (0.004) & (0.13) & - & (0.004)\\
Âge aux examens $\times$ Cohorte - 2012 $\times$ CSP - Manquante & 0.035 & - & $-$0.003 & 0.047 & - & $-$0.004 & 0.015 & - & $-$0.001\\
 & (0.145) & - & (0.009) & (0.145) & - & (0.009) & (0.145) & - & (0.009)\\
 &  &  &  &  &  &  &  &  & \\
Contrôles & Oui & Oui & Oui & Oui & Oui & Oui & Oui & Oui & Oui\\
Observations & 42030 & 54319 & 42015 & 42030 & 54319 & 42015 & 42030 & 54319 & 42015\\
R$^2$ ajusté & 0.241 & - & - & 0.257 & - & - & 0.179 & - & -\\*
\end{longtable}
\end{ThreePartTable}
\endgroup{}
\end{landscape}

\quad Sur les résultats du DNB (Tableau \ref{tab:agemodelsmtrel}), les effets de l'âge aux examens et de l'âge relatif sont qualitativement les mêmes mais de plus grande ampleur par rapport aux effets correspondant trouvés pour le CM2 puisque les coefficients estimés associés à l'âge aux examens se trouvent entre 0.8 et 0.9 UET et ceux associés à l'âge relatif entre 0.6 et 0.8 UET. Il n'y a cette fois pas de différence entre les effets estimés en français et en mathématiques\footnote{Probabilité critique de 0.67 issue d'un test d'égalité des effets de l'âge absolu en français et en mathématiques. Cette probabilité est de 0.44 en ce qui concerne les effets de l'âge relatif.}.

\quad Dans la même logique qu'au CM2, les ampleurs des coefficients du Tableau \ref{tab:agemodelsmtrel} qui apparaissent très différents des résultats à moyen terme qui ne tiennent pas compte des effets éventuels de l'âge relatif ne représentent pas une incohérence puisque les effets mesurés auparavant représentent mécaniquement la somme de l'effet de l'âge aux examens en soi et de l'effet de l'âge relatif.

\quad Quelques hypothèses permettent d'expliquer ces différences d'ampleur entre les effets estimés au CM2 et ceux estimés au DNB. D'une part, l'effet de l'âge aux examens \emph{per se} qui est supérieur est probablement dû à la pratique du redoublement entre le CM2 (y compris) et la 3\textsuperscript{ème} (y compris)\footnote{Les arguments fournis par Grenet (\protect\hyperlink{ref-GRE:09}{2009}) qui tentent d'expliquer pourquoi des effets de l'âge restent visibles à moyen terme en France vont dans ce sens.}. En effet, redoubler une fois pendant cette période équivaut à avoir un an de plus aux examens du DNB. On peut alors supposer que le redoublement entre la fin du primaire et le collège a un effet positif à La Réunion.\\
D'autre part, l'effet de l'âge relatif peut être plus prononcé à la fin du collège puisque des élèves plus âgés sont plus aptes à s'influencer directement\footnote{On peut penser qu'ils ont la possibilité de mieux s'exprimer, vu qu'ils sont plus âgés, par exemple.}. Une autre serait que les tailles de classes légèrement plus élevées au collège impliquent plus de difficulté pour un enseignant de gérer ses élèves, augmentant ainsi en moyenne les perturbations des plus jeunes ou le détournement de l'attention de l'enseignant en faveur des plus jeunes de la classe (\protect\hyperlink{ref-GIV:20}{Givord, 2020, p. 42}).

\newpage
\begingroup\fontsize{8}{10}\selectfont

\begin{ThreePartTable}
\begin{TableNotes}
\item \textit{Sources :} Fichiers CM2 (2009 à 2012), calculs de l'auteur.
\item \textit{Notes :} Une colonne correspond à une régression. Les notes sont normalisées sur l'année scolaire. Écart-types entre parenthèses. Ils sont calculés par wild bootstrap avec 1001 réplications. Les contrôles utilisés sont le sexe, la CSP, le régime scolaire et l'année scolaire.
\item CSP : Catégorie Socio-Professionnelle.
\item Significativité : 10\% * 5\% ** 1\% ***.
\end{TableNotes}
\begin{longtable}[t]{llll}
\caption{\label{tab:agemodelsmtrel}Estimations des effets séparés à moyen terme de l'âge relatif}\\
\toprule
\multicolumn{1}{c}{} & \multicolumn{3}{c}{Variable dépendante :} \\
\cmidrule(l{3pt}r{3pt}){2-4}
 & \makecell{\makecell[l]{Note totale (écrits) \\ \ } \\ (1) } & \makecell{\makecell[l]{Note en \\ français (écrits)} \\ (2) } & \makecell{\makecell[l]{Note en \\ mathématiques (écrits)} \\ (3) }\\
\midrule
\endfirsthead
\caption[]{\label{tab:agemodelsmtrel}Estimations des effets séparés à moyen terme de l'âge relatif (suite)}\\
\toprule
\multicolumn{1}{c}{} & \multicolumn{3}{c}{Variable dépendante :} \\
\cmidrule(l{3pt}r{3pt}){2-4}
 & \makecell{\makecell[l]{Note totale (écrits) \\ \ } \\ (1) } & \makecell{\makecell[l]{Note en \\ français (écrits)} \\ (2) } & \makecell{\makecell[l]{Note en \\ mathématiques (écrits)} \\ (3) }\\
\midrule
\endhead

\endfoot
\bottomrule
\insertTableNotes
\endlastfoot
Âge aux examens & 0.971$^{***}$ & 0.836$^{***}$ & 0.892$^{***}$\\
 & (0.093) & (0.092) & (0.095)\\
Âge relatif (classe) & $-$0.794$^{***}$ & $-$0.661$^{***}$ & $-$0.76$^{***}$\\
 & (0.091) & (0.09) & (0.093)\\
 &  &  & \\
Contrôles & Oui & Oui & Oui\\
Observations & 41567 & 41497 & 41343\\*
\end{longtable}
\end{ThreePartTable}
\endgroup{}

\quad Les effets de l'âge relatif ne semblent pas différents en fonction du sexe, comme le démontre le Tableau \ref{tab:agemodelsmtrelsexemod}. Cette observation est cohérente avec les constats précédents indiquant une absence d'effets hétérogènes de l'âge aux examens à moyen terme.

\begingroup\fontsize{8}{10}\selectfont

\begin{ThreePartTable}
\begin{TableNotes}
\item \textit{Sources :} Fichiers CM2 (2009 à 2012), calculs de l'auteur.
\item \textit{Notes :} Une colonne correspond à une régression. Les notes sont normalisées sur l'année scolaire. Écart-types entre parenthèses. Ils sont calculés par wild bootstrap avec 1001 réplications. Les contrôles utilisés sont le sexe, la CSP, le régime scolaire et l'année scolaire.
\item CSP : Catégorie Socio-Professionnelle.
\item Significativité : 10\% * 5\% ** 1\% ***.
\end{TableNotes}
\begin{longtable}[t]{llll}
\caption{\label{tab:agemodelsmtrelsexemod}Estimations des effets séparés à moyen terme de l'âge relatif, hétérogènes selon le sexe}\\
\toprule
\multicolumn{1}{c}{} & \multicolumn{3}{c}{Variable dépendante :} \\
\cmidrule(l{3pt}r{3pt}){2-4}
 & \makecell{\makecell[l]{Note totale (écrits) \\ \ } \\ (1) } & \makecell{\makecell[l]{Note en \\ français (écrits)} \\ (2) } & \makecell{\makecell[l]{Note en \\ mathématiques (écrits)} \\ (3) }\\
\midrule
\endfirsthead
\caption[]{\label{tab:agemodelsmtrelsexemod}Estimations des effets séparés à moyen terme de l'âge relatif, hétérogènes selon le sexe (suite)}\\
\toprule
\multicolumn{1}{c}{} & \multicolumn{3}{c}{Variable dépendante :} \\
\cmidrule(l{3pt}r{3pt}){2-4}
 & \makecell{\makecell[l]{Note totale (écrits) \\ \ } \\ (1) } & \makecell{\makecell[l]{Note en \\ français (écrits)} \\ (2) } & \makecell{\makecell[l]{Note en \\ mathématiques (écrits)} \\ (3) }\\
\midrule
\endhead

\endfoot
\bottomrule
\insertTableNotes
\endlastfoot
Âge aux examens & 0.947$^{***}$ & 0.896$^{***}$ & 0.866$^{***}$\\
 & (0.128) & (0.131) & (0.129)\\
Âge aux examens $\times$ Sexe - Garçon & 0.046 & $-$0.118 & 0.052\\
 & (0.183) & (0.182) & (0.187)\\
Âge relatif (classe) & $-$0.774$^{***}$ & $-$0.722$^{***}$ & $-$0.735$^{***}$\\
 & (0.126) & (0.128) & (0.127)\\
Âge relatif (classe) $\times$ Sexe - Garçon & $-$0.04 & 0.12 & $-$0.049\\
 & (0.18) & (0.179) & (0.185)\\
 &  &  & \\
Contrôles & Oui & Oui & Oui\\
Observations & 41567 & 41497 & 41343\\*
\end{longtable}
\end{ThreePartTable}
\endgroup{}

\quad De la même manière, selon le Tableau \ref{tab:agemodelsmtrelpcsregmod}, il est il n'y a pas d'hétérogénéité des effets de l'âge relatif en fonction de la catégorie sociale puisque les coefficients devant les interactions entre l'âge relatif et les indicatrices de catégorie sociale sont tous non significatifs.

\begingroup\fontsize{8}{10}\selectfont

\begin{ThreePartTable}
\begin{TableNotes}
\item \textit{Sources :} Fichiers CM2 (2009 à 2012), calculs de l'auteur.
\item \textit{Notes :} Une colonne correspond à une régression. Les notes sont normalisées sur l'année scolaire. Écart-types entre parenthèses. Ils sont calculés par wild bootstrap avec 1001 réplications. Les contrôles utilisés sont le sexe, la CSP, le régime scolaire et l'année scolaire.
\item CSP : Catégorie Socio-Professionnelle.
\item Significativité : 10\% * 5\% ** 1\% ***.
\end{TableNotes}
\begin{longtable}[t]{llll}
\caption{\label{tab:agemodelsmtrelpcsregmod}Estimations des effets séparés à moyen terme de l'âge relatif, hétérogènes selon la catégorie sociale}\\
\toprule
\multicolumn{1}{c}{} & \multicolumn{3}{c}{Variable dépendante :} \\
\cmidrule(l{3pt}r{3pt}){2-4}
 & \makecell{\makecell[l]{Note totale (écrits) \\ \ } \\ (1) } & \makecell{\makecell[l]{Note en \\ français (écrits)} \\ (2) } & \makecell{\makecell[l]{Note en \\ mathématiques (écrits)} \\ (3) }\\
\midrule
\endfirsthead
\caption[]{\label{tab:agemodelsmtrelpcsregmod}Estimations des effets séparés à moyen terme de l'âge relatif, hétérogènes selon la catégorie sociale (suite)}\\
\toprule
\multicolumn{1}{c}{} & \multicolumn{3}{c}{Variable dépendante :} \\
\cmidrule(l{3pt}r{3pt}){2-4}
 & \makecell{\makecell[l]{Note totale (écrits) \\ \ } \\ (1) } & \makecell{\makecell[l]{Note en \\ français (écrits)} \\ (2) } & \makecell{\makecell[l]{Note en \\ mathématiques (écrits)} \\ (3) }\\
\midrule
\endhead

\endfoot
\bottomrule
\insertTableNotes
\endlastfoot
Âge aux examens & 1.057$^{***}$ & 0.918$^{***}$ & 0.925$^{***}$\\
 & (0.12) & (0.122) & (0.119)\\
Âge aux examens $\times$ CSP - Moyenne & $-$0.249 & $-$0.213 & $-$0.142\\
 & (0.232) & (0.229) & (0.238)\\
Âge aux examens $\times$ CSP - Favorisée & $-$0.271 & $-$0.071 & $-$0.283\\
 & (0.349) & (0.342) & (0.359)\\
Âge aux examens $\times$ CSP - Très favorisée & $-$0.042 & $-$0.229 & 0.252\\
 & (0.306) & (0.281) & (0.341)\\
Âge aux examens $\times$ CSP - Autres & $-$0.289 & $-$0.306 & $-$0.273\\
 & (1.015) & (0.994) & (1.133)\\
Âge relatif (classe) & $-$0.901$^{***}$ & $-$0.752$^{***}$ & $-$0.808$^{***}$\\
 & (0.117) & (0.12) & (0.117)\\
Âge relatif (classe) $\times$ CSP - Moyenne & 0.312 & 0.259 & 0.17\\
 & (0.228) & (0.225) & (0.234)\\
Âge relatif (classe) $\times$ CSP - Favorisée & 0.2 & 0.007 & 0.239\\
 & (0.347) & (0.339) & (0.356)\\
Âge relatif (classe) $\times$ CSP - Très favorisée & 0.159 & 0.271 & $-$0.13\\
 & (0.297) & (0.273) & (0.331)\\
Âge relatif (classe) $\times$ CSP - Autres & $-$0.08 & $-$0.035 & 0.04\\
 & (1.004) & (0.99) & (1.124)\\
 &  &  & \\
Contrôles & Oui & Oui & Oui\\
Observations & 41567 & 41497 & 41343\\*
\end{longtable}
\end{ThreePartTable}
\endgroup{}

\hypertarget{ageconcl}{%
\section{Conclusion}\label{ageconcl}}

L'étude présente utilise des données exhaustives sur La Réunion pour mesurer l'effet d'être âgé un an de plus aux épreuves nationales de CM2 et de DNB en utilisant l'approche par fonction de contrôle. L'avantage de cette approche est qu'elle est basée sur l'utilisation de la date de naissance dans l'année comme instrument de l'âge réel mais ne nécessite pas l'hypothèse de monotonie pour identifier l'effet moyen du traitement (\protect\hyperlink{ref-HAM:KOL:12}{Hámori \& Köllő, 2012} ; \protect\hyperlink{ref-WOO:15}{Wooldridge, 2015}). Nous mobilisons également des modèles de régressions sur une discontinuité. Nous analysons l'hétérogénéité des effets de l'âge en fonction des caractéristiques observables et nous tentons de séparer l'effet de l'âge absolu en soi de l'effet de l'âge relatif.

\quad En fin de CM2, nous trouvons des effets d'ampleur comparables aux résultats internationaux. L'effet confondu de l'âge d'entrée à l'école, de la position de l'âge dans la composition de la classe ainsi que de la maturité est positif, significatif et d'ampleur considérable à bas âge. Nous trouvons un effet plus fort chez les filles et chez les enfants issus de catégorie sociale très favorisée. Certains de ces effets hétérogènes n'auraient pas pu être mis en évidence avec de simples estimations par variables instrumentales. À bas âge, les effets ne semblent pas différer selon la matière (français ou mathématiques).\\
Avec ces seuls résultats, bien qu'une inégalité liée à l'âge à bas âge soit confirmée, il est difficile de fournir des préconisations précises (\protect\hyperlink{ref-DHU:eal:17}{Dhuey et al., 2017}). De telles préconisations requièrent en effet d'identifier le rôle distinct des trois types d'effet de l'âge.\\
La recommandation principale que nous pouvons faire est qu'il est nécessaire de sensibiliser les enseignants sur les effets de l'âge. L'enseignant devra alors prendre en compte l'âge de ses élèves dans ses appréciations : plutôt que de comparer un enfant à des camarades plus âgés lors de commentaires envers l'enfant ou les parents, l'enseignant devra mettre en évidence soit la progression de l'enfant soit ses performances par rapport à son âge (\protect\hyperlink{ref-CRA:eal:14}{Crawford et al., 2014}). Cette initiative permettrait de diminuer les risques de perte de confiance en soi de l'enfant dès son plus jeune âge (\protect\hyperlink{ref-GIV:20}{Givord, 2020}), qui influence ses performances futures et ses aspirations scolaires (dont découle son choix de spécialisation en classe de seconde, par exemple).\\
Les parents doivent également être sensibilisés de la même manière, lors des réunions collectives ou individuelles, par exemple.

\quad Pour éviter de pénaliser injustement les plus jeunes au CM2 à cause de leur âge, des politiques de normalisation des notes (aux examens officiels) en fonction du mois de naissance sont également envisageables. À notre connaissance, cela a été popularisé par Crawford et al. (\protect\hyperlink{ref-CRA:eal:07}{2007}) (Angleterre). Concrètement, pour un élève né à un mois de naissance donné (en décembre, par exemple), il s'agit de rajouter le montant de la pénalité uniquement dû à son âge aux examens par rapport à un élève né en janvier dans sa note officielle. Ce sont ces notes normalisées qui devraient alors être utilisées dans les décisions institutionnelles (décision de redoublement, placement les sections spéciales comme la SEGPA, discussion avec les parents). Bien sûr, ces notes ne reflètent pas le niveau de capital humain de l'enfant en soi mais ont le mérite d'éviter les mêmes effets cités plus hauts tels que la perte de confiance en soi très tôt dans l'éducation. Il faudrait arrêter d'utiliser des notes normalisées au moment où les effets de l'âge ne sont en effet plus visibles (sur les vraies notes). Cela suggère de repérer précisément ce moment (2\textsuperscript{nde}, première, ou terminale, par exemple). Cette question fera l'objet de futures recherches.\\
Une politique de notes normalisées en fonction du mois de naissance peut être mise en place en sensibilisant le personnel enseignant à l'intérêt de cette pratique, en indiquant systématiquement le mois de naissance des élèves dans les copies d'examen, et en systématisant la normalisation par tous les correcteurs.

Afin de mieux comprendre ces problématiques, les futures recherches devraient tenter de mesurer précisément les effets de l'âge sur des mesures non cognitives de confiance en soi, ambitions, santé mentale, etc..

\quad Bien que d'autres études sur la France (ou mieux, sur La Réunion) soient nécessaires pour les confirmer ou les infirmer, nos résultats impliquent que l'âge d'entrée (toujours confondu avec l'âge relatif d'entrée en primaire) prime sur la maturité au moment des examens, du moins jusqu'à la fin du primaire. De plus, être plus jeune dans la classe s'avère être un avantage, contrairement à l'intuition. Ces deux résultats nuancés permettent de formuler plus assurément qu'avancer la date seuil d'entrée plus tôt dans l'année peut être un moyen efficace de lutter contre l'inégalité à bas âge. En effet, une telle politique augmenterait non seulement l'âge d'entrée de chaque élève mais l'exposerait aussi à des camarades de classes relativement plus âgés (ce qui leur est avantageux). Par ailleurs, différencier les dates seuils d'entrée en fonction des régions de France avec des expérimentations permettrait aux futures recherches de mesurer l'effet de l'âge d'entrée net de l'effet d'âge aux examens et de fournir ainsi des recommandations plus précises.\\
Les avantages que tirent les plus jeunes de la classe des plus âgés indiquent de plus qu'il serait bénéfique pour la performance des élèves de limiter autant que possible les classes de groupes d'âge. Cette implication est valable que la cause derrière ces résultats soit l'ajustement du comportement de l'enseignant ou les effets de pairs.\\
Les résultats sur les effets de l'âge relatif indiquent également que la pratique du retard d'entrée est à considérer avec prudence puisque décaler l'inscription de son enfant équivaut à un enfant plus âgé à l'entrée (avantage, effets de l'âge \emph{per se}) mais aussi confrontés à des pairs plus jeunes (désavantage, effets de l'âge relatif).

\quad Nos résultats indiquent également que les effets diminuent mais persistent vraisemblablement jusqu'en fin de collège. Cela suppose que les 4 années de collège ne sont probablement pas suffisantes pour éliminer les différences de performances dues aux différences d'âge. Cela peut être problématique à plus long terme puisque c'est à partir du lycée que les parcours scolaires des élèves diffèrent (général et technologique ou professionnel, entre autres). Plus d'attention devrait alors être fournie aux plus jeunes de la classe tout le long du collège.\\
Les effets de l'âge aux examens nets de l'âge relatif semblent même avoir augmenté en fin de collège mais cette augmentation est compensée par les effets négatifs de l'âge relatif. Sur ces effets séparés et en fin de collège, des différences sont perçues selon la matière concernée : les effets en mathématiques sont significativement plus prononcés que ceux en français.

\quad Pour assurer la qualité de futures recherches, les tests à bas âge pourraient être uniformisés (en structures et en difficulté) à travers le temps.

\quad D'autres pistes de politiques publiques existent pour diminuer la pénalité de performance causée par l'âge, mais nous laissons le débat ouvert puisque nous notre travail ne contient pas d'éléments nous permettant d'en discuter. Ce sont les politiques liées à l'emploi du temps et à la taille des classes. Au primaire, il se pourrait en effet que les plus jeunes aient une limite de temps de concentration dont les plus âgés n'en souffrent pas. Des classes plus petites pourraient également permettre aux enseignants de mieux adapter leurs enseignements selon les différentes classes d'âge (\protect\hyperlink{ref-GIV:20}{Givord, 2020}).

\hypertarget{pe}{%
\chapter{Effets de pairs en éducation : applications sur données réunionnaises}\label{pe}}

\chaptermark{Effets de pairs en éducation à La Réunion}

\newpage

\hypertarget{peintro}{%
\section{Introduction}\label{peintro}}

En France, un élève de 3\textsuperscript{ème} passe 26 heures par semaine avec ses camarades de classe (ses pairs)\footnote{\url{https://www.education.gouv.fr/les-horaires-par-cycle-au-college-9884}.}. Ces interactions jouent potentiellement un rôle dans la détermination des résultats scolaires de cet élève via des mécanismes d'entraide ou de compétition, par exemple (\protect\hyperlink{ref-DAV:04}{Davezies, 2004}). L'effet de telles interactions est appelé effet de pairs.

\quad D'une part, l'étude de l'existence, de la forme et de l'ampleur des effets de pairs en éducation peut aider à faire avancer le débat sur les politiques de mixité scolaire et sociale en France. Par exemple, si la présence d'élèves de niveau scolaire initial élevé parmi ses camarades de classe améliore les résultats d'un élève de faible niveau scolaire sans que la présence de camarades faibles n'ait d'effets néfastes sur les élèves forts, les établissements ont intérêt à mixer les différents niveaux scolaires dans chaque classe pour espérer améliorer leurs résultats. Ce genre de politique possède de plus l'avantage d'avoir un coût de mise en place pratiquement nul.\\
D'autre part, l'anticipation des effets de pairs par les parents d'élèves peut contribuer à accentuer la ségrégation scolaire. Les parents peuvent avoir tendance à éviter d'inscrire leurs enfants dans les établissements composés d'une forte proportion d'élèves faibles à cause de la croyance que ces derniers influenceraient négativement les résultats de leurs enfants. Dans ce scénario, les élèves forts auront tendance à se regrouper dans les mêmes types d'établissements. Or, si en réalité, la présence des élèves faibles n'est pas néfaste pour les élèves forts et si la présence des élèves forts est bénéfique pour les élèves faibles, le regroupement des élèves forts entre eux constitue une forme d'injustice puisqu'il prive les élèves faibles de l'avantage de la présence de pairs plus forts (\protect\hyperlink{ref-BOU:MAI:18}{Boutchenik \& Maillard, 2018}).\\
Par ailleurs, les résultats scolaires étant corrélés avec la catégorie sociale, une ségrégation scolaire correspond à une certaine ségrégation sociale (\protect\hyperlink{ref-MON:eal:19}{Monso et al., 2019}).

\quad Dans Manski (\protect\hyperlink{ref-MAN:93}{1993}), un des papiers pionniers des études sur les effets de pairs, deux types d'effets de pairs sont définis : effets de pairs endogènes et exogènes. Les effets de pairs endogènes sont les effets des comportements des pairs sur le comportement de l'individu. En éducation, il s'agit par exemple des effets de la note des pairs à un examen donné sur la note individuelle à ce même examen. Pour comprendre cette définition, une interprétation possible est de voir la note des pairs comme le reflet de leurs comportements au cours de l'année scolaire (attention en cours, efforts, comportements de perturbations, entraides directes, etc.) où l'individu et ses pairs ont été placés dans la même classe. Les effets exogènes concernent les effets des caractéristiques prédéterminées, c'est-à-dire avant que les comportements susmentionnés n'aient eu lieu, sur le comportement. En guise d'exemple, ces effets peuvent être l'effet sur la note de l'élève d'une part importante de pairs issus de catégorie sociale favorisée ou ayant un niveau scolaire initial élevé.\\
Les effets corrélés de groupe constituent un troisième type d'effet à ne pas confondre avec les effets de pairs. Ces effets existent parce que l'individu et ses pairs sont exposés à des facteurs communs liés au fait qu'ils sont des camarades. Pour des pairs définis au niveau de la classe, ces effets corrélés sont par exemple la qualité matérielle de la classe ou encore la qualité des enseignants.

\quad Théoriquement, les effets de pairs peuvent prendre différentes formes\footnote{Proposées originalement par Hoxby \& Weingarth (\protect\hyperlink{ref-HOX:WEI:05}{2005}). Ces formes sont également présentées dans Sacerdote (\protect\hyperlink{ref-SAC:11}{2011}) (Tableau 4.1) ou Epple \& Romano (\protect\hyperlink{ref-EPP:ROM:11}{2011}).}. La plus simple est la forme linéaire en moyenne qui suppose que le comportement ou les caractéristiques des pairs n'impactent l'individu que via la moyenne et que l'effet est le même indépendamment du type de l'individu (on dit par abus de langage que l'effet est homogène). Cela suppose que tant que la moyenne du niveau des pairs est fixée, avoir des camarades très faibles et très forts, par exemple, ne fait aucune différence sur la note d'un élève par rapport au fait d'avoir des camarades uniquement de niveau moyen. Et ce propos est vérifié peu importe le niveau de l'élève considéré.
Le modèle dit de \emph{bad apple} suppose que la présence d'un élève faible/désavantagé est néfaste pour tout le monde. L'effet n'est plus linéaire en la moyenne mais reste toujours homogène. Cela peut être dû à des problèmes de discipline\footnote{Ou incitation des autres élèves au manque de discipline.} ou à une appropriation excessive par l'élève faible/désavantagé de l'attention de l'enseignant.\\
Son opposé, le modèle dit de \emph{shining light} suppose que la présence d'un bon élève est bénéfique pour tout le monde. Contrairement au modèle de \emph{bad apple}, il est ardu d'imaginer quelles en sont les raisons.\\
Les modèles dits de \emph{focus} et de \emph{rainbow} constituent les derniers modèles d'effets homogènes. Dans le modèle de \emph{focus}, l'homogénéité des comportements/caractéristiques (indépendamment de la moyenne) est avantageux pour tout type d'élève\footnote{Typiquement, si ce modèle est vrai pour la France, les classes de niveau, qu'elles regroupent uniquement les meilleurs entre eux ou les plus faibles entre eux (voir Section \ref{peinst}) ont un effet bénéfique sur les résultats.}. Cela peut s'expliquer par exemple par le fait que les enseignants peuvent facilement adapter leurs enseignements aux besoins de tous si la classe est homogène. Une autre explication est l'homophilie : c'est mieux, toutes choses égales par ailleurs, d'être entouré par ses semblables.
Son opposé, le modèle de \emph{rainbow}, suppose que l'hétérogénéité des comportements/caractéristiques est avantageux pour tout type d'individu\footnote{Dans ce modèle, tout type de classes de niveau est néfaste pour les notes.}.\\
Un premier modèle d'effets de pairs hétérogènes qui peut être confondu avec le modèle de \emph{focus} est celui de \emph{boutique}. Dans ce modèle, même au sein d'un groupe homogène, le bénéfice augmente avec la similarité entre les pairs et l'élève.
Le deuxième modèle où les effets sont hétérogènes est le modèle dit de \emph{single crossing} dans lequel les pairs les plus forts sont bénéfiques pour tous mais ce bénéfice augmente avec le niveau de l'élève.\\
Le modèle de \emph{indivious comparison} constitue le dernier modèle d'effets hétérogènes. Il se traduit par un effet négatif de la présence de pairs ayant un niveau relativement supérieur. Cette forme peut se présenter si les élèves se découragent par la présence de plus forts qu'eux ou encore si l'enseignant calibre le niveau de ses cours à celui des plus forts de la classe\footnote{Autrement dit, s'il y a des élèves moyens et des élèves forts dans la classe, l'enseignant choisit de faire des cours de haut niveau ; et s'il y a des élèves moyens et des élèves faibles, l'enseignant choisit de faire des cours de niveau moyen.}.
Il est important de noter que ces modèles ne sont pas mutuellement exclusifs (\protect\hyperlink{ref-BUR:SAS:13}{Burke \& Sass, 2013}, par exemple).

\quad L'identification empirique des effets de pairs en éducation est rendue difficile du fait d'au moins trois problèmes : le problème de la réflexion, la formation endogène des groupes des pairs et la prise en compte des effets corrélés. Le problème de la réflexion est dû à la simultanéité de la variable dépendante (une variable de résultats scolaire, par exemple) et la variable explicative d'intérêt (cette même variable calculée chez les pairs). Dans le cas des camarades de classe, la simultanéité pose problème dans la mesure où au cours de l'année scolaire, le comportement de l'individu est influencé par celui de ses pairs mais l'inverse est également vrai. Si une corrélation positive est observée entre la note de l'individu et celle des pairs, il est alors difficile de connaître dans quelles mesures le comportement des pairs a déterminé celui de l'individu et \emph{vice versa}.\\
La formation endogène des groupes se traduit par la tendance des individus ayant des résultats potentiels similaires à se regrouper entre eux. Ce problème peut se présenter entre les établissements ou entre les classes au sein de chaque établissement. Le premier cas correspond pratiquement à l'existence des établissements élitistes ou encore au choix endogène des parents aisés en faveur de ces établissements. Le second cas correspond aux classes de niveau constituées au sein d'un établissement. Ces classes regroupent ceux qui auront vraisemblablement de bons résultats (forts) entre eux et ceux qui auront vraisemblablement de moins bons résultats (moins forts) entre eux. Que la formation endogène des groupes s'opère au niveau des établissements ou au niveau des classes, si nous observons que les résultats des élèves forts sont supérieurs à ceux des élèves moins forts, il est difficile de savoir si cela est dû aux effets de pairs (les pairs forts sont bénéfiques) ou bien si les élèves forts auraient de toute façon eu des résultats supérieurs, indépendamment de leurs pairs.\\
Distinguer les effets de pairs des effets corrélés de groupe est plutôt un problème lié aux données. D'une part, il est possible de prendre en compte les effets corrélés de groupe en ayant une base riche en variables explicatives pertinentes au niveau de la classe\footnote{Notamment des informations détaillées sur l'enseignant, telles que son âge, son genre, son expérience dans l'enseignement, son niveau d'études, etc..}. D'autre part, si le chercheur peut identifier les individus de même groupe, il lui est possible de prendre en compte toute caractéristique inobservable spécifique au groupe via des effets fixes.

\quad D'autres éléments qui dépendent du contexte et des données disponibles amènent les chercheurs à adapter leurs méthodologies au cas par cas. Un des premiers concerne le niveau de groupement utilisé pour définir les pairs. En éducation (mais pas exclusivement), les pairs peuvent être définis au niveau du voisinage (\protect\hyperlink{ref-GOU:MAU:07}{Goux \& Maurin, 2007}), de l'école (\protect\hyperlink{ref-LAV:eal:12}{Lavy et al., 2012}), de la classe (\protect\hyperlink{ref-HOX:WEI:05}{Hoxby \& Weingarth, 2005}), du dortoir (\protect\hyperlink{ref-SAC:01}{Sacerdote, 2001}) ou du réseau d'amis (\protect\hyperlink{ref-BRA:eal:20}{Bramoullé et al., 2020}). La nature supposée des interactions et donc les effets de pairs estimés peuvent différer selon ces niveaux de groupement (\protect\hyperlink{ref-BUR:SAS:13}{Burke \& Sass, 2013} ; \protect\hyperlink{ref-CAR:eal:09}{Carrell et al., 2009}). Un deuxième élément, qui a fait l'objet de relativement peu d'attention est la présence des valeurs manquantes dans les caractéristiques prédéterminées de l'individu lorsque ces caractéristiques sont à récupérer depuis d'autres bases de données. Ces valeurs manquantes rendent impossible le calcul des vraies caractéristiques moyennes chez les pairs. Particulièrement, si les valeurs manquantes ne sont pas aléatoires, la moyenne de la variable correspondante chez les pairs souffre d'un biais de sélection.

\quad Ce papier exploite les fichiers exhaustifs de résultats aux épreuves nationales de l'éducation secondaire (principalement) et primaire en France afin de contribuer à la connaissance, relativement limitée pour le territoire français (\protect\hyperlink{ref-MON:eal:19}{Monso et al., 2019}), des effets du niveau et des comportements scolaires des pairs sur les performances en fin de collège (principalement) et en fin de primaire à La Réunion. Autrement dit, nous tentons à la fois d'estimer un mélange d'effets de pairs exogènes et endogènes (via des équations de forme réduite, voir Section \ref{pelitt}) et de mesurer séparément les effets de pairs endogènes. Dans les deux cas, nous sommes en mesure d'identifier les pairs au niveau de la classe, un niveau pertinent puisque la majorité des interactions scolaires des élèves se produisent au sein de la classe. Dans les équations de forme réduite, nous traitons les problèmes de la formation endogène des groupes et des effets corrélés en comparant les individus au sein d'un même établissement scolaire et en invoquant l'hypothèse d'assignation aléatoire des élèves entre les classes (\protect\hyperlink{ref-AMM:PIS:09}{Ammermueller \& Pischke, 2009} ; \protect\hyperlink{ref-BOU:MAI:18}{Boutchenik \& Maillard, 2018}, par exemple). Nous justifions notre hypothèse. Contrairement à la majorité des papiers similaires au nôtre, nous prenons en compte la présence de valeurs manquantes dans la mesure du niveau scolaire des pairs (\protect\hyperlink{ref-SOJ:13}{Sojourner, 2013}). Nous explorons dans la mesure du possible les formes théoriques des effets de pairs décrites ci-dessus qui collent le mieux à nos données et au contexte de La Réunion. Des recommandations assez claires en termes d'allocation des élèves dans les classes selon leurs niveaux scolaires sont tirées.\\
Tenter de mesurer séparément les effets endogènes nous permet de comparer les effets de pairs à deux niveaux d'éducation très différents, la fin du primaire et la fin du collège. Cette démarche possède également l'avantage de reposer sur un ensemble d'hypothèses plus réduit grâce à une méthodologie développée relativement récemment et peu appliquée à des cas similaires au nôtre (\protect\hyperlink{ref-BRA:eal:09}{Bramoullé et al., 2009} ; \protect\hyperlink{ref-BOU:eal:14}{Boucher et al., 2014} ; \protect\hyperlink{ref-IZA:DIC:20}{Izaguirre \& Di Capua, 2020}). La méthodologie consiste à spécifier la relation entre le comportement de l'individu et celui des pairs par des modèles de type auto-régression spatiale et repose principalement sur la variation des tailles de classe pour identifier les effets de pairs. Elle permet d'utiliser directement des effets fixes de classes, avantage considérable discuté dans le corps du texte.

\quad Avec une spécification linéaire en moyenne de la forme réduite, nous trouvons qu'une unité d'écart-type de niveau scolaire en plus des pairs procure un avantage d'environ 0.2 unité d'écart-type sur la performance en fin de collège\footnote{Par ailleurs, des résultats annexes de régression indiquent qu'il y a bien formation endogène des groupes aussi bien à travers les établissements qu'à travers les classes au sein de chaque établissement ; ce qui justifie les différentes étapes de notre stratégie d'identification.}.\\
Nous rejetons clairement la spécification linéaire en moyenne, notamment en trouvant des effets plus robustes et plus faciles à expliquer avec des spécifications des effets de pairs hétérogènes selon le niveau scolaire de l'élève et des spécifications hétérogènes et non linéaires à la fois. Avec les spécifications hétérogènes des effets de pairs, nous trouvons de manière robuste que les élèves les plus faibles ne sont pas sensibles à ces effets et que ces derniers augmentent avec le niveau de l'élève. Ces résultats fournissent un premier indice que les enseignants adaptent le niveau de leurs enseignements en fonction du niveau de la classe, ce qui a d'ailleurs été mis en évidence par Davezies (\protect\hyperlink{ref-DAV:04}{2004}) sur données françaises. Nous obtenons des résultats cohérents dans les spécifications à la fois hétérogènes et non linéaires des effets de pairs. En effet, nous trouvons que les élèves les plus faibles ne sont pas impactés, que ces derniers ont un effet négatif sur les élèves plus forts qu'eux sauf les plus forts, et que les pairs les plus forts sont bénéfiques pour tous les élèves ayant un niveau au moins moyen, le bénéfice étant croissant avec le niveau de l'élève. Par rapport aux formes théoriques, les effets de pairs en éducation à La Réunion correspondent vraisemblablement, en partie\footnote{En partie puisque les élèves les plus faibles ne sont pas sensible aux pairs, quel que soit le niveau de ces derniers.} à un modèle de \emph{single crossing}.\\
Au niveau des effets endogènes, nos coefficients sont de grande ampleur en fin de primaire et en fin de collège. Entre le primaire et le collège, ces effets doublent en mathématiques et se réduisent de moitié en français. Ne pouvant estimer les effets endogènes qu'avec des équations linéaires en moyenne (voir corps du texte), nous compensons cette limite par une analyse par type d'établissement et trouvons que ces effets sont non significatifs (bien que de grande ampleur) au sein des écoles\footnote{Sauf en mathématiques.} et collèges privés ; et que les effets sont les plus forts dans les collèges publics hors éducation prioritaires, surtout au collège et en mathématiques. Ces différences d'effets de pairs endogènes entre le primaire et le collège peuvent refléter le fait que les élèves acquièrent plus de capacités à interagir en grandissant. Des effets plus forts dans les établissements publics, surtout en éducation prioritaire, correspondent à l'intuition que les élèves y étant scolarisés ont tendance à être de catégorie sociale plus modeste (par rapport aux élèves des établissements publics hors éducation prioritaire et surtout privés) et donc plus dépendants du contexte scolaire pour l'accumulation de capital humain.

\quad La Section \ref{pelitt} de ce papier reconstitue l'état des connaissances sur les effets du niveau et des comportements des pairs dans l'éducation secondaire. Elle discute également des méthodologies retenues pour y parvenir. La Section \ref{peinst} complète son homologue dans le chapitre précédent avec les éléments du système éducatif de La Réunion nécessaires au présent chapitre. Notamment, la constitution des classes y est discutée plus en détail. La Section \ref{pedata} vient également en complément de son homologue du chapitre précédent et aborde les aspects des données utilisées spécifiques au chapitre présent. La Section \ref{pemethods} présente les stratégies d'identification dans le cas de la forme réduite et dans l'estimation des effets de pairs endogènes, au vu des données disponibles. La Section \ref{peres} présente et discute des résultats qui en découlent et la Section \ref{peconcl} conclue le chapitre.

\hypertarget{pelitt}{%
\section{Revue de littérature}\label{pelitt}}

Dans le dernier quart du 20\textsuperscript{ème} siècle, Coleman (\protect\hyperlink{ref-COL:68}{1968}) trouve une corrélation forte entre la composition sociale de l'établissement et la réussite des élèves aux États-Unis. Il est probable que cette corrélation est due aux effets de pairs. Comprendre précisément la nature et la forme des effets de pairs permet de connaître l'allocation optimale des élèves dans les territoires, les établissements ou les classes pour un objectif d'augmentation des performances éducatives ou encore d'expliquer la réussite ou l'échec des politiques telles que la déségrégation. Il y a alors un accroissement exponentiel de l'intérêt des chercheurs pour le rôle des caractéristiques des camarades dans la fonction de production de l'éducation.\\
Par exemple, au début du 21\textsuperscript{ème} siècle, aux États-Unis, Hoxby (\protect\hyperlink{ref-HOX:00}{2000}) mobilise des données sur les établissements scolaires publics du Texas\footnote{Couvrant les élèves de la troisième à la sixième année d'éducation à partir du primaire et les années scolaires de 1990-1991 à 1998-1999.} pour étudier la relation entre la proportion de filles dans la classe sur les résultats des garçons. De leur côté, Boozer \& Cacciola (\protect\hyperlink{ref-BOO:CAC:01}{2001}) profitent du projet STAR (\emph{Student Teacher Achievement Ratio}) dont l'objet est d'évaluer l'effet de la taille de classe, pour estimer des effets de pairs. Dans chaque établissement scolaire du projet, les élèves sont affectés aléatoirement à des classes de tailles différentes. De ce fait, la variation du niveau scolaire de la classe est également aléatoire. Les auteurs exploitent cette variation expérimentale pour distinguer les effets de pairs des effets de taille de classe.\\
Sur données française, Davezies (\protect\hyperlink{ref-DAV:04}{2004}) exploite le panel des écoliers 1997 et les évaluations bilans de CM2 de 2003 pour mesurer le lien causal d'une part entre la composition sociale de l'établissement et les résultats et d'autre part entre le résultat des pairs de la classe et celui de l'élève. Goux \& Maurin (\protect\hyperlink{ref-GOU:MAU:07}{2007}) quant à eux s'intéressent à des effets de voisinage en éducation\footnote{Plus précisément, une partie de leur étude se demande si la proportion d'élèves ayant déjà redoublé parmi les voisins de quartier a une influence sur le redoublement.}. Plus récemment, Boutchenik \& Maillard (\protect\hyperlink{ref-BOU:MAI:18}{2018}) étudient l'effet du niveau en fin de 3\textsuperscript{ème} des pairs de la classe de terminale sur les résultats au Baccalauréat.\\
Ces quelques exemples affichent à la fois l'intérêt des chercheurs pour le domaine et l'hétérogénéité de leurs approches. Nous développons ce second point ci-dessous.

\quad La littérature des effets de pairs en éducation est constituée d'une part de papiers qui n'estiment qu'une forme réduite, sans ambition de séparer les effets de pairs exogènes et endogènes (\protect\hyperlink{ref-SAC:11}{Sacerdote, 2011}) ; et d'autre part d'une littérature plus ou moins récente qui tente de le faire (\protect\hyperlink{ref-YEU:NGU:16}{Yeung \& Nguyen-Hoang, 2016}). La première catégorie se contente d'estimer des effets mixtes probablement à cause de la limite des développements méthodologiques de l'époque ou par manque de données.

Dans la première catégorie, il y a en particulier une famille d'études cherchant à modéliser une performance éducative en fonction d'une mesure passée chez les pairs de cette variable chez les pairs. Dans ce cas, le problème de la réflexion ne se pose pas mais les auteurs n'identifient qu'un mix d'effets endogènes et exogènes. Par exemple, Lavy et al. (\protect\hyperlink{ref-LAV:eal:12}{2012}) s'intéressent à l'effet de la note moyenne des pairs obtenue en fin de primaire sur la note obtenue à la neuvième année d'éducation du premier cycle en Angleterre (équivalent du 3\textsuperscript{ème} du système français). Le papier de Gibbons \& Telhaj (\protect\hyperlink{ref-GIB:TEL:16}{2016}) constitue un autre exemple en Angleterre avec une approche légèrement différente du problème.\\
La littérature apparaît relativement plus fournie aux États-Unis avec Hanushek et al. (\protect\hyperlink{ref-HAN:eal:03}{2003}), Hoxby \& Weingarth (\protect\hyperlink{ref-HOX:WEI:05}{2005}) ou Burke \& Sass (\protect\hyperlink{ref-BUR:SAS:13}{2013}), en particulier\footnote{Environ 40\% des papiers revus par Yeung \& Nguyen-Hoang (\protect\hyperlink{ref-YEU:NGU:16}{2016}) qui concernent la modélisation d'une mesure en fonction de sa mesure passée chez les pairs couvre au moins un état des États-Unis.}. Le premier utilise des données de panel couvrant 3 cohortes successives qui commencent par les entrants de la troisième année d'éducation du premier cycle de 1992 au Texas, pour estimer l'effet, pour un élève, de la note obtenue deux ans plus tôt par ses camarades d'école actuels, sur sa propre note. Les auteurs du second article s'intéressent à la Caroline du Nord. L'étude de Burke \& Sass (\protect\hyperlink{ref-BUR:SAS:13}{2013}) couvre quant à elle tous les établissements scolaires publics de la Floride et les élèves de la troisième à la dixième année d'éducation du premier cycle y étant inscrits.

\quad Face aux difficultés de l'identification empirique des effets de pairs (Section \ref{peintro}), les stratégies utilisées varient significativement en fonction du contexte et des données disponibles. Certains auteurs ont la possibilité de constituer des groupes de pairs formés de manière aléatoire, ce qui élimine toute corrélation du niveau des pairs avec les facteurs incitant les individus similaires à se regrouper entre eux. C'est le cas de Duflo et al. (\protect\hyperlink{ref-DUF:eal:11}{2011}), par exemple. Les auteurs s'intéressent à 120 écoles primaires du Kenya. Pour 60 d'entre elles, les classes sont explicitement composées selon le niveau des élèves et pour les 60 autres, les classes sont composées aléatoirement. Au sein de ces dernières, il est possible d'étudier les effets de pairs. En France et pour une université d'élite, Brodaty \& Gurgand (\protect\hyperlink{ref-BRO:GUR:16}{2016}) se basent également sur la composition aléatoire contrôlée des groupes.

\quad Lorsque cette composition aléatoire n'est pas possible, les économistes peuvent exploiter une expérience naturelle générant une variation vraisemblablement exogène dans le niveau des pairs, permettant d'identifier séparément les effets de pairs des effets de sélection ou des effets corrélés. C'est le cas de Imberman et al. (\protect\hyperlink{ref-IMB:eal:12}{2012}) qui utilisent l'occurrence de deux ouragans en 2005 en Louisiane et au Texas pour identifier les effets de pairs en éducation. Ces deux ouragans ont forcé les habitants des territoires susmentionnés à immigrer. Cette vague d'immigration a entraîné une variation substantielle du niveau des pairs des zones d'accueil. Cette variation est exogène puisqu'elle est due à des évènements naturels et peut être utilisée pour identifier les effets de pairs. Les réformes affectant substantiellement la composition scolaire des groupes peuvent également être utilisées pour la mesure des effets de pairs. C'est ce que font Hoxby \& Weingarth (\protect\hyperlink{ref-HOX:WEI:05}{2005}) dans le \emph{Wake County}, aux États-Unis. Au début des années 2000, ce comté passe d'une politique de constitution des écoles basée sur l'ethnie à une autre basée sur le revenu. D'une année à l'autre, la composition des pairs a significativement varié de manière vraisemblablement exogène conditionnellement à toutes caractéristiques de l'élève affectant potentiellement sa réussite, ce qui a permis aux auteurs d'identifier les effets de pairs.

\quad Le troisième type de stratégie consiste à utiliser des effets fixes pour prendre en compte la formation endogène des groupes. Par exemple, Ammermueller \& Pischke (\protect\hyperlink{ref-AMM:PIS:09}{2009}) mobilisent les données du PIRLS (\emph{Programme in International Reading Literacy Study}) sur 6 pays européens\footnote{L'Allemagne, la France, l'Islande, les Pays-Bas, la Norvège et la Suède.} pour étudier l'effet de la qualité des pairs, définie comme le nombre de livres à leur domicile, sur les résultats de tests standardisés de lecture. Ces auteurs peuvent identifier les élèves de la même classe et du même établissement et utilisent des effets fixes d'établissements pour comparer les élèves à établissement identique, vu qu'ils ne sont pas affectés aléatoirement aux établissements scolaires. Dans le même esprit, nous pouvons mentionner Sojourner (\protect\hyperlink{ref-SOJ:13}{2013}) pour les États-Unis ou encore Boutchenik \& Maillard (\protect\hyperlink{ref-BOU:MAI:18}{2018}) pour la France. Avec ce type de stratégie, il est nécessaire de pouvoir justifier qu'au sein des écoles, l'allocation des élèves, des enseignants et des ressources à travers les classes peut être considérée comme aléatoire. Lorsque cette hypothèse est irréaliste pour l'ensemble de l'échantillon, les auteurs identifient et excluent de l'échantillon d'estimation les établissements dans lesquels les classes ne semblent pas être formés indépendamment de la mesure d'intérêt chez les pairs. Cette stratégie de restriction a été appliquée par Ammermueller \& Pischke (\protect\hyperlink{ref-AMM:PIS:09}{2009}) et Boutchenik \& Maillard (\protect\hyperlink{ref-BOU:MAI:18}{2018}). Sojourner (\protect\hyperlink{ref-SOJ:13}{2013}) n'en n'a pas eu besoin puisqu'il exploite les données issues du projet STAR dans lequel les élèves sont explicitement alloués aléatoirement dans les classes.\\
Toujours dans la catégorie de la stratégie des effets fixes, à la place de la variation intra-établissement inter-classes de la mesure d'intérêt chez les pairs, certains papiers s'appuient plutôt sur une variation intra-établissement inter-cohortes. Hoxby (\protect\hyperlink{ref-HOX:00}{2000}), un des premiers à exploiter une telle idée, s'intéresse en partie à l'effet de la proportion de filles dans la classe, qui varie naturellement d'une année à l'autre, sur les performances éducatives. C'est également le cas de Hanushek et al. (\protect\hyperlink{ref-HAN:eal:03}{2003}), par exemple. Les auteurs profitent du fait qu'ils possèdent plusieurs cohortes à travers lesquelles le niveau des pairs varie forcément. Cette variation est naturelle et est donc vraisemblablement exogène.\\
Lavy et al. (\protect\hyperlink{ref-LAV:eal:12}{2012}) utilisent quant à eux le fait qu'ils ont à disposition les notes de l'élève et des pairs pour plusieurs matières. Sous la condition que les élèves plus aptes dans une matière par rapport à une autre ne se regroupent pas entre eux, les auteurs peuvent identifier les effets de pairs en exploitant la variation intra-individu inter-matières.

\quad Des variables instrumentales peuvent également être mobilisées pour contourner la formation endogène des groupes. C'est le cas de Goux \& Maurin (\protect\hyperlink{ref-GOU:MAU:07}{2007}) qui instrumentent la proportion de redoublants parmi les pairs (définis comme des voisins de quartier) par des moyennes de dates de naissance dans l'année chez les voisins. Plus récemment au Brésil, Arruda Raposo \& Gonçalves (\protect\hyperlink{ref-RAP:GON:20}{2020}) se basent sur la même stratégie en instrumentant la performance des pairs par la proportion de pairs nés au second semestre.

\quad Dans des cas spécifiques où le fait d'avoir des camarades de plus haut niveau dépend d'un seuil bien précis de la valeur d'une variable observable chez les élèves, des stratégies de régressions sur une discontinuité peuvent permettre de mesurer les effets de pairs. En guise d'exemple, l'étude de Landaud et al. (\protect\hyperlink{ref-LAN:eal:18}{2018}) se concentrent sur les lycées parisiens parmi lesquels les demandes d'admission en seconde sont supérieures aux places disponibles. Ce sont les lycées de plus haut niveau. Ils sélectionnent alors les candidats sur la base d'un seuil précis de la note au collège. Les élèves ayant une note au collège juste inférieure et juste supérieure au seuil sont vraisemblablement très similaires mais les seconds bénéficient de pairs plus forts que les premiers. Ce type de contexte permet d'identifier l'effet causal d'avoir de meilleurs pairs. Des stratégies similaires sont appliquées par Abdulkadiroğlu et al. (\protect\hyperlink{ref-ABD:eal:14}{2014}) aux États-Unis ou encore par Lucas \& Mbiti (\protect\hyperlink{ref-LUC:MBI:14}{2014}) au Kenya.

\quad Dans la catégorie de papiers qui tentent de séparer les effets endogènes \emph{per se} des effets exogènes, les chercheurs correspondant se heurtent en plus au problème de la réflexion et disposent de deux options. La première est d'utiliser une stratégie par variable instrumentale. Les instruments proposés sont très variables. Davezies (\protect\hyperlink{ref-DAV:04}{2004}) instrumente par exemple la note contemporaine des pairs par les proportions de mois de naissance dans l'année\footnote{C'est-à-dire sans tenir compte de l'année de naissance. Le chapitre qui précède en discute longuement.} chez les pairs. Zabel (\protect\hyperlink{ref-ZAB:08}{2008, p. 204}), avec des données de panel instrumente la note contemporaine des pairs par la note passée de ces derniers. Avec des données sur des élèves en 8\textsuperscript{ème} année de primaire dans la partie rurale du Bangladesh, Asadullah (\protect\hyperlink{ref-ASA:CHA:08}{2008}) mobilise la proportion de pairs (au niveau de l'école) exposés à des puits contaminés par de l'arsenic comme instrument de la note des pairs. Il a également déjà été envisagé d'utiliser les caractéristiques prédéterminées des pairs comme instrument de leur note (\protect\hyperlink{ref-DUF:eal:11}{Duflo et al., 2011, p. 1754}).\\
La seconde option, dont la méthodologie a été développée plus récemment (\protect\hyperlink{ref-BRA:eal:09}{Bramoullé et al., 2009}), consiste à voir la relation entre la note des pairs et celle de l'élève comme une auto-régression spatiale. Le cas des pairs définis au niveau de la classe constitue un cas particulier de ce modèle dans lequel les élèves d'une même classe s'influencent tous entre eux et ne sont influencés par personne en dehors de la classe. Dans ce cas, il a été montré que les effets de pairs endogènes peuvent être identifiés séparément si le chercheur retient le modèle linéaire en moyenne, dispose d'au moins trois tailles de classe différentes, exclut l'individu du calcul de la moyenne chez les pairs, et peut justifier que la taille de classe est exogène conditionnellement aux effets fixes de classe (\protect\hyperlink{ref-BOU:eal:14}{Boucher et al., 2014} ; \protect\hyperlink{ref-IZA:DIC:20}{Izaguirre \& Di Capua, 2020}).

\quad Dans la présente étude, nous pouvons identifier les pairs au niveau de la classe de 3\textsuperscript{ème}. Lorsque nous nous intéressons aux notes, la classe, en termes de niveau de groupement, est un \emph{second best}, juste après le réseau d'amis (\protect\hyperlink{ref-PAL:20}{Paloyo, 2020}) mais elle reste plus pertinente que l'école ou les dortoirs. Dans un premier temps, nous estimons une forme réduite dans laquelle nous expliquons la note de l'élève par une note passée des pairs, en fin de primaire. Ce faisant, nous ne sommes pas confrontés au problème de la réflexion et sommes capables de proposer des préconisations exploitables. Afin de contourner le problème de sélection des élèves dans les écoles, nous utilisons des effets fixes d'établissement. Pour s'affranchir des effets corrélés de classe, nous nous appuyons sur l'hypothèse d'assignation aléatoire des élèves dans les classes. Pour crédibiliser cette hypothèse, nous excluons de l'échantillon d'estimation les classes qui sont explicitement formées en fonction des performances potentielles des élèves. Via un test statistique tiré de la littérature, nous enlevons en plus des établissements suspectés de grouper les élèves dans les classes en fonction de leurs résultats potentiels pour des raisons autres que celle susmentionnée. Puisque la structure de nos données implique des valeurs manquantes non aléatoires dans la mesure du niveau des pairs, nous empruntons le modèle développé par Sojourner (\protect\hyperlink{ref-SOJ:13}{2013}) pour surmonter ce problème. Notre étude permet de confirmer quelques prédictions théoriques qui s'y trouvent.\\
Plus généralement, notre étude complète la littérature sur les effets de pairs en éducation relativement peu fournie en France et inexistante à La Réunion. Nous contribuons ainsi au débat concernant la nécessité de constituer des classes de niveau ou non.

\quad Dans un second temps, puisque notre contexte et nos données satisfont les hypothèses nécessaires, nous tentons d'estimer séparément les effets de pairs endogènes \emph{per se} en utilisant la méthodologie susmentionnée, décrite par Boucher et al. (\protect\hyperlink{ref-BOU:eal:14}{2014}). Nous pouvons ainsi analyser si des mesures éducatives ciblées pourraient avoir des conséquences positives sur tous les élèves ; et si oui, dans quel contexte. De plus, en exploitant indépendamment des données en fin de primaire, cette approche a l'avantage de nous permettre de comparer les effets de pairs à deux niveaux d'éducation différents.\\
Notre papier est à notre connaissance le premier à utiliser les deux approches, à savoir l'estimation de la forme réduite ainsi que des effets endogènes eux-mêmes.\\

\quad En termes de résultats empiriques obtenus dans la littérature, toutes stratégies d'identifications confondues, les études utilisant un modèle linéaire en moyenne donnent des résultats très hétérogènes\footnote{Les effets sur des variables de comportements telles que l'utilisation de l'emploi du temps, la tricherie, l'indiscipline, etc. existent plus clairement (\protect\hyperlink{ref-BRO:10}{Brodaty, 2010} ; \protect\hyperlink{ref-SAC:11}{Sacerdote, 2011}).}. Certains papiers trouvent des effets significatifs et importants (\protect\hyperlink{ref-AMM:PIS:09}{Ammermueller \& Pischke, 2009}), d'autres significatifs mais modérés (\protect\hyperlink{ref-BOU:MAI:18}{Boutchenik \& Maillard, 2018}) et d'autres ne trouvent pas d'effet en général (\protect\hyperlink{ref-ANG:LAN:04}{Angrist \& Lang, 2004}). Le modèle linéaire en moyenne nous apprend toutefois que les effets sont généralement supérieurs au niveau de la classe que de l'école (\protect\hyperlink{ref-BUR:SAS:13}{Burke \& Sass, 2013}) et que le niveau scolaire des pairs prime sur leurs caractéristiques socio-économiques dans la fonction de production d'éducation (\protect\hyperlink{ref-HOX:WEI:05}{Hoxby \& Weingarth, 2005} ; \protect\hyperlink{ref-GIB:TEL:16}{Gibbons \& Telhaj, 2016} ; \protect\hyperlink{ref-FOU:eal:17}{Fougère et al., 2017}). Lorsque les chercheurs s'intéressent à la différence des effets selon le niveau de l'individu, ils trouvent généralement que les effets sont plus forts chez les individus de niveau plus faible (\protect\hyperlink{ref-DAV:04}{Davezies, 2004} ; \protect\hyperlink{ref-BOU:MAI:18}{Boutchenik \& Maillard, 2018}, pour la France). L'hypothèse qui rationalise ce résultat est que les élèves de plus faible niveau ont probablement un environnement familial moins favorable à leurs études, ce qui les rend plus dépendant du contexte scolaire.\\
Avec des modèles plus flexibles, les résultats sont moins ambigus. Les effets sont globalement non linéaires (\protect\hyperlink{ref-MEN:eal:18}{Mendolia et al., 2018}). Plus précisément, les chercheurs trouvent souvent un effet positif des pairs forts (\protect\hyperlink{ref-BUR:SAS:13}{Burke \& Sass, 2013} ; \protect\hyperlink{ref-IMB:eal:12}{Imberman et al., 2012}, par exemple). Les individus sur lesquels les pairs sont bénéfiques peuvent varier d'un cadre institutionnel à l'autre. Par exemple, Hoxby \& Weingarth (\protect\hyperlink{ref-HOX:WEI:05}{2005}) trouvent que les pairs forts ne bénéficient qu'aux individus forts alors que Imberman et al. (\protect\hyperlink{ref-IMB:eal:12}{2012}) trouvent que les pairs forts bénéficient à tous\footnote{L'Angleterre semble être une exception notable puisque Lavy et al. (\protect\hyperlink{ref-LAV:eal:12}{2012}) Gibbons \& Telhaj (\protect\hyperlink{ref-GIB:TEL:16}{2016}) ne trouvent pas d'effet convainquant des pairs forts.}. Les résultats sur l'influence des pairs faibles sont plus flous (\protect\hyperlink{ref-MON:eal:19}{Monso et al., 2019}).\\
\quad En ce qui concerne les effets de pairs endogènes, Boucher et al. (\protect\hyperlink{ref-BOU:eal:14}{2014}) et Izaguirre \& Di Capua (\protect\hyperlink{ref-IZA:DIC:20}{2020}) trouvent des effets robustes et significatifs en mathématiques\footnote{La méta-analyse de Yeung \& Nguyen-Hoang (\protect\hyperlink{ref-YEU:NGU:16}{2016}) conclut également que les effets de pairs endogènes existent et sont d'ampleur modérée. Mais ce papier inclut en plus dans le terme ``effet endogène'' l'effet de la note passée des pairs qui est censé être une approximation de la note contemporaine. Dans notre étude, nous considérons l'effet de la note passée des pairs comme un mix d'effets de pairs endogène et exogène.}. Les effets dans d'autres matières ne semblent pas être avérés.

\hypertarget{peinst}{%
\section{Contexte institutionnel}\label{peinst}}

Au sein d'une classe de primaire, la totalité des enseignements des matières est dispensée par un seul enseignant tandis qu'au collège, les cours sont assurés par plusieurs enseignants en fonction de la matière.

\quad Le CM2 correspondant à la fin du primaire, nous pouvons considérer les notes aux évaluations nationales de CM2 comme des niveaux d'entrée au collège. Nous considérons uniquement la note totale au CM2 et pas séparément la note en français ou en mathématiques\footnote{Considérer séparément les effets du niveau de français ou de mathématiques des pairs ne modifient pas nos conclusions.}.

\quad Au collège, nous distinguons des sections normales et des sections spéciales, en termes de niveau scolaire. Seules les sections de la 3\textsuperscript{ème} à La Réunion nous intéressent. Nous utilisons les fichiers de résultats au CM2 (chapitre précédent) pour définir et pour repérer les sections avantagées ou désavantagées en termes de niveau scolaire. Les sections désavantagées (avantagées) sont celles avec un niveau au CM2 inférieur (supérieur) à celui de la section normale.
Le Tableau \ref{tab:pestatssection} propose une image plus détaillée des sections et du niveau scolaire correspondant à La Réunion pour les trois années scolaires disponibles dans nos données (Section \ref{pedata}). Les sections sont classées par ordre décroissant de la note totale au CM2 afin de visualiser leur hiérarchie en niveau scolaire des élèves.
Les sections désavantagées sont constituées des élèves dits en SEGPA (Section d'Enseignement Général et Professionnel Adapté), en DIMA (Dispositif d'Initiation aux Métiers en Alternance), en ULIS (Unité Localisée d'Inclusion Scolaire), en Prépa Pro (Préparation Professionnelle) et, de manière surprenante, en Théâtre. Les sections avantagées regroupent les élèves en Histoire des arts et en option Internationale, Européenne orientale, Bilangue et Sportive.\\
En termes de proportions des différentes sections, la colonne 2 montre que les deux sections spéciales les plus importantes sont celles de l'Européenne orientale (17\% ou 18\% de l'ensemble des candidats au DNB de l'année en question) et de la Préparation professionnelle (11 \% ou 12\%). La proportion d'élèves pour lesquels nous pouvons retrouver la note au CM2\footnote{Et plus généralement, les informations au CM2 (voir Section \ref{pedata}).} varie considérablement entre les sections et les années\footnote{Pour avoir une image globale, les pourcentages entre parenthèses de la colonne 3 vont de 0\% à 100\%. Les premier et troisième quartiles sont respectivement de 67\% et 85\%. La médiane (80\%) est supérieure à la moyenne (69\%), suggérant l'existence d'une ou plusieurs valeurs extrêmement basses. Il s'agit du 0\% de retrouvés chez les SEGPA de 2014.}. Les colonnes 4 et 5 présentent la moyenne de note au CM2 et en français respectivement par section, pour l'année considérée. La colonne 5 s'assure de la cohérence de la hiérarchie des niveaux décrite par la colonne 4, vu que la note totale au CM2 n'est pas comparable d'une année d'évaluation de CM2 à l'autre\footnote{À cause de la trop grande différence en structure des évaluations de mathématiques. Cette différence en structure s'est ensuite traduite par une inflation significative des notes en mathématiques d'une année à l'autre.} alors que la note en français l'est. La hiérarchie de la colonne 5 est très largement cohérente avec celle de la colonne 4. Les colonnes 6 et 7 montrent également que les catégories sociales sont généralement en cohérence avec la hiérarchie de la note au CM2. En effet, la proportion d'élèves issus de catégories sociales défavorisées (très favorisées) dans les sections avantagées (désavantagées) est inférieure (supérieure) à celle dans la section normale.

\newpage  
\begingroup\fontsize{8}{10}\selectfont

\begin{ThreePartTable}
\begin{TableNotes}
\item \textit{Sources :} Fichiers DNB (2014 à 2016), fichiers CM2 (2010 à 2012), fichiers CONSTAT (2013 à 2016), calculs de l'auteur.
\item \textit{Lecture :} Parmi les candidats au DNB de 2014, 9 sont dans la section internationale, représentant 0.1% des candidats. Parmi les 267 candidats en section sportive au DNB de 2014, nous pouvons récupérer la note au CM2 pour 209 (78.3%) d'entre eux et 154, soit 57.7% sont issus de catégorie sociale défavorisée.
\item \textit{Notes : } La note totale au CM2 est ordonnée par ordre décroissante (par année scolaire). Les sections ayant une note supérieure (inférieure) à celle de la section normale sont les sections avantagées (désavantagées). Les cellules vides des colonnes 4 et 5 signifient qu'il n'a pas été possible de retrouver la note au CM2 des 4 élèves de SEGPA.
\item SEGPA : Section d'Enseignement Général et Professionnel Adapté, ULIS : Unité Locale d'Inclusion Scolaire, DIMA : Dispositif d'Initiation aux Métiers de l'Alternance.
\end{TableNotes}
\begin{longtable}[t]{lllllll}
\caption{\label{tab:pestatssection}Les sections en 3ème à La Réunion}\\
\toprule
\multicolumn{2}{c}{ } & \multicolumn{3}{c}{Note au CM2} & \multicolumn{2}{c}{Effectif (proportion) de} \\
\cmidrule(l{3pt}r{3pt}){3-5} \cmidrule(l{3pt}r{3pt}){6-7}
\makecell{\makecell{Sections \\ \ } \\ (1) } & \makecell{\makecell{Effectif (proportion) \\ \ } \\ (2) } & \makecell{\makecell{Effectif (proportion) \\ de retrouvés} \\ (3) } & \makecell{\makecell{Totale \\ \ } \\ (4) } & \makecell{\makecell{En français \\ \ } \\ (5) } & \makecell{\makecell{Défavorisés \\ \ } \\ (6) } & \makecell{\makecell{Très favorisés \\ \ } \\ (7) }\\
\midrule
\endfirsthead
\caption[]{\label{tab:pestatssection}Les sections en 3ème à La Réunion (suite)}\\
\toprule
\makecell{\makecell{Sections \\ \ } \\ (1) } & \makecell{\makecell{Effectif (proportion) \\ \ } \\ (2) } & \makecell{\makecell{Effectif (proportion) \\ de retrouvés} \\ (3) } & \makecell{\makecell{Totale \\ \ } \\ (4) } & \makecell{\makecell{En français \\ \ } \\ (5) } & \makecell{\makecell{Défavorisés \\ \ } \\ (6) } & \makecell{\makecell{Très favorisés \\ \ } \\ (7) }\\
\midrule
\endhead

\endfoot
\bottomrule
\insertTableNotes
\endlastfoot
\addlinespace[0.3em]
\multicolumn{7}{l}{\textbf{2014}}\\
\hline
\hspace{1em}Européenne orientale & 2398 (0.174) & 1899 (0.792) & 64.53 & 42.20 & 950 (0.396) & 540 (0.225)\\
\hspace{1em}Bilangue & 1096 (0.08) & 881 (0.804) & 64.41 & 41.96 & 490 (0.447) & 208 (0.19)\\
\hspace{1em}Internationale & 9 (0.001) & 7 (0.778) & 61.57 & 42.43 & 1 (0.111) & 2 (0.222)\\
\hspace{1em}Histoire des arts & 21 (0.002) & 17 (0.81) & 55.94 & 37.29 & 4 (0.19) & 8 (0.381)\\
\hspace{1em}Sportive & 267 (0.019) & 209 (0.783) & 53.43 & 34.31 & 154 (0.577) & 29 (0.109)\\
\hspace{1em}Normale & 8305 (0.604) & 5849 (0.704) & 46.83 & 31.16 & 4911 (0.591) & 754 (0.091)\\
\hspace{1em}Théâtre & 15 (0.001) & 14 (0.933) & 36.43 & 23.00 & 10 (0.667) & 1 (0.067)\\
\hspace{1em}Préparation professionnelle & 1584 (0.115) & 1060 (0.669) & 34.62 & 23.28 & 1204 (0.76) & 27 (0.017)\\
\hspace{1em}ULIS & 4 (0) & 1 (0.25) & 29.00 & 26.00 & 3 (0.75) & 0 (0)\\
\hspace{1em}DIMA & 42 (0.003) & 23 (0.548) & 28.52 & 19.09 & 29 (0.69) & 0 (0)\\
\hspace{1em}SEGPA & 4 (0) & 0 (0) & - & - & 3 (0.75) & 0 (0)\\
\addlinespace[0.3em]
\multicolumn{7}{l}{\textbf{2015}}\\
\hline
\hspace{1em}Internationale & 27 (0.002) & 25 (0.926) & 78.60 & 47.04 & 4 (0.148) & 12 (0.444)\\
\hspace{1em}Européenne orientale & 2680 (0.183) & 2253 (0.841) & 68.72 & 42.36 & 1028 (0.384) & 611 (0.228)\\
\hspace{1em}Bilangue & 1112 (0.076) & 946 (0.851) & 67.14 & 41.58 & 513 (0.461) & 202 (0.182)\\
\hspace{1em}Sportive & 261 (0.018) & 225 (0.862) & 53.26 & 32.60 & 139 (0.533) & 21 (0.08)\\
\hspace{1em}Normale & 8804 (0.602) & 7028 (0.798) & 48.12 & 29.99 & 5114 (0.581) & 842 (0.096)\\
\hspace{1em}Théâtre & 10 (0.001) & 9 (0.9) & 43.78 & 27.44 & 10 (1) & 0 (0)\\
\hspace{1em}Préparation professionnelle & 1707 (0.117) & 1384 (0.811) & 35.66 & 22.50 & 1333 (0.781) & 30 (0.018)\\
\hspace{1em}ULIS & 14 (0.001) & 3 (0.214) & 28.33 & 20.33 & 9 (0.643) & 1 (0.071)\\
\hspace{1em}SEGPA & 9 (0.001) & 3 (0.333) & 17.00 & 12.33 & 7 (0.778) & 0 (0)\\
\addlinespace[0.3em]
\multicolumn{7}{l}{\textbf{2016}}\\
\hline
\hspace{1em}Internationale & 28 (0.002) & 28 (1) & 81.00 & 50.18 & 4 (0.143) & 19 (0.679)\\
\hspace{1em}Européenne orientale & 2682 (0.188) & 2186 (0.815) & 71.61 & 43.55 & 1015 (0.378) & 616 (0.23)\\
\hspace{1em}Bilangue & 1139 (0.08) & 931 (0.817) & 70.16 & 42.54 & 542 (0.476) & 192 (0.169)\\
\hspace{1em}Sportive & 264 (0.019) & 228 (0.864) & 58.07 & 34.14 & 141 (0.534) & 16 (0.061)\\
\hspace{1em}Normale & 8392 (0.589) & 6474 (0.771) & 50.02 & 30.07 & 5029 (0.599) & 751 (0.089)\\
\hspace{1em}Théâtre & 13 (0.001) & 12 (0.923) & 44.92 & 25.75 & 8 (0.615) & 0 (0)\\
\hspace{1em}Préparation professionnelle & 1694 (0.119) & 1330 (0.785) & 37.53 & 22.31 & 1287 (0.76) & 25 (0.015)\\
\hspace{1em}SEGPA & 10 (0.001) & 4 (0.4) & 22.00 & 13.00 & 10 (1) & 0 (0)\\
\hspace{1em}ULIS & 15 (0.001) & 3 (0.2) & 16.67 & 5.33 & 10 (0.667) & 0 (0)\\*
\end{longtable}
\end{ThreePartTable}
\endgroup{}

\hypertarget{pedata}{%
\section{Données}\label{pedata}}

Nous utilisons dans ce chapitre les fichiers de résultats aux épreuves du DNB décrites dans le chapitre précédent\footnote{Pour résumer, ce sont des données en coupe couvrant tous les établissements publics et privés sous contrat des années scolaires 2013-2014, 2014-2015 et 2015-2016. Chaque année compte 14000 observations environ avec une augmentation d'année en année partiellement en raison d'un \emph{baby boom} au début des années 2000.}.

\quad Notre variable dépendante est la note globale, en français ou en mathématiques aux épreuves écrites. Rappelons qu'il y a environ 3\% de valeurs manquantes au sein de cette variable et que même si les valeurs manquantes ne sont pas aléatoires\footnote{Par rapport aux candidats avec des valeurs non manquantes des notes aux épreuves écrites du DNB, les élèves pour lesquels les notes sont manquantes ont un niveau au CM2 inférieurs, sont constitués de plus de garçons, plus d'élève de catégorie sociale défavorisée et plus d'élèves en régime scolaire externe.}, leur faible proportion n'est pas de nature à biaiser les résultats de notre étude.

\quad L'INE est l'identifiant qui permet de lier la base DNB à la base CM2. Parmi les candidats au DNB, nous arrivons à récupérer les notes au CM2 d'environ 78\% d'entre eux. Outre les raisons habituelles d'attrition, la principale raison de cette perte d'information est la limite de la période couverte par les données de CM2 combinée avec le phénomène de redoublement et de saut de classe entre le CM2 (y compris) et la 3\textsuperscript{ème} (y compris). Plus précisément, notons \(s = \{- 2, - 1, 0, 1, 2\}\) la variable qui vaut 0 si un élève n'a ni redoublé ni sauté de classe, 1 (ou 2) s'il a sauté une (ou 2) classe(s), - 1 (ou - 2) s'il a redoublé une (ou deux) fois\footnote{Nous nous limitons à ces 5 valeurs de \(s\) car redoubler 3 fois ou sauter 3 classes ne se produit quasiment jamais.}. Pour une année \(t\) de DNB, l'année de CM2 d'un élève est \(t - 4 + s\). Nous ne pouvons donc récupérer sa note que si \((t - 4 + s) \in \{2010, 2011, 2012\} \ \forall t \in \{2014, 2015, 2016\}\). Par exemple, nous pouvons théoriquement récupérer la note au CM2 d'un candidat au DNB de 2015 qui a redoublé une fois puisque \((2015 - 4 - 1 = 2010) \in \{2010, 2011, 2012\}\). Par contre, nous perdons tous les candidats au DNB de 2014 qui ont redoublé au moins une fois au collège puisque \((2014 - 4 - 1 = 2009) \notin \{2010, 2011, 2012\}\) et \((2014 - 4 - 2 = 2009) \notin \{2010, 2011, 2012\}\).

\quad Nous récupérons également au sein des fichiers de CM2, via l'INE, les identifiants d'écoles primaires et de classes de CM2. Cela nous permet d'identifier explicitement pour chaque élève que nous pouvons lier à la base CM2, parmi les pairs de 3\textsuperscript{ème}, ceux qui étaient dans la même classe de CM2. Nous appelons ce sous-groupe de pairs les anciens pairs et son complément (parmi les pairs) les nouveaux pairs. L'utilité de cette information est explicitée dans la Section \ref{peresfrrob}.

\quad Au primaire comme au collège, les tailles de classe calculées (chapitre précédent) diffèrent en fonction du type d'établissement.
Au CM2, la taille moyenne de classe dans les écoles privées est de 26 élèves. Elle est de 23 dans les écoles hors éducation prioritaire et de 22 dans les écoles en éducation prioritaire. En 3\textsuperscript{ème}, ces chiffres sont respectivement de 29, 25 et 23. Ces informations constituent un premier indice que les élèves moins forts sont vraisemblablement assignés à des classes de plus petite taille. Nous discutons de l'importance de ce fait dans les résultats (Section \ref{peresfr} et \ref{peresendo}).

\quad Les identifiants des établissements nous permettront de prendre en compte les effets fixes nécessaires dans les estimations (Section \ref{pemethodss13}). Le Tableau \ref{tab:pestats} présente les statistiques descriptives de différents échantillons de candidats au DNB. La colonne (1) correspond à l'échantillon n'excluant aucun élève, aucune classe ou aucun établissement\footnote{Dans la Section \ref{pemethodsrestr}, les colonnes 2 à 5 du tableau sont discutées. Les différentes exclusions sont en lien direct avec notre stratégie d'identification.}. Cette colonne nous montre que nous comptons 82 globalement établissements à La Réunion, pour les trois années scolaires considérées et qu'il y a en moyenne 7 classes par établissement.

\quad Pour rappel et pour comparaison avec les autres colonnes, les statistiques sur les variables individuelles sont présentées dans la colonne 1 du Tableau \ref{tab:pestats}.

\begin{landscape}\begingroup\fontsize{8}{10}\selectfont

\begin{ThreePartTable}
\begin{TableNotes}
\item \textit{Sources :} Fichiers DNB (2014 à 2016), fichiers CM2 (2010 à 2012), fichiers CONSTAT (2013 à 2016), calculs de l'auteur.
\item \textit{Notes :} Moyennes et proportions. Écart-types entre parenthèses sauf mention contraire. Les notes sont exprimées en rang de percentile. Les moyennes de notes au CM2 sont calculées sur les retrouvés uniquement. KS : Kolmogorov-Smirnov. CSP : Catégorie Socio-Professionnelle. CM2 : Cours Moyen 2\textsuperscript{ème} année. DNB : Diplôme National du Brevet.
\end{TableNotes}
\begin{longtable}[t]{lllllll}
\caption{\label{tab:pestats}Statistiques descriptives}\\
\toprule
\multicolumn{3}{c}{ } & \multicolumn{2}{c}{Classes atypiques} & \multicolumn{2}{c}{ } \\
\cmidrule(l{3pt}r{3pt}){4-5}
  & \makecell{\makecell{Complètes \\ \ } \\ (1) } & \makecell{\makecell{Sans classes atypiques \\ \ } \\ (2) } & \makecell{\makecell{Avantagées \\ \ } \\ (3) } & \makecell{\makecell{Désavantagées \\ \ } \\ (4) } & \makecell{\makecell{Sans classes atypiques, \\ après tests KS} \\ (5) } & \makecell{\makecell{Établissements exclus \\ après tests KS} \\ (6) }\\
\midrule
\endfirsthead
\caption[]{\label{tab:pestats}Statistiques descriptives (suite)}\\
\toprule
  & \makecell{\makecell{Complètes \\ \ } \\ (1) } & \makecell{\makecell{Sans classes atypiques \\ \ } \\ (2) } & \makecell{\makecell{Avantagées \\ \ } \\ (3) } & \makecell{\makecell{Désavantagées \\ \ } \\ (4) } & \makecell{\makecell{Sans classes atypiques, \\ après tests KS} \\ (5) } & \makecell{\makecell{Établissements exclus \\ après tests KS} \\ (6) }\\
\midrule
\endhead

\endfoot
\bottomrule
\insertTableNotes
\endlastfoot
\addlinespace[0.3em]
\multicolumn{7}{l}{\textbf{}}\\
\hspace{1em}Nombre d'établissements & 82 & 82 & 16 & 78 & 77 & 5\\
\hspace{1em}Nombre de classes & 7.22 (2.05) & 6.02 (1.89) & 1.33 (0.55) & 1.07 (0.29) & 5.95 (1.9) & 7.07 (1.49)\\
\hspace{1em}Taille de classe & 24.55 (3.15) & 24.8 (3) & 26.78 (3.34) & 21.99 (2.81) & 24.77 (2.99) & 25.22 (3.06)\\
\addlinespace[0.3em]
\multicolumn{7}{l}{\textbf{}}\\
\hspace{1em}Note totale (DNB) & 10.28 (3.64) & 10.21 (3.74) & 12.58 (3.21) & 9.88 (2.48) & 10.19 (3.71) & 10.58 (4.12)\\
\hspace{1em}Note en français & 11.16 (3.89) & 11.17 (3.95) & 13.95 (3.25) & 10.17 (3.06) & 11.17 (3.93) & 11.16 (4.28)\\
\hspace{1em}Note en mathématiques & 9.33 (4.43) & 9.23 (4.51) & 11.13 (4.22) & 9.2 (3.49) & 9.18 (4.48) & 9.87 (4.8)\\
\addlinespace[0.3em]
\multicolumn{7}{l}{\textbf{}}\\
\hspace{1em}Note totale (CM2) & 52.55 (20.82) & 54 (20.41) & 66.34 (17.82) & 35.9 (15.09) & 53.82 (20.29) & 56.46 (21.69)\\
\hspace{1em}Effectif (proportion) de retrouvés & 33032 (0.78) & 27902 (0.78) & 797 (0.84) & 3701 (0.75) & 25937 (0.78) & 1965 (0.75)\\
\addlinespace[0.3em]
\multicolumn{7}{l}{\textbf{}}\\
\hspace{1em}Âge au DNB & 15.16 (0.51) & 15.13 (0.49) & 14.99 (0.39) & 15.46 (0.55) & 15.13 (0.49) & 15.14 (0.52)\\
\addlinespace[0.3em]
\multicolumn{7}{l}{\textbf{Position}}\\
\hspace{1em}En retard & 0.19 & 0.16 & 0.03 & 0.44 & 0.16 & 0.18\\
\hspace{1em}À l'heure & 0.79 & 0.81 & 0.92 & 0.56 & 0.82 & 0.78\\
\hspace{1em}En avance & 0.02 & 0.02 & 0.05 & 0 & 0.02 & 0.03\\
\addlinespace[0.3em]
\multicolumn{7}{l}{\textbf{Sexe}}\\
\hspace{1em}Filles & 0.5 & 0.51 & 0.59 & 0.43 & 0.51 & 0.5\\
\hspace{1em}Garçons & 0.5 & 0.49 & 0.41 & 0.57 & 0.49 & 0.5\\
\addlinespace[0.3em]
\multicolumn{7}{l}{\textbf{CSP}}\\
\hspace{1em}Défavorisés & 0.56 & 0.54 & 0.42 & 0.77 & 0.55 & 0.49\\
\hspace{1em}Moyens & 0.24 & 0.25 & 0.27 & 0.17 & 0.25 & 0.27\\
\hspace{1em}Favorisés & 0.08 & 0.08 & 0.1 & 0.03 & 0.08 & 0.07\\
\hspace{1em}Très favorisés & 0.12 & 0.12 & 0.21 & 0.02 & 0.12 & 0.17\\
\hspace{1em}Autres & 0.01 & 0.01 & 0 & 0.01 & 0.01 & 0.01\\
\addlinespace[0.3em]
\multicolumn{7}{l}{\textbf{Régime scolaire}}\\
\hspace{1em}Demi-pensionnaires & 0.6 & 0.61 & 0.67 & 0.49 & 0.62 & 0.51\\
\hspace{1em}Internes & 0 & 0 & 0 & 0 & 0 & 0\\
\hspace{1em}Externes & 0.4 & 0.39 & 0.33 & 0.51 & 0.38 & 0.49\\
 &  &  &  &  &  & \\
Observations & 42606 & 35975 & 949 & 4908 & 33345 & 2630\\*
\end{longtable}
\end{ThreePartTable}
\endgroup{}
\end{landscape}

\hypertarget{pemethods}{%
\section{Méthodologie}\label{pemethods}}

\hypertarget{pemethodss13}{%
\subsection{\texorpdfstring{Estimation d'une forme réduite avec le modèle de Sojourner (\protect\hyperlink{ref-SOJ:13}{2013})}{Estimation d'une forme réduite avec le modèle de Sojourner (2013)}}\label{pemethodss13}}

Nous cherchons à mesurer l'influence du niveau scolaire des pairs en 3\textsuperscript{ème} sur les résultats au DNB. Les données à notre disposition sont parfaitement adaptées au modèle de Sojourner (\protect\hyperlink{ref-SOJ:13}{2013}) puisque nous disposons d'une note antérieure des candidats au DNB (la note au CM2), avec 22\% de valeurs manquantes (voir Section \ref{pedata})\footnote{Les données utilisées par Sojourner (\protect\hyperlink{ref-SOJ:13}{2013}) contiennent 40\% de valeurs manquantes de la note antérieure.}. Ces valeurs manquantes ne semblent pas aléatoires, vu que le Tableau \ref{tab:pecompretr} illustre des différences en note au DNB et en caractéristiques prédéterminées (c'est-à-dire déterminées avant les résultats au DNB) entre les élèves pour lesquels nous observons la note au CM2 (les retrouvés) et ceux pour lesquels nous ne l'observons pas (les non retrouvés). Les non retrouvés ont de moins bons résultats au DNB, sont plus âgés, ont tendance à être en retard, sont plus souvent des garçons, sont plus souvent issus de catégories sociales défavorisées et suivent plus souvent un régime scolaire externe. Toutefois, à part pour la position (proportion d'élèves en retard), ces différences statistiquement significatives ne semblent pas économiquement importantes et proviennent probablement de la taille importante de l'échantillon. Autrement dit, les élèves retrouvés et non retrouvés ne sont probablement pas pratiquement si différents.\\
Un des avantages du modèle est que ce caractère non aléatoire des valeurs manquantes sur le niveau initial n'empêche pas l'estimation convergente des effets de pairs.

\begingroup\fontsize{8}{10}\selectfont

\begin{ThreePartTable}
\begin{TableNotes}
\item \textit{Sources :} Fichiers DNB (2014-2016), fichiers CM2 (2010-2012), calculs de l'auteur.
\item \textit{Notes :} Les notes au DNB sont des notes sur 20. Les probabilités critiques correspondent à des tests de comparaison de moyennes pour les variables quantitatives et à des tests de comparaison de proportion pour les variables qualitatives. CSP : catégorie socio-professionnelle.
\item Significativité : 10\% * 5\% ** 1\% ***.
\end{TableNotes}
\begin{longtable}[t]{llll}
\caption{\label{tab:pecompretr}Comparaison retrouvés vs. non retrouvés}\\
\toprule
  & \makecell{\makecell{Retrouvés \\ \ } \\ (1) } & \makecell{\makecell{Non retrouvés \\ \ } \\ (2) } & \makecell{\makecell{(1) = (2) : \\ probabilité critique} \\ (3) }\\
\midrule
\endfirsthead
\caption[]{\label{tab:pecompretr}Comparaison retrouvés vs. non retrouvés (suite)}\\
\toprule
  & \makecell{\makecell{Retrouvés \\ \ } \\ (1) } & \makecell{\makecell{Non retrouvés \\ \ } \\ (2) } & \makecell{\makecell{(1) = (2) : \\ probabilité critique} \\ (3) }\\
\midrule
\endhead

\endfoot
\bottomrule
\insertTableNotes
\endlastfoot
Note aux écrits (DNB) & 12.127 & 11.602 & 0***\\
Note en français (DNB) & 11.22 & 10.962 & 0***\\
Note en mathématiques (DNB) & 9.426 & 9 & 0***\\
Âge DNB & 15.131 & 15.283 & 0***\\
En retard & 0.157 & 0.302 & 0***\\
Garçons & 0.49 & 0.517 & 0***\\
CSP défavorisées & 0.558 & 0.578 & 0***\\
Régimes scolaires externes & 0.394 & 0.407 & 0.058*\\
 &  &  & \\
Observations & 33032 & 9574 & \\*
\end{longtable}
\end{ThreePartTable}
\endgroup{}

Pour alléger les notations, considérons une année de DNB donnée quelconque. Soit \(i\), \(c\) et \(e\) les indices représentant respectivement l'élève, la classe et l'établissement. Pour analyser conjointement les effets fixes et les moyennes des variables chez les pairs, nous écrirons systématiquement ces indices. Nous empruntons la démarche de Sojourner (\protect\hyperlink{ref-SOJ:13}{2013, p. 577}) et modélisons la performance scolaire en fin de collège en fonction des caractéristiques des pairs, des individus et de la classe. Pour prendre en compte de potentiels effets corrélés de groupes (comme la qualité matérielle de la classe et les caractéristiques de l'enseignant), nous devrions \emph{a priori} introduire des effets fixes de classe. L'équation d'intérêt peut s'écrire comme suit :

\begin{equation}
\label{eq:pes130}
y_{ice}^{dnb} = \rho_{ce} + \rho_1 \bar{v}_{-ice} + f_{ice} + \omega_{ice}.
\end{equation}

Dans l'équation \eqref{eq:pes130}, \(y^{dnb}_{ice}\) représente la note au DNB de l'individu \(i\) qui se trouve dans la classe \(c\) de l'établissement \(e\). Elle est exprimée en unité d'écart-type (UET), la moyenne et l'écart-type étant calculés par année scolaire de 3\textsuperscript{ème}\footnote{Ce qui évite tout souci de comparabilité des examens entre les années.}. La grande majorité (99.3\%) des individus de l'échantillon sont observés une fois. 289 élèves sont observés deux fois. Il s'agit d'élèves ayant redoublé leur 3\textsuperscript{ème}. La variable \(\bar{v}_{-ice}\) représente une mesure d'intérêt chez les pairs. Elle se calcule comme suit :

\[
\bar{v}_{-ice} = \frac{\displaystyle\sum_{j \in ce} \nu_{jce} - \nu_{ice}}{t_{ce} - 1}.
\]

L'indice \(-ice\) reflète le fait que l'individu en question est exclu du calcul de la moyenne chez les pairs. Le terme \(t_{ce}\) désigne la taille de la classe \(c\) de l'établissement \(e\).

\quad La définition générale de \(\nu_{ice}\) nous permet d'être flexible selon la nature et la spécification des effets de pairs que nous souhaitons étudier. Par exemple, si nous souhaitons étudier les effets de pairs linéaires en moyenne, \(\nu_{ice} \equiv y^{cm2}_{ice}\), \(y^{cm2}_{ice}\) désignant la note au CM2 de l'élève. Si nous souhaitons étudier les effets de pairs non linéaires, \(\nu_{ice} \equiv (Q1_{ice}, Q2_{ice}, Q4_{ice}, Q5_{ice})'\), \(Qn_{ice}, n = \{1, 2, 4, 5\}\) étant la variable dichotomique de la présence de l'individu dans le quintile \(n\)\footnote{\(Q3_{ice} = 1\) est la valeur de référence.}.

\quad Pour alléger les notations, nous définissons \(f_{ice} \equiv \rho_2 y^{cm2}_{ice} + x_{ice}' \rho_3 + \bar{x}_{-ice}' \rho_4 + \rho_5 t_{ce}\). Les caractéristiques individuelles sont représentées par le vecteur \(x_{ice}\) et leurs moyennes chez les pairs par \(\bar{x}_{-ice}\). Le terme \(f_{ice}\) condense simplement le rôle de ces caractéristiques dans la détermination de \(y^{dnb}_{ice}\). Cela n'affecte pas notre raisonnement.

Le paramètre \(\rho_{ce}\) représente l'effet fixe au niveau de la classe discuté plus haut. Le terme d'erreur \(\omega_{ice}\) regroupe tous les facteurs inobservés par le chercheur qui déterminent \(y^{dnb}_{ice}\). Notre paramètre d'intérêt est \(\rho_1\). Dans le cas où \(\bar{v}_{-ice}\) désigne la moyenne de la note au CM2 des pairs de 3\textsuperscript{ème}, \(\rho_1\) est l'effet de l'augmentation d'une unité d'écart-type du niveau scolaire moyen des pairs de 3\textsuperscript{ème} sur les notes au DNB.

\quad Les pairs étant définis au niveau de la classe, utiliser des effets fixes de classe fait perdre la quasi-totalité de la variation de \(\bar{v}_{-ice}\) et empêche l'identification de \(\rho_1\) (\protect\hyperlink{ref-AMM:PIS:09}{Ammermueller \& Pischke, 2009}). En effet, \(\bar{v}_{-ice}\) (une moyenne sur la classe en enlevant un individu) est très proche de la moyenne de la classe \(\bar{v}_{ce}\). Utiliser des effets fixes de classe revient à estimer l'équation \eqref{eq:pes130} en transformant les variables en écarts à la moyenne de la classe. Par conséquent, \(\bar{v}_{-ice} - \bar{v}_{ce}\) est très proche de zéro pour tous les individus et \(\rho_1\) sera donc difficilement identifié.

\quad Une hypothèse permet de surmonter ce problème. Il s'agit de l'hypothèse selon laquelle les inobservables sont en moyenne égales au niveau de chaque classe et donc au niveau de l'établissement. L'assignation aléatoire des classes, au sein de chaque établissement, donne du crédit à cette hypothèse. Sous cette hypothèse, l'idée est qu'au sein de chaque établissement, les caractéristiques des différentes classes (qualité matérielle, caractéristiques des enseignants, par exemple) et des élèves (facultés, caractéristiques familiales, par exemple) sont identiques et équivalent donc aux caractéristiques au niveau de l'établissement. Nous pouvons alors utiliser des effets fixes d'établissement à la place des effets fixes de classe. Cela surmonte la perte de variation susmentionnée tout en prenant en compte les effets corrélés au niveau de la classe. Nous présentons les procédures empiriques qui permettent de justifier cette hypothèse dans la Section \ref{pemethodsrestr}. En notant \(\rho_e\) l'effet fixe établissement, l'équation d'intérêt devient :

\begin{equation}
\label{eq:pes13}
y_{ice}^{dnb} = \rho_e + \rho_1 \bar{v}_{-ice} + f_{ice} + \omega_{ice}.
\end{equation}

Désormais, nous définissons \(\bar{\nu}^{obs}_{-ice}\) et \(\bar{\nu}^m_{-ice}\) la partie observée et la partie non observée de \(\bar{v}_{-ice}\) ; et \(p_{-ice}\) la proportion de retrouvés parmi les pairs. Pour chaque individu, la vraie moyenne chez les pairs est une moyenne pondérée de sa partie observée et manquante, les poids étant \(p_{-ice}\) et son complément \((1 - p_{-ice})\), respectivement. Formellement,

\begin{equation}
\label{eq:pevdecomp}
\bar{v}_{-ice} = \bar{v}^{obs}_{ice} p_{-ice} + \bar{v}^m_{-ice} (1 - p_{-ice}).
\end{equation}

\quad Le modèle de Sojourner (\protect\hyperlink{ref-SOJ:13}{2013}) propose une version dans laquelle toutes les observations (y compris celles avec des valeurs manquantes de \(y^{cm2}_{ice}\)) peuvent être utilisées pour l'estimation de \(\rho_1\) et une autre qui n'exploite que la partie observée (\(obs_{ice} = 1\), ci-dessous). Notre étude se limite uniquement au second cas.

\quad Dans ce qui suit, nous notons l'ensemble du conditionnement de l'espérance \(C \equiv (e, \bar{v}^{obs}_{-ice}, p_{-ice}, y^{cm2}_{ice}, x_{ice}, \bar{x}_{-ice}, t_{ce}, obs_{ice} = 1)\), \(obs_{ice}\) étant une indicatrice de l'observation de la note au CM2 de l'individu \(i\). L'identification et l'estimation de \(\rho_1\) peuvent être illustrées en commençant par la régression de population obtenue à partir des équations \eqref{eq:pes13} et \eqref{eq:pevdecomp} :

\begin{equation}
\label{eq:pes13pop}
\begin{aligned}
&E(y^{dnb}_{ice} \mid C) &= 
E \big[ \rho_e &+ \rho_1 (\bar{v}^{ob}_{-ice} p_{-ice} + \bar{v}^m_{-ice} (1 - p_{-ice})) \\ 
&&&+ f_{ice} + \omega_{ice} \mid C \big] \\
&&= \rho_e &+ \rho_1 \bar{v}^{obs}_{-ice} p_{-ice} + \rho_1 (1 - p_{-ice}) E(\bar{v}^m_{-ice} \mid C) \\
&&&+ f_{ice} + E(w_{ice} \mid C).
\end{aligned}
\end{equation}

Il est montré dans le papier de référence de notre modèle que \(\rho_1\) n'est pas identifié à partir de l'équation \eqref{eq:pes13pop} uniquement. Deux hypothèses sont nécessaires pour y arriver, la première requiert que la moyenne chez les pairs de la partie manquante de \(v_{ice}\) ne dépende que de l'école et la seconde, plus standard, requiert l'absence de corrélation, au sein de chaque établissement et pour les retrouvés, entre les inobservables \(\omega_{ice}\) et les observables \(\bar{v}^{obs}_{-ice}, p_{-ice}, y^{cm2}_{ice}, x_{ice}, \bar{x}_{-ice}, t_{ce}\). Il est clair que l'assignation aléatoire des classes au sein de chaque établissement donne également du crédit à ces deux autres hypothèses.

\quad Pour montrer comment ces hypothèses amènent à l'identification du paramètre d'intérêt, il convient de les écrire formellement :

\begin{equation}
\label{eq:pemi1}
\begin{aligned}
&E(\bar{v}^m_{-ice} \mid e, \bar{v}^{obs}_{-ice}, p_{-ice}, y^{cm2}_{ice}, x_{ice}, \bar{x}_{-ice}, t_{ce}, obs_{ice}) = 
E(v^m \mid e) \\
&\Rightarrow
E(\bar{v}^m_{-ice} \mid C) = E(v^m \mid e)
\end{aligned}
\end{equation}

et

\begin{equation}
\label{eq:pemi2}
E(\omega_{ice} \mid C) = E(\omega_{ice} \mid e, obs_{ice} = 1).
\end{equation}

\quad Sous les hypothèses illustrées par les équations \eqref{eq:pemi1} et \eqref{eq:pemi2} et à partir de la dernière égalité de l'équation \eqref{eq:pes13pop}, nous avons :

\begin{equation}
\label{eq:pes13popident}
\begin{aligned}
E(y^{dnb}_{ice} \mid C) = 
\rho_e &+ \rho_1 \bar{v}^{obs}_{-ice} p_{-ice} - \rho_1 E(v^m \mid e) p_{-ice} + \rho_1 E(v^m \mid e) \\
&+ f_{ice} + 
E(w_{ice} \mid e, obs = 1)
\end{aligned}
\end{equation}

Si \(I_e\) désigne un vecteur d'indicatrices d'appartenance à l'établissement \(e\), l'équation \eqref{eq:pes13popident} est identifiée dans la mesure où un effet fixe d'établissement et des régresseurs \(p_{-ice} \times I_e\) sont utilisés dans l'estimation afin d'absorber l'effet des quantités inobservées \(\rho_1 E(v^m \mid e)\) et \(- \rho_1 E(v^m \mid e) p_{-ice}\), respectivement.

\quad Un souci avec cette procédure d'estimation est que les régresseurs \(p_{-ice} I_e\) sont très corrélés avec \(\rho_1 \bar{v}^{obs}_{-ice}\) et les effets fixes d'établissement, faisant potentiellement drastiquement augmenter la variance des paramètres estimés. Une manière de surmonter cela est d'utiliser des blocs d'établissements au lieu de \(I_e\) dans \(p_{-ice} I_e\) au prix d'introduire du bais dans les estimations. Ces blocs peuvent être obtenus en groupant les établissement selon leur position dans la distribution du niveau scolaire moyen par établissement (\protect\hyperlink{ref-SOJ:13}{Sojourner, 2013, p. 581}). Le papier de référence montre que le meilleur compromis biais-variance est de grouper les établissements dans un seul bloc, c'est-à-dire d'utiliser \(p_{-ice}\) comme régresseur à la place de \(p_{-ice} I_e\). Des simulations préliminaires avec nos données le confirment et nous retenons cette option.

\quad Afin d'étudier les effets du niveau scolaire des pairs, nous estimons donc des équations du type :

\begin{equation}
\label{eq:pes13reg0}
y^{dnb}_{ice} = \rho_e + \rho_1 \bar{v}^{obs}_{-ice} p_{-ice} + f_{ice}+ \rho_p p_{-ice} + \omega_{ice}.
\end{equation}

Nous commençons par nous intéresser au cas \(v_{ice} \equiv y^{cm2}_{ice}\), qui correspond à une spécification linéaire en moyenne des effets de pairs, notre équation d'intérêt estimable s'écrit alors :

\begin{equation}
\label{eq:pes13reg}
y^{dnb}_{ice} = \rho_e + \rho_1 \bar{v}^{obs}_{-ice} p_{-ice} + \rho_2 y^{cm2}_{ice} + x_{ice}' \rho_3 + \bar{x}_{-ice}' \rho_4 + \rho_5 t_{ce} + \rho_p p_{-ice} + \omega_{ice}.
\end{equation}

L'équation \eqref{eq:pes13reg} s'estime par moindres carrés ordinaires après avoir transformé les variables en leurs écarts par rapport à la moyenne de l'établissement. Elle impose que les effets du niveau scolaire des pairs est le même indépendamment du niveau de l'individu et n'agissent que via la moyenne du niveau scolaire.

\quad Il est établi que ce modèle n'est pas intéressant en termes de politiques publiques puisque s'il reflète la réalité, aucune politique de réarrangement des classes selon le niveau scolaire n'augmenterait la performance totale de l'établissement (\protect\hyperlink{ref-HOX:WEI:05}{Hoxby \& Weingarth, 2005}, par exemple). En effet, au sein d'un établissement, les gains obtenus par les classes accueillant des élèves en moyenne plus forts sont exactement compensés par les pertes subies par les classes d'origine de ces élèves. Une large part des résultats de la littérature confirment en plus l'existence d'effets de pairs hétérogènes (qui varient avec le niveau de l'individu) ou non linéaires (qui varient avec le niveau des pairs) (\protect\hyperlink{ref-BUR:SAS:13}{Burke \& Sass, 2013} ; \protect\hyperlink{ref-HAN:eal:03}{Hanushek et al., 2003} ; \protect\hyperlink{ref-BOU:MAI:18}{Boutchenik \& Maillard, 2018}). À l'instar de Hoxby \& Weingarth (\protect\hyperlink{ref-HOX:WEI:05}{2005}), Burke \& Sass (\protect\hyperlink{ref-BUR:SAS:13}{2013}) ou Sojourner (\protect\hyperlink{ref-SOJ:13}{2013, p. 598}), nous estimons deux équations supplémentaires pour prendre en compte des effets de pairs hétérogènes dans le premier cas et hétérogènes et non linéaires dans le second. Ils permettent de tenter de connaître la forme la plus compatible des effets de pairs (Section \ref{peintro}) en éducation à La Réunion avec les données.

\quad Pour mesurer les effets de pairs hétérogènes, nous construisons une mesure discrète du niveau scolaire : le quintile de la note au CM2, désigné par \(Qn_{ice}, n = \{1, 2, ..., 5\}\). Ces quintiles sont calculés par année scolaire de 3\textsuperscript{ème} pour les élèves dont la note de CM2 est connue ; et non par année scolaire au CM2. Ce choix est justifié par le fait que nous devrons calculer des proportions de pairs (de 3\textsuperscript{ème}) appartenant à ces quintiles dans les modèles d'effets de pairs non linéaires ci-dessous. La somme de ces proportions pour un individu vaudra alors exactement 100\%. Cela ne serait pas le cas si les quintiles étaient calculés par année scolaire au CM2. La variable indicatrice \(Q1_{ice}\), par exemple, vaut 1 si l'individu \(i\) de la classe \(c\) de l'établissement \(e\), pour une année de 3\textsuperscript{ème} donnée, appartient au premier quintile de la distribution de la note au CM2 pendant l'année scolaire en question. Nous retenons cette division en 5 parties pour avoir des notions de niveau scolaire ``très faible'' (\(Q1\)), ``faible'' (\(Q2\)), ``moyen'' (\(Q3\)), ``fort'' (\(Q4\)) et ``très fort'' (\(Q5\))\footnote{Nous analysons la robustesse de nos résultats à un découpage alternatif de la distribution du niveau scolaire.}. L'équation correspondant à ce modèle d'effets de pairs hétérogènes est donc ~:

\begin{equation}
\label{eq:pes13h5reg}
y^{dnb}_{ice} = \rho_e + \displaystyle\sum^5_{n = 1} \rho_{1Qn} (\bar{v}^{obs}_{-ice} p_{-ice})Qn_{ice} + \rho_2 y_{ice}^{cm2} + x_{ice}' \rho_3 + \bar{x}_{-ice}' \rho_4 + \rho_5 t_{ce} + \rho_p p_{-ice} + w_{ice}. 
\end{equation}

Dans l'équation \eqref{eq:pes13h5reg}, nous avons toujours \(v^{obs}_{ice} \equiv y^{cm2}_{ice}\) mais l'effet de pair peut désormais varier en fonction du niveau de l'individu. Le paramètre \(\rho_{1Qn}\) désigne alors l'effet de l'augmentation d'une UET du niveau scolaire moyen des pairs pour un élève du quintile \(Qn\).

\quad Pour analyser les effets de pairs hétérogènes et non linéaires, nous nous demandons comment différents niveaux des pairs influencent différents niveaux d'individus. Pour ce faire, nous définissons \(\bar{Qm}_{-ice}\) comme la proportion de pairs appartenant au quintile \(Qm\) de la note au CM2. Cela équivaut à autoriser les effets de pairs à être non linéaires en moyenne, c'est-à-dire que chaque partie de la distribution du niveau scolaire des pairs a un effet spécifique. Puis, nous autorisons chaque effet non linéaire à varier en fonction du niveau de l'individu. L'équation estimable prend alors la forme :

\begin{equation}
\label{eq:pes13hnolinq5reg}
y^{dnb}_{ice} = \rho_e + \displaystyle\sum_{m \in \{1, 2, 4, 5\}} 
[\displaystyle\sum_{n = 1}^5 \rho_{1\bar{Qm}, Qn} (\bar{Qm}_{-ice} p_{-ice}) Qn_{ice}] + \rho_2 y_{ice}^{cm2} + x_{ice}' \rho_3 + \bar{x}_{-ice}' \rho_4 + \rho_5 t_{ce} + \rho_p p_{-ice} + w_{ice}. 
\end{equation}

L'équation \eqref{eq:pes13hnolinq5reg} n'est identifiée qu'en omettant toute interaction avec la proportion des pairs dans un des quintiles. Notre choix se porte sur \(\bar{Q3}_{-ice}\). De plus, dans les résultats, nous interprétons plutôt les \(\frac{\rho_{1\bar{Qm},Qn}}{10}\). Ces choix rendent plus intuitif l'interprétation des coefficients puisque \(\frac{\rho_{1\bar{Qm}, Qn}}{10}\) représente l'effet de l'augmentation de 10 points de pourcentage de la proportion de pairs au quintile \(Qm\), en échange de pairs au \(Q3\) (valeur de référence) de la même proportion, pour un individu au quintile \(Qn\). Des choix similaires sont faits par Hoxby \& Weingarth (\protect\hyperlink{ref-HOX:WEI:05}{2005}) et Burke \& Sass (\protect\hyperlink{ref-BUR:SAS:13}{2013}).\\
Nous confirmons que le modèle de Sojourner (\protect\hyperlink{ref-SOJ:13}{2013}) peut être utilisé pour l'estimation des effets de pairs hétérogènes et non linéaires (p.~598).

\quad Les limites que nous avons pu identifier de la modélisation présentée jusqu'ici sont les suivantes. L'idéal aurait été de partir d'un modèle théorique formel (\protect\hyperlink{ref-EPP:ROM:11}{Epple \& Romano, 2011}) pour arriver aux équations estimées. Cela aurait permis de prédire clairement les résultats en fonction des paramètres du modèle théorique. Aussi, l'équation estimable \eqref{eq:pes13hnolinq5reg} n'est pas un cas général des équations \eqref{eq:pes13h5reg} (effets de pairs hétérogènes) et \eqref{eq:pes13reg}\footnote{Tout de même, l'équation \eqref{eq:pes13h5reg} est un cas plus général de l'équation \eqref{eq:pes13reg} dans la mesure où dans la seconde, l'effet de pairs est contraint d'être le même pour les individus de tout niveau.}.\\

\quad En pratique, nous disposons de trois années scolaires de 3\textsuperscript{ème} et empilons les données correspondantes pour estimer les différentes équations. Les effets fixes de classes (d'établissements) doivent être compris comme des effets de classes-années (d'établissements-années).

\hypertarget{pemethodsrestr}{%
\subsection{L'hypothèse d'assignation aléatoire des classes au sein de chaque établissement : une procédure de restriction d'échantillon pour la crédibiliser}\label{pemethodsrestr}}

Dans nos données de départ, au sein d'un établissement, les élèves ne peuvent pas être considérés comme assignés aux classes de manière aléatoire. En effet, en plus de l'existence des sections avantagées et désavantagées révélée dans la Section \ref{peinst}, quasiment tous les établissements regroupent explicitement des élèves de ces sections dans les mêmes classes\footnote{Une nuance à faire est qu'une partie des élèves des sections désavantagées et avantagées se trouvent dans les mêmes salles de classe que les élèves des sections normales.}. C'est le cas d'une classe qui ne contient que des élèves en préparation professionnelle, par exemple. Désignons ces classes par classes atypiques. Il existe alors des classes atypiques qui regroupent exclusivement des élèves de sections désavantagées (classes atypiques désavantagées désormais) ou avantagées (classes atypiques avantagées désormais). Ces classes sont détectées dans la quasi-totalité des établissements. \\
\quad Sur les 3 années, parmi les 1776 classes-années identifiées, 241 (13.6\%) sont des classes atypiques désavantagées et 61 (3.4\%) des classes atypiques avantagées. Le reste est constitué de classes composées uniquement de sections normales (33.8\%), de sections normales et avantagées (47\%), de sections désavantagées et normales (0.9\%) et des trois types de sections (1.1\%).

\quad Le regroupement exclusif des sections désavantagées (respectivement avantagées) entre elles correspond à une sélection des élèves avec de moins bonnes (respectivement meilleures) performances au DNB. Aussi, le regroupement des élèves similaires en inobservables entre eux entraîne une corrélation positive entre le terme d'erreur de l'équation estimée et la note des pairs. Il semble alors difficile d'anticiper d'avance le biais causé par les classes atypiques.
Pour s'affranchir de ce biais, nous retirons les classes atypiques de l'échantillon avant d'effectuer les estimations\footnote{Ammermueller \& Pischke (\protect\hyperlink{ref-AMM:PIS:09}{2009}) ou encore Boutchenik \& Maillard (\protect\hyperlink{ref-BOU:MAI:18}{2018}) emploient une stratégie similaire.}. L'idée est de crédibiliser l'hypothèse d'assignation aléatoire des classes au sein de chaque établissement-année. Nos résultats peuvent alors perdre en validité externe puisqu'ils ne concernent que les élèves hors classes atypiques. Toutefois, du fait qu'il existe une grande majorité de classes mixant sections normales avec sections désavantagées et avantagées (ci-dessus), nous pensons que la perte en validité externe en question n'est pas très grave.

\quad L'échantillon restant est résumé par la deuxième colonne du Tableau \ref{tab:pestats}. Nous constatons une augmentation claire de la note au CM2 entre l'échantillon original (colonne 1) et celui sans classes atypiques vu que celle-ci passe de 52 à 54 points. Cette augmentation est due à la prévalence des classes atypiques désavantagées par rapport aux classes atypiques avantagées. La proportion d'élèves en retard passe de 19\% à 16\% entre les deux échantillons. La proportion de CSP défavorisées baisse de 2 points de pourcentage dès que les classes atypiques sont exclues.

\quad Une limite potentielle de l'exclusion des classes atypiques est qu'il peut y avoir des regroupements selon le niveau scolaire qui ne se traduisent pas par la formation de classes atypiques. Afin de contourner cette limite et de donner le maximum de crédit à l'hypothèse d'assignation aléatoire des classes, nous effectuons des tests de Kolmogorov-Smirnov (KS désormais) similaires à ceux réalisés par Brodaty \& Gurgand (\protect\hyperlink{ref-BRO:GUR:16}{2016}) et nous excluons les établissements dans lesquels nous détectons statistiquement des distributions de niveau scolaire trop différentes par rapport aux distributions que nous aurions obtenues sous l'hypothèse d'assignation aléatoire.\\
Le test compare au moins deux classes au sein d'un établissement. Les établissements ne comportant qu'une seule classe de 3\textsuperscript{ème} ne sont pas naturellement pas concernés.\\
Plus précisément, soit \(n_{c}\) le nombre de classes identifiées au sein d'un établissement-année, hors classes spécifiques. À une classe \(c\) donnée correspond une moyenne de note au CM2 parmi les retrouvés, \(\bar{y}^{cm2}_c\). À l'établissement-année correspond alors une distribution de moyennes de notes de CM2 par classe : \(D = \{\bar{y}^{cm2}_c\}_{c = 1, 2, ...n_c}\). Le test consiste à simuler \(R\) affectations aléatoires des élèves dans les classes au sein de l'établissement considéré puis à comparer la distribution simulée de notes moyennes au CM2 par classe avec la vraie distribution \(D\). Nous retenons \(R = 1000\).\\
Si nous désignons par \(D^r\) la distribution de notes moyennes au CM2 par classe calculée à partir de la \(r^{ème}\) simulation d'affectation aléatoire des élèves au sein des classes, nous nous retrouvons avec échantillon simulé \(D^* = \{D^r\}_{r = 1, ..., R}\), de taille \(n_c \times R\). Il s'agira alors d'évaluer si \(D\) et \(D^*\) sont similaires (l'hypothèse nulle est telle que \(D\) et \(D^*\) suivent la même distribution). Pour ce faire, la statistique de KS, qui correspond à la différence maximale (en valeur absolue) entre les fonctions de répartition de \(D\) et de \(D^*\), est calculée. Cette statistique suit une loi de KS.

\quad Au bout de la série de tests KS, pour chaque collège-année, nous obtenons une statistique de test (ci-dessus) et une probabilité critique. Dans nos données, quasiment tous les collèges sont observés sur les trois années scolaires. Cela influe notre règle de décision de l'exclure ou non de l'échantillon d'estimation qui utilise les travaux de Boutchenik \& Maillard (\protect\hyperlink{ref-BOU:MAI:18}{2018}) (voir leur annexe) pour décider correctement. Nous choisissons d'exclure un établissement lorsqu'il a été détecté comme effectuant le regroupement au moins deux années scolaires sur les trois (désignons ce choix par \(k = 2\) pour des raisons pratiques). La probabilité critique discutée plus haut correspondante avec une marge d'erreur de 5\% est de 0.1354. Autrement dit, il s'agira d'exclure les établissements qui, après la série de tests KS, rejettent au moins deux fois (dans deux années scolaires) l'hypothèse nulle selon cette probabilité critique.\\
Nous aurions pu retenir \(k = 1\) ou \(k = 3\). Le premier peut faire exclure les établissements qui auraient été détectés par nos tests sans qu'il n'y ait de problème d'endogénéité de formation des classes et le dernier nous paraît trop restrictive\footnote{Évidemment, la sensibilité des résultats en fonction de \(k\) est analysée.}.

\quad Les résultats de la restriction sont résumés par les colonnes (5) et (6) du Tableau \ref{tab:pestats}. 5 établissements ont été détectés comme regroupant les élèves dans les classes selon leur niveau scolaire (colonne 6). Ce sont tous des collèges publics. Dans ces établissements, sur les trois années, soit 15 établissements-années, 7 sont hors éducation prioritaire. Cela semble suggérer que le fait de pratiquer d'éventuels regroupement d'élèves dans les classes de 3\textsuperscript{ème} selon les classes ne dépend pas du statut d'éducation prioritaire. Le nombre d'observations des données résumées par la colonne (5) du Tableau \ref{tab:pestats} représentent 92.7\% des données sans classes atypiques (colonne 2).

\quad Il est clair que les 5 établissements exclus (colonne 6) diffèrent de la totalité (sans classes atypiques, colonne 2) puisque les élèves des premiers ont en moyenne plus de classes par établissement, des tailles de classe légèrement supérieures, un niveau scolaire supérieur\footnote{Rejet de l'hypothèse nulle d'un test d'égalité de moyenne de la note au CM2. Il faut toutefois rester prudent face à cette différence car les notes au CM2 ne sont pas comparables à travers les années scolaires de CM2. Pour être sûr qu'il y a bien une différence de niveau entre les élèves des établissements exclus et la totalité (hors classes atypiques), il faudrait s'assurer que nous trouvons les mêmes proportions de cohortes de CM2 entre les deux échantillons. Hors classes atypiques, parmi les retrouvés, nous comptons 31.5\% d'élèves qui étaient en CM2 en 2010, 36\% en 2011 et 32\% en 2012. Dans les établissements exclus, ces chiffres sont respectivement de 31\%, 36\% et 32.6\%. Ces deux groupes de proportions étant très similaires, nous pouvons être certain que les élèves des établissements exclus ont bien un niveau scolaire supérieur.}, plus d'élèves en retard (de manière étonnante) et moins d'élèves défavorisés.\\
Ces spécificités des élèves des établissements exclus apportent du crédit aux tests statistiques utilisés pour les exclure.

\quad Les estimations seront principalement effectuées sur l'échantillon sans classes atypiques et sans les établissements dans lesquels nous avons détecté des traces statistiques de regroupement des élèves dans les classes en fonction de leur niveau au CM2 au moins deux fois sur les trois années scolaires (colonne 4 du Tableau \ref{tab:pestats}).

\hypertarget{pemethodsbeal}{%
\subsection{Identification et estimation des effets de pairs endogènes}\label{pemethodsbeal}}

Les modèles précédents, basés sur des formes réduites, nous permettent au mieux d'estimer un mix des effets endogènes et exogènes, sans pouvoir les distinguer. Dans cette section, nous justifions que les effets endogènes peuvent être identifiés avec nos données, en suivant la méthodologie présentée dans Boucher et al. (\protect\hyperlink{ref-BOU:eal:14}{2014}) et originalement proposée par Lee (\protect\hyperlink{ref-LEE:07}{2007}). À notre connaissance, notre étude est la première à tenter d'estimer à la fois les effets de pairs en forme réduite et les effets endogènes \emph{per se}.

\quad Le fait de s'être préalablement intéressé aux équations de forme réduite avant celles des effets de pairs endogènes présente de multiples intérêts. Tout d'abord, estimer les effets endogènes se trouve être particulièrement difficile du fait du problème de la réflexion, ce qui justifie le fait d'estimer préalablement les effets mixtes pour s'assurer que des effets de pairs endogènes existent potentiellement (\protect\hyperlink{ref-MON:eal:19}{Monso et al., 2019}). Ensuite, la forme réduite fournit une possibilité d'analyser les différentes formes théoriques que pourraient prendre les effets de pairs, ce que ne permet pas le modèle que nous mobilisons pour les effets endogènes. Puis, les effets de pairs mesurés avec la forme réduite ont des implications en politiques publiques liées à la mixité scolaire, comme discuté dans la Section \ref{peintro}. Enfin, si les effets de pairs endogènes existent, leurs implications seraient d'une autre nature : des politiques éducatives ciblées à une partie de la population (moins coûteuses) desquelles l'ensemble de la population en bénéficiera à terme via les effets de pairs endogènes.

\quad Un avantage considérable de la méthodologie de Boucher et al. (\protect\hyperlink{ref-BOU:eal:14}{2014}) est qu'il est possible d'utiliser des effets fixes de classe. Rappelons que s'il y a une tendance à ce que les meilleurs enseignants soient assignés aux classes les plus fortes (hypothèse que nous pouvons difficilement tester), les effets de pairs mesurés selon la méthodologie présentée dans la Section \ref{pemethodss13} sont surestimés. La méthodologie développée dans cette section ne souffre pas de cette limite du fait des effets fixes de classe.\\
Un autre avantage de la présente approche est que nous pouvons étudier séparément les effets de pairs au CM2 et en 3\textsuperscript{ème}, ce qui permet de les comparer dans deux niveaux d'éducation très différents.

\quad Formellement, notre équation structurelle est la suivante (\protect\hyperlink{ref-BOU:eal:14}{Boucher et al., 2014} ; \protect\hyperlink{ref-IZA:DIC:20}{Izaguirre \& Di Capua, 2020}) :

\begin{equation}
\label{eq:pebeal}
y^{dnb}_{ice} = \beta_{ce} + \beta_1 \bar{y}^{dnb}_{-ice} + x'_{ice} \beta_2 + \bar{x}_{-ice}' \beta_3 + \psi_{ice}.
\end{equation}

Dans l'équation \eqref{eq:pebeal}, les indices et les variables \(x_{ice}\) et \(\bar{x}_{-ice}\) sont les mêmes que ceux définis dans la Section \ref{pemethodss13}. La variable \(\bar{y}^{dnb}_{-ice}\) représente la moyenne chez les pairs de 3\textsuperscript{ème} de la note au DNB. Le terme d'erreur usuel est \(\psi_{ice}\). Le terme \(\beta_{ce}\) absorbe l'effet de tout facteur inobservé spécifique à la classe et déterminant de la note au DNB. Le fait que des élèves similaires ont tendance à se regrouper entre eux est également pris en compte par cet effet fixe de classe. Notons que l'idéal aurait été de prendre également la note au CM2 comme régresseur mais les valeurs manquantes (voir Section \ref{pemethodss13}) constituent un obstacle puisque le modèle n'apporte des corrections que par rapport aux valeurs manquantes sur la variable dépendante (\protect\hyperlink{ref-BOU:eal:14}{Boucher et al., 2014, p. 96}).\\
Le paramètre \(\beta_1\) constitue notre paramètre d'intérêt. Il désigne l'effet de l'augmentation d'une UET du résultat moyen des pairs au DNB sur le résultat de l'individu. Comme mentionné dans la Section \ref{peintro}, il faut le comprendre comme l'effet du comportement moyen des pairs au cours de l'année scolaire sur les résultats de l'individu.

\quad L'identification repose sur une première hypothèse telle que la taille de classe est exogène, \emph{conditionnellement à \(\beta_{ce}\)}. Autrement dit, il autorise le fait que les élèves sont assignés à des classes de différentes tailles en fonction de facteurs de réussite scolaire que le chercheur n'observe pas. Mais une fois ces inobservables pris en compte, la taille de classe doit pouvoir être considérée comme exogène. Il nous semble difficile d'imaginer des situations où cette hypothèse ne serait pas crédible en France puisqu'à notre connaissance, la taille de classe ne dépend que des facteurs liés à la réussite scolaire.

\quad Afin d'analyser la question de l'identification, nous appliquons une transformation en écart par rapport à la moyenne de la classe à l'équation \eqref{eq:pebeal}. Pour toute variable \(a_{ice}\), notons \(\bar{a}_{ce}\) sa moyenne sur la classe. Par ailleurs, il est aisé de montrer que \((\overline{\bar{a}_{-ice}})_{ce} = \bar{a}_{ce}\)\footnote{\((\overline{\bar{a}_{-ice}})_{ce} = \frac{1}{t_{ce}} \displaystyle\sum_{i \in ce} (\frac{\displaystyle\sum_{i \in ce} a_{ice} - a_{ice}}{t_{ce} - 1}) = \frac{\displaystyle\sum_{i \in ce} t_{ce} \bar{a}_{ce} - \displaystyle\sum_{i \in ce} a_{ice}}{t_{ce}(t_{ce} - 1)} = \frac{t_{ce}^2 \bar{a}_{ce} - t_{ce} \bar{a}_{ce}}{t_{ce} (t_{ce} - 1)} = \frac{t_{ce} (t_{ce} - 1)}{t_{ce}(t_{ce} - 1}) \bar{a}_{ce} = \bar{a}_{ce}.\)}.

À partir de l'équation \eqref{eq:pebeal}, nous montrons dans l'Annexe \ref{pebealwithin} que la forme réduite en écart par rapport à la moyenne sur la classe est :

\begin{equation}
\label{eq:pebealwithin}
(y_{ice}^{dnb} - \bar{y}^{dnb}_{ce}) = (x_{ice}' - \bar{x}_{ce}') \frac{(\beta_2 - \frac{\beta_3}{t_{ce} - 1})}{(1 + \frac{\beta_1}{t_{ce} - 1})} + \frac{1}{(1 + \frac{\beta_1}{t_{ce} - 1})} (\psi_{ice} - \bar{\psi}_{ce}).
\end{equation}

Au sein de l'équation \eqref{eq:pebealwithin}, le terme composite associé à \((x_{ice}' - \bar{x}_{ce})\) est composé de trois paramètres \(\beta_1, \beta_2 \text{ et } \beta_3\)\footnote{Pour être plus précis, \(\beta_2\) et \(\beta_3\) sont des vecteurs de paramètres mais cela ne change rien à notre raisonnement.}. Pour les identifier séparément et de manière unique à partir de cette forme réduite, il faut au moins trois valeurs de taille de classe (\(t_{ce}\)). Nos données satisfont largement cette condition. Les tailles de classe varient de 2 à 33, ont une moyenne de 24 et un écart-type de 3.6\footnote{La valeur étonnante de 2 correspond à des classes de DIMA et de SEGPA.}.\\
\quad Le fait d'exclure l'individu dans le calcul de la moyenne chez les pairs est une condition implicite qui amène à l'identification puisque dans le cas contraire, l'effet des moyennes chez les pairs seraient tous absorbés par l'effet fixe de classe.

\quad En supposant que le terme d'erreur suit une loi normale, nous estimons les paramètres de l'équation \eqref{eq:pebealwithin} par maximum de vraisemblance conditionnelle (\protect\hyperlink{ref-LEE:07}{Lee, 2007} ; \protect\hyperlink{ref-BOU:eal:14}{Boucher et al., 2014}).

\hypertarget{peres}{%
\section{Résultats et discussions}\label{peres}}

\hypertarget{peresfr}{%
\subsection{Les effets du niveau initial des pairs : résultats de la forme réduite}\label{peresfr}}

Dans cette sous-section, nous présentons les résultats d'estimation des équations présentées dans la Section \ref{pemethodss13} avec les spécifications des effets de pairs linéaires en moyenne (Section \ref{peresfrlem}), hétérogènes (Section \ref{peresfrh5}), et hétérogènes et non linéaires (Section \ref{peresfrhnolinq5}). Ce schéma de présentation suit Hoxby \& Weingarth (\protect\hyperlink{ref-HOX:WEI:05}{2005}), entre autres. Il est pratique de s'intéresser en premier lieu à la spécification linéaire en moyenne vu qu'elle permet d'étudier simplement l'importance de la stratégie d'identification (prise en compte des données manquantes, effets établissements-années, suppression des classes atypiques et suppression des établissements avec regroupement des élèves dans les classes en fonction de leurs niveaux scolaires). Mobiliser ces trois types de modèles permet de s'assurer de la cohérence qualitative des résultats à travers ces-dits modèles.

\hypertarget{peresfrlem}{%
\subsubsection{Spécification des effets de pairs linéaires en moyenne}\label{peresfrlem}}

Le Tableau \ref{tab:pemodels0}, dont la version complète se trouve en Annexe \ref{pemodels0compl} (Tableau \ref{tab:pemodels0compl}), a pour but de démontrer l'importance des différents éléments de notre stratégie d'identification. Sauf mention contraire, les données utilisées sont celles sans classes atypiques et sans établissements groupant les élèves dans les classes en fonction de leurs niveaux scolaires.
Notons que Sojourner (\protect\hyperlink{ref-SOJ:13}{2013}) inclut les variables d'âge et de position dans ses régressions. Selon le chapitre précédent, nous avons des raisons de penser que l'âge et la position au DNB sont potentiellement endogènes, même conditionnellement à la note au CM2, dans la mesure où cette dernière serait un proxy imparfait des inobservables liés à la fois à l'âge et la position au DNB ainsi qu'à la réussite scolaire (\protect\hyperlink{ref-FRU:14}{Fruehwirth, 2014, p. 521}). Si ces deux variables sont étroitement liées au niveau scolaire des pairs, les inclure dans nos régression induirait un biais dans nos estimations des effets de pairs. Sinon, nos résultats devraient être robustes à leur inclusion ou non. Sojourner (\protect\hyperlink{ref-SOJ:13}{2013}) inclut également la taille de classe dans ses régressions. En France, nous savons qu'en moyenne, les élèves plus faibles sont assignés à des classes plus petites\footnote{Une simple régression (non reportée) de la taille de classe de 3\textsuperscript{ème} en fonction de la note au CM2 et de toutes les variables de contrôle rencontrées dans le Tableau \ref{tab:pemodels0} le confirme pour La Réunion et ce constat est robuste à l'inclusion d'effets fixes établissements-années ou non.}. Cette variable peut également induire un biais (vers le haut) dans nos estimations. Afin d'avoir une idée du signe et de l'ampleur de ces biais potentiels, nous montrerons de manière quasi-systématique les résultats de régression avec et sans ces trois variables potentiellement endogènes.

\quad Les deux premières colonnes du Tableau \ref{tab:pemodels0} présentent les résultats d'estimations sans prise en compte des données manquantes\footnote{C'est-à-dire sans inclure \(p\) et sans la faire interagir avec la note au CM2 des pairs dans les régressions.}, avec (colonne 1) et sans (colonne 2) effets fixes établissements-années et sans les trois variables potentiellement endogènes discutées plus haut (âge, position et taille de classe). Nous remarquons en premier la différence frappante entre les effets de pairs estimés avec ou sans effets fixes : 0.216 contre 0.069 UET, respectivement. Cette différence met en évidence le phénomène de formation endogène des pairs à travers les établissements et donc l'utilité de la stratégie basée sur les effets fixes. Les coefficients associés aux variables de contrôles calculées chez les pairs (Tableau \ref{pemodels0compl} de l'Annexe \ref{pemodels0compl}) diffèrent également de manière conséquente, ce qui renforce la proposition précédente. Par exemple, une augmentation de 10 points de pourcentage de la part de garçons parmi les pairs apparaît comme ayant un effet de - 0.018 UET si la formation endogène des pairs n'est pas prise en compte (sans effets fixes) tandis que cet effet devient nul si elle l'est. Les coefficients des variables de contrôle individuels sont en revanche stables.

\quad Nous évaluons ensuite le biais lié à la prise en compte des données manquantes en comparant les résultats obtenus selon que \(p\) est incluse comme régresseur et interagit en même temps avec la note au CM2 des pairs (colonne 3 et 4) ou rien de tout cela (colonnes 1 et 2). L'effet de pairs obtenu dans le second cas est de 0.076 UET et 0.264 UET sans et avec effets fixes, respectivement. Ces chiffres sont légèrement supérieurs à ceux obtenus dans les régressions sans \(p\). Ce constat correspond à la prédiction selon laquelle les effets de pairs mesurés sans prise en compte des données manquantes tendent vers zéro au fur et à mesure que la proportion de pairs non retrouvés augmente (\protect\hyperlink{ref-SOJ:13}{Sojourner, 2013, p. 579}). Dans notre cas, la part des non retrouvés est de 22\%. Cela suggère alors que cette valeur induit un biais relativement limité. Le constat est d'autant plus cohérent avec le fait que la proportion de non retrouvés dans le papier de référence est d'environ 40\% avec un biais de sous-estimation de non prise en compte des données manquantes considérablement plus grand (voir \protect\hyperlink{ref-SOJ:13}{Sojourner, 2013}, Tableau 5).
De manière rassurante, en comparant la colonne (1) avec la colonne (3) et la colonne (2) avec la colonne (4), nous constatons les coefficients des variables de contrôles individuels et chez les pairs sont similaires que l'on prenne les données manquantes en compte ou non.

\quad Les colonnes (5) et (6) correspondent à des régressions dans lesquelles l'échantillon est augmenté des élèves des classes atypiques avantagées et désavantagées, respectivement. Nous constatons que le rajout des classes atypiques avantagées modifie à peine les résultats (comparaison de la colonne 4 avec la colonne 5) tandis que celui des classes atypiques désavantagées induit un biais de sous-estimation pouvant jusqu'à changer le signe de l'effet de pairs (comparaison de la colonne 4 avec la colonne 6). En plus, les autres coefficients de la colonne 6 (avec classes atypiques désavantagées) sont profondément différents de ceux de la colonne 4 (sans classes atypiques). Cette asymétrie des biais induits par les deux types de classes atypiques peut provenir de deux facteurs. Le premier est le fait que les classes atypiques avantagées ne représentent qu'une très faible part (2\%) de l'ensemble des classes, contrairement aux classes atypiques désavantagées (13\%). Le second est le fait que, contrairement aux élèves des classes atypiques avantagées, ceux des classes atypiques désavantagées sont plus ``éloignés'', en termes de déterminants de performances scolaires, des élèves des classes non atypiques.\\
Il est possible de tester empiriquement la première possibilité. La procédure et son résultat est montré par la Figure \ref{fig:petestsbiaisasymeff} de l'Annexe \ref{petestsbiaisasym}. L'idée est de mesurer les effets de pairs lorsque nous rajoutons l'échantillon d'élèves des classes atypiques désavantagées de même ordre d'effectifs que ceux des classes atypiques avantagées. Si c'est le fait que les classes atypiques désavantagées sont nombreuses qui est à la source du grand biais de sous-estimation lorsqu'elles sont rajoutées dans les estimations, nous devrions trouver un biais de sous-estimation plus petit lorsque nous rajoutons moins de classes atypiques désavantagées. Le résultat principal (Figure \ref{fig:petestsbiaisasymeff}) est que les effets sont effectivement moins sous-estimés par rapport au cas où nous rajoutons toutes les classes atypiques désavantagées. Il est alors raisonnable de penser que s'il y avait eu beaucoup plus de classes atypiques avantagées, les effets de pairs de la colonne (5) du Tableau (\ref{tab:pemodels0}) auraient été fortement surestimés. Ces conclusions justifient l'exclusion des classes atypiques, avantagées comme désavantagées, de nos échantillons d'estimation.\\
Nous pouvons évaluer la seconde possibilité mais nous ne pourrons pas savoir laquelle des deux possibilités d'explication de l'asymétrie des biais induits par les deux classes atypiques l'emporte éventuellement sur l'autre. L'Annexe \ref{peecdfscore} (Figure \ref{fig:peecdfscore}) semble confirmer que les élèves des classes atypiques désavantagées sont bien plus ``éloignées'' en termes de niveau scolaire des élèves des classes non atypiques, par rapport aux élèves des classes atypiques avantagées.

\quad Les trois dernières colonnes du Tableau \ref{tab:pemodels0} sont les mêmes que les colonnes (4) à (6), respectivement, à la différence que l'âge, la position au DNB (avec leurs moyennes chez les pairs) ainsi que la taille de classe sont incluses comme régresseurs. Ces trois colonnes indiquent que, par rapport au cas où les trois variables ne sont pas utilisées (colonne 4 à 6), les effets de pairs sont revus à la baisse d'une valeur de l'ordre de 0.07 UET. La seule différence importante avec la cas où les variables potentiellement endogènes ne sont pas utilisées est que dans ce cas, le rajout des classes atypiques désavantagées dans les données d'estimation produit une mesure non significative des effets de pairs (colonne 9, ligne \emph{Note au CM2 des pairs \(\times \ p\)} à comparer avec la colonne 6 de la même ligne).\\
Nous faisons remarquer l'effet apparent d'une personne en plus dans la classe qui est de 0.018 UET (colonne 7 du Tableau \ref{tab:pemodels0compl} de l'Annexe \ref{pemodels0compl}). Cette mesure est surestimée car elle est contaminée par la sélection mentionnée ci-dessus selon laquelle les élèves ayant des résultats potentiels moins élevés ont tendance à être placés dans des classes de plus petite taille. Piketty et al. (\protect\hyperlink{ref-PIK:VAL:06}{2006}) proposent d'instrumenter la taille de classe par la taille de classe théorique si les écoles suivaient strictement les règles de limite de taille de classe. Nous ne pouvons pas recourir à cette solution puisque l'instrument proposé ne varie que par établissement(-année), ce qui est strictement incompatible avec les effets fixes établissements-années, une de nos stratégies d'identification centrales. Comparer les effets de pairs avec l'effet de la taille de classe aurait certes été intéressant en termes de politiques publiques car en fonction des effets trouvés, un arbitrage entre politiques de mixité scolaire et politiques de réduction de taille de classe est à analyser.

\quad Les régresseurs potentiellement endogènes que sont l'âge, la position au DNB et la taille de classe ne remettent pas en cause nos conclusions.

\quad Une UET de plus de la note au CM2 semble procurer de manière très robuste\footnote{C'est-à-dire indépendamment des régressions effectuées, comme l'illustre la ligne \emph{Note au CM2} de toutes les colonnes du Tableau \ref{tab:pemodels0}.} un avantage entre 0.6 et 0.68 UET sur les épreuves écrites du DNB. Ce montant est très important mais il est vraisemblablement surestimé dans la mesure où les inobservables pré-CM2 corrélés positivement à la performance au CM2 le sont également à la performance au DNB\footnote{Ce sont typiquement les variables telles que le talent, la motivation ou encore l'implication des parents dans la scolarité de leur enfant ; en supposant qu'elles ne varient pas au cours du temps.} (\protect\hyperlink{ref-TOD:WOL:03}{Todd \& Wolpin, 2003}).

\quad En résumé, nous trouvons qu'une augmentation d'une UET du niveau moyen au CM2 des pairs implique une augmentation d'environ 0.2 UET des résultats aux épreuves écrites du DNB. L'ampleur de cet effet est comparable aux ampleurs trouvées dans les papiers qui s'intéressent à l'effet d'une note antérieure des pairs et qui utilisent des stratégies basées sur les effets fixes autres que les effets fixes enseignants (\protect\hyperlink{ref-HAN:eal:03}{Hanushek et al., 2003} ; \protect\hyperlink{ref-HOX:WEI:05}{Hoxby \& Weingarth, 2005} ; \protect\hyperlink{ref-AMM:PIS:09}{Ammermueller \& Pischke, 2009}). Lorsque les effets fixes enseignants peuvent être pris en compte, les effets de pairs linéaires en moyennes restent le plus souvent significatifs mais de plus faible ampleur (\protect\hyperlink{ref-SAC:11}{Sacerdote, 2011}, Tableau 4.2).

\begin{landscape}\begingroup\fontsize{8}{10}\selectfont

\begin{ThreePartTable}
\begin{TableNotes}
\item \textit{Sources :} Fichiers DNB (2014 à 2016), fichiers CM2 (2010 à 2012), fichiers CONSTAT (2013 à 2016), calculs de l'auteur.
\item \textit{Notes :} Une colonne correspond à une régression. Les notes au CM2 et au DNB sont normalisées sur l'année scolaire de CM2 et de DNB, respectivement. Écart-types robustes entre parenthèses. Des effets fixes collèges-année sont utilisés. CSP : Catégorie Socio-Professionnelle. CM2 : Cours Moyen 2\textsuperscript{ème} année. DNB : Diplôme National du Brevet.
\item Significativité : 10\% * 5\% ** 1\% ***.
\end{TableNotes}
\begin{longtable}[t]{llllllllll}
\caption{\label{tab:pemodels0}Résultats principaux sur les effets de pairs linéaires en moyenne}\\
\toprule
\multicolumn{1}{c}{} & \multicolumn{9}{c}{\makecell{Variable dépendante : \\ Note DNB totale (écrits)}} \\
\cmidrule(l{3pt}r{3pt}){2-10}
\multicolumn{1}{c}{} & \multicolumn{2}{c}{\makecell{Simple régression, sans \\ var.endo.}} & \multicolumn{4}{c}{\makecell{Sojourner (2013), sans \\ var.endo.}} & \multicolumn{3}{c}{\makecell{Sojourner (2013), avec \\ var.endo.}} \\
\cmidrule(l{3pt}r{3pt}){2-3} \cmidrule(l{3pt}r{3pt}){4-7} \cmidrule(l{3pt}r{3pt}){8-10}
 & \makecell{\makecell{Sans \\ effets fixes \\ \ } \\ (1) } & \makecell{\makecell{Avec \\ effets fixes \\ \ } \\ (2) } & \makecell{\makecell{Sans \\ effets fixes \\ \ } \\ (3) } & \makecell{\makecell{Avec \\ effets fixes \\ \ } \\ (4) } & \makecell{\makecell{Avec \\ effets fixes, \\ + sec.av.} \\ (5) } & \makecell{\makecell{Avec \\ effets fixes \\ + sec.désav.} \\ (6) } & \makecell{\makecell{Avec \\ effets fixes \\ \ } \\ (7) } & \makecell{\makecell{Avec \\ effets fixes, \\ + sec.av.} \\ (8) } & \makecell{\makecell{Avec \\ effets fixes \\ + sec.désav.} \\ (9) }\\
\midrule
\endfirsthead
\caption[]{\label{tab:pemodels0}Résultats principaux sur les effets de pairs linéaires en moyenne (suite)}\\
\toprule
 & \makecell{\makecell{Sans \\ effets fixes \\ \ } \\ (1) } & \makecell{\makecell{Avec \\ effets fixes \\ \ } \\ (2) } & \makecell{\makecell{Sans \\ effets fixes \\ \ } \\ (3) } & \makecell{\makecell{Avec \\ effets fixes \\ \ } \\ (4) } & \makecell{\makecell{Avec \\ effets fixes, \\ + sec.av.} \\ (5) } & \makecell{\makecell{Avec \\ effets fixes \\ + sec.désav.} \\ (6) } & \makecell{\makecell{Avec \\ effets fixes \\ \ } \\ (7) } & \makecell{\makecell{Avec \\ effets fixes, \\ + sec.av.} \\ (8) } & \makecell{\makecell{Avec \\ effets fixes \\ + sec.désav.} \\ (9) }\\
\midrule
\endhead

\endfoot
\bottomrule
\insertTableNotes
\endlastfoot
Note au CM2 des pairs & 0.069$^{***}$ & 0.216$^{***}$ & - & - & - & - & - & - & -\\
 & (0.014) & (0.026) & - & - & - & - & - & - & -\\
Note au CM2 des pairs $\times \ p$ & - & - & 0.076$^{***}$ & 0.264$^{***}$ & 0.266$^{***}$ & $-$0.095$^{**}$ & 0.195$^{***}$ & 0.198$^{***}$ & $-$0.017\\
 & - & - & (0.017) & (0.034) & (0.03) & (0.037) & (0.037) & (0.033) & (0.042)\\
Note au CM2 & 0.665$^{***}$ & 0.674$^{***}$ & 0.666$^{***}$ & 0.674$^{***}$ & 0.671$^{***}$ & 0.633$^{***}$ & 0.638$^{***}$ & 0.635$^{***}$ & 0.608$^{***}$\\
 & (0.005) & (0.009) & (0.005) & (0.009) & (0.009) & (0.008) & (0.009) & (0.009) & (0.008)\\
p & - & - & 0.073$^{*}$ & 0.147$^{*}$ & 0.133$^{*}$ & 0.302$^{***}$ & 0.144$^{*}$ & 0.126 & 0.413$^{***}$\\
 & - & - & (0.037) & (0.082) & (0.08) & (0.104) & (0.079) & (0.077) & (0.102)\\
 &  &  &  &  &  &  &  &  & \\
Observations & 25387 & 25387 & 25387 & 25387 & 26735 & 29109 & 25387 & 26735 & 29109\\
R$^2$ ajusté & 0.546 & 0.505 & 0.546 & 0.506 & 0.512 & 0.431 & 0.518 & 0.525 & 0.445\\*
\end{longtable}
\end{ThreePartTable}
\endgroup{}
\end{landscape}

La majorité des études trouvent des effets de pairs significatifs indépendamment de la matière et les exceptions concluent que les effets du niveau des pairs ne sont significatifs qu'en mathématiques (\protect\hyperlink{ref-YEU:NGU:16}{Yeung \& Nguyen-Hoang, 2016} ; \protect\hyperlink{ref-BRU:eal:10}{Brunello et al., 2010}). Des effets de pairs significatifs en mathématiques, par exemple, et non dans d'autres matières signifieraient que les cours de mathématiques offrent plus d'interactions des élèves par rapport aux autres matières (\protect\hyperlink{ref-IZA:DIC:20}{Izaguirre \& Di Capua, 2020}).

\quad Le Tableau \ref{tab:pemodelsfrmaths} rapporte les résultats d'estimation des effets de pairs aux épreuves écrites de français et de mathématiques. Les effets mesurés oscillent entre environ 0.2 UET et 0.25 UET et ne sont pas significativement différents en fonction de la variable dépendante. Si la spécification linéaire en moyenne est la bonne, nous pouvons penser que les interactions (directes ou indirectes) entre les pairs lors des cours de français et de mathématiques ont les mêmes effets. Le Tableau \ref{tab:pemodelsssmoy} de l'Annexe \ref{pemodelsssmoy} reproduit l'exercice avec les résultats aux écrits d'histoire-et-géographie (colonnes 1 et 2), de dictée (colonnes 3 et 4) et de rédaction (colonnes 5 et 6) comme variables dépendantes alternatives. Les effets mesurés sont essentiellement les mêmes à part ceux en rédaction qui sont légèrement inférieurs mais restent statistiquement et économiquement significatifs (0.17 UET et 0.15 UET selon les régresseurs utilisés)\footnote{De manière plus annexe, la note au CM2 de l'individu joue un rôle moins élevé sur les résultats en rédaction (0.4 UET) comparé à ceux des autres matières (environ 0.6 UET). Cela suggère que les capacités de rédaction de l'individu se développent surtout pendant le collège.}.

\newpage
\begingroup\fontsize{8}{10}\selectfont

\begin{ThreePartTable}
\begin{TableNotes}
\item \textit{Sources :} Fichiers DNB (2014 à 2016), fichiers CM2 (2010 à 2012), fichiers CONSTAT (2013 à 2016), calculs de l'auteur.
\item \textit{Notes :} Une colonne correspond à une régression. Les notes au CM2 et au DNB sont normalisées sur l'année scolaire de CM2 et de DNB, respectivement. Écart-types robustes entre parenthèses. Des effets fixes collèges-année sont utilisés. Contrôles potentiellement endogènes : âge, position et taille de classe au DNB. Autres contrôles : sexe, catégorie socio-professionnelle et régime scolaire. CM2 : Cours Moyen 2\textsuperscript{ème} année.
\item Significativité : 10\% * 5\% ** 1\% ***.
\end{TableNotes}
\begin{longtable}[t]{lllll}
\caption{\label{tab:pemodelsfrmaths}Les effets de pairs linéaires en moyenne sur les notes en français et en mathématiques}\\
\toprule
\multicolumn{1}{c}{} & \multicolumn{4}{c}{Variable dépendante : Note en} \\
\cmidrule(l{3pt}r{3pt}){2-5}
\multicolumn{1}{c}{} & \multicolumn{2}{c}{Français (écrits)} & \multicolumn{2}{c}{Mathématiques (écrits)} \\
\cmidrule(l{3pt}r{3pt}){2-3} \cmidrule(l{3pt}r{3pt}){4-5}
 & \makecell{Sans var.endo. \\ (1) } & \makecell{Avec var.endo. \\ (2) } & \makecell{Sans var.endo. \\ (3) } & \makecell{Avec var.endo. \\ (4) }\\
\midrule
\endfirsthead
\caption[]{\label{tab:pemodelsfrmaths}Les effets de pairs linéaires en moyenne sur les notes en français et en mathématiques (suite)}\\
\toprule
 & \makecell{Sans var.endo. \\ (1) } & \makecell{Avec var.endo. \\ (2) } & \makecell{Sans var.endo. \\ (3) } & \makecell{Avec var.endo. \\ (4) }\\
\midrule
\endhead

\endfoot
\bottomrule
\insertTableNotes
\endlastfoot
Note au CM2 des pairs $\times \ p$ & 0.242$^{***}$ & 0.185$^{***}$ & 0.247$^{***}$ & 0.199$^{***}$\\
 & (0.032) & (0.033) & (0.035) & (0.038)\\
Note au CM2 & 0.599$^{***}$ & 0.565$^{***}$ & 0.607$^{***}$ & 0.578$^{***}$\\
 & (0.008) & (0.008) & (0.01) & (0.01)\\
p & 0.228$^{***}$ & 0.227$^{***}$ & 0.059 & 0.061\\
 & (0.072) & (0.07) & (0.085) & (0.083)\\
 &  &  &  & \\
Contrôles individuels & Oui & Oui & Oui & Oui\\
Contrôles chez les pairs & Oui & Oui & Oui & Oui\\
Contrôles potentiellement endogènes & Non & Oui & Non & Oui\\
Observations & 25347 & 25347 & 25282 & 25282\\
R$^2$ ajusté & 0.435 & 0.446 & 0.396 & 0.407\\*
\end{longtable}
\end{ThreePartTable}
\endgroup{}

Les effets peuvent différer selon le sexe puisque la fin du collège correspond à un âge de renforcement des identifications personnelles et des interactions sociales. Ainsi, à la manière de Lavy et al. (\protect\hyperlink{ref-LAV:eal:12}{2012}), nous tentons de savoir dans quelle mesure les effets de pairs diffèrent en fonction du sexe. Pour ce faire, des interactions entre la note au CM2 des pairs, la proportion de pairs retrouvés \(p\) et des indicatrices de sexe sont rajoutées dans les régressions.

\quad Le Tableau \ref{tab:pemodelssexemod}, ainsi que le Tableau \ref{tab:pemodelssexemodssmoy} de l'Annexe \ref{pemodelssexemodssmoy} montrent que les effets de pairs sont légèrement plus élevés chez les filles excepté en mathématiques. Ce résultat est robuste à l'inclusion itérative de variables de contrôle.

\begin{landscape}\begingroup\fontsize{8}{10}\selectfont

\begin{ThreePartTable}
\begin{TableNotes}
\item \textit{Sources :} Fichiers DNB (2014 à 2016), fichiers CM2 (2010 à 2012), fichiers CONSTAT (2013 à 2016), calculs de l'auteur.
\item \textit{Notes :} Une colonne correspond à une régression. Les notes au CM2 et au DNB sont normalisées sur l'année scolaire de CM2 et de DNB, respectivement. Écart-types robustes entre parenthèses. Des effets fixes collèges-année sont utilisés. Contrôles potentiellement endogènes : âge, position et taille de classe au DNB. Autres contrôles : sexe, catégorie socio-professionnelle et régime scolaire. CM2 : Cours Moyen 2\textsuperscript{ème} année.
\item Significativité : 10\% * 5\% ** 1\% ***.
\end{TableNotes}
\begin{longtable}[t]{lllllll}
\caption{\label{tab:pemodelssexemod}Effets de pairs linéaires en moyenne, hétérogènes selon le sexe}\\
\toprule
\multicolumn{1}{c}{} & \multicolumn{6}{c}{Variable dépendante : Note} \\
\cmidrule(l{3pt}r{3pt}){2-7}
\multicolumn{1}{c}{} & \multicolumn{2}{c}{Totale (écrits)} & \multicolumn{2}{c}{En français (écrits)} & \multicolumn{2}{c}{En Mathématiques (écrits)} \\
\cmidrule(l{3pt}r{3pt}){2-3} \cmidrule(l{3pt}r{3pt}){4-5} \cmidrule(l{3pt}r{3pt}){6-7}
 & \makecell{Sans var.endo. \\ (1) } & \makecell{Avec var.endo. \\ (2) } & \makecell{Sans var.endo. \\ (3) } & \makecell{Avec var.endo. \\ (4) } & \makecell{Sans var.endo. \\ (5) } & \makecell{Avec var.endo. \\ (6) }\\
\midrule
\endfirsthead
\caption[]{\label{tab:pemodelssexemod}Effets de pairs linéaires en moyenne, hétérogènes selon le sexe (suite)}\\
\toprule
 & \makecell{Sans var.endo. \\ (1) } & \makecell{Avec var.endo. \\ (2) } & \makecell{Sans var.endo. \\ (3) } & \makecell{Avec var.endo. \\ (4) } & \makecell{Sans var.endo. \\ (5) } & \makecell{Avec var.endo. \\ (6) }\\
\midrule
\endhead

\endfoot
\bottomrule
\insertTableNotes
\endlastfoot
Note au CM2 des pairs $\times \ p$ & 0.286$^{***}$ & 0.22$^{***}$ & 0.286$^{***}$ & 0.231$^{***}$ & 0.221$^{***}$ & 0.176$^{***}$\\
 & (0.035) & (0.038) & (0.035) & (0.035) & (0.036) & (0.039)\\
Note au CM2 des pairs $\times \ p$ $\times \ $Sexe - Garçon & $-$0.044 & $-$0.051$^{*}$ & $-$0.088$^{***}$ & $-$0.096$^{***}$ & 0.052$^{*}$ & 0.047\\
 & (0.027) & (0.027) & (0.029) & (0.029) & (0.031) & (0.031)\\
Note au CM2 & 0.674$^{***}$ & 0.638$^{***}$ & 0.598$^{***}$ & 0.565$^{***}$ & 0.607$^{***}$ & 0.578$^{***}$\\
 & (0.009) & (0.009) & (0.008) & (0.008) & (0.01) & (0.01)\\
p & 0.146$^{*}$ & 0.142$^{*}$ & 0.226$^{***}$ & 0.223$^{***}$ & 0.06 & 0.063\\
 & (0.082) & (0.079) & (0.072) & (0.07) & (0.085) & (0.083)\\
 &  &  &  &  &  & \\
Contrôles individuels & Oui & Oui & Oui & Oui & Oui & Oui\\
Contrôles chez les pairs & Oui & Oui & Oui & Oui & Oui & Oui\\
Contrôles potentiellement endogènes & Non & Oui & Non & Oui & Non & Oui\\
Observations & 25387 & 25387 & 25347 & 25347 & 25282 & 25282\\
R$^2$ ajusté & 0.506 & 0.519 & 0.435 & 0.446 & 0.396 & 0.407\\*
\end{longtable}
\end{ThreePartTable}
\endgroup{}
\end{landscape}

Nous ne trouvons pas de tendance claire sur d'éventuelles différences des effets de pairs en fonction de la catégorie sociale. Les résultats sont montrés en Annexe \ref{pemodelspcsregmodssmoy} , par le Tableau \ref{tab:pemodelspcsregmod} pour toutes les matières, le français et les mathématiques et par le Tableau \ref{tab:pemodelspcsregmodssmoy} pour les autres matières). Puisque nos données couvrent la population de La Réunion sur trois années scolaires, nous avons de bonnes raisons de penser que cette absence de tendance suggère une vraie absence de différence des effets de pairs selon la strate sociale à La Réunion, dans la spécification linéaire en moyenne.
Ces résultats sont relativement en accord avec ceux de Lavy et al. (\protect\hyperlink{ref-LAV:eal:12}{2012, p. 404}) qui ne trouvent aucune différence des effets de pairs en fonction de l'éligibilité au repas gratuit, qui est une variable d'approximation du revenu familial.

La littérature trouve toutefois le plus souvent des effets de pairs supérieurs pour les élèves plutôt de catégorie sociale défavorisée (\protect\hyperlink{ref-GRI:RAS:14}{Griffith \& Rask, 2014}, par exemple).

\hypertarget{peresfrh5}{%
\subsubsection{Spécification des effets de pairs hétérogènes}\label{peresfrh5}}

Une spécification linéaire en moyenne des effets de pairs est simple mais n'est pas intéressante en termes de politiques puisqu'elle implique qu'aucune politique de réarrangement des classes (intra-établissement) selon le niveau scolaire des élèves n'augmenterait pas la performance totale (voir Section \ref{pemethodss13}). De plus, des modèles plus flexibles (effets de pairs hétérogènes selon le niveau de l'individu et/ou non linéaires) expliquent souvent mieux les données utilisées (\protect\hyperlink{ref-MON:eal:19}{Monso et al., 2019}). Ces deux constats constituent notre motivation à étudier les effets de pairs hétérogènes et non linéaires dans le présent paragraphe et dans le suivant.

\quad Afin d'analyser dans quelle mesure le niveau scolaire moyen des pairs varie avec le niveau de l'individu, nous faisons interagir le niveau des pairs avec des variables indicatrices du quintile du niveau de l'individu. Chaque coefficient associé à l'interaction correspond alors à l'effet de pair pour l'individu du quintile concerné.
La Figure \ref{fig:pemodelsh5corrgraph} rapporte les résultats de cet exercice. De manière globale, il en ressort que les effets de pairs n'existent pas pour les individus les plus faibles (Q1) et augmentent avec le niveau de l'individu. Les effets augmentent légèrement plus avec le niveau en mathématiques que dans les autres matières\footnote{La Figure \ref{fig:pemodelsh5corrssmoygraph} de l'Annexe \ref{pemodelsh5corrssmoy} réplique les exercices pour les autres matières. Elle met en valeur la robustesse de notre constat.}.

\quad Ces effets de pairs qui augmentent en fonction du niveau initial de l'individu sont en contradiction avec les enseignements de la littérature en éducation. Notamment, sur données françaises, Davezies (\protect\hyperlink{ref-DAV:04}{2004}) trouve qu'en primaire, être scolarisé dans des écoles à forte proportion d'élèves modestes (donc de niveau vraisemblablement plus faible que la moyenne) est deux fois plus impactant pour des élèves qui sont eux-mêmes modestes. Pour le lycée, Boutchenik \& Maillard (\protect\hyperlink{ref-BOU:MAI:18}{2018}), dont l'étude présente beaucoup de similarités avec la nôtre, trouvent clairement des effets de pairs décroissants en fonction du niveau de l'individu.
Il est difficile d'expliquer cette différence entre nos résultats avec ceux de la littérature. Une première possibilité est que l'effet mesuré capte surtout l'adaptation du niveau d'enseignement de l'enseignant au niveau moyen de la classe, ce que nous ne pouvons pas prendre en compte. Plus précisément, il se peut que l'enseignant dispense des enseignements de niveau plus élevé s'il y a une quantité considérable de bons élèves. De cette manière, plus l'élève est bon, plus il arrive à suivre, ce qui explique le fait qu'une augmentation du niveau moyen des pairs bénéficie plus aux plus forts. Cela rentre d'ailleurs en cohérence avec le modèle théorique proposé par Duflo et al. (\protect\hyperlink{ref-DUF:eal:11}{2011}) et une partie de leurs résultats pour le Kenya. \\
Une deuxième possibilité concerne la spécification du modèle. Le modèle d'effets de pairs hétérogènes restreint toujours les effets de pairs à n'agir qu'à travers la moyenne et ne nous permet pas d'étudier d'éventuels complémentarités entre les pairs et individus de certains niveaux. Par exemple, nous ne pouvons pas savoir si plus de pairs forts dans la classe bénéficie/nuit aux individus forts/faibles (\protect\hyperlink{ref-HOX:WEI:05}{Hoxby \& Weingarth, 2005} ; \protect\hyperlink{ref-BUR:SAS:13}{Burke \& Sass, 2013}). Cette question est abordée dans la Section \ref{peresfrhnolinq5}.

\begin{figure}[H]

{\centering \includegraphics[width=1\linewidth]{000_files/figure-latex/pemodelsh5corrgraph-1} 

}

\caption{Effets de pairs hétérogènes}\label{fig:pemodelsh5corrgraph}
\end{figure}

Les Figures \ref{fig:pemodelsh5corrsexemodgraph} (variables dépendantes principales) et \ref{fig:pemodelsh5corrsexemodssmoygraph} (autres variables dépendantes) de l'Annexe \ref{pemodelsh5corrsexemodssmoy} nous confirme que les garçons sont légèrement moins sensibles aux effets de pairs et nous précisent que ce sont les garçons les plus faibles. Nous avons également vérifié que cette conclusion est robuste aux variables de contrôles utilisées.\\
Dans la même logique, les Figures \ref{fig:pemodelsh5corrpcsregmodgraph} et \ref{fig:pemodelsh5corrpcsregmodssmoygraph} de l'Annexe \ref{pemodelsh5corrpcsregmodssmoy} nous renseignent qu'en fait, les élèves de classe sociale très favorisée, même s'ils font partie des plus forts, sont moins sensibles aux effets de pairs. Cela traduit simplement le fait que les élèves de cette catégorie sont moins dépendants du contexte scolaire, vu qu'ils sont dans un contexte familial vraisemblablement plus favorable aux performances scolaires par rapport aux élèves des catégories sociales moins élevées. Les mathématiques constituent une exception, comme dans les résultats précédents.

\quad Nous proposons quelques possibilités de formes théoriques des effets de pairs qui peuvent correspondre ou non à nos résultats jusque-là. Puisque les effets de pairs apparaissent clairement hétérogènes en fonction du niveau scolaire de l'élève, nous pouvons penser que le modèle linéaire en moyenne n'est pas adapté. Nous ne nous trouvons vraisemblablement pas dans une situation de \emph{bad apple} ou similaire (un élève ou des pairs faibles sont néfastes pour tout le monde) ou de \emph{shining light} (un élève ou des pairs forts sont bénéfiques pour tout le monde), vu qu'elles supposent également que les effets de pairs sont les mêmes pour tous les individus et qu'elles sont difficiles à expliquer (coûteuses en hypothèses).\\
Puisque le modèle de Sojourner (\protect\hyperlink{ref-SOJ:13}{2013}) ne nous permet théoriquement pas d'analyser l'effet de l'hétérogénéité de la classe en termes de niveau\footnote{Qui consiste pratiquement à étudier l'effet de l'écart-type du niveau des pairs (\protect\hyperlink{ref-VID:NEC:06}{Vigdor \& Nechyba, 2007}, par exemple).}, nous ne pouvons malheureusement pas savoir si nous sommes dans une situation de \emph{focus} (une classe homogène en niveau, peu importe lequel, est bon pour tout le monde) ou de \emph{rainbow} (une classe hétérogène en niveau est bon pour tout le monde).\\
De ce que nous pouvons tester, il nous reste les situations de \emph{indivious comparison} (la présence de pairs plus forts est néfaste pour l'individu) ou de \emph{single crossing} (l'effet des pairs les plus forts est positif et s'accroît légèrement en fonction du niveau de l'individu). Nous reviendrons sur ce point à la Section \ref{peresfrhnolinq5}.

\hypertarget{peresfrhnolinq5}{%
\subsubsection{Spécification des effets de pairs hétérogènes et non linéaires}\label{peresfrhnolinq5}}

Nous relâchons désormais l'hypothèse de la linéarité des effets de pairs, ce qui nous permet de savoir comment des pairs de différents niveaux influencent des individus de différents niveaux (équation \ref{eq:pes13hnolinq5reg}). Les résultats d'estimation de ce type de modèle sont présentés graphiquement, comme dans les deux panels de la Figure \ref{fig:pemodelshnolinq5corrgraph}. Les deux panels diffèrent selon que nous utilisons les variables de contrôle potentiellement endogènes (panel B) ou non (panel A). Pour chaque panel, une ligne de graphiques de cette figure correspond à une régression et à chaque ligne correspond une variable dépendante différente. Les proportions de pairs dans les différents quintiles sont représentées sur l'axe des abscisses. Chaque colonne de graphiques correspond à un quintile de la note au CM2 de l'élève. Le coefficient associé à l'interaction entre la proportion de pairs d'un niveau donné avec une indicatrice de niveau de l'individu, est représenté par les points dans les cases. Pour que les coefficients apparaissent plus intuitifs, nous choisissons de montrer l'effet de l'augmentation de 10 points de pourcentage (au lieu de 100 points de pourcentage) de pairs d'un niveau donné en échange des pairs de niveau moyen (\(\bar{Q3}\)). Les barres verticales représentent les intervalles de confiance à 95\% calculés à partir d'écart-types robustes à l'hétéroscédasticité de forme inconnue.

\quad Les individus les plus faibles (colonne de graphiques \(Q1\)) ne semblent tirer aucun bénéfice de leurs pairs, quel que soit le niveau de ces derniers\footnote{Cela est cohérent avec les résultats de la spécification hétérogène des effets de pairs de la Section \ref{peresfrh5}.}. Pour les individus légèrement plus forts (\(Q2\)), la présence de plus de pairs les plus faibles (\(\bar{Q1}\)) leur est néfaste. Les pairs d'autres niveaux ne semblent pas non plus affecter les individus au Q2.
Les pairs les plus faibles sont également néfastes pour l'individu de niveau moyen (colonne de graphiques \(Q3\)). Ce qui change par rapport aux autres individus c'est qu'avoir plus de pairs parmi les plus forts (\(\bar{Q5}\)) semblent aider les élèves de niveau moyen. Ce bénéfice est légèrement supérieur en mathématiques.
À travers les deux panels, les mêmes observations valent pour les individus au quintile \(Q4\) à la différence que les bénéfices d'avoir des pairs au quintile \(Q5\) sont plus prononcés.
Les individus les plus forts (\(Q5\)) ne sont affectés par aucun type de pairs sauf leurs semblables en termes de niveau scolaire : les pairs les plus forts, qui sont bénéfiques.
La Figure \ref{fig:pemodelshnolinq5corrsDexossmoygraph} de l'Annexe \ref{pemodelshnolinq5corrssmoy}, pour les autres matières démontre essentiellement les mêmes tendances.

\quad \textit{A priori}, nous nous trouvons alors dans un modèle de \emph{single crossing} au lieu d'un \emph{indivious comparison}, vu que les effets positifs des pairs les plus forts augmente légèrement en fonction du niveau scolaire. Nos résultats rejoignent en partie ceux de Burke \& Sass (\protect\hyperlink{ref-BUR:SAS:13}{2013}) sur l'influence positive et croissante des pairs forts sur les individus au moins de niveau moyen. Par contre, ces auteurs trouvent que les pairs forts sont néfastes pour les individus faibles, et de manière plus surprenante que les individus forts sont positivement influencés par les pairs faibles.\\
Nos résultats sont toujours compatibles avec le fait que les enseignants augmentent le niveau des cours en fonction du niveau de la classe mais imposent quand même une limite, sinon nous aurions très probablement observé des effets négatifs de l'augmentation de pairs forts chez les individus les plus faibles. En effet, si le niveau des cours dépasse une certaine limite, les plus faibles ne pourront vraisemblablement plus suivre et/ou se décourager. Le fait que les individus les plus forts ne sont pas non plus gênés par la présence de pairs plus faibles est également compatible avec l'hypothèse d'adaptation du comportement de l'enseignant au lieu d'interactions directes entre les élèves.

\begin{figure}[H]

{\centering \includegraphics[width=1\linewidth]{000_files/figure-latex/pemodelshnolinq5corrgraph-1} 

}

\caption{Effets de pairs hétérogènes et non linéaires}\label{fig:pemodelshnolinq5corrgraph}
\end{figure}

De manière annexe, les Figures \ref{fig:pemodelshnolinq5corrsDexosexemodgraph} et \ref{fig:pemodelshnolinq5corrsDexosexemodssmoygraph} (Annexe \ref{pemodelshnolinq5corrsexemodssmoy}) nous apprennent que parmi les plus faibles, les garçons pâtissent plus de l'adaptation du niveau des cours par l'enseignant que les filles. Ces constats sont un peu plus en accord avec les résultats des papiers sur données américaines (\protect\hyperlink{ref-HOX:WEI:05}{Hoxby \& Weingarth, 2005} ; \protect\hyperlink{ref-BUR:SAS:13}{Burke \& Sass, 2013}). Ils trouvent que les plus faibles sont négativement impactés par les pairs les plus forts. Dans notre cas, ce sont uniquement les garçons les plus faibles qui valident cette observation.
Nous retrouvons toujours la sensibilité moindre des élèves de catégorie sociale élevée dans les dernières colonnes Figures \ref{fig:pemodelshnolinq5corrsDexopcsregmodgraph} et \ref{fig:pemodelshnolinq5corrsDexopcsregmodssmoygraph} (Annexe \ref{pemodelshnolinq5corrpcsregmodssmoy}). Il semble difficile de tirer d'autres leçons de ces deux figures.

\hypertarget{peresfrrob}{%
\subsubsection{Robustesse des résultats en forme réduite des effets de pairs}\label{peresfrrob}}

Ce paragraphe discute de la robustesse de nos résultats en fonction du choix de \(k\) (voir Section \ref{pemethodsrestr}), du découpage du niveau scolaire et de la distinction, parmi les pairs de 3\textsuperscript{ème}, entre les anciens pairs de CM2 et les nouveaux pairs.

\quad Tout d'abord, notre règle de détection et d'exclusion des établissements qui regrouperaient les élèves de 3\textsuperscript{ème} en fonction de leurs notes au CM2 semble arbitraire (\(k = 2\), voir Section \ref{pemethodsrestr}). Pour rappel, \(k = 1\) signifie qu'un établissement, toutes années confondues, est exclu de l'échantillon d'estimation dès que nous détectons sur au moins une année un regroupement. Cette règle est plus sévère. De la même manière, \(k = 3\) signifie qu'un établissement n'est exclu de l'échantillon d'estimation que s'il semble regrouper ses élèves 3 années de suite. Cette règle est moins sévère.\\
Lorsque \(k = 1\), le test implique d'exclure uniquement un établissement contre 5 et 9 établissements lorsque \(k = 2\) et \(k = 3\), respectivement. Le fait qu'une règle plus sévère implique d'exclure moins d'établissement est étonnant. La cause en est une probabilité critique\footnote{La probabilité critique liée à l'hypothèse nulle qu'un établissement-année donné ne pratique pas de regroupement.} plus petite au fur et à mesure que la règle d'exclusion est sévère (Annexe de \protect\hyperlink{ref-BOU:MAI:18}{Boutchenik \& Maillard, 2018}).\\
Nos résultats sont robustes à la règle d'exclusion retenue\footnote{Ils ne sont pas montrés par souci de concision. Ils sont disponibles sous requêtes.}. Ce test de robustesse en particulier est pertinent dans la mesure où les trois règles d'exclusion examinées sont les seules possibles dans notre cas vu que nos données ne comportent que trois années scolaires. Cela veut dire que si les résultats apparaissent robustes en fonction de ces règles, ce n'est pas à cause d'une sélection de règles les confortant.

\quad Ensuite, la justification du découpage de la note au CM2 en quintile, qui consiste à pouvoir distinguer des niveaux ``très faible'' (\(Q1\)), ``faible'' (\(Q2\)), ``moyen'' (\(Q3\)), ``fort'' (\(Q4\)), ``très fort'' (\(Q5\)), peut sembler insuffisante. Il ne semble pas y avoir de découpage privilégié dans la littérature. Par exemple, Hoxby \& Weingarth (\protect\hyperlink{ref-HOX:WEI:05}{2005}) découpent la note antérieure en déciles et Burke \& Sass (\protect\hyperlink{ref-BUR:SAS:13}{2013}) en quintiles mais regroupent ensuite les quintiles Q3 à Q4 dans la catégorie de niveau ``moyen''. Gibbons \& Telhaj (\protect\hyperlink{ref-GIB:TEL:16}{2016}), quant à eux, découpent la note antérieure en deux (en dessous de la médiane ou non). Sojourner (\protect\hyperlink{ref-SOJ:13}{2013}) quant à lui, utilise des terciles.
Notre découpage fait un compromis entre la nuance des résultats et leur visibilité\footnote{Dans une figure telle que la Figure \ref{fig:pemodelshnolinq5corrgraph}, des découpages plus fins entraîneront des soucis de visibilité.}.

\quad Pour nous assurer que nos résultats sont valables à des niveaux plus globaux, nous utilisons un découpage moins nuancé de la note antérieure, celui utilisé par Burke \& Sass (\protect\hyperlink{ref-BUR:SAS:13}{2013}). Comme mentionné ci-dessus, selon ce découpage, les individus au quintile Q1 (faibles) et Q5 (forts) sont laissés tels quels ; et les individus aux quintiles Q2 à Q4 sont regroupés dans une seule catégorie (moyens).\\
Le Tableau \ref{tab:pemodelshbcorr} montre les résultats d'estimation des effets de pairs hétérogènes selon le niveau de l'individu et en suivant ce nouveau découpage. La seule matière sur laquelle il peut y avoir un effet de pair pour les individus de niveau faible est le français (colonne 3). Cet effet devient non significatif dès l'inclusion de l'âge, la position et la taille de classe (colonne 4). En mathématiques ou toutes matières confondues, les effets de pairs ne jouent vraisemblablement pas sur les individus de niveau faible. Nous retrouvons ces constats dans le cas du découpage principal (Figure \ref{fig:pemodelsh5corrgraph}). Aidé du Tableau \ref{tab:pemodelshbcorrssmoy} de l'Annexe \ref{pemodelshbcorrssmoy} qui réplique l'exercice pour les autres variables dépendantes, nous pouvons globalement maintenir que les effets de pairs n'existent pas pour les élèves les plus faibles et augmentent avec le niveau de l'élève.

\begin{landscape}\begingroup\fontsize{8}{10}\selectfont

\begin{ThreePartTable}
\begin{TableNotes}
\item \textit{Sources :} Fichiers DNB (2014 à 2016), fichiers CM2 (2010 à 2012), fichiers CONSTAT (2013 à 2016), calculs de l'auteur.
\item \textit{Notes :} Une colonne correspond à une régression. Les notes au CM2 et au DNB sont normalisées sur l'année scolaire de CM2 et de DNB, respectivement. Q1 à Q5 : quintiles de note au CM2 calculés sur les retrouvés par année de DNB. Écart-types robustes entre parenthèses. Des effets fixes collèges-année sont utilisés. Contrôles potentiellement endogènes : âge, position et taille de classe au DNB. Autres contrôles : découpage alternatif de la note au CM2, sexe, catégorie socio-professionnelle et régime scolaire. CM2 : Cours Moyen 2\textsuperscript{ème} année.
\item Significativité : 10\% * 5\% ** 1\% ***.
\end{TableNotes}
\begin{longtable}[t]{lllllll}
\caption{\label{tab:pemodelshbcorr}Effets de pairs hétérogènes - découpage alternatif du niveau au CM2}\\
\toprule
\multicolumn{1}{c}{} & \multicolumn{6}{c}{Variable dépendante : Note} \\
\cmidrule(l{3pt}r{3pt}){2-7}
\multicolumn{1}{c}{} & \multicolumn{2}{c}{Totale (écrits)} & \multicolumn{2}{c}{En français (écrits)} & \multicolumn{2}{c}{En mathématiques (écrits)} \\
\cmidrule(l{3pt}r{3pt}){2-3} \cmidrule(l{3pt}r{3pt}){4-5} \cmidrule(l{3pt}r{3pt}){6-7}
 & \makecell{Sans var.endo \\ (1) } & \makecell{Avec var.endo \\ (2) } & \makecell{Sans var.endo \\ (3) } & \makecell{Avec var.endo \\ (4) } & \makecell{Sans var.endo \\ (5) } & \makecell{Avec var.endo \\ (6) }\\
\midrule
\endfirsthead
\caption[]{\label{tab:pemodelshbcorr}Effets de pairs hétérogènes - découpage alternatif du niveau au CM2 (suite)}\\
\toprule
 & \makecell{Sans var.endo \\ (1) } & \makecell{Avec var.endo \\ (2) } & \makecell{Sans var.endo \\ (3) } & \makecell{Avec var.endo \\ (4) } & \makecell{Sans var.endo \\ (5) } & \makecell{Avec var.endo \\ (6) }\\
\midrule
\endhead

\endfoot
\bottomrule
\insertTableNotes
\endlastfoot
Note au CM2 des pairs $\times \ p \ \times$ Q1 & 0.036 & $-$0.076 & 0.104$^{**}$ & 0.016 & 0.017 & $-$0.071\\
 & (0.048) & (0.049) & (0.048) & (0.046) & (0.052) & (0.055)\\
Note au CM2 des pairs $\times \ p \ \times$ (Q2 à Q4) & 0.331$^{***}$ & 0.25$^{***}$ & 0.306$^{***}$ & 0.243$^{***}$ & 0.291$^{***}$ & 0.228$^{***}$\\
 & (0.034) & (0.037) & (0.034) & (0.035) & (0.035) & (0.037)\\
Note au CM2 des pairs $\times \ p \ \times$ Q5 & 0.383$^{***}$ & 0.342$^{***}$ & 0.218$^{***}$ & 0.193$^{***}$ & 0.458$^{***}$ & 0.427$^{***}$\\
 & (0.049) & (0.053) & (0.042) & (0.043) & (0.058) & (0.06)\\
p & 0.121 & 0.102 & 0.217$^{***}$ & 0.207$^{***}$ & 0.028 & 0.013\\
 & (0.081) & (0.077) & (0.072) & (0.069) & (0.085) & (0.082)\\
 &  &  &  &  &  & \\
Contrôles individuels & Oui & Oui & Oui & Oui & Oui & Oui\\
Contrôles chez les pairs & Oui & Oui & Oui & Oui & Oui & Oui\\
Contrôles potentiellement endogènes & Non & Oui & Non & Oui & Non & Oui\\
Observations & 25387 & 25387 & 25347 & 25347 & 25282 & 25282\\
R$^2$ ajusté & 0.507 & 0.521 & 0.436 & 0.447 & 0.399 & 0.41\\*
\end{longtable}
\end{ThreePartTable}
\endgroup{}
\end{landscape}

\quad Avec ce découpage alternatif, les effets de pairs mesurés selon la spécification hétérogène et non linéaire sont illustrés par la Figure \ref{fig:pemodelshnolinqbcorrsDexo}. Un coefficient correspond à l'effet de l'augmentation de 10 points de pourcentage de pairs au niveau indiqué sur l'axe des abscisses en échange de pairs au niveau \(Q3\) à \(Q5\).
Il est également clair que les résultats sont robustes au découpage de la note au CM2 puisque les élèves faibles sont peu sensibles aux effets de pairs. De manière très distincte, l'élève de niveau moyen est impacté négativement par les pairs faibles et positivement par les pairs forts. L'impact positif des pairs forts est légèrement supérieur en mathématiques. Les élèves forts ne sont pas gênés par les pairs faibles mais sont avantagés par la présence de leurs semblables que sont les pairs forts. L'impact positif des pairs forts est toujours légèrement supérieur en mathématiques. La Figure \ref{fig:pemodelshnolinqbcorrsDexossmoy} de l'Annexe \ref{pemodelshnolinqbcorr} (autres variables dépendantes) présente des résultats identiques.
Il n'y a aucune différence qualitative entre les résultats de la Figure \ref{fig:pemodelshnolinq5corrgraph} et ceux trouvés avec le découpage en quintile de la note au CM2.

\begin{figure}[H]

{\centering \includegraphics[width=1\linewidth]{000_files/figure-latex/pemodelshnolinqbcorrsDexo-1} 

}

\caption{Effets de pairs hétérogènes et non linéaires, sans contrôles potentiellement endogènes - découpage alternatif de la note au CM2}\label{fig:pemodelshnolinqbcorrsDexo}
\end{figure}

\quad Enfin, une grande partie des papiers sur les effets de pairs en éducation résolvent le problème de la réflexion (Section \ref{peintro}) en choisissant une note antérieure comme mesure de la qualité des pairs (\protect\hyperlink{ref-YEU:NGU:16}{Yeung \& Nguyen-Hoang, 2016}). Lavy et al. (\protect\hyperlink{ref-LAV:eal:12}{2012}) rappellent que cette stratégie peut ne pas totalement éliminer le problème de la réflexion à cause de la présence parmi les pairs, d'anciens pairs et de nouveaux pairs (voir Section \ref{pedata}). Dans notre cas, les anciens pairs sont les pairs de 3\textsuperscript{ème} qui proviennent de la même classe de CM2\footnote{En toute logique, les nouveaux pairs sont les pairs de 3\textsuperscript{ème} qui n'ont pas été dans la même classe de CM2 que l'individu considéré.}. Un candidat au DNB donné a très bien pu contribuer à déterminer le niveau au CM2 de ses anciens pairs lorsqu'ils étaient en CM2. Il y a alors un potentiel problème de réflexion, vu que nous cherchons à mesurer l'influence du niveau initial des pairs sur la performance (au DNB) du candidat.\\
Nos données contiennent les identifiants des classes et écoles au CM2, ce qui nous permet d'identifier, parmi les pairs de 3\textsuperscript{ème} d'un élève, ses anciens et nouveaux pairs. Une limite directe de cette information est que nous ne pouvons identifier les anciens et les nouveaux pairs que parmi les individus retrouvés. Une autre limite probablement moins importante est l'existence des classes multiniveaux au CM2 puisqu'un élève \(i\) de CM2 et un élève \(j\) de CM1 dans la même salle de classe (multiniveau) peuvent se retrouver dans la même classe de 3\textsuperscript{ème}\footnote{Si \(i\) redouble une fois ou si \(j\) saute une classe, par exemple.} sans que nous ne puissions le savoir vu que nous ne disposons pas des identifiants de classe de CM1.\\
Empiriquement, sur les données complètes, parmi les retrouvés, un élève de 3\textsuperscript{ème} retrouve en moyenne dans sa classe 92.5\% de nouveaux pairs. Ces chiffres sont de 92.7\% et 92.8\% sur les données sans classes atypiques et après exclusion d'établissements tels que décrits dans la Section \ref{pemethodsrestr}, respectivement. La probabilité de retrouver ses anciens camarades de classe de CM2 en 3\textsuperscript{ème} est alors très faible et ne dépend pas du type de classe de 3\textsuperscript{ème} (classes atypiques ou formées en tenant compte du niveau au CM2). Cela laisse penser \emph{a priori} que le biais lié à la présence des anciens pairs n'affectera pas considérablement nos conclusions.\\
Pour traiter ce problème potentiel lié aux anciens pairs, nous suivons Lavy et al. (\protect\hyperlink{ref-LAV:eal:12}{2012}) et Gibbons \& Telhaj (\protect\hyperlink{ref-GIB:TEL:16}{2016}) et séparons la mesure du niveau des pairs en deux, celui des anciens pairs et celui des nouveaux pairs, puis nous nous intéressons principalement aux effets du niveau des nouveaux pairs. Ces derniers n'ont pas pu être influencés au CM2 par l'individu et n'engendrent pas de problème de réflexion. Nous contrôlons également par le niveau des anciens pairs ainsi que par des effets fixes d'écoles(-années du CM2) au CM2. Ces derniers permettent d'éliminer d'éventuels effets persistants des caractéristiques des écoles primaires d'origine liées à la fois à la qualité des nouveaux pairs et aux performances en troisième.
Le Tableau \ref{tab:pemodelsnppsfe} présente les résultats correspondants à la spécification linéaire en moyenne. Nous trouvons que les effets des nouveaux pairs ne sont pas significativement différents par rapport au cas où nous ne les distinguions pas (Tableaux \ref{tab:pemodels0} et \ref{tab:pemodelsfrmaths}). Par exemple, une UET en plus sur le niveau au CM2 des anciens pairs entraîne une augmentation de 0.24 UET de la performance au DNB (colonne 1). Ce chiffre était de 0.26 UET dans la spécification sans distinction des nouveaux et anciens pairs. Les autres effets présentés sont tous très proches de leurs homologues dans les résultats principaux.\\
Ce qui est intéressant de noter ici c'est le léger effet négatif et très stable en ampleur (autour de \(- 0.045\) UET) qu'engendre le niveau des anciens pairs. Ce genre d'effet n'est pas approfondi par la littérature, à notre connaissance\footnote{Lavy et al. (\protect\hyperlink{ref-LAV:eal:12}{2012}) ne rapportent pas les coefficients liés aux anciens pairs. Ly \& Riegert (\protect\hyperlink{ref-LY:RIE:14}{2014}) analysent un concept apparemment similaire. Ces auteurs prouvent qu'en début de lycée, avoir plus de pairs qui étaient des pairs de l'année précédente est bénéfique pour les lycéens français. Notre propos diffère de cette dernière étude d'abord sur le fait que nous identifions de plus anciens pairs, c'est-à-dire des pairs au CM2. De plus, notre récupération d'information est imparfaite à cause de l'impossibilité de faire correspondre à tous les candidats du DNB une classe de CM2 (voir Section \ref{pedata}). Surtout, nous parlons du niveau des anciens pairs et non de leur quantité dans la classe de 3\textsuperscript{ème}.}. Nous nous sommes assurés que cet effet négatif est robuste que la proportion d'anciens pairs (parmi les pairs) soit prise en compte dans les régressions ou non. Le Tableau \ref{tab:pemodelsnppsfeh5nppsfessmoy} de l'Annexe \ref{pemodelsnppsfessmoy} qui montre les résultats pour les autres matières au DNB vient renforcer nos observations.

\begin{landscape}\begingroup\fontsize{8}{10}\selectfont

\begin{ThreePartTable}
\begin{TableNotes}
\item \textit{Sources :} Fichiers DNB (2014 à 2016), fichiers CM2 (2010 à 2012), fichiers CONSTAT (2013 à 2016), calculs de l'auteur.
\item \textit{Notes :} Une colonne correspond à une régression. Les notes au CM2 et au DNB sont normalisées sur l'année scolaire de CM2 et de DNB, respectivement. Q1 à Q5 : quintiles de note au CM2 calculés sur les retrouvés par année de DNB. Écart-types robustes entre parenthèses. Des effets fixes collèges-année et écoles(primaires)-année(au CM2) sont utilisés. Contrôles potentiellement endogènes : âge, position et taille de classe au DNB. Autres contrôles : note au CM2 (panel A)/quintile de note au CM2 (panel B), sexe, catégorie socio-professionnelle et régime scolaire. CM2 : Cours Moyen 2\textsuperscript{ème} année.
\item Significativité : 10\% * 5\% ** 1\% ***.
\end{TableNotes}
\begin{longtable}[t]{lllllll}
\caption{\label{tab:pemodelsnppsfe}Effets de pairs linéaires en moyenne, séparément pour les nouveaux pairs et anciens pairs du CM2}\\
\toprule
\multicolumn{1}{c}{} & \multicolumn{6}{c}{Variable dépendante : Note} \\
\cmidrule(l{3pt}r{3pt}){2-7}
\multicolumn{1}{c}{} & \multicolumn{2}{c}{Totale (écrits)} & \multicolumn{2}{c}{En français (écrits)} & \multicolumn{2}{c}{En Mathématiques (écrits)} \\
\cmidrule(l{3pt}r{3pt}){2-3} \cmidrule(l{3pt}r{3pt}){4-5} \cmidrule(l{3pt}r{3pt}){6-7}
 & \makecell{Sans var.endo. \\ (1) } & \makecell{Avec var.endo. \\ (2) } & \makecell{Sans var.endo. \\ (3) } & \makecell{Avec var.endo. \\ (4) } & \makecell{Sans var.endo. \\ (5) } & \makecell{Avec var.endo. \\ (6) }\\
\midrule
\endfirsthead
\caption[]{\label{tab:pemodelsnppsfe}Effets de pairs linéaires en moyenne, séparément pour les nouveaux pairs et anciens pairs du CM2 (suite)}\\
\toprule
 & \makecell{Sans var.endo. \\ (1) } & \makecell{Avec var.endo. \\ (2) } & \makecell{Sans var.endo. \\ (3) } & \makecell{Avec var.endo. \\ (4) } & \makecell{Sans var.endo. \\ (5) } & \makecell{Avec var.endo. \\ (6) }\\
\midrule
\endhead

\endfoot
\bottomrule
\insertTableNotes
\endlastfoot
Note au CM2 nouveaux pairs $\times \ p$ & 0.244$^{***}$ & 0.198$^{***}$ & 0.223$^{***}$ & 0.188$^{***}$ & 0.232$^{***}$ & 0.203$^{***}$\\
 & (0.019) & (0.022) & (0.021) & (0.024) & (0.022) & (0.024)\\
Note au CM2 anciens pairs $\times \ p$ & $-$0.045$^{***}$ & $-$0.046$^{***}$ & $-$0.049$^{***}$ & $-$0.048$^{***}$ & $-$0.044$^{***}$ & $-$0.043$^{***}$\\
 & (0.009) & (0.009) & (0.01) & (0.01) & (0.01) & (0.01)\\
p & 0.155$^{***}$ & 0.155$^{***}$ & 0.243$^{***}$ & 0.245$^{***}$ & 0.07 & 0.074\\
 & (0.049) & (0.049) & (0.053) & (0.053) & (0.055) & (0.055)\\
 &  &  &  &  &  & \\
Contrôles individuels & Oui & Oui & Oui & Oui & Oui & Oui\\
Contrôles chez les pairs & Oui & Oui & Oui & Oui & Oui & Oui\\
Contrôles potentiellement endogènes & Non & Oui & Non & Oui & Non & Oui\\
Observations & 25387 & 25387 & 25347 & 25347 & 25282 & 25282\\
R$^2$ ajusté & 0.54 & 0.548 & 0.463 & 0.471 & 0.42 & 0.426\\*
\end{longtable}
\end{ThreePartTable}
\endgroup{}
\end{landscape}

La Figure \ref{fig:pemodelsh5corrnppsfesDexograph} montre les effets hétérogènes du niveau des nouveaux et anciens pairs selon le niveau au CM2. Les effets de l'augmentation du niveau des nouveaux pairs sont très comparables à ceux dans le cas sans distinction des nouveaux et anciens pairs, à savoir que les effets sont toujours les moins importants pour les élèves les plus faibles et augmentent avec le niveau de l'élève.\\
En cohérence avec ce que nous avons montré précédemment, la plupart des effets du niveau des anciens pairs sont négatifs et de faible ampleur.\\
De la même manière, la Figure \ref{fig:pemodelsh5corrnppsfesDexossmoygraph} de l'Annexe \ref{pemodelsh5corrnppsfessmoy} qui montre les effets hétérogènes du niveau des nouveaux pairs et anciens pairs sur les notes aux écrits d'histoire-et-géographie, de dictée et de rédaction vient confirmer nos résultats.

\begin{figure}[H]

{\centering \includegraphics[width=1\linewidth]{000_files/figure-latex/pemodelsh5corrnppsfesDexograph-1} 

}

\caption{Effets de pairs hétérogènes, séparément pour les nouveaux pairs et anciens pairs du CM2, sans variables potentiellement endogènes}\label{fig:pemodelsh5corrnppsfesDexograph}
\end{figure}

\quad Enfin, la Figure \ref{fig:pemodelshnolinq5corrnppsfesDexograph} présente les effets hétérogènes et non linéaires du niveau des nouveaux pairs et des anciens pairs. Pour les nouveaux pairs, une comparaison détaillée des effets présentés dans cette figure avec celle obtenue sans séparation des nouveaux et anciens pairs (Figure \ref{fig:pemodelshnolinq5corrgraph}) ne fait ressortir aucune différence qualitative des résultats. En résumé, les élèves les plus faibles sont insensibles à la présence de nouveaux pairs, quel que soit le niveau de ces derniers. Pour les élèves des autres niveaux, les nouveaux pairs les plus faibles sont néfastes sauf les plus forts et les pairs les plus forts sont bénéfiques pour tous les élèves au moins de niveau moyen, ce bénéfice étant croissant avec le niveau de l'individu et légèrement plus fort en mathématiques qu'en français. La Figure met également en évidence le caractère négligeable des effets du niveau des anciens pairs. Ces constats sont maintenus lorsque nous regardons les résultats de l'exercice sur les autres matières (Figure \ref{fig:pemodelshnolinq5corrnppsfesDexossmoygraph} de l'Annexe \ref{pemodelshnolinq5corrnpssmoy}).

\begin{figure}[H]

{\centering \includegraphics[width=1\linewidth]{000_files/figure-latex/pemodelshnolinq5corrnppsfesDexograph-1} 

}

\caption{Effets de pairs hétérogènes et non linéaires, séparément pour les nouveaux pairs et anciens pairs, sans variables potentiellement endogènes}\label{fig:pemodelshnolinq5corrnppsfesDexograph}
\end{figure}

.

\hypertarget{peresendo}{%
\subsection{Les effets de pairs endogènes}\label{peresendo}}

Comme expliqué auparavant, les résultats de la forme réduite (Section \ref{peresfr}), nous assurent qu'il existe soit des effets de pairs exogènes uniquement, soit des effets endogènes uniquement, soit les deux. Chaque type d'effet a des implications de politiques publiques différentes.\\
Dans cette section, nous rapportons les effets de pairs endogènes issus de l'estimation de l'équation \eqref{eq:pebealwithin} par maximum de vraisemblance conditionnelle (\protect\hyperlink{ref-LEE:07}{Lee, 2007} ; \protect\hyperlink{ref-BOU:eal:14}{Boucher et al., 2014} ; \protect\hyperlink{ref-IZA:DIC:20}{Izaguirre \& Di Capua, 2020}). Les données utilisées sont les données complètes (colonne 1 du Tableau \ref{tab:pestats}). Il n'y a pas de problème lié au caractère aléatoire ou non de l'assignation des classes ni à la formation endogène des pairs à travers les écoles et collèges puisqu'il nous est possible de prendre directement en compte des effets fixes de classe.\\
Un avantage considérable des résultats de cette section est la possibilité de comparer les effets de pairs au CM2 et au DNB. Nous sommes à notre connaissance le premier papier à faire cette comparaison avec les effets de pairs endogènes.

\quad Le modèle mobilisé ne nous permet pas d'inclure des interactions en fonction de la note au DNB. Nous ne pouvons donc pas effectuer d'analyses d'effets de pairs hétérogènes et non linéaires comme dans la Section \ref{peresfr}. Pour avoir une idée de la différence des effets de pairs endogènes en fonction du niveau scolaire, nous réestimons l'équation \eqref{eq:pebealwithin} sur différents sous-échantillons d'établissement : les établissements privés, les établissements publics hors éducation prioritaires (HEP) et les établissements publics en éducation prioritaires (EP). Le type d'établissement est vraisemblablement corrélé avec les résultats potentiels des élèves. Empiriquement, sur les données complètes et parmi les retrouvés, la moyenne de la note au CM2 des élèves scolarisés dans les collèges privés est de 66 points sur 100. Elle est de 54 points dans les collèges publics hors éducation prioritaires et de 49 points dans les collèges publics en éducation prioritaire. Cette hiérarchie est maintenue en prenant la note en français au CM2, qui ne subit pas l'inflation de note d'année en année (au CM2) contrairement à la note totale. Nous ne pouvons pas faire le même raisonnement pour les écoles au CM2 puisque nous n'avons pas de mesure de note antérieure. Toutefois, nous pouvons considérer les proportions des différentes catégories sociales par type d'écoles, vu que la catégorie sociale est vraisemblablement corrélée avec les résultats potentiels au CM2. À La Réunion, 20\%, 8\% et 4.5\% des élèves de CM2 scolarisés respectivement dans les écoles privées, dans les écoles publiques hors éducation prioritaire et dans les écoles publiques en éducation prioritaire sont issus de catégorie sociale très favorisée.

\quad Les résultats tout établissement confondu et pour chaque établissement sont développés en Section \ref{peresendohomo} et \ref{peresendohetero}, respectivement.

\hypertarget{peresendohomo}{%
\subsubsection{Effets homogènes en fonction du type d'établissement}\label{peresendohomo}}

Le Tableau \ref{tab:pepcmlmodelscm2} rapporte les résultats d'estimation des effets endogènes comme décrits ci-dessus mais sur les données de CM2\footnote{Les statistiques descriptives liées à ces données se trouvent dans le chapitre précédent.}. Les variables explicatives sont les mêmes que dans les résultats précédents, hormis la taille de classe et le régime scolaire\footnote{Cette dernière variable n'est pas disponible dans les bases de données au CM2.}. En cohérence avec les résultats des sections précédentes, nous distinguons les cas où l'âge et la position sont utilisés comme variables explicatives en plus (colonnes 2, 4 et 6). Les variables de contrôles utilisées ont l'avantage d'assez bien correspondre à ce que la littérature utilise (\protect\hyperlink{ref-IZA:DIC:20}{Izaguirre \& Di Capua, 2020}).\\
La nature des effets endogènes estimés, contrairement à ceux de la Section \ref{peresfr}, change légèrement selon la variable dépendante utilisée. Par exemple, l'effet de 1.46 de la colonne (1) représente l'effet du comportement global des pairs du CM2 pendant toute l'année tandis que l'effet de 1.29 de la colonne (3) correspond à l'effet de la partie du comportement des pairs importante pour leurs notes en français\footnote{Le comportement des pairs pendant les cours de français, par exemple.}.

\quad Une augmentation d'une UET de la note au CM2 des pairs au CM2 implique une augmentation de la note individuelle au CM2 de l'ordre de 1.3 UET. Les effets ne sont pas significativement différents en français et en mathématiques. Ces ampleurs sont très supérieures à celles trouvées dans la littérature sur le sujet, qui est relativement rare. En mathématiques, Izaguirre \& Di Capua (\protect\hyperlink{ref-IZA:DIC:20}{2020}) (Amérique latine et Caraïbes) trouvent pour les élèves du primaire un effet significatif de l'ordre de 0.3 UET en mathématiques et d'environ 0.1 UET en langues. Boucher et al. (\protect\hyperlink{ref-BOU:eal:14}{2014}) trouvent un effet en mathématiques de l'ordre de 0.8 UET et des effets non significatifs dans les autres matières.\\
Une manière d'évaluer la crédibilité de ces chiffres est d'observer les coefficients associés aux variables de contrôles individuels et chez les pairs. Le fait d'être un garçon plutôt qu'une fille est désavantageux, particulièrement en français\footnote{Cela entre cohérence avec les résultats de la Section \ref{peresfr} ainsi qu'avec l'état de connaissances actuelles à la différence où les garçons sont plus souvent meilleurs en mathématiques dans les autres études (\protect\hyperlink{ref-HYD:eal:90}{Hyde et al., 1990}).}. L'effet de la catégorie sociale de l'individu est en accord avec ceux trouvés avec la forme réduite (Section \ref{peresfr}) et avec la littérature (\protect\hyperlink{ref-SPO:14}{Sposato, 2014}, par exemple) : être dans une catégorie sociale plus élevée procure un avantage significatif sur les performances éducatives. Les effets de pairs exogènes montrés dans ce tableau sont généralement non significatifs une fois séparés des effets endogènes. Ce résultat est également qualitativement le même à ceux retrouvés dans Boucher et al. (\protect\hyperlink{ref-BOU:eal:14}{2014}) et Izaguirre \& Di Capua (\protect\hyperlink{ref-IZA:DIC:20}{2020}).

\newpage  
\begingroup\fontsize{6}{8}\selectfont

\begin{ThreePartTable}
\begin{TableNotes}
\item \textit{Sources :} Fichiers CM2 (2010 à 2012), calculs de l'auteur.
\item \textit{Notes :} Une colonne correspond à une régression. La note contemporaine des pairs est de même nature que la variable dépendante (exemple : note totale dans la colonne (1) et note en français aux écrits dans la colonne (3)). Écart-types entre parenthèses. Des effets fixes de classe sont utilisés. Contrôles potentiellement endogènes : âge et position au CM2. Autres contrôles : sexe et catégorie socio-professionnelle (CSP).
\item Significativité : 10\% * 5\% ** 1\% ***.
\end{TableNotes}
\begin{longtable}[t]{lllllll}
\caption{\label{tab:pepcmlmodelscm2}Effets de pairs endogènes au CM2}\\
\toprule
\multicolumn{1}{c}{} & \multicolumn{6}{c}{Variable dépendante : Note} \\
\cmidrule(l{3pt}r{3pt}){2-7}
\multicolumn{1}{c}{} & \multicolumn{2}{c}{Totale} & \multicolumn{2}{c}{En français} & \multicolumn{2}{c}{En mathématiques} \\
\cmidrule(l{3pt}r{3pt}){2-3} \cmidrule(l{3pt}r{3pt}){4-5} \cmidrule(l{3pt}r{3pt}){6-7}
 & \makecell{Sans var.endo. \\ (1) } & \makecell{Avec var.endo. \\ (2) } & \makecell{Sans var.endo. \\ (3) } & \makecell{Avec var.endo. \\ (4) } & \makecell{Sans var.endo. \\ (5) } & \makecell{Avec var.endo. \\ (6) }\\
\midrule
\endfirsthead
\caption[]{\label{tab:pepcmlmodelscm2}Effets de pairs endogènes au CM2 (suite)}\\
\toprule
 & \makecell{Sans var.endo. \\ (1) } & \makecell{Avec var.endo. \\ (2) } & \makecell{Sans var.endo. \\ (3) } & \makecell{Avec var.endo. \\ (4) } & \makecell{Sans var.endo. \\ (5) } & \makecell{Avec var.endo. \\ (6) }\\
\midrule
\endhead

\endfoot
\bottomrule
\insertTableNotes
\endlastfoot
\addlinespace[0.3em]
\multicolumn{7}{l}{\textbf{}}\\
\hspace{1em}Note contemporaine des pairs & 1.466$^{***}$ & 1.355$^{***}$ & 1.288$^{***}$ & 1.202$^{***}$ & 1.306$^{***}$ & 1.169$^{***}$\\
\hspace{1em} & (0.156) & (0.15) & (0.151) & (0.146) & (0.147) & (0.142)\\
\addlinespace[0.3em]
\multicolumn{7}{l}{\textbf{}}\\
\hspace{1em}Âge aux examens & - & 0.096$^{***}$ & - & 0.076$^{***}$ & - & 0.118$^{***}$\\
\hspace{1em} & - & (0.026) & - & (0.026) & - & (0.026)\\
\hspace{1em}Position - À l'heure & - & 1.06$^{***}$ & - & 1.059$^{***}$ & - & 0.931$^{***}$\\
\hspace{1em} & - & (0.032) & - & (0.032) & - & (0.032)\\
\hspace{1em}Position - En avance & - & 1.883$^{***}$ & - & 1.821$^{***}$ & - & 1.742$^{***}$\\
\hspace{1em} & - & (0.067) & - & (0.067) & - & (0.068)\\
\hspace{1em}Sexe - Garçon & $-$0.282$^{***}$ & $-$0.211$^{***}$ & $-$0.395$^{***}$ & $-$0.323$^{***}$ & $-$0.083$^{***}$ & $-$0.024\\
\hspace{1em} & (0.016) & (0.014) & (0.016) & (0.014) & (0.015) & (0.015)\\
\hspace{1em}CSP - Moyenne & 0.307$^{***}$ & 0.223$^{***}$ & 0.311$^{***}$ & 0.225$^{***}$ & 0.267$^{***}$ & 0.196$^{***}$\\
\hspace{1em} & (0.021) & (0.02) & (0.021) & (0.02) & (0.021) & (0.02)\\
\hspace{1em}CSP - Favorisée & 0.57$^{***}$ & 0.436$^{***}$ & 0.531$^{***}$ & 0.395$^{***}$ & 0.556$^{***}$ & 0.444$^{***}$\\
\hspace{1em} & (0.037) & (0.034) & (0.037) & (0.034) & (0.037) & (0.035)\\
\hspace{1em}CSP - Très favorisée & 0.825$^{***}$ & 0.649$^{***}$ & 0.794$^{***}$ & 0.617$^{***}$ & 0.768$^{***}$ & 0.616$^{***}$\\
\hspace{1em} & (0.036) & (0.033) & (0.036) & (0.033) & (0.035) & (0.033)\\
\hspace{1em}CSP - Autre & $-$0.076 & $-$0.028 & $-$0.085 & $-$0.036 & $-$0.048 & $-$0.009\\
\hspace{1em} & (0.094) & (0.086) & (0.094) & (0.086) & (0.093) & (0.087)\\
\hspace{1em}CSP - Manquante & $-$0.133$^{***}$ & $-$0.07$^{***}$ & $-$0.138$^{***}$ & $-$0.072$^{***}$ & $-$0.113$^{***}$ & $-$0.06$^{***}$\\
\hspace{1em} & (0.023) & (0.021) & (0.023) & (0.021) & (0.023) & (0.022)\\
\addlinespace[0.3em]
\multicolumn{7}{l}{\textbf{Moyenne chez les pairs}}\\
\hspace{1em}Âge aux examens & - & $-$0.105 & - & $-$0.164 & - & 0.073\\
\hspace{1em} & - & (0.43) & - & (0.43) & - & (0.437)\\
\hspace{1em}Position - À l'heure & - & 1.009$^{*}$ & - & 0.937$^{*}$ & - & 1.215$^{**}$\\
\hspace{1em} & - & (0.527) & - & (0.526) & - & (0.534)\\
\hspace{1em}Position - En avance & - & 1.255 & - & 1.045 & - & 1.832$^{*}$\\
\hspace{1em} & - & (1.054) & - & (1.054) & - & (1.067)\\
\hspace{1em}Sexe - Garçon & 0.218 & 0.26 & 0.302 & 0.34 & $-$0.06 & $-$0.006\\
\hspace{1em} & (0.259) & (0.239) & (0.262) & (0.242) & (0.254) & (0.239)\\
\hspace{1em}CSP - Moyenne & $-$0.049 & $-$0.089 & 0.179 & 0.13 & $-$0.251 & $-$0.283\\
\hspace{1em} & (0.352) & (0.327) & (0.351) & (0.326) & (0.35) & (0.332)\\
\hspace{1em}CSP - Favorisée & 1.311$^{**}$ & 0.991 & 0.826 & 0.499 & 1.936$^{***}$ & 1.695$^{***}$\\
\hspace{1em} & (0.656) & (0.606) & (0.658) & (0.608) & (0.652) & (0.614)\\
\hspace{1em}CSP - Très favorisée & 1.139$^{*}$ & 0.8 & 1.131$^{*}$ & 0.782 & 1.145$^{*}$ & 0.878\\
\hspace{1em} & (0.675) & (0.62) & (0.675) & (0.62) & (0.67) & (0.629)\\
\hspace{1em}CSP - Autre & $-$1.681 & $-$1.657 & $-$2.441$^{*}$ & $-$2.412$^{*}$ & $-$0.235 & $-$0.245\\
\hspace{1em} & (1.389) & (1.274) & (1.393) & (1.277) & (1.379) & (1.292)\\
\hspace{1em}CSP - Manquante & 1.049$^{***}$ & 0.828$^{**}$ & 0.993$^{***}$ & 0.772$^{**}$ & 0.882$^{**}$ & 0.733$^{**}$\\
\hspace{1em} & (0.374) & (0.349) & (0.374) & (0.349) & (0.371) & (0.354)\\
 &  &  &  &  &  & \\
Observations & 42015 & 42015 & 42015 & 42015 & 42015 & 42015\\*
\end{longtable}
\end{ThreePartTable}
\endgroup{}

\quad Au DNB, selon le Tableau \ref{tab:pepcmlmodels}, les effets de pairs endogènes ont au moins doublé en mathématiques (colonnes 5 et 6) et ont environ diminué de moitié en français (colonnes 3 et 4). En analysant en plus les effets endogènes sur les autres matières (Tableau \ref{tab:pepcmlmodelsssmoy} de l'Annexe \ref{pepcmlmodelsssmoy}), nous trouvons que les comportements des pairs qui déterminent la note en histoire-et-géographie (colonnes 1 et 2) sont très importants. De plus, les effets de pairs en français ne semblent provenir que des comportements des pairs déterminants la note de dictée et non la rédaction\footnote{Les résultats en sous-items dans le cas du CM2 ne sont pas montrés à cause du fait que la méthode d'estimation suppose que les notes ont une distribution normale. La distribution des notes agrégées au CM2 (totale, en français et en mathématiques) prend la forme d'une courbe en cloche mais ce n'est pas le cas pour les sous-items.}.\\
Les effets de pairs exogènes semblent cependant plus importants au DNB par rapport au CM2, sauf en mathématiques.\\
En accord avec le chapitre précédent sur l'augmentation des effets de l'âge relatif entre le CM2 et la 3\textsuperscript{ème}, des effets de pairs plus importants en fin de collège par rapport à la fin du primaire suggèrent plus grande capacité des élèves à interagir au collège.

\begingroup\fontsize{6}{8}\selectfont

\begin{ThreePartTable}
\begin{TableNotes}
\item \textit{Sources :} Fichiers DNB (2014 à 2016), fichiers CONSTAT (2013 à 2016), calculs de l'auteur.
\item \textit{Notes :} Une colonne correspond à une régression. La note contemporaine des pairs est de même nature que la variable dépendante (exemple : Note totale dans la colonne (1) et note en français aux écrits dans la colonne (3)). Les valeurs manquantes sur les variables dépendantes sont exclues avant toute estimation. Écart-types entre parenthèses. Des effets fixes de classe sont utilisés. Contrôles potentiellement endogènes : âge et position au DNB. Autres contrôles : sexe, catégorie socio-professionnelle (CSP) et régime scolaire.
\item Significativité : 10\% * 5\% ** 1\% ***.
\end{TableNotes}
\begin{longtable}[t]{lllllll}
\caption{\label{tab:pepcmlmodels}Effets de pairs endogènes au DNB}\\
\toprule
\multicolumn{1}{c}{} & \multicolumn{6}{c}{Variable dépendante : Note} \\
\cmidrule(l{3pt}r{3pt}){2-7}
\multicolumn{1}{c}{} & \multicolumn{2}{c}{Totale (écrits)} & \multicolumn{2}{c}{En français (écrits)} & \multicolumn{2}{c}{En mathématiques (écrits)} \\
\cmidrule(l{3pt}r{3pt}){2-3} \cmidrule(l{3pt}r{3pt}){4-5} \cmidrule(l{3pt}r{3pt}){6-7}
 & \makecell{Sans var.endo. \\ (1) } & \makecell{Avec var.endo. \\ (2) } & \makecell{Sans var.endo. \\ (3) } & \makecell{Avec var.endo. \\ (4) } & \makecell{Sans var.endo. \\ (5) } & \makecell{Avec var.endo. \\ (6) }\\
\midrule
\endfirsthead
\caption[]{\label{tab:pepcmlmodels}Effets de pairs endogènes au DNB (suite)}\\
\toprule
 & \makecell{Sans var.endo. \\ (1) } & \makecell{Avec var.endo. \\ (2) } & \makecell{Sans var.endo. \\ (3) } & \makecell{Avec var.endo. \\ (4) } & \makecell{Sans var.endo. \\ (5) } & \makecell{Avec var.endo. \\ (6) }\\
\midrule
\endhead

\endfoot
\bottomrule
\insertTableNotes
\endlastfoot
\addlinespace[0.3em]
\multicolumn{7}{l}{\textbf{}}\\
\hspace{1em}Note contemporaine des pairs & 2.825$^{***}$ & 3.354$^{***}$ & 0.553$^{*}$ & 0.803$^{**}$ & 2.982$^{***}$ & 2.854$^{***}$\\
\hspace{1em} & (0.492) & (0.516) & (0.324) & (0.338) & (0.441) & (0.429)\\
\addlinespace[0.3em]
\multicolumn{7}{l}{\textbf{}}\\
\hspace{1em}Âge aux examens & - & $-$0.02 & - & $-$0.056 & - & $-$0.123$^{**}$\\
\hspace{1em} & - & (0.054) & - & (0.049) & - & (0.055)\\
\hspace{1em}Position - À l'heure & - & 1.038$^{***}$ & - & 0.786$^{***}$ & - & 0.763$^{***}$\\
\hspace{1em} & - & (0.072) & - & (0.061) & - & (0.068)\\
\hspace{1em}Position - En avance & - & 0.728$^{***}$ & - & 0.376$^{*}$ & - & 0.561$^{**}$\\
\hspace{1em} & - & (0.245) & - & (0.224) & - & (0.249)\\
\hspace{1em}Sexe - Garçon & $-$0.213$^{***}$ & $-$0.245$^{***}$ & $-$0.435$^{***}$ & $-$0.456$^{***}$ & $-$0.007 & $-$0.021\\
\hspace{1em} & (0.041) & (0.041) & (0.038) & (0.038) & (0.042) & (0.041)\\
\hspace{1em}CSP - Moyenne & 0.447$^{***}$ & 0.353$^{***}$ & 0.353$^{***}$ & 0.294$^{***}$ & 0.304$^{***}$ & 0.239$^{***}$\\
\hspace{1em} & (0.05) & (0.05) & (0.044) & (0.045) & (0.048) & (0.048)\\
\hspace{1em}CSP - Favorisée & 0.55$^{***}$ & 0.498$^{***}$ & 0.436$^{***}$ & 0.412$^{***}$ & 0.464$^{***}$ & 0.427$^{***}$\\
\hspace{1em} & (0.058) & (0.058) & (0.053) & (0.055) & (0.058) & (0.059)\\
\hspace{1em}CSP - Très favorisée & 0.563$^{***}$ & 0.464$^{***}$ & 0.382$^{***}$ & 0.321$^{***}$ & 0.611$^{***}$ & 0.526$^{***}$\\
\hspace{1em} & (0.101) & (0.1) & (0.093) & (0.091) & (0.104) & (0.101)\\
\hspace{1em}CSP - Autre & 0.692$^{***}$ & 0.778$^{***}$ & 0.214 & 0.196 & 0.389 & 0.406\\
\hspace{1em} & (0.208) & (0.204) & (0.3) & (0.293) & (0.297) & (0.287)\\
\addlinespace[0.3em]
\multicolumn{7}{l}{\textbf{Moyenne chez les pairs}}\\
\hspace{1em}Âge aux examens & - & 0.354 & - & $-$0.875 & - & $-$1.585\\
\hspace{1em} & - & (1.175) & - & (1.061) & - & (1.176)\\
\hspace{1em}Position - À l'heure & - & 8.619$^{***}$ & - & 5.295$^{***}$ & - & 4.885$^{***}$\\
\hspace{1em} & - & (1.43) & - & (1.261) & - & (1.394)\\
\hspace{1em}Position - En avance & - & $-$15.935$^{***}$ & - & $-$16.907$^{***}$ & - & $-$16.467$^{***}$\\
\hspace{1em} & - & (5.773) & - & (5.297) & - & (5.876)\\
\hspace{1em}Sexe - Garçon & 0.746 & $-$0.788 & $-$0.129 & $-$1.258 & 1.588$^{*}$ & 0.462\\
\hspace{1em} & (0.894) & (0.889) & (0.819) & (0.817) & (0.914) & (0.898)\\
\hspace{1em}CSP - Moyenne & 3.597$^{***}$ & 2.362$^{**}$ & 2.49$^{***}$ & 1.905$^{**}$ & 1.631 & 1.071\\
\hspace{1em} & (1.053) & (1.065) & (0.947) & (0.964) & (1.042) & (1.046)\\
\hspace{1em}CSP - Favorisée & 2.226$^{*}$ & 2.145$^{*}$ & 1.66 & 2.08$^{*}$ & 1.934 & 2.293$^{*}$\\
\hspace{1em} & (1.219) & (1.249) & (1.136) & (1.178) & (1.251) & (1.27)\\
\hspace{1em}CSP - Très favorisée & $-$5.096$^{**}$ & $-$5.332$^{**}$ & $-$5.181$^{**}$ & $-$4.868$^{**}$ & $-$1.834 & $-$1.539\\
\hspace{1em} & (2.406) & (2.38) & (2.19) & (2.162) & (2.472) & (2.402)\\
\hspace{1em}CSP - Autre & 17.192$^{***}$ & 16.693$^{***}$ & 5.324 & 2.974 & 9.702 & 7.977\\
\hspace{1em} & (4.294) & (4.207) & (6.399) & (6.26) & (6.277) & (6.047)\\
 &  &  &  &  &  & \\
Observations & 41565 & 41565 & 41496 & 41496 & 41342 & 41342\\*
\end{longtable}
\end{ThreePartTable}
\endgroup{}

\hypertarget{peresendohetero}{%
\subsubsection{Effets hétérogènes en fonction du type d'établissement}\label{peresendohetero}}

\quad Le Tableau \ref{tab:pepcmlmodelscm2statutreseau} présente les résultats d'estimation des effets de pairs endogènes séparément pour les établissements privés, publics HEP et publics EP. Au CM2 et en mathématiques (panel A, colonnes 5 et 6), les effets endogènes sont les plus importants dans les écoles privées. Ce fait est directement en accord avec les résultats de Izaguirre \& Di Capua (\protect\hyperlink{ref-IZA:DIC:20}{2020, p. 79}). Les magnitudes des coefficients en français dans les écoles privées (colonnes 3 et 4 du panel A) sont considérables mais ils ne sont pas significatifs. Dans les écoles publiques HEP (panel B), les effets de pairs existent et ne sont pas significativement différents selon la matière. Les effets mesurés dans les écoles publiques en EP (panel C) sont tous significatifs et légèrement supérieurs à ceux des écoles HEP.\\
Ces résultats suggèrent qu'il peut toujours être intéressant de cibler une partie des élèves en fin de primaire, indépendamment du type d'établissement, pour des politiques visant à améliorer les résultats en mathématiques. Les effets de pairs endogènes sont censés transférer les comportements des élèves ciblés vers leurs camarades de classes non ciblés.

\begin{landscape}\begingroup\fontsize{8}{10}\selectfont

\begin{ThreePartTable}
\begin{TableNotes}
\item \textit{Sources :} Fichiers CM2 (2010 à 2012), calculs de l'auteur.
\item \textit{Notes :} Une colonne correspond à une régression. La note contemporaine des pairs est de même nature que la variable dépendante (exemple : Note totale dans la colonne (1) et note en français aux écrits dans la colonne (3)). Écart-types entre parenthèses. Des effets fixes de classe sont utilisés. Contrôles potentiellement endogènes : âge et position CM2. Autres contrôles : sexe et catégorie socio-professionnelle (CSP). (H)EP : (Hors-)Éducation Prioritaire.
\item Significativité : 10\% * 5\% ** 1\% ***.
\end{TableNotes}
\begin{longtable}[t]{lllllll}
\caption{\label{tab:pepcmlmodelscm2statutreseau}Effets de pairs endogènes au CM2, par type d'école}\\
\toprule
\multicolumn{1}{c}{} & \multicolumn{6}{c}{Variable dépendante : Note} \\
\cmidrule(l{3pt}r{3pt}){2-7}
\multicolumn{1}{c}{} & \multicolumn{2}{c}{Totale (écrits)} & \multicolumn{2}{c}{En français (écrits)} & \multicolumn{2}{c}{En mathématiques (écrits)} \\
\cmidrule(l{3pt}r{3pt}){2-3} \cmidrule(l{3pt}r{3pt}){4-5} \cmidrule(l{3pt}r{3pt}){6-7}
 & \makecell{Sans var.endo. \\ (1) } & \makecell{Avec var.endo. \\ (2) } & \makecell{Sans var.endo. \\ (3) } & \makecell{Avec var.endo. \\ (4) } & \makecell{Sans var.endo. \\ (5) } & \makecell{Avec var.endo. \\ (6) }\\
\midrule
\endfirsthead
\caption[]{\label{tab:pepcmlmodelscm2statutreseau}Effets de pairs endogènes au CM2, par type d'école (suite)}\\
\toprule
 & \makecell{Sans var.endo. \\ (1) } & \makecell{Avec var.endo. \\ (2) } & \makecell{Sans var.endo. \\ (3) } & \makecell{Avec var.endo. \\ (4) } & \makecell{Sans var.endo. \\ (5) } & \makecell{Avec var.endo. \\ (6) }\\
\midrule
\endhead

\endfoot
\bottomrule
\insertTableNotes
\endlastfoot
\addlinespace[0.3em]
\multicolumn{7}{l}{\textbf{Panel A : Écoles privées}}\\
\hline
\hspace{1em}Note contemporaine des pairs & 2.268 & 2.774$^{*}$ & 1.23 & 1.824 & 3.235$^{**}$ & 3.724$^{**}$\\
\hspace{1em} & (1.399) & (1.485) & (1.209) & (1.31) & (1.578) & (1.673)\\
\hspace{1em} &  &  &  &  &  \vphantom{5} & \\
\hspace{1em}Contrôles individuels & Oui & Oui & Oui & Oui & Oui & \vphantom{2} Oui\\
\hspace{1em}Contrôles chez les pairs & Oui & Oui & Oui & Oui & Oui & \vphantom{2} Oui\\
\hspace{1em}Contrôles potentiellement endogènes & Non & Oui & Non & Oui & Non & \vphantom{2} Oui\\
\hspace{1em}Observations & 3415 & 3415 & 3415 & 3415 & 3415 & 3415\\
 &  &  &  &  &  \vphantom{4} & \\
\addlinespace[0.3em]
\multicolumn{7}{l}{\textbf{Panel B : Écoles publiques HEP}}\\
\hline
\hspace{1em}Note contemporaine des pairs & 1.423$^{***}$ & 1.311$^{***}$ & 1.351$^{***}$ & 1.288$^{***}$ & 1.107$^{***}$ & 0.949$^{***}$\\
\hspace{1em} & (0.222) & (0.214) & (0.221) & (0.215) & (0.201) & (0.193)\\
\hspace{1em} &  &  &  &  &  \vphantom{3} & \\
\hspace{1em}Contrôles individuels & Oui & Oui & Oui & Oui & Oui & \vphantom{1} Oui\\
\hspace{1em}Contrôles chez les pairs & Oui & Oui & Oui & Oui & Oui & \vphantom{1} Oui\\
\hspace{1em}Contrôles potentiellement endogènes & Non & Oui & Non & Oui & Non & \vphantom{1} Oui\\
\hspace{1em}Observations & 18517 & 18517 & 18517 & 18517 & 18517 & 18517\\
 &  &  &  &  &  \vphantom{2} & \\
\addlinespace[0.3em]
\multicolumn{7}{l}{\textbf{Panel C : Écoles publiques EP}}\\
\hline
\hspace{1em}Note contemporaine des pairs & 1.788$^{***}$ & 1.605$^{***}$ & 1.572$^{***}$ & 1.388$^{***}$ & 1.659$^{***}$ & 1.528$^{***}$\\
\hspace{1em} & (0.237) & (0.226) & (0.227) & (0.216) & (0.23) & (0.224)\\
\hspace{1em} &  &  &  &  &  \vphantom{1} & \\
\hspace{1em}Contrôles individuels & Oui & Oui & Oui & Oui & Oui & Oui\\
\hspace{1em}Contrôles chez les pairs & Oui & Oui & Oui & Oui & Oui & Oui\\
\hspace{1em}Contrôles potentiellement endogènes & Non & Oui & Non & Oui & Non & Oui\\
\hspace{1em}Observations & 20083 & 20083 & 20083 & 20083 & 20083 & 20083\\
 &  &  &  &  &  & \\*
\end{longtable}
\end{ThreePartTable}
\endgroup{}
\end{landscape}

Le Tableau \ref{tab:pepcmlmodelsstatutreseau} illustre les effets de pairs endogènes au DNB en fonction du type d'établissement. Nous constatons que quasiment aucun effet ne ressort significatif dans les collèges privés (panel A). Toutefois, les ampleurs sont grandes. Au sein des collèges publics HEP (panel B), les effets de pairs sont tous positifs, importants et significatifs sans qu'il n'y ait de différence en fonction de la matière. De manière plus intéressante, le panel C du tableau indique que les effets de pairs, comparés aux autres types d'établissement, sont les plus prononcés dans les établissements en EP. Ces deux derniers constats sont relativement les mêmes qu'avec les élèves du CM2. Les pairs sont davantage importants dans la détermination du comportement de l'individu en lien avec les résultats en mathématiques (colonnes 5 et 6 du panel C).\\
Les résultats sur les autres matières (Tableau \ref{tab:pepcmlmodelsssmoystatutreseau} de l'Annexe \ref{pepcmlmodelsssmoystatutreseau}) ne nous en apprend apparemment pas plus puisque nous pouvons juste observer que des effets peuvent exister en histoire-et-géographie au sein des collèges privés et que le résultat en rédaction n'est pas impacté par les pairs sauf dans les collèges publics en éducation prioritaire.

\quad Globalement, ces résultats suggèrent aussi qu'en fin de collège, des mesures éducatives ciblées sur une partie des élèves des établissements publics (HEP en priorité) peuvent être prometteuses, surtout en mathématiques.

\begin{landscape}\begingroup\fontsize{8}{10}\selectfont

\begin{ThreePartTable}
\begin{TableNotes}
\item \textit{Sources :} Fichiers DNB (2014 à 2016), fichiers CONSTAT (2013 à 2016), calculs de l'auteur.
\item \textit{Notes :} Une colonne correspond à une régression. La note contemporaine des pairs est de même nature que la variable dépendante (exemple : Note totale dans la colonne (1) et note en français aux écrits dans la colonne (3)). Les valeurs manquantes sur les variables dépendantes sont exclues avant toute estimation. Écart-types entre parenthèses. Des effets fixes de classe sont utilisés. Contrôles potentiellement endogènes : âge et position au DNB. Autres contrôles : sexe, catégorie socio-professionnelle (CSP) et régime scolaire. (H)EP : (Hors-)Éducation Prioritaire.
\item Significativité : 10\% * 5\% ** 1\% ***.
\end{TableNotes}
\begin{longtable}[t]{lllllll}
\caption{\label{tab:pepcmlmodelsstatutreseau}Effets de pairs endogènes au DNB, par type de collège}\\
\toprule
\multicolumn{1}{c}{} & \multicolumn{6}{c}{Variable dépendante : Note} \\
\cmidrule(l{3pt}r{3pt}){2-7}
\multicolumn{1}{c}{} & \multicolumn{2}{c}{Totale (écrits)} & \multicolumn{2}{c}{En français (écrits)} & \multicolumn{2}{c}{En mathématiques (écrits)} \\
\cmidrule(l{3pt}r{3pt}){2-3} \cmidrule(l{3pt}r{3pt}){4-5} \cmidrule(l{3pt}r{3pt}){6-7}
 & \makecell{Sans var.endo. \\ (1) } & \makecell{Avec var.endo. \\ (2) } & \makecell{Sans var.endo. \\ (3) } & \makecell{Avec var.endo. \\ (4) } & \makecell{Sans var.endo. \\ (5) } & \makecell{Avec var.endo. \\ (6) }\\
\midrule
\endfirsthead
\caption[]{\label{tab:pepcmlmodelsstatutreseau}Effets de pairs endogènes au DNB, par type de collège (suite)}\\
\toprule
 & \makecell{Sans var.endo. \\ (1) } & \makecell{Avec var.endo. \\ (2) } & \makecell{Sans var.endo. \\ (3) } & \makecell{Avec var.endo. \\ (4) } & \makecell{Sans var.endo. \\ (5) } & \makecell{Avec var.endo. \\ (6) }\\
\midrule
\endhead

\endfoot
\bottomrule
\insertTableNotes
\endlastfoot
\addlinespace[0.3em]
\multicolumn{7}{l}{\textbf{Panel A : Collèges privés}}\\
\hline
\hspace{1em}Note contemporaine des pairs & 7.231$^{*}$ & 6.575 & 1.353 & 0.808 & 1.302 & 0.992\\
\hspace{1em} & (4.332) & (4.309) & (2.421) & (2.242) & (2.295) & (2.22)\\
\hspace{1em} &  &  &  &  &  \vphantom{5} & \\
\hspace{1em}Contrôles individuels & Oui & Oui & Oui & Oui & Oui & \vphantom{2} Oui\\
\hspace{1em}Contrôles chez les pairs & Oui & Oui & Oui & Oui & Oui & \vphantom{2} Oui\\
\hspace{1em}Contrôles potentiellement endogènes & Non & Oui & Non & Oui & Non & \vphantom{2} Oui\\
\hspace{1em}Observations & 3415 & 3415 & 3415 & 3415 & 3415 & 3415\\
 &  &  &  &  &  \vphantom{4} & \\
\addlinespace[0.3em]
\multicolumn{7}{l}{\textbf{Panel B : Collèges publics HEP}}\\
\hline
\hspace{1em}Note contemporaine des pairs & 3.913$^{***}$ & 3.7$^{***}$ & 1.671$^{***}$ & 2.385$^{***}$ & 1.576$^{***}$ & 1.413$^{***}$\\
\hspace{1em} & (0.843) & (0.812) & (0.632) & (0.775) & (0.536) & (0.515)\\
\hspace{1em} &  &  &  &  &  \vphantom{3} & \\
\hspace{1em}Contrôles individuels & Oui & Oui & Oui & Oui & Oui & \vphantom{1} Oui\\
\hspace{1em}Contrôles chez les pairs & Oui & Oui & Oui & Oui & Oui & \vphantom{1} Oui\\
\hspace{1em}Contrôles potentiellement endogènes & Non & Oui & Non & Oui & Non & \vphantom{1} Oui\\
\hspace{1em}Observations & 18159 & 18159 & 18140 & 18140 & 18089 & 18089\\
 &  &  &  &  &  \vphantom{2} & \\
\addlinespace[0.3em]
\multicolumn{7}{l}{\textbf{Panel C : Collèges publics EP}}\\
\hline
\hspace{1em}Note contemporaine des pairs & 6.624$^{***}$ & 8.185$^{***}$ & 2.748$^{***}$ & 3.55$^{***}$ & 6.779$^{***}$ & 6.185$^{***}$\\
\hspace{1em} & (1.114) & (1.316) & (0.689) & (0.798) & (1.105) & (1.045)\\
\hspace{1em} &  &  &  &  &  \vphantom{1} & \\
\hspace{1em}Contrôles individuels & Oui & Oui & Oui & Oui & Oui & Oui\\
\hspace{1em}Contrôles chez les pairs & Oui & Oui & Oui & Oui & Oui & Oui\\
\hspace{1em}Contrôles potentiellement endogènes & Non & Oui & Non & Oui & Non & Oui\\
\hspace{1em}Observations & 19991 & 19991 & 19941 & 19941 & 19838 & 19838\\
 &  &  &  &  &  & \\*
\end{longtable}
\end{ThreePartTable}
\endgroup{}
\end{landscape}

\hypertarget{peconcl}{%
\section{Conclusion}\label{peconcl}}

Ce travail s'intéresse aux effets du niveau initial des pairs de la classe (forme réduite) et de leurs comportements (effets de pairs endogènes) sur les résultats en fin de collège à La Réunion. Concernant l'estimation d'une forme réduite, la stratégie d'identification s'appuie sur la variation intra-établissement inter-classes du niveau des pairs tout en prenant en compte les valeurs manquantes dans la variable de niveau scolaire de l'élève et donc de ses pairs (\protect\hyperlink{ref-SOJ:13}{Sojourner, 2013}). Nous analysons plusieurs spécifications possibles des effets de pairs (\protect\hyperlink{ref-HOX:WEI:05}{Hoxby \& Weingarth, 2005}). Des classes et établissements ont été exclus des estimations afin de s'affranchir du biais dû à la formation endogène des classes, à établissement donné (\protect\hyperlink{ref-BOU:MAI:18}{Boutchenik \& Maillard, 2018}, par exemple). Les effets endogènes sont quant à eux estimés principalement grâce au fait que la taille des classes présente suffisamment de variation (\protect\hyperlink{ref-BOU:eal:14}{Boucher et al., 2014}).

\quad En forme réduite, nous trouvons des effets de pairs linéaires en moyenne d'ampleur considérable, de l'ordre de 0.2 UET.
Avec des analyses plus précises, nous trouvons d'abord que les élèves les plus faibles ne sont généralement pas sensibles aux effets de pairs et que les élèves de niveau moyen et forts (sans les plus forts) pâtissent de la présence de pairs faibles. Alors, afin d'augmenter les performances des élèves des collèges grâce aux effets de pairs, les responsables ont de bonnes raisons de séparer autant que possible les élèves les plus faibles des élèves de niveau plus élevé, hormis les plus forts. Cela justifie dans une certaine mesure la constitution des classes de niveau pour les plus faibles (des classes constituées uniquement d'élèves de SEGPA, entre autres). Toutefois, ces classes de niveau engendrent pratiquement une ségrégation des classes sociales défavorisées puisque ces dernières sont surreprésentées parmi les élèves les plus faibles.

Nous trouvons ensuite que les élèves ayant un niveau au moins moyen bénéficient de la présence des pairs les plus forts et que le bénéfice est maximal pour les élèves les plus forts. Les élèves les plus forts doivent alors être appariés avec des élèves moins forts, les meilleurs d'entre ces derniers en priorité. Vu sous cet angle, les classes de niveau pour les élèves les plus forts (des classes constituées uniquement d'élèves en option internationale, par exemple) engendrent un manque à gagner pour les élèves moins forts qui auraient bénéficié de la présence de ces élèves forts. Il y a toutefois un arbitrage à faire puisque les effets sont au maximum entre les élèves les plus forts entre eux, ce qui implique qu'assigner les élèves les plus forts à des élèves d'autres niveaux constitue aussi un manque à gagner pour les premiers.

\quad Il est important de mentionner que ces recommandations ne sont pas à appliquer à grande échelle dans l'état actuel de ce qui est connu sur les effets du niveau scolaire des pairs. Notamment, il a été constaté qu'une fois les groupes reconstitués de manière ``optimale'' (par rapport aux effets de pairs mesurés), les individus avaient des comportements d'ajustement tels que ceux qui se ressemblaient (ici, ceux qui ont des niveaux scolaires similaires) constituaient de petits sous-groupes relativement isolés. Les résultats attendus par rapport à l'amélioration des performances via les effets de pairs n'ont donc pas été obtenus (\protect\hyperlink{ref-CAR:eal:13}{Carrell et al., 2013} ; \protect\hyperlink{ref-BOO:eal:17}{Booij et al., 2017}). Ce phénomène est aussi appelé le problème de l'ajustement endogène.

\quad Pour les effets endogènes, même si nous trouvons des ampleurs considérables, nos résultats sont conformes à ce que trouve la littérature. Nous trouvons des effets de pairs endogènes positifs, significatifs et de grande ampleur. En fin de collège, ils doublent en mathématiques et diminuent de moitié en français par rapport au primaire. Globalement et en accord avec la littérature, les effets sont supérieurs en mathématiques. Ces résultats, combinés avec des analyses par type d'établissement, nous amènent à penser que les effets de pairs peuvent constituer un levier prometteur dans les collèges publics pour améliorer les résultats en mathématiques. Un type d'intervention particulièrement intéressant, inspiré de Avvisati et al. (\protect\hyperlink{ref-AVV:eal:14}{2014}) consiste à améliorer l'implication des parents dans l'éducation de leur enfant. Ces auteurs ont montré qu'il était possible d'améliorer la proximité entre les parents et l'école ainsi que le degré d'implication des parents (assistance pendant les devoirs à la maison, intéressement à l'expérience scolaire de son enfant, etc.) et que cela a un effet positif sur les comportements scolaires (assiduité et discipline). Seulement une partie des parents des enfants par classe ont bénéficié des incitations mais des effets positifs ont été observés pour les enfants des non bénéficiaires, via des effets d'interactions sociales. Toutefois, ce genre d'intervention n'a pas eu d'effet sur les notes des bénéficiaires ou non bénéficiaires. À notre connaissance, pour espérer obtenir une augmentation des performances par établissement, les notes en mathématiques apparaissent comme un canal naturel, vu que les élèves de La Réunion semblent particulièrement faibles en mathématiques et que les deux facteurs étudiés jusqu'ici dans cette thèse (âge, caractéristiques et comportements des camarades de classe) semblent avoir plus d'effets en mathématiques qu'en français, au collège. En s'inspirant en partie de propositions suédoises (\protect\hyperlink{ref-BOE:HEL:09}{Boesen \& Helenius, 2009}), nous pouvons proposer premièrement de sensibiliser une partie des parents\footnote{L'intérêt de ne considérer qu'une partie se trouve toujours dans la réduction des coûts.} à l'intérêt des mathématiques et à transmettre cet intérêt à leur enfant. Ensuite, les technologies d'apprentissage assistés par ordinateurs\footnote{Ils ont la particularité de proposer un apprentissage individuellement adapté au niveau de l'élève.} constituent une piste relativement solide pour l'amélioration des facultés elles-mêmes (\protect\hyperlink{ref-BAR:eal:09}{Barrow et al., 2009}). Nous avons alors des raisons de penser que, en complément des mesures liées aux parents ci-dessus, faire bénéficier une partie élèves inscrits dans les écoles et collèges publics de ces technologies aura un impact considérable sur leurs résultats en mathématiques et à terme sur ceux de leurs camarades de classes. Bien sûr, il convient de commencer à tester de telles mesures à petite échelle.

\hypertarget{g20}{%
\chapter{Effets de la révision des mathématiques via une plateforme en ligne sur les résultats en mathématiques : cas d'étudiants en première année à l'Université de La Réunion}\label{g20}}

\chaptermark{Efficacité d'une plateforme de révision des mathématiques}

\newpage

\hypertarget{g20intro}{%
\section{Introduction}\label{g20intro}}

Le faible niveau en mathématiques des élèves français par rapport à d'autres pays européens est un fait. La dernière enquête TIMSS de 2019 montre par exemple qu'en mathématiques, les élèves de CM1 de la France se trouvent en dessous des niveaux moyens de l'Union européenne et de l'OCDE (\protect\hyperlink{ref-COL:LEC:20}{Colmant \& Le Cam, 2020}). De plus, le niveau en mathématiques a chuté avec le temps, surtout pour les élèves de 4\textsuperscript{ème} (\protect\hyperlink{ref-SAL:LEC:20}{Salles \& Le Cam, 2020}). Cela est vrai dans une moindre mesure pour des élèves plus âgés (15 ans) qui ont participé aux enquêtes PISA (\protect\hyperlink{ref-BER:eal:19}{Bernigole et al., 2019} ; \protect\hyperlink{ref-OCD:19}{OCDE, 2019}). Cela a aussi été constaté dans plusieurs pays à l'entrée des études supérieures. Les étudiants n'ont alors globalement pas le niveau en mathématiques attendu à l'entrée de l'université (\protect\hyperlink{ref-KAJ:LOV:05}{Kajander \& Lovric, 2005} ; \protect\hyperlink{ref-ENG:eal:15}{Engelbrecht \& Harding, 2015}).

Ce dernier constat peut être partiellement expliqué par la massification de l'accès aux études supérieures (\protect\hyperlink{ref-DEP:21}{DEPP, 2021b, p. 148}, par exemple) qui élargit mécaniquement le spectre de niveau en mathématiques des étudiants. Cette massification est elle-même liée à une baisse des exigences du niveau d'entrée, notamment par souci de lutte contre les inégalités d'accès aux études supérieures (\protect\hyperlink{ref-BEC:eal:21}{Bechichi et al., 2021}). L'image dégradée des mathématiques et de ses enseignants ainsi que la transmission du manque d'intérêt des mathématiques d'une génération à l'autre est une autre raison valable (\protect\hyperlink{ref-KAJ:LOV:05}{Kajander \& Lovric, 2005}). Un autre facteur majeur du niveau insuffisant des étudiants en mathématiques à l'entrée de l'université est la transition entre le secondaire et le supérieur, une problématique avérée depuis longtemps (\protect\hyperlink{ref-DEG:98}{De Guzmán et al., 1998}) et particulièrement mise en avant dans les pays anglo-saxons (\protect\hyperlink{ref-ENG:00}{Engeneering Council, 2000}). Nous renvoyons à Gueudet \& Vandebrouck (\protect\hyperlink{ref-GUE:VAN:22}{2022}) pour une exposition plus détaillée du problème.
Les moyens investis sur les élèves du secondaire ne s'avèrent alors pas rentables par rapport à leurs performances effectives en mathématiques (\protect\hyperlink{ref-VIL:eal:18}{Villani et al., 2018}).

\quad En plus d'être crucial dans la vie quotidienne et dans la réussite sur le marché du travail (\protect\hyperlink{ref-HAN:eal:15}{Hanushek et al., 2015}), avoir un certain niveau en mathématiques est fondamental pour certaines filières des études supérieures (\protect\hyperlink{ref-FAUL:eal:14}{Faulkner et al., 2014}). Le déclin du niveau en mathématiques des étudiants et élèves est alors problématique puisqu'il implique pour la France un capital humain de moins bonne qualité.

\quad Dans ce contexte, l'enseignement des mathématiques a été déclaré comme une priorité nationale en France. Toutefois, les différentes politiques se concentrent surtout dans l'enseignement primaire et secondaire. Ceci peut être due à une raison opérationnelle puisque les responsables de l'enseignement primaire et secondaire sont uniformément soumis aux programmes de l'éducation nationale tandis que ceux de l'enseignement supérieur ont plus de flexibilité\footnote{De plus, les actions dans le primaire et le secondaire sont contraintes par les priorités données aux examens officiels (brevet et bac). Ces contraintes n'existent pas dans le supérieur.}.

Ce relatif manque d'attention sur le supérieur se retrouve à l'international. Les différentes interventions en mathématiques dans le monde se concentrent majoritairement sur des élèves de l'enseignement primaire ou secondaire (voir Section \ref{g20litt}). Avec le développement exponentiel de la technologie en général, l'utilisation de cette dernière dans lesdites interventions, y compris en France, se multiplient également (\protect\hyperlink{ref-XIE:eal:20}{Xie et al., 2020} ; \protect\hyperlink{ref-GUE:VAN:22}{Gueudet \& Vandebrouck, 2022}). Il en va de même pour l'évaluation de leur efficacité.

\quad Ce chapitre s'intéresse à l'efficacité d'un dispositif visant à maintenir le contact avec les mathématiques des sortants du lycée pour que ces derniers puissent mieux suivre les cours de mathématiques de l'université, dans un contexte de transition secondaire-université (\protect\hyperlink{ref-VEN:JAE:13}{Venezia \& Jaeger, 2013}). Le dispositif prend la forme d'une jeune plateforme de révision des mathématiques du lycée pour des néo-bacheliers de l'Université de La Réunion ayant choisi la licence d'Économie-Gestion et d'Administration Économique et Sociale (AES). La plateforme propose des cours en ligne sous forme de vidéos d'explication et des exercices corrigés. Chaque utilisateur possède un compte individuel et peut accéder à la plateforme même en dehors des horaires d'études. Nous recueillons les résultats des individus concernés aux évaluations de travaux dirigés et aux examens de mathématiques du premier semestre 2020-2021. Puisque les étudiants choisissent l'utilisation et l'intensité d'utilisation de la plateforme, nous utilisons un protocole d'encouragement (\protect\hyperlink{ref-DUF:eal:07}{Duflo et al., 2007} ; \protect\hyperlink{ref-BEL:eal:17}{Bellity et al., 2017}) pour fournir une mesure fiable de l'impact de l'utilisation de la plateforme sur les résultats en mathématiques. Il s'agit d'influer sur l'intensité d'utilisation de la plateforme par les participants grâce à des messages électroniques envoyés à un groupe de participants déterminé de manière explicitement aléatoire. L'effet d'intérêt s'estime alors par variables instrumentales en instrumentant les mesures d'utilisation de la plateforme par le fait d'être incité ou non.\\
Nous mobilisons des modèles adaptés au nombre important d'étudiants n'ayant pas pu obtenir le moindre point aux évaluations de mathématiques.

Notre contribution à la littérature existante est triple. La première concerne le manque d'attention et donc de connaissance des effets des diverses interventions en mathématiques dans le supérieur (Section \ref{g20litt}). La deuxième repose sur le fait que nous nous intéressons à l'efficacité d'un dispositif spécifique aux mathématiques puisque pour la France, l'étude de Bellity et al. (\protect\hyperlink{ref-BEL:eal:17}{2017}) (est la seule qui) s'intéresse à l'efficacité d'une plateforme d'entraînement des capacités écrites. La troisième contribution concerne la méthodologie que nous utilisons. La plupart des études qui portent sur les interventions en mathématiques dans le supérieur ne possèdent pas de groupe de contrôle constitué de manière aléatoire ou n'utilisent pas les méthodes d'évaluation d'impact qui s'approchent d'une telle situation.

\quad Nous trouvons qu'utiliser plus la plateforme a permis aux étudiants d'améliorer grandement leurs performances aux évaluations de travaux dirigés mais ne leur a pas aidé pour les examens de mathématiques à la fin du premier semestre. Nous justifions que l'effet positif et important sur les notes aux travaux dirigés correspond à un comportement stratégique d'une petite partie d'étudiants motivés mais pas particulièrement forts (en mathématiques). De telles explications sont relativement nouvelles dans cette littérature et donnent des pistes d'amélioration pour la mise en place d'interventions similaires en vue d'aider effectivement une plus grande partie des étudiants à augmenter leur capital de connaissances et de compétences en mathématiques et d'améliorer le taux de réussite en première année de licence d'Économie-Gestion et d'AES à La Réunion.

\quad Le reste du chapitre sera structuré comme suit. La Section \ref{g20litt} propose un parcours de la littérature sur les interventions en mathématiques. La Section \ref{g20inst} expose les éléments du système éducatif français liés à notre étude tout en complétant ceux présentés dans les chapitres précédents. Le protocole d'encouragement y est ensuite présenté en détails. La Section \ref{g20data} présente les données statistiques de l'étude. La Section \ref{g20methods} décrit le modèle économétrique utilisé, compte tenu du protocole d'encouragement et les données statistiques. La Section \ref{g20res} concerne les résultats et discussions de ce chapitre et la Section \ref{g20concl} le conclue.

\hypertarget{g20litt}{%
\section{Revue de la littérature sur les interventions en mathématiques}\label{g20litt}}

Pour commencer, il est éclairant de caractériser les interventions en mathématiques selon plusieurs éléments : le niveau d'éducation de la population cible (maternelle, primaire, secondaire ou supérieur), le type d'intervention (cours de mise à niveau, tutorat, programme d'apprentissage, e-learning ou apprentissage assisté par ordinateur ludifié\footnote{C'est-à-dire présenté sous forme de jeux.} ou non, entre autres) et les modalités de mise en œuvre (avec ou sans participation des enseignants, uniquement ou non au sein de l'école/université, nécessite l'interaction des bénéficiaires entre eux, etc.).

\quad Par rapport au niveau d'éducation de la population cible, probablement sous l'influence des travaux de Heckman et ses co-auteurs sur l'efficacité des interventions éducatives le plus tôt possible (\protect\hyperlink{ref-HEC:12}{Heckman, 2012}), de nombreuses interventions en mathématiques ont été testées sur des enfants de niveau maternelle ou primaire et ont fait l'objet d'évaluations rigoureuses, c'est-à-dire qu'un groupe de contrôle crédible a été constitué via l'assignation aléatoire ou une stratégie d'identification qui l'approxime (\protect\hyperlink{ref-ANG:PIS:08}{Angrist \& Pischke, 2009}).

Par exemple, Tienken \& Wilson (\protect\hyperlink{ref-TIE:WIL:07}{2007}) s'intéressent à l'effet de l'utilisation, par les enseignants, de sites web et de logiciels de présentation, pour l'enseignement des mathématiques, sur des notes standardisées de deux centaines d'élèves en 7\textsuperscript{ème} année d'éducation d'une école du New Jersey. L'année suivante, sur pratiquement le même échantillon et avec la même intervention, Tienken \& Maher (\protect\hyperlink{ref-TIE:MAH:08}{2008}) s'intéressent à l'effet de cette dernière sur les élèves en 8\textsuperscript{ème} année d'éducation. En Inde et donc dans un autre contexte socio-démographique et économique, Banerjee et al. (\protect\hyperlink{ref-BAN:eal:07}{2007}) évaluent l'effet de deux interventions différentes sur des notes d'élèves en troisième et en quatrième année d'éducation primaire. La première intervention prend la forme d'un tutorat à destination des plus faibles et la seconde prend la forme d'une instruction assistée par ordinateur uniquement disponible dans les écoles. Des études dans la même thématique, sur des élèves en bas âge existent dans d'autres pays, comme la Chine (\protect\hyperlink{ref-MO:eal:14}{Mo et al., 2014}) ou la Belgique (\protect\hyperlink{ref-PRA:DES:14}{Praet \& Desoete, 2014}), entre autres. Les auteurs de la première se demandent si des sessions de remédiation composées principalement de jeux mathématiques aident des élèves en 3\textsuperscript{ème} et 5\textsuperscript{ème} année d'éducation primaire d'écoles rurales d'une province relativement défavorisée de la Chine, en termes de performances en mathématiques. Les auteurs du second se concentrent sur des activités de comptage ou de comparaison sur ordinateur, à destination des enfants de maternelle inscrits dans 5 écoles d'un même district scolaire\footnote{Les résultats, dont la qualité est conditionnée par les méthodologies mobilisées, sont discutés plus bas.}.

Cette littérature, pour les étudiants en début d'études supérieures, est moins fournie, en termes d'études empiriques\footnote{Cette plus faible attention des chercheurs sur les individus du supérieur a d'ailleurs été constatée en ce qui concerne des interventions liées à la littératie (\protect\hyperlink{ref-BEL:eal:17}{Bellity et al., 2017}).}. Par ailleurs, la quasi-totalité des papiers sur les étudiants à l'université se limitent à ceux en première année, celle où le taux d'échec le plus élevé est observé (\protect\hyperlink{ref-BEL:eal:17}{Bellity et al., 2017}). En guise d'exemple, Msomi \& Bansilal (\protect\hyperlink{ref-MSO:BAN:18}{2018}) étudient de manière qualitative la corrélation qui pourrait exister entre l'utilisation d'une plateforme d'apprentissage en ligne et la réussite des étudiants en première année dans une université technologique d'Afrique du Sud. De leur côté, Howard et al. (\protect\hyperlink{ref-HOW:eal:19}{2019}) étudient la corrélation entre des activités de quiz hebdomadaires avec feedback\footnote{Le fait de faire des retours au participant sur son travail sur la plateforme, par rapport à ces exploits et ces faiblesses.}, pendant 10 semaines, et les notes aux examens d'étudiants en première au sein d'une université en Irlande.

Par rapport à ce premier aspect (le niveau d'éducation de la population cible), le présent travail contribue à cette littérature relativement récente sur l'efficacité des interventions en mathématiques dans les études supérieures. Si le travail de Bellity et al. (\protect\hyperlink{ref-BEL:eal:17}{2017}) porte sur une intervention en écriture du français, notre étude est à notre connaissance le premier de ce type dans le contexte français, en ce qui concerne spécifiquement les mathématiques.

\quad Une autre manière de catégoriser les interventions est de distinguer si l'utilisation de la technologie constitue le cœur de l'intervention ou non. Parmi les mesures dont la technologie est au plus une modalité de mise en œuvre, les cours de remise à niveau constituent une pratique assez courante, bien que sujette à débat (\protect\hyperlink{ref-BAH:08}{Bahr, 2008}). Le débat existe en raison du coût de telles interventions et du manque de preuve scientifique de leur efficacité. Ce type d'intervention est le plus souvent destiné aux élèves qui souhaitent s'inscrire à l'université mais qui n'ont pas le niveau nécessaire en mathématiques pour espérer suivre les cours dans les filières où les mathématiques sont indispensables (\protect\hyperlink{ref-LOG:eal:16}{Logue et al., 2016}).\\
Pour des coûts potentiellement réduits et des élèves plus jeunes, il est également possible de proposer des tutorats. C'est un des objets de l'article de Banerjee et al. (\protect\hyperlink{ref-BAN:eal:07}{2007}) pour l'Inde. Plus précisément, de jeunes tuteurs étaient engagés à courte durée pour aider les plus faibles du primaire hors des heures de cours habituelles. La spécificité de cette intervention est que les tuteurs devaient, si nécessaire, aider les bénéficiaires sur des compétences très basiques. Par exemple, si un élève est en fin de primaire mais qu'il ne sait pas lire, le tuteur devait l'apprendre à lire bien que le programme de fin de primaire aille bien au-delà de la lecture.\\
Dans l'étude de Mo et al. (\protect\hyperlink{ref-MO:eal:14}{2014}), des enseignants devaient superviser étroitement, régulièrement et pendant toute l'année scolaire des élèves bénéficiaires du programme de jeux mathématiques. Une telle intervention peut être considérée comme une forme de tutorat. Dans le même esprit, Topping et al. (\protect\hyperlink{ref-TOP:eal:11}{2011}) s'intéressent à l'efficacité d'un tutorat précisément structuré dans différentes configurations\footnote{Selon l'hétérogénéité de l'âge des élèves, selon l'intensité du tutorat et selon que des activités de lecture soient rajoutées ou non.} pour des élèves du primaire en Écosse.

Dans la littérature, il est notable que les interventions qui reposent principalement sur la technologie (des ordinateurs ou des tablettes et des logiciels, le plus souvent) attirent relativement plus l'attention des chercheurs et des établissements d'éducation. Au moins deux éléments pourraient expliquer cet intérêt. Le premier est l'intuition selon laquelle l'utilisation de la technologie rend les interventions elles-mêmes plus attractives pour les élèves (\protect\hyperlink{ref-BAR:eal:09}{Barrow et al., 2009}), plus particulièrement lorsqu'elles sont ludifiées (\protect\hyperlink{ref-PUT:eal:20}{Putz et al., 2020}). Le second est le fait que les technologies d'apprentissages pourraient servir de substitut lorsqu'il existe un manque d'enseignants qualifiés.
Lister toutes les interventions à base de technologie ne nous paraît pas intéressant mais les distinguer selon leur modalité de mise en œuvre peut s'avérer éclairant. Notamment, le déploiement des interventions peut d'un côté être limité dans l'établissement scolaire. C'est le cas du second programme évalué par Banerjee et al. (\protect\hyperlink{ref-BAN:eal:07}{2007}) en plus du tutorat. Il consistait à mettre à disposition des écoles indiennes de l'expérimentation des ordinateurs contenant des jeux mathématiques. Les bénéficiaires disposaient de deux heures par semaine pour les utiliser. Des instructeurs étaient présents lors de ces deux heures, ce qui est normal vu que les concernés sont de jeunes élèves (équivalents du CE2 et du CM1 du système français).\\
C'est également le cas de l'intervention qui fait l'objet de l'étude de Bozzoli et al. (\protect\hyperlink{ref-BOZ:eal:20}{2020}), qui consistait à procurer des tablettes aux bénéficiaires\footnote{Tous les élèves d'écoles primaires désignées comme bénéficiaires.} pour que ces derniers utilisent une application visant à développer des compétences fondamentales en mathématiques pendant 20 minutes par jour, au rythme de 4 fois par semaine et pendant 10 semaines. Similairement, l'utilisation d'un logiciel d'apprentissage des mathématiques pendant 8 à 15 semaines, au rythme de trois fois par semaine, par des élèves de primaire a été supervisée par des enseignants dans l'étude de Burns et al. (\protect\hyperlink{ref-BUR:eal:12}{2012}).\\
Les enseignants peuvent jouer un rôle plus important dans le déploiement de ce type d'intervention. Les articles de Tienken \& Wilson (\protect\hyperlink{ref-TIE:WIL:07}{2007}) et Tienken \& Maher (\protect\hyperlink{ref-TIE:MAH:08}{2008}) correspondent notamment à deux cas où le programme (utilisation de sites web et de logiciels de présentation, comme susmentionné) est destiné aux enseignants pour que ces derniers améliorent leur pédagogie.

En outre, la technologie peut permettre un apprentissage plus individualisé dans le sens où le bénéficiaire peut en profiter même hors de l'établissement scolaire ou universitaire et/ou elle s'adapte au niveau initial et à la progression de son utilisateur. Notamment, le programme auquel s'intéressent Barrow et al. (\protect\hyperlink{ref-BAR:eal:09}{2009}) est uniquement utilisable au sein des établissements mais s'adapte au niveau de l'individu. L'article de Roschelle et al. (\protect\hyperlink{ref-ROS:eal:16}{2016}) constitue un autre exemple dans lequel le programme est individualisé et est cette fois accessible en dehors de l'établissement, les bénéficiaires disposant d'ordinateurs qu'ils pouvaient ramener chez eux. De manière similaire, aux Pays-Bas, l'efficacité d'un logiciel qui peut également prendre la forme d'une application mobile, est étudiée par Haelermans \& Ghysels (\protect\hyperlink{ref-HAE:GHY:17}{2017}).

L'intervention étudiée dans ce chapitre est une plateforme de révision individuel en ligne, accessible pendant 4 semaines uniquement par ordinateur même en dehors de l'établissement et qui ne nécessite aucune participation des enseignants. Le contenu de la plateforme ne s'adapte pas au niveau initial de l'individu ni à sa progression.

\quad La fiabilité des résultats d'analyses de l'efficacité de ces différentes interventions dépend naturellement de la qualité de la mesure de la variable dépendante, de la ``prise effective du traitement''\footnote{Autrement dit, si l'intervention est effectivement parvenue aux supposés bénéficiaires (\protect\hyperlink{ref-BUR:eal:12}{Burns et al., 2012, p. 187}).}, de la sélection et la taille de l'échantillon et de la méthode d'évaluation mobilisée. Les notes constituent le type de mesure usuel des acquis en mathématiques dans la littérature. Les acteurs (chercheurs, établissements) semblent généralement être bien armés en termes d'instrument de mesure avec des tests standardisés faisant explicitement l'objet d'une recherche scientifique a priori\footnote{TEMA-3, utilisé pour les enfants, par exemple. TEMA signifie \emph{Tests for Early Mathematics Ability}. Le suffixe \emph{-3} correspond à la dernière édition. Cet instrument de mesure a été notamment mobilisé par Jaciw et al. (\protect\hyperlink{ref-JAC:eal:16}{2016}) et Karademir \& Akman (\protect\hyperlink{ref-KAR:AKM:19}{2019}).}. Dans le cas des interventions ``technologiques'' discutées plus haut, il est aussi d'usage que la mesure soit effectuée ``par la technologie'' elle-même (\protect\hyperlink{ref-HAE:GHY:17}{Haelermans \& Ghysels, 2017}). Les examens préparés habituellement par les établissements constituent également une bonne mesure, tant qu'ils sont uniformes vis à vis des étudiants et qu'ils ne sont ni trop difficiles ni trop faciles (\protect\hyperlink{ref-ROS:eal:16}{Roschelle et al., 2016} ; \protect\hyperlink{ref-HOW:eal:19}{Howard et al., 2019}).\\
Les différentes interventions parviennent globalement à la population cible, notamment au vu de la durée de ces dernières d'au moins une dizaine de semaines.
En termes d'échantillon, les interventions dans le supérieur sont le plus souvent limitées à un établissement\footnote{Voir Howard et al. (\protect\hyperlink{ref-HOW:eal:19}{2019}), Logue et al. (\protect\hyperlink{ref-LOG:eal:16}{2016}) ou Du Preez et al. (\protect\hyperlink{ref-DUP:eal:08}{2008}) en guise d'exemples.} tandis que le primaire ou le secondaire ne semblent pas vérifier de ce constat.
Aussi, parmi les études qui concernent le supérieur, contrairement à celles qui couvrent le primaire ou le secondaire, rares sont celles qui disposent d'un groupe de contrôle crédible. Dans cette littérature, cela s'obtient en utilisant l'assignation aléatoire de l'accès du traitement ou d'une incitation à son utilisation, au niveau individuel (\protect\hyperlink{ref-KAR:AKM:19}{Karademir \& Akman, 2019}), au niveau des classes (\protect\hyperlink{ref-BAR:eal:09}{Barrow et al., 2009}) ou au niveau des écoles (\protect\hyperlink{ref-CLE:eal:11}{Clements et al., 2011}). Il est alors nécessaire de rester prudent quant à l'interprétation des résultats correspondants que nous évoquons plus bas.

L'intervention étudiée dans ce chapitre est évaluée avec un protocole d'encouragement (voir Section \ref{g20inst}), ce qui constitue une contribution relativement significative vu l'utilisation limitée des méthodes d'évaluation d'impact dans le supérieur sur le sujet.

\quad Dans l'éducation secondaire et surtout primaire, les interventions de différentes natures sont globalement efficaces\footnote{En se focalisant sur les études exploitant au moins un groupe de contrôle crédible.}. Par exemple, Banerjee et al. (\protect\hyperlink{ref-BAN:eal:07}{2007}) trouvent un effet fort à court terme des deux interventions allant d'environ 0.2 UET à environ 0.4 UET selon la période, la ville ou la variable dépendante analysée. En cohérence avec certains résultats de la littérature et avec les analyses empiriques menées les chapitres \ref{age} et \ref{pe} de la présente thèse, ces auteurs trouvent des effets plus forts en mathématiques et chez les élèves initialement plus faibles. Dans un autre contexte, en Argentine, Bozzoli et al. (\protect\hyperlink{ref-BOZ:eal:20}{2020}) trouvent que donner des tablettes contenant une application de jeux mathématiques améliore la mémoire active des enfants inscrits à tous les niveaux du primaire. Sur des lycéens, Barrow et al. (\protect\hyperlink{ref-BAR:eal:09}{2009}) trouvent également un effet positif et fort d'une application plus individualisée sur les notes aux examens nationaux aux États-Unis. Les articles de Roschelle et al. (\protect\hyperlink{ref-ROS:eal:16}{2016}), de Tienken \& Wilson (\protect\hyperlink{ref-TIE:WIL:07}{2007}) et de Tienken \& Maher (\protect\hyperlink{ref-TIE:MAH:08}{2008}) constituent des démonstrations de l'efficacité des interventions en mathématiques pour des collégiens.

Pour les étudiants en première année d'université, l'efficacité des interventions sur les compétences ne semble pas être prouvée, dans le sens où de travaux supplémentaires sont nécessaires. En guise d'illustration, Leong \& Alexander (\protect\hyperlink{ref-LEO:AL:14}{2014}) ne s'intéressent qu'à l'attitude des étudiants envers une plateforme de devoirs en ligne. Les auteurs trouvent à l'aide de méthodes qualitatives et quantitatives\footnote{Hors des méthodes d'évaluation d'impact (\protect\hyperlink{ref-ANG:PIS:08}{Angrist \& Pischke, 2009}).} que les étudiants les plus faibles ont une attitude positive envers la plateforme. De manière similaire, Moreno-Guerrero et al. (\protect\hyperlink{ref-MOR:eal:20}{2020}) concluent à des effets positifs de l'utilisation d'outils digitaux (moodle, mails, emploi du temps en ligne) sur la motivation, l'autonomie, la participation, la maîtrise de concepts mathématiques et les notes en mathématiques. Cependant, ce sont probablement des résultats surestimés vu que leur groupe de contrôle n'est pas comparable avec les groupes de bénéficiaires. La plupart des autres papiers peuvent être soumis à ce type de biais (\protect\hyperlink{ref-MSO:BAN:18}{Msomi \& Bansilal, 2018} ; \protect\hyperlink{ref-HOW:eal:19}{Howard et al., 2019} ; \protect\hyperlink{ref-JAC:PRE:16}{Jacobs \& Pretorius, 2016}).

\hypertarget{g20inst}{%
\section{Contexte institutionnel et présentation du protocole quasi-expérimental}\label{g20inst}}

\hypertarget{g20instuniv}{%
\subsection{Du lycée aux portes de l'université}\label{g20instuniv}}

Par souci de continuité, si dans les chapitres \ref{age} et \ref{pe}, nous avons brièvement présenté les derniers niveaux du primaire et du collège, nous proposons ici une courte présentation du dernier niveau du lycée (la terminale) avant celle de la première année universitaire.

\quad La terminale correspond à la troisième année du lycée, soit à la sixième et dernière année de l'enseignement secondaire français. Contrairement au collège (voir chapitres précédents), les élèves du lycée doivent choisir un parcours dès la première année de lycée. Ils ont le choix entre le parcours général et technologique (GT) et le parcours professionnel (Pro) qui est plus spécialisé. À partir de la première et donc en terminale, le parcours GT est séparé en deux filières distinctes (filière générale et filière technologique). Dans notre période d'étude (avant la rentrée de la terminale de 2020-2021), la filière générale se divise en séries économique et social (ES), scientifique (S) et littéraire (L). De la même manière, la filière technologique est constituée de huit séries\footnote{\url{https://www.education.gouv.fr/reussir-au-lycee/le-lycee-41642}.}, à savoir les sciences et technologies de laboratoire (STL), les sciences et technologies de l'industrie et du développement durable (ST2D), les sciences et technologies du design et des arts appliqués (STD2A), les sciences et technologies du management et de la gestion (STMG), les sciences et technologies de la santé et du social (ST2S), les sciences et techniques du théâtre, de la musique et de la danse (S2TMD), les sciences et technologies de l'hôtellerie et de la restauration (STHR) et les sciences et technologies de l'agronomie et du vivant (STAV)\footnote{Cette dernière série concerne uniquement les lycées agricoles.}. La filière professionnelle quant à elle ne se divise pas en seulement quelques séries mais comporte environ 200 spécialités directement liées aux métiers visés\footnote{Un référentiel de spécialités peut être trouvé sur le site \url{https://eduscol.education.fr/1916/le-baccalaureat-professionnel}.}.

\quad Les épreuves du bac sont composées d'épreuves anticipées de français à passer en première, d'épreuves de contrôles continus tout au long de l'année de terminale et des épreuves finales qui ont lieu vers la fin de l'année scolaire de terminale, soit dans la troisième semaine du mois de juin. Les épreuves du bac professionnel constituent une exception puisqu'elles ne contiennent pas d'épreuves anticipées. La forme des épreuves et les matières concernées sont spécifiques aux filières et aux séries. En raison de la crise sanitaire, toutes les épreuves finales de la session de 2020 sont exceptionnellement remplacées par des épreuves de contrôle continu\footnote{\url{https://www.education.gouv.fr/baccalaureat-tout-savoir-sur-la-session-2020-308372}.}. Cela constitue très probablement une cause de l'augmentation considérable du taux de réussite en 2020 (95\%) par rapport aux années précédentes qui présentaient une croissance modérée avec le temps du taux de réussite\footnote{Il est de 85.6\% en 2010, 87.9\% en 2015, 88.2\% en 2018 et 88\% en 2019 (\protect\hyperlink{ref-DEP:21}{DEPP, 2021b, p. 223}).}.

\quad Antérieurement à la crise sanitaire, les notes aux épreuves de contrôles continus et aux épreuves terminales comptent respectivement pour 40\% et 60\% de la note au bac. Un élève obtient le diplôme du bac s'il obtient au moins 50\% de la note maximale. Les résultats du bac sont disponibles sur le site du ministère de l'éducation nationale vers le début du mois de juillet.

\quad L'obtention du bac est nécessaire pour s'inscrire à l'université\footnote{Selon les textes officiels (\url{https://www.univ-reunion.fr/scolarite/admissions}), d'autres types diplômes sont acceptés par l'université mais leurs effectifs sont largement négligeables en pratique.} mais les procédures de pré-inscription commencent vers le milieu de l'année scolaire de terminale. Plus précisément, dès la troisième semaine de janvier, les lycéens qui souhaitent effectuer des études supérieures doivent passer par la plateforme appelée \emph{Parcoursup} afin d'indiquer leurs vœux de formation. Ces lycéens reçoivent les réponses des établissements d'enseignement supérieur entre la troisième semaine de mai et la troisième semaine de juillet. Les calendriers d'inscription en première année d'études supérieures varient en fonction des formations.

\quad L'expérimentation a été menée pour les seules filières d'Économie-Gestion et d'AES de l'Université de La Réunion. Les formations liées à ces deux filières sont proposées à travers deux campus : au Nord de l'île dans la commune de Saint-Denis et au sud dans la commune du Tampon. Les procédures d'inscription en première année correspondantes se passent entre début mai et début juin, pour une rentrée à la dernière semaine d'août. Le profil des candidats peut différer considérablement d'une filière à l'autre.

\quad Les mathématiques constituent une unité d'enseignement majeure dans les deux filières. Le contenu de l'enseignement des mathématiques est différent en fonction de la filière mais le volume horaire est le même en cours magistral (30 heures) et en travaux dirigés (12 heures). Les cours magistraux sont également assurés par le même enseignant à travers les deux campus et les deux filières, ce qui n'est pas le cas pour les travaux dirigés (TD) pour des raisons pratiques\footnote{Considérant les effectifs des inscrits dans la licence d'Économie et Gestion (Section \ref{g20data}), il y a en tout une vingtaine de groupes de TD. Il n'est pratiquement pas possible qu'un même enseignant assure toutes les séances de TD.}. À l'Université de La Réunion, les mathématiques sont enseignées dès le premier semestre. En mathématiques comme dans d'autres matières, les étudiants ont une note d'évaluation aux TD et une note d'évaluation à un examen final (que nous dénommons désormais par examens). La note en mathématiques du semestre est la moyenne de la note aux TD et la note aux examens. Notre période d'étude est limitée au premier semestre de l'année universitaire 2020-2021. L'emploi du temps des séances de cours magistraux et de TD peut être ajusté pour des raisons d'organisation (disponibilité des chargés de TD, par exemple) mais les examens ont lieu vers la fin du semestre, en décembre.

\hypertarget{g20instproto}{%
\subsection{Le protocole d'encouragement}\label{g20instproto}}

Nous nous demandons dans quelles mesures réviser individuellement les mathématiques du lycée sous forme de vidéos et d'exercices sur une plateforme en ligne influe sur les résultats du premier semestre en mathématiques des étudiants en première année de licence d'Économie-Gestion.
Pour mesurer cet effet, l'idéal aurait été d'attribuer aléatoirement des intensités d'utilisation de cette plateforme aux participants de l'expérimentation, ce qui n'est pas possible puisque les participants choisissent eux-mêmes cette intensité. Ce choix est aussi vraisemblablement lié aux compétences \emph{a priori} en mathématiques des individus. Nous aurions théoriquement pu attribuer aléatoirement l'accès à la plateforme aux participants. Cela pose pratiquement des problèmes d'éthique (\protect\hyperlink{ref-SHA:82}{Schafer, 1982}). Un des problèmes standards est que, si en réalité la plateforme avait grandement amélioré les résultats en mathématiques pour des bénéficiaires choisis aléatoirement, il est injuste de ne pas avoir donné l'accès à la plateforme au groupe de contrôle.

Pour justifier le fait que l'accès à la plateforme soit attribué de manière aléatoire aux étudiants, compte tenu du temps imparti pour l'évaluation, il aurait fallu que tous les étudiants bénéficient d'une autre intervention qui vise spécifiquement à l'amélioration des résultats en mathématiques. Dans ce cas, nous aurions défini un groupe de traitement qui bénéficie à la fois de l'autre intervention et de l'accès à la plateforme et un groupe contrôle qui bénéficie uniquement de l'autre intervention. Une telle autre intervention n'existe pas dans notre période d'étude.\\
De plus, mis à part ces problèmes d'éthique, il aurait été difficile de mesurer l'effet de l'intensité de l'utilisation de la plateforme en attribuant aléatoirement uniquement l'accès à cette plateforme, puisque dans le groupe de traitement, les plus motivés utiliseront plus la plateforme et auront tendance à avoir de meilleurs résultats en mathématiques (biais classique de variable omise).

\quad Nous utilisons alors un protocole d'encouragement (\protect\hyperlink{ref-DUF:eal:07}{Duflo et al., 2007} ; \protect\hyperlink{ref-HOL:88}{Holland, 1988}) afin d'atteindre notre objectif de recherche. L'idée principale est de donner l'accès à la plateforme à tous les étudiants du champ d'étude mais d'agir sur l'intensité d'utilisation de la plateforme avec des incitations attribuées de manière aléatoire. Si les hypothèses d'identification (voir Section \ref{g20methods}) sont crédibles, nous pouvons mesurer l'effet causal de la révision des mathématiques du lycée via la plateforme sur les résultats en mathématiques à l'université pour les étudiants dont les incitations influent sur leur intensité d'utilisation de la plateforme (\protect\hyperlink{ref-IMB:ANG:94}{Imbens \& Angrist, 1994} ; \protect\hyperlink{ref-ANG:IMB:95}{Angrist \& Imbens, 1995}).

Concrètement, en utilisant un mail renseigné dans Parcoursup par chaque participant, nous avons contacté par mail tous les néo-bacheliers inscrits sur Parcoursup pour la rentrée universitaire 2020-2021 pour leur demander s'ils étaient intéressés par une plateforme de révision des mathématiques du lycée, dénommée ``J'ai 20 en maths'' et s'ils étaient d'accord pour que nous transmettions leur mail à la plateforme en question pour création des comptes personnels d'utilisation. Parcoursup compte 1142 néo-bacheliers. Nous avons obtenu 306 réponses positives de la part de ces derniers. \emph{A priori}, ces 306 individus ne sont pas représentatifs de l'ensemble des néo-bacheliers inscrits sur Parcoursup puisque nous pouvons supposer que les plus motivés (et donc ceux qui ont des résultats potentiels en mathématiques plus élevés) seront surreprésentés parmi les répondants. Cela se justifie empiriquement avec le Tableau \ref{tab:g20compint0int1} qui compare les caractéristiques prédéterminées des individus non intéressés avec celles des individus intéressés. Les séries ES et S, les femmes, les personnes nées à l'étranger et celles venant d'un établissement d'un territoire français autre que La Réunion sont surreprésentés chez les individus intéressés\footnote{Les différences de note au bac, de nationalité et de statut de boursier sont statistiquement significatives mais de faible ampleur.}.
Nous avons ensuite transféré les mails des individus intéressés aux responsables de la plateforme afin que les comptes individuels soient créés et qu'ils puissent bénéficier des services proposés. En se connectant à son compte, un participant de l'expérimentation a essentiellement accès à deux activités : regarder des vidéos et visionner des exercices corrigés (que le participant peut effectuer ou non). Les notions mathématiques révisées sont présentées dans l'Annexe \ref{g20notions}. Les participants ont gardé leur accès à la plateforme jusqu'au 10 Septembre 2020, qui correspond à 4 semaines d'accès.\\
Nous connaissons le nombre de connexion, le nombre de vidéos visionnées entièrement (``Vidéos (complètes)'' dans les tableaux et figures) et le nombre de vidéos visionnées, potentiellement entièrement (Section \ref{g20data}). La Figure \ref{fig:g20treats} montre que les participants ont faiblement utilisé la plateforme (panels B, C et D). La distribution du nombre de connexions (panel A) suggère que les étudiants qui se sont connectés l'ont principalement fait pour regarder les exercices, puisque la forme de la distribution du nombre de connexions (panel A) et celle du nombre d'exercices (panel D) sont similaires.

\newpage
\begingroup\fontsize{7}{9}\selectfont

\begin{ThreePartTable}
\begin{TableNotes}
\item \textit{Sources :} Base Parcoursup - L1 AES et Eco-Ge, plateforme '\textit{J'ai 20 en maths}', APOGEE (version extraite en mai 2021), calculs de l'auteur.
\item \textit{Notes :} Moyennes et proportions. Écart-types entre parenthèses. Les probabilités critiques correspondent à des tests de comparaison de moyennes pour les variables quantitatives et à des tests de comparaison de proportions pour les variables qualitatives. ES : Économique et social, S : Scientifique, AES : Administration Économique et Sociale. Eco-Ge : Économie-Gestion.
\item Significativité : 10\% * 5\% ** 1\% ***.
\end{TableNotes}
\begin{longtable}[t]{llllll}
\caption{\label{tab:g20compint0int1}Comparaison entre les néo-bacheliers non intéressés et intéressés par la plateforme de Parcoursup}\\
\toprule
\multicolumn{1}{c}{ } & \multicolumn{2}{c}{Non intéressés} & \multicolumn{2}{c}{Intéressés} & \multicolumn{1}{c}{ } \\
\cmidrule(l{3pt}r{3pt}){2-3} \cmidrule(l{3pt}r{3pt}){4-5}
  & \makecell{\makecell{Moyenne \\ (Écart-type)} \\ (1) } & \makecell{\makecell{Proportion \\ manquantes} \\ (2) } & \makecell{\makecell{Moyenne \\ (Écart-type)} \\ (3) } & \makecell{\makecell{Proportion \\ manquantes} \\ (4) } & \makecell{\makecell{(1) = (3) \\ probabilité critique} \\ (5) }\\
\midrule
\endfirsthead
\caption[]{\label{tab:g20compint0int1}Comparaison entre les néo-bacheliers non intéressés et intéressés par la plateforme de Parcoursup (suite)}\\
\toprule
  & \makecell{\makecell{Moyenne \\ (Écart-type)} \\ (1) } & \makecell{\makecell{Proportion \\ manquantes} \\ (2) } & \makecell{\makecell{Moyenne \\ (Écart-type)} \\ (3) } & \makecell{\makecell{Proportion \\ manquantes} \\ (4) } & \makecell{\makecell{(1) = (3) \\ probabilité critique} \\ (5) }\\
\midrule
\endhead

\endfoot
\bottomrule
\insertTableNotes
\endlastfoot
\addlinespace[0.3em]
\multicolumn{6}{l}{\textbf{Note au bac}}\\
\hspace{1em}Totale & 11.71 (1.41) & 0.03 & 11.99 (1.59) & 0.03 & 0***\\
\hspace{1em}En français (écrits) & 9.97 (3.31) & 0.32 & 10.27 (3.28) & 0.2 & 0.24\\
\hspace{1em}En mathématiques & 11.11 (3.21) & 0.07 & 11.42 (3.26) & 0.07 & 0.16\\
\addlinespace[0.3em]
\multicolumn{6}{l}{\textbf{Série au bac}}\\
\hspace{1em}Pro & 0.28 & 0 & 0.15 & 0 & 0 ***\\
\hspace{1em}Techno & 0.32 &  & 0.28 &  & \\
\hspace{1em}ES & 0.3 &  & 0.39 &  & \\
\hspace{1em}S & 0.06 &  & 0.13 &  & \\
\hspace{1em}Autre & 0.04 &  & 0.05 &  & \\
\addlinespace[0.3em]
\multicolumn{6}{l}{\textbf{Mention au bac}}\\
\hspace{1em}Aucune & 0.61 & 0.01 & 0.51 & 0.02 & 0.01 ***\\
\hspace{1em}Assez bien & 0.3 &  & 0.35 &  & \\
\hspace{1em}Bien & 0.07 &  & 0.12 &  & \\
\hspace{1em}Très bien & 0.01 &  & 0.02 &  & \\
\addlinespace[0.3em]
\multicolumn{6}{l}{\textbf{ }}\\
\hspace{1em}Âge à la rentrée & 18.63 (1.29) & 0 & 18.74 (2.26) & 0 & 0.34\\
\addlinespace[0.3em]
\multicolumn{6}{l}{\textbf{Sexe}}\\
\hspace{1em}Fille & 0.61 & 0 & 0.7 & 0 & 0.01 ***\\
\hspace{1em}Homme & 0.39 &  & 0.3 &  & \\
\addlinespace[0.3em]
\multicolumn{6}{l}{\textbf{Pays de naissance}}\\
\hspace{1em}À l'étranger & 0.05 & 0 & 0.11 & 0 & 0 ***\\
\hspace{1em}En France & 0.95 &  & 0.89 &  & \\
\addlinespace[0.3em]
\multicolumn{6}{l}{\textbf{Nationalité}}\\
\hspace{1em}Hors union européenne & 0.03 & 0 & 0.06 & 0 & 0.1 *\\
\hspace{1em}Française & 0.97 &  & 0.94 &  & \\
\addlinespace[0.3em]
\multicolumn{6}{l}{\textbf{Statut de boursier}}\\
\hspace{1em}Non boursier & 0.37 & 0 & 0.36 & 0 & 0.69\\
\hspace{1em}Boursier du secondaire & 0.63 &  & 0.64 &  & \\
\addlinespace[0.3em]
\multicolumn{6}{l}{\textbf{Statut de l'établissement d'origine}}\\
\hspace{1em}Public & 0.96 & 0.01 & 0.95 & 0.01 & 0.56\\
\hspace{1em}Privé & 0.04 &  & 0.05 &  & \\
\addlinespace[0.3em]
\multicolumn{6}{l}{\textbf{Type de l'établissement d'origine}}\\
\hspace{1em}Sans postbac & 0.06 & 0.01 & 0.05 & 0.01 & 0.68\\
\hspace{1em}Avec postbac & 0.94 &  & 0.95 &  & \\
\addlinespace[0.3em]
\multicolumn{6}{l}{\textbf{Département de l'établissement d'origine}}\\
\hspace{1em}La Réunion & 0.87 & 0.01 & 0.77 & 0 & 0 ***\\
\hspace{1em}France hors Réunion & 0.12 &  & 0.2 &  & \\
\hspace{1em}Étranger & 0.01 &  & 0.03 &  & \\
\addlinespace[0.3em]
\multicolumn{6}{l}{\textbf{Campus}}\\
\hspace{1em}Tampon & 0.38 & 0 & 0.39 & 0 & 0.71\\
\hspace{1em}Saint-Denis & 0.62 &  & 0.61 &  & \\
\addlinespace[0.3em]
\multicolumn{6}{l}{\textbf{Filière}}\\
\hspace{1em}AES & 0.72 & 0 & 0.75 & 0 & 0.35\\
\hspace{1em}Eco-Ge & 0.28 &  & 0.25 &  & \\
\addlinespace[0.3em]
\multicolumn{6}{l}{\textbf{ }}\\
\hspace{1em}Observations & 836 &  & 306 &  & \\*
\end{longtable}
\end{ThreePartTable}
\endgroup{}
\newpage

\begin{figure}[H]

{\centering \includegraphics[width=1\linewidth]{000_files/figure-latex/g20treats-1} 

}

\caption{Utilisation de la plateforme par les participants}\label{fig:g20treats}
\end{figure}

Parmi les 306 néo-bacheliers participants, 144 ont été assignés aléatoirement à un groupe incité et 162 à un groupe non incité. Notons bien que ceux qui ont répondu positivement à notre premier message sont naturellement informés de l'existence de la plateforme.
Nous avons envoyé par mail\footnote{Il s'agit toujours du mail renseigné dans Parcoursup, à distinguer du mail étudiant.} aux participants du groupe incité 4 messages d'incitation espacés d'une dizaine de jours. Les individus non incités n'ont rien reçu de tout cela. Même s'il ne s'agit pas de l'échantillon d'estimation (voir Section \ref{g20data}), il est intéressant de constater dans le Tableau \ref{tab:g20compintz0z1} de l'Annexe \ref{g20compintinscz0z1} que les participants incités et non incités sont bien comparables.

\quad Tous les participants ne se sont pas \emph{in fine} inscrits à l'université et tous ceux qui se sont inscrits ne sont pas forcément restés dans la formation pour passer les examens de mathématiques en fin de semestre. Ce sont des étudiants qui se sont réorientés ou qui ont abandonné, par exemple. Plus de détails sont fournis dans la Section \ref{g20data}.

\hypertarget{g20data}{%
\section{Données}\label{g20data}}

Pour aboutir aux différents échantillons d'estimation, nous mobilisons trois sources distinctes de données : Parcoursup, la plateforme ``J'ai 20 en maths'' et APOGEE\footnote{Signifiant Application pour l'Organisation et la Gestion des Enseignements et des Étudiants (\url{https://www.legifrance.gouv.fr/jorf/id/JORFTEXT000000186826}). Sa fonction globale est résumée par son appellation. L'aspect de cet outil qui nous intéresse est qu'elle est source de données individuelles sur les identifiants des étudiants (numéro étudiant, nom, prénom(s), date de naissance) et leurs résultats (notes, admission) aux évaluations de toutes les matières enseignées à l'université.}. Deux à deux, les colonnes (1) à (6) du Tableau \ref{tab:g20stats} décrivent ces données dans cet ordre. Dans ces colonnes, une cellule vide signifie que la variable (en ligne) n'est pas originalement disponible dans la base correspondante (en colonne). Les données décrites par les colonnes (3) et (4) correspondent à une jointure faite entre les données de la plateforme et celles de Parcoursup pour visualiser la différence entre les inscrits de Parcoursup et ceux qui sont intéressés par la plateforme. La clef de jointure est le mail renseigné dans Parcoursup.

\quad Pour rappel, la base Parcoursup montrée dans le Tableau \ref{tab:g20stats} concerne uniquement les néo-bacheliers ayant \emph{a priori} choisi la Licence d'Économie-Gestion ou la licence d'AES de l'Université de La Réunion. Ces individus correspondent à la population d'individus qui souhaitent suivre une de ces formations à La Réunion et qui ont effectué un minimum de démarche d'admission à l'université.\\
Dans cette base, nous distinguons des identifiants individuels, des informations sur les caractéristiques individuelles des inscrits, sur leurs établissements d'origine, sur leurs profils et résultats au bac et sur leurs vœux de formation parmi les deux licences uniquement.

Les identifiants individuels de la base Parcoursup sont l'adresse mail (Section \ref{g20instproto}), un numéro unique Parcoursup, le nom et les prénoms.

Quant aux caractéristiques individuelles des inscrits dans Parcoursup, nous connaissons le sexe, la date de naissance, le pays de naissance, la nationalité et le statut de boursier.\\
Il est intéressant de constater la prévalence des femmes (63\%) parmi les inscrits aux deux licences (colonne 1 du Tableau \ref{tab:g20stats}). Ceci correspond à une réalité française dans toutes les disciplines des études supérieures sauf en sciences fondamentales et applications et en sciences et technologies des activités physiques et sportives (\protect\hyperlink{ref-INS:20}{INSEE, 2020, p. 99}, par exemple).\\
Comme dans les chapitres précédents, nous utilisons la date de naissance pour calculer l'âge à la rentrée universitaire. Normalement, un étudiant a 18 ans à sa première année d'université. Vu que nous nous intéressons aux néo-bacheliers, il est normal d'observer des moyennes d'âge d'environ 18.5 ans ou un peu plus. Les individus qui font augmenter la moyenne d'âge sont ceux qui sont entré tardivement à l'école, ont redoublé ou ont arrêté temporairement leurs études dans le secondaire. Le premier panel de la Figure \ref{fig:g20age26aoutcompl} de l'Annexe \ref{g20age26aoutnotescompl} montre la distribution de l'âge à la rentrée des néo-bacheliers inscrits dans Parcoursup. Il est remarquable de constater des valeurs extrêmes d'âge de 30 ans ou plus. Leur effectif est toutefois largement négligeable et ils ne sont pas concernés par nos estimations puisqu'ils ne se sont pas inscrits à l'université (trois derniers panels de la Figure \ref{fig:g20age26aoutcompl}).

La grande majorité des néo-bacheliers de Parcoursup sont nés en France (93\%) et de nationalité française (96\%)\footnote{Les modalités présentes de la variable de nationalité dans notre base se limitent à ``Française'' et ``Hors union européenne''. Il n'y a pas de valeurs manquantes. Cela suggère qu'il n'y a pas d'individus de nationalité d'un pays de l'Union Européenne autre que la France.}. Ils sont 63\% à être boursiers de l'enseignement secondaire.

Les variables sur l'établissement d'origine sont le statut (public ou privé), le type (propose des formations postbac\footnote{Nous n'avons pas plus de précision mais ce sont principalement les formations de Brevet de Technicien Supérieur (BTS) et de préparation aux grandes écoles.} ou non) et le département. Les individus viennent en grande majorité (96\%) d'un établissement public. Même si la grande majorité des inscrits viennent d'un établissement de La Réunion (84\%), il existe une proportion non négligeable (14\%) d'individus qui viennent d'un établissement français en dehors de La Réunion.

En ce qui concerne le profil et les résultats au bac, les séries professionnelles représentent 25\% des néo-bacheliers. Ce chiffre est d'environ 30\% pour chacune des séries technologique et ES. Ceux en série S ne représentent que 8\% et les autres séries\footnote{Séries L ou brevet professionnel, par exemple.} comptent pour 4\%.
Nous disposons également de la note globale, de la note aux épreuves écrites de français et de la note en mathématiques des individus. Pour les séries littéraires, les notes en français et en mathématiques sont toutes non renseignées. Pour les séries professionnelles, les notes en français ne sont pas renseignées.
Pour ces différentes raisons, nous retenons principalement la note globale dans nos modèles. Cette variable contient 3\% de valeurs manquantes. En 2020, ils ont une moyenne d'un peu moins de 12 sur 20.\\
La mention au bac est également disponible sans que nous n'ayons eu à la construire. Elle est globalement cohérente avec la note au bac. Elle contient moins de valeurs manquantes que la note au bac mais au vu de leur effectif, surtout dans nos échantillons d'estimation, cela n'affecte pas nos résultats. La part des inscrits n'ayant obtenu aucune mention est de 58\%. Ceux ayant obtenu les mentions assez bien, bien et très bien représentent respectivement 32\%, 9\% et 2\%.

La base Parcoursup nous permet enfin de connaître la filière et le campus choisis par les inscrits de Parcoursup. Parmi les néo-bacheliers, ils sont 73\% en AES et 27\% en Eco-Ge ; 38\% au Sud et 62\% au Nord.

Les statistiques descriptives pour les inscrits de Parcoursup intéressés par la plateforme (voir Section \ref{g20instproto}) sont données par les colonnes (3) et (4) du Tableau \ref{tab:g20stats}. Nous rappelons que les inscrits intéressés comptent significativement plus de séries ES et S, de femmes, d'individus nés à l'étranger et venant d'un territoire français autre que La Réunion\footnote{Des analyses supplémentaires des différences entre les différentes sous-échantillons sont faites dans la Section \ref{g20methods}.}.

\quad Dans la base issue de la plateforme ``J'ai 20 en maths'', pour chaque jour de l'expérimentation, nous connaissons le nombre de connexions, d'exercices corrigés vus\footnote{Nous ne savons pas si le participant a effectué l'exercice ou non.}, le nombre de vidéos visionnées et le nombre de vidéos visionnées entièrement. Comme indicateurs d'utilisation de la plateforme, nous retenons la somme par individu, du 01\textsuperscript{er} juillet au 10 septembre 2020, des vidéos visionnées entièrement, des vidéos visionnées et des exercices vus. Pour tenter d'avoir une mesure synthétique, même imparfaite de l'utilisation de la plateforme, nous considérons également la somme jusqu'au 10 septembre 2020 des vidéos visionnées entièrement \emph{ou} d'exercices vus ainsi que la somme des vidéos visionnées \emph{ou} d'exercices vus. Nous ne retenons pas le nombre de connexion dans nos régressions puisqu'il contient la même information que les autres mesures ci-dessus.\\
Parmi les néo-bacheliers intéressés, 20\% ne se sont pas connectés à la plateforme. Pour ces individus, nous ramenons les différentes mesures de l'utilisation de la plateforme ci-dessous à zéro. Cela permet de garder les observations correspondantes dans les estimations. Nous avons vérifié que l'ampleur des coefficients estimés est robuste à l'exclusion de ces observations. Cette robustesse est attendue puisque la part d'inscrits qui se sont connectés au moins une fois est significativement plus grande parmi les incités (88\%) que parmi les non incités (74\%).\\
Les participants ont très peu visionné les vidéos jusqu'à la fin puisque seulement 12\% ont visionné au moins une vidéo jusqu'à la fin, pour une moyenne de 0.71. Le nombre moyen d'exercices vus de 10.8 qui apparaît important est dû à des valeurs extrêmes de l'ordre de 200. La médiane des exercices vus est de 2 et 37\% des participants n'ont vu aucun exercice.
Au final, les participants \emph{a priori} intéressés par la plateforme l'ont très peu utilisée.

\quad Les fichiers extraits d'APOGEE contiennent principalement les notes des étudiants. Les identifiants individuels sont le numéro étudiant (qui n'est pas le numéro Parcoursup), le nom, un prénom et la date de naissance. La filière est également disponible. Nous nous sommes limités aux notes en mathématiques du premier semestre de l'année 2020-2021. Étant donné la courte durée de l'expérimentation de 4 semaines, il nous paraît plutôt improbable que d'éventuels effets puissent subsister jusqu'aux examens du second semestre. Nous laissons l'analyse de l'impact de l'utilisation de la plateforme pendant une plus longue durée à d'éventuelles futures recherches.\\
Les statistiques descriptives correspondantes sont montrées par les colonnes (5) et (6) du Tableau \ref{tab:g20stats}. Plus précisément, ces fichiers contiennent les notes aux TD et aux examens de mathématiques et la moyenne de ces deux dernières (notes à l'UE de mathématiques) pour tous les étudiants inscrits à ces évaluations. Les notes aux TD et aux examens compte chacune pour 50\% de la note à l'UE. Ce sont des notes sur 20 points. De manière plus annexe, nous avons aussi pu extraire d'APOGEE les notes aux UE de gestion et d'économie (en tant que matières). La Section \ref{g20methods} en détaille l'utilité.\\
Nous comptons 1763 étudiants inscrits aux examens.
Dans les fichiers APOGEE, les notes à l'UE sont mises à 0 pour les étudiants qui ne se sont pas présentés aux examens. Pour les notes aux TD et au contrôle terminal, ce sont des valeurs manquantes. Cela explique le faible pourcentage de valeurs manquantes de 1\% dans la note à l'UE et les pourcentages plus forts de 21\% et de 30\% dans les notes aux TD et aux examens, respectivement. Ces pourcentages élevés correspondent en partie au fait que des étudiants \emph{a priori} inscrits dans la formation ne la suivent finalement pas.
Au vu des moyennes de 6 points sur 20 aux TD et de 3 points sur 20 aux examens, le niveau en mathématiques de la promotion est faible\footnote{Les moyennes de mathématiques au bac largement plus élevées illustrent bien la difficulté de la transition secondaire-supérieur pour les étudiants.}.

Même si les examens ne sont pas comparables entre les deux filières et même si les évaluations aux TD le sont encore moins puisqu'elles diffèrent en fonction du groupe de TD (voir Section \ref{g20inst}), il peut être intéressant de comparer les différentes notes entre les filières.
Les néo-bacheliers en Eco-Ge ont clairement de meilleures notes mais demeurent de faible niveau\footnote{Ils ont une moyenne de 3 points, une médiane de 2.5 points et comptent 24\% d'étudiants ayant obtenu au moins 10 points au contrôle terminal. Ces chiffres sont respectivement de 2.5 points, de 0.5 points et de 7\% chez les étudiants d'AES.}. Il n'y a pas de raison de croire que l'examen des étudiants en Eco-Ge est significativement plus facile que celui des étudiants en AES. Les sujets des étudiants en Économie et en AES aux examens en décembre 2020 que nous avons pu observer renforce ce propos. La différence de note reflète alors très probablement la différence de niveau des étudiants entre les deux filières.

\quad Parmi les 306 individus intéressés, 260 (colonne 7) se sont effectivement inscrits. Nous repérons ces derniers grâce aux fichiers d'APOGEE décrits ci-dessus.

\quad Pour construire nos échantillons d'estimation, il a été nécessaire de passer par deux jointures successives. La première est une jointure entre la base issue de la plateforme et la base Parcoursup. Cette jointure est très directe puisque la clef de jointure est le mail provenant de Parcoursup qui a ensuite été transmis aux responsables de la plateforme pour création de compte (voir Section \ref{g20inst})\footnote{Il convient de mentionner que cela a été fait par un collègue enseignant chercheur.}. La seconde jointure se fait entre le résultat de la première jointure et la base APOGEE. Dans celle-ci, nous récupérons l'ensemble des étudiants à la fois participants à l'expérimentation et inscrits aux examens de mathématiques du premier semestre 2020. Cette seconde jointure est moins directe puisque nous utilisons le nom, le(s) prénom(s) et la date de naissance comme clefs de jointure\footnote{Dans la base APOGEE, seul un prénom est renseigné. La base Parcoursup renseigne quant à elle jusqu'à trois prénoms. Nous nous sommes assurés de ne pas perdre d'individus dans les jointures correspondantes.}. Le résultat est une base de 260 individus contenant les informations sur les individus, sur leur établissement d'origine (variables de contrôle), sur leur utilisation de la plateforme (variables explicatives d'intérêt) et sur leurs résultats de mathématiques (variables dépendantes). Les statistiques correspondantes sont montrées dans les colonnes (7) et (8) du Tableau \ref{tab:g20stats}.

Les parts des valeurs manquantes dans les notes aux TD (16\%) et aux examens (13\%) sont non négligeables. D'après le Tableau \ref{tab:g20compinscvenutd0venutd1} de l'Annexe \ref{g20compinscvenutd0venutd1}, ceux qui ont une note aux TD sont significativement meilleurs au bac et ont plus souvent un bac S et ES. Les chiffres sur l'utilisation de la plateforme sont également clairement différents mêmes si ces différences ne sont pas détectées par nos tests statistiques. Par exemple, la proportion de ceux qui ont vu au moins une vidéo entière ou un exercice est de 68\% chez ceux qui ont une note aux TD contre 57\% chez ceux qui n'ont n'en pas.\\
En observant le Tableau \ref{tab:g20compinscvenuctqcm0venuctqcm1} de l'Annexe \ref{g20compinscvenuctqcm0venuctqcm1}, la sélection de ceux qui ont une note aux examens est moins sévère mais existe toujours. Les conditions d'attribution de la bourse\footnote{Nous supposons que les boursiers du secondaire restent majoritairement boursiers en première année d'université.} qui incitent fortement les étudiants à se présenter aux examens (et non pas aux TD) constituent une explication naturelle de la différence de l'intensité de la sélection entre ceux qui ont une note aux TD et ceux qui ont une note aux examens.

Les colonnes (9) et (10), (11) et (12) du Tableau \ref{tab:g20stats} donnent les statistiques descriptives de l'échantillon d'estimation pour ceux qui ont une note aux TD et pour ceux qui ont une note aux examens, respectivement.

\begin{landscape}\begingroup\fontsize{5}{7}\selectfont

\begin{ThreePartTable}
\begin{TableNotes}
\item \textit{Sources :} Base Parcoursup - L1 AES et Eco-Ge, plateforme '\textit{J'ai 20 en maths}', APOGEE (version extraite en mai 2021), calculs de l'auteur.
\item \textit{Notes :} Moyennes et proportions. Écart-types entre parenthèses. Les valeurs des variables liées à la plateforme (vidéos visionnées entièrement ou non et/ou exercices) sont mises à 0 pour ceux qui ne se sont pas connectés. UE : Unité d'enseignement, ES : Économique et social, S : Scientifique, AES : Administration Économique et Sociale, Eco-Ge : Économie-Gestion. TD : Travaux Dirigés. UE : Unité d'Enseignement.
\end{TableNotes}
\begin{longtable}[t]{lllllllllllll}
\caption{\label{tab:g20stats}Statistiques descriptives}\\
\toprule
\multicolumn{1}{c}{ } & \multicolumn{2}{c}{Parcoursup (tous)} & \multicolumn{2}{c}{Parcoursup (intéressés)} & \multicolumn{2}{c}{Notes de mathématiques} & \multicolumn{2}{c}{Inscrits} & \multicolumn{2}{c}{\makecell{Inscrits - Ayant une note \\ aux TD}} & \multicolumn{2}{c}{\makecell{Inscrits - Ayant une note \\ aux examens}} \\
\cmidrule(l{3pt}r{3pt}){2-3} \cmidrule(l{3pt}r{3pt}){4-5} \cmidrule(l{3pt}r{3pt}){6-7} \cmidrule(l{3pt}r{3pt}){8-9} \cmidrule(l{3pt}r{3pt}){10-11} \cmidrule(l{3pt}r{3pt}){12-13}
  & \makecell{\makecell{Moyenne \\ (Écart-type)} \\ (1) } & \makecell{\makecell{Proportion \\ manquantes} \\ (2) } & \makecell{\makecell{Moyenne \\ (Écart-type)} \\ (3) } & \makecell{\makecell{Proportion \\ manquantes} \\ (4) } & \makecell{\makecell{Moyenne \\ (Écart-type)} \\ (5) } & \makecell{\makecell{Proportion \\ manquantes} \\ (6) } & \makecell{\makecell{Moyenne \\ (Écart-type)} \\ (7) } & \makecell{\makecell{Proportion \\ manquantes} \\ (8) } & \makecell{\makecell{Moyenne \\ (Écart-type)} \\ (9) } & \makecell{\makecell{Proportion \\ manquantes} \\ (10) } & \makecell{\makecell{Moyenne \\ (Écart-type)} \\ (11) } & \makecell{\makecell{Proportion \\ manquantes} \\ (12) }\\
\midrule
\endfirsthead
\caption[]{\label{tab:g20stats}Statistiques descriptives (suite)}\\
\toprule
  & \makecell{\makecell{Moyenne \\ (Écart-type)} \\ (1) } & \makecell{\makecell{Proportion \\ manquantes} \\ (2) } & \makecell{\makecell{Moyenne \\ (Écart-type)} \\ (3) } & \makecell{\makecell{Proportion \\ manquantes} \\ (4) } & \makecell{\makecell{Moyenne \\ (Écart-type)} \\ (5) } & \makecell{\makecell{Proportion \\ manquantes} \\ (6) } & \makecell{\makecell{Moyenne \\ (Écart-type)} \\ (7) } & \makecell{\makecell{Proportion \\ manquantes} \\ (8) } & \makecell{\makecell{Moyenne \\ (Écart-type)} \\ (9) } & \makecell{\makecell{Proportion \\ manquantes} \\ (10) } & \makecell{\makecell{Moyenne \\ (Écart-type)} \\ (11) } & \makecell{\makecell{Proportion \\ manquantes} \\ (12) }\\
\midrule
\endhead

\endfoot
\bottomrule
\insertTableNotes
\endlastfoot
\addlinespace[0.3em]
\multicolumn{13}{l}{\textbf{Notes}}\\
\hspace{1em}Aux TD & - & - & - & - & 6.63 (6.26) & 0.3 & 7.73 (6.56) & 0.16 & 7.73 (6.56) & 0 & 7.97 (6.55) & 0.08\\
\hspace{1em}Aux examens & - & - & - & - & 3.11 (4.52) & 0.21 & 4.19 (5.03) & 0.13 & 4.49 (5.11) & 0.05 & 4.19 (5.03) & 0\\
\hspace{1em}À l'UE & - & - & - & - & 3.55 (4.75) & 0.01 & 5.02 (5.36) & 0 & 5.95 (5.37) & 0 & 5.72 (5.4) & 0\\
\addlinespace[0.3em]
\multicolumn{13}{l}{\textbf{Plateforme}}\\
\hspace{1em}Connecté & - & - & 0.77 (0.42) & 0 & - & - & 0.8 (0.4) & 0 & 0.82 (0.39) & 0 & 0.81 (0.39) & 0\\
\hspace{1em}Vidéos (complètes) ou exercices $\geq 1$ & - & - & 0.63 (0.48) & 0 & - & - & 0.67 (0.47) & 0 & 0.68 (0.47) & 0 & 0.68 (0.47) & 0\\
\hspace{1em}Vidéos (complètes) ou exercices & - & - & 11.51 (26.74) & 0 & - & - & 12.86 (28.37) & 0 & 13.96 (29.9) & 0 & 13.58 (29.53) & 0\\
\hspace{1em}Vidéos ou exercices $\geq 1$ & - & - & 0.63 (0.48) & 0 & - & - & 0.67 (0.47) & 0 & 0.69 (0.46) & 0 & 0.68 (0.47) & 0\\
\hspace{1em}Vidéos ou exercices & - & - & 15.42 (34.09) & 0 & - & - & 17.09 (35.56) & 0 & 18.67 (37.39) & 0 & 18.09 (36.89) & 0\\
\hspace{1em}Vidéos (complètes) $\geq 1$ & - & - & 0.12 (0.33) & 0 & - & - & 0.13 (0.33) & 0 & 0.13 (0.34) & 0 & 0.14 (0.35) & 0\\
\hspace{1em}Vidéos (complètes) & - & - & 0.71 (2.92) & 0 & - & - & 0.8 (3.15) & 0 & 0.76 (3.09) & 0 & 0.79 (3.09) & 0\\
\hspace{1em}Vidéos $\geq 1$ & - & - & 0.3 (0.46) & 0 & - & - & 0.34 (0.47) & 0 & 0.37 (0.48) & 0 & 0.35 (0.48) & 0\\
\hspace{1em}Vidéos & - & - & 4.62 (11.91) & 0 & - & - & 5.02 (12.14) & 0 & 5.47 (12.65) & 0 & 5.3 (12.46) & 0\\
\hspace{1em}Exercices $\geq 1$ & - & - & 0.63 (0.48) & 0 & - & - & 0.67 (0.47) & 0 & 0.68 (0.47) & 0 & 0.68 (0.47) & 0\\
\hspace{1em}Exercices & - & - & 10.8 (25.31) & 0 & - & - & 12.07 (26.82) & 0 & 13.2 (28.47) & 0 & 12.79 (28.1) & 0\\
\addlinespace[0.3em]
\multicolumn{13}{l}{\textbf{Notes au bac}}\\
\hspace{1em}Note au bac & 11.78 (1.46) & 0.03 & 11.99 (1.59) & 0.03 & - & - & 12 (1.62) & 0.01 & 12.11 (1.66) & 0 & 12.11 (1.65) & 0.01\\
\hspace{1em}En français (écrits) & 10.06 (3.3) & 0.29 & 10.27 (3.28) & 0.2 & - & - & 10.29 (3.35) & 0.18 & 10.48 (3.38) & 0.13 & 10.33 (3.38) & 0.16\\
\hspace{1em}En mathématiques & 11.19 (3.23) & 0.07 & 11.42 (3.26) & 0.07 & - & - & 11.28 (3.26) & 0.05 & 11.42 (3.3) & 0.04 & 11.39 (3.33) & 0.05\\
\addlinespace[0.3em]
\multicolumn{13}{l}{\textbf{Série au bac}}\\
\hspace{1em}Pro & 0.25 & 0 & 0.15 & 0 & - & - & 0.14 & 0 & 0.1 & 0 & 0.12 & 0\\
\hspace{1em}Techno & 0.31 & 0 & 0.28 & 0 & - & - & 0.27 & 0 & 0.27 & 0 & 0.25 & 0\\
\hspace{1em}ES & 0.33 & 0 & 0.39 & 0 & - & - & 0.42 & 0 & 0.44 & 0 & 0.43 & 0\\
\hspace{1em}S & 0.08 & 0 & 0.13 & 0 & - & - & 0.14 & 0 & 0.17 & 0 & 0.16 & 0\\
\hspace{1em}Autre & 0.04 & 0 & 0.05 & 0 & - & - & 0.03 & 0 & 0.03 & 0 & 0.04 & 0\\
\addlinespace[0.3em]
\multicolumn{13}{l}{\textbf{Mention au bac}}\\
\hspace{1em}Aucune & 0.57 & 0.01 & 0.5 & 0.02 & - & - & 0.5 & 0 & 0.46 & 0 & 0.47 & 0\\
\hspace{1em}Assez bien & 0.32 & 0.01 & 0.35 & 0.02 & - & - & 0.35 & 0 & 0.37 & 0 & 0.36 & 0\\
\hspace{1em}Bien & 0.09 & 0.01 & 0.12 & 0.02 & - & - & 0.13 & 0 & 0.14 & 0 & 0.14 & 0\\
\hspace{1em}Très bien & 0.02 & 0.01 & 0.02 & 0.02 & - & - & 0.02 & 0 & 0.03 & 0 & 0.03 & 0\\
\addlinespace[0.3em]
\multicolumn{13}{l}{\textbf{ }}\\
\hspace{1em}Âge à la rentrée & 18.66 (1.61) & 0 & 18.74 (2.26) & 0 & - & - & 18.52 (0.86) & 0 & 18.45 (0.67) & 0 & 18.52 (0.89) & 0\\
\addlinespace[0.3em]
\multicolumn{13}{l}{\textbf{Sexe}}\\
\hspace{1em}Fille & 0.63 & 0 & 0.7 & 0 & - & - & 0.7 & 0 & 0.7 & 0 & 0.7 & 0\\
\hspace{1em}Homme & 0.37 & 0 & 0.3 & 0 & - & - & 0.3 & 0 & 0.3 & 0 & 0.3 & 0\\
\addlinespace[0.3em]
\multicolumn{13}{l}{\textbf{Pays de naissance}}\\
\hspace{1em}À l'étranger & 0.07 & 0 & 0.11 & 0 & - & - & 0.1 & 0 & 0.09 & 0 & 0.08 & 0\\
\hspace{1em}En France & 0.93 & 0 & 0.89 & 0 & - & - & 0.9 & 0 & 0.91 & 0 & 0.92 & 0\\
\addlinespace[0.3em]
\multicolumn{13}{l}{\textbf{Nationalité}}\\
\hspace{1em}Hors union européenne & 0.04 & 0 & 0.06 & 0 & - & - & 0.05 & 0 & 0.04 & 0 & 0.03 & 0\\
\hspace{1em}Française & 0.96 & 0 & 0.94 & 0 & - & - & 0.95 & 0 & 0.96 & 0 & 0.97 & 0\\
\addlinespace[0.3em]
\multicolumn{13}{l}{\textbf{Statut de boursier}}\\
\hspace{1em}Non boursier & 0.37 & 0 & 0.36 & 0 & - & - & 0.37 & 0 & 0.36 & 0 & 0.34 & 0\\
\hspace{1em}Boursier du secondaire & 0.63 & 0 & 0.64 & 0 & - & - & 0.63 & 0 & 0.64 & 0 & 0.66 & 0\\
\addlinespace[0.3em]
\multicolumn{13}{l}{\textbf{Statut de l'établissement d'origine}}\\
\hspace{1em}Public & 0.95 & 0.01 & 0.95 & 0.01 & - & - & 0.95 & 0 & 0.94 & 0 & 0.94 & 0\\
\hspace{1em}Privé & 0.05 & 0.01 & 0.05 & 0.01 & - & - & 0.05 & 0 & 0.06 & 0 & 0.06 & 0\\
\addlinespace[0.3em]
\multicolumn{13}{l}{\textbf{Type de l'établissement d'origine}}\\
\hspace{1em}Sans postbac & 0.06 & 0.01 & 0.05 & 0.01 & - & - & 0.05 & 0 & 0.05 & 0 & 0.06 & 0\\
\hspace{1em}Avec postbac & 0.94 & 0.01 & 0.94 & 0.01 & - & - & 0.95 & 0 & 0.95 & 0 & 0.94 & 0\\
\addlinespace[0.3em]
\multicolumn{13}{l}{\textbf{Département de l'établissement d'origine}}\\
\hspace{1em}La Réunion & 0.84 & 0.01 & 0.77 & 0 & - & - & 0.79 & 0 & 0.8 & 0 & 0.78 & 0\\
\hspace{1em}France hors Réunion & 0.14 & 0.01 & 0.2 & 0 & - & - & 0.18 & 0 & 0.17 & 0 & 0.19 & 0\\
\hspace{1em}Étranger & 0.01 & 0.01 & 0.03 & 0 & - & - & 0.03 & 0 & 0.03 & 0 & 0.03 & 0\\
\addlinespace[0.3em]
\multicolumn{13}{l}{\textbf{Campus}}\\
\hspace{1em}Tampon & 0.38 & 0 & 0.39 & 0 & - & - & 0.4 & 0 & 0.42 & 0 & 0.4 & 0\\
\hspace{1em}Saint-Denis & 0.62 & 0 & 0.61 & 0 & - & - & 0.6 & 0 & 0.58 & 0 & 0.6 & 0\\
\addlinespace[0.3em]
\multicolumn{13}{l}{\textbf{Filière (Parcoursup)}}\\
\hspace{1em}AES & 0.73 & 0 & 0.75 & 0 & - & - & 0.74 & 0 & 0.74 & 0 & 0.73 & 0\\
\hspace{1em}Eco-Ge & 0.27 & 0 & 0.25 & 0 & - & - & 0.26 & 0 & 0.26 & 0 & 0.27 & 0\\
\addlinespace[0.3em]
\multicolumn{13}{l}{\textbf{Filière}}\\
\hspace{1em}AES & - & - & - & - & 0.74 & 0 & 0.74 & 0 & 0.74 & 0 & 0.73 & 0\\
\hspace{1em}Eco-Ge & - & - & - & - & 0.26 & 0 & 0.26 & 0 & 0.26 & 0 & 0.27 & 0\\
\addlinespace[0.3em]
\multicolumn{13}{l}{\textbf{ }}\\
\hspace{1em}Observations & 1142 &  & 306 &  & 1214 &  & 260 &  & 218 &  & 225 & \\*
\end{longtable}
\end{ThreePartTable}
\endgroup{}
\end{landscape}

\hypertarget{g20methods}{%
\section{Modélisation économétrique}\label{g20methods}}

Afin de savoir si réviser les mathématiques du lycée sur la plateforme \emph{``J'ai 20 en maths''} impacte effectivement les performances en mathématiques des étudiants en première année de Licence d'Économie-Gestion et d'AES, nous nous proposons d'estimer l'équation d'intérêt suivante :

\begin{equation}
\label{eq:g20ols}
y_i^{l1} = \sigma_0 + \sigma_1 g_i + x'_i \sigma_2 + \phi_i. 
\end{equation}

Dans l'équation \eqref{eq:g20ols}, \(y_i^{l1}\) peut désigner une note (aux TD ou à l'examen final) de mathématiques au premier semestre de l'année universitaire 2020-2021 de l'étudiant \(i\). Elle peut également désigner la note aux UE de gestion ou d'économie. Nous revenons sur l'intérêt de cette considération ci-après. La variable \(g_i\) est notre variable explicative d'intérêt. Elle désigne une des mesures de l'utilisation de la plateforme décrites dans la section précédente.

Sauf mention contraire, le vecteur \(x'_i\) est constitué des variables de contrôle suivantes : la note au bac, la série au bac, le sexe, le pays de naissance, le statut de boursier, le statut de l'établissement d'origine et le département de l'établissement d'origine (voir Section \ref{g20data}, Tableau \ref{tab:g20stats}, par exemple). Nous omettons la nationalité puisque dans nos données, elle est très corrélée avec le pays de naissance.\\
Dans la même logique, nous omettons le type de l'établissement d'origine vu que cette variable est très corrélée avec le statut de l'établissement d'origine dans nos données. À notre sens, le statut de l'établissement d'origine est plus pertinent par rapport à son type puisqu'il ne semble pas y avoir de faits stylisés indiquant une différence de niveau systématique entre les lycées issus d'établissements avec ou sans formation postbac (type d'établissement) alors qu'il est connu que les lycées privés attirent des élèves \emph{a priori} plus forts en France (\protect\hyperlink{ref-PEL:12}{Pellet, 2012}, par exemple)\footnote{Cela est également vrai dans un contexte plus large que le contexte français (\protect\hyperlink{ref-BAR:ROU:09}{Rouse \& Barrow, 2009}).}.\\
Le terme d'erreur \(\phi_i\) désigne par définition l'ensemble des facteurs inobservés par le chercheur qui déterminent \(y_i^{l1}\). Nous rappelons que le talent et la motivation constituent deux exemples typiques dans la littérature.\\
Le paramètre d'intérêt est \(\sigma_1\). Son interprétation exacte dépend de la mesure considérée \(g_i\) parmi celles décrites dans la Section \ref{g20data}.

\quad Les étudiants ne choisissent pas au hasard à quel point ils utilisent la plateforme. Les plus capables et/ou les plus motivés vont vraisemblablement plus l'utiliser. Or, ces derniers ont tendance à avoir des caractéristiques inobservées favorables aux notes de mathématiques.
Cela se traduit par une corrélation positive entre \(g_i\) et \(\phi_i\). Dans ce chapitre, nous profitons de l'existence du protocole pour justifier ce postulat en régressant les variables d'utilisation de la plateforme sur les variables de contrôle pour les participants non incités\footnote{Nous utilisons l'échantillon des néo-bacheliers inscrits (colonnes 7 et 8 du Tableau \ref{tab:g20stats}) pour maximiser la puissance statistique. Les conclusions sont les mêmes sur les échantillons de ceux qui ont une note aux TD ou aux examens.}. Comme l'incitation a été attribuée de manière aléatoire, les non incités représentent une bonne approximation de la population des inscrits dans un monde sans protocole. S'il y a une corrélation positive entre l'utilisation de la plateforme et les observables favorables aux performances en mathématiques, il est raisonnable de penser que cette corrélation existe également entre l'utilisation de la plateforme et les inobservables. Les résultats desdites régressions montrées dans le Tableau \ref{tab:g20ovbmodels} de l'Annexe \ref{g20ovbmodels} vont dans ce sens puisqu'indépendamment de la mesure d'utilisation de la plateforme considérée (colonnes 1 à 5), à série égale, ceux qui ont eu une meilleure note au bac ont plus utilisé la plateforme. De la même manière, à note au bac égale, les titulaires d'un bac ES et S ont plus utilisé la plateforme, excepté lorsque l'utilisation est mesurée par le nombre de vidéos visionnées entièrement. Les coefficients devant les autres observables sont relativement de plus faible ampleur et sont non significatifs.

Estimer l'équation \eqref{eq:g20ols} par moindres carrés ordinaires ferait alors ressortir \emph{a priori} une estimation biaisée vers le haut de \(\sigma_1\).\\
Le protocole d'encouragement (Section \ref{g20instproto}) a été mis en place pour surmonter ce problème. Du point de vue de la modélisation, l'incitation, qui a été assignée aléatoirement aux participants, est censée générer une variation exogène dans l'intensité d'utilisation de la plateforme. Elle est alors par construction une variable instrumentale de la variable explicative d'intérêt. Nous écrivons l'équation de première étape comme suit :

\begin{equation}
\label{eq:g20pe}
g_i = \tau_0 + \tau_1 z_i + x'_i \tau_2 + \psi_i.
\end{equation}

Au sein de l'équation \eqref{eq:g20pe}, \(z_i\) est une indicatrice qui vaut 1 si l'étudiant \(i\) est incité et 0 sinon. Nous n'avons pas d'autre spécification de l'incitation. La variable \(\psi_i\) est le terme d'erreur usuel. Le paramètre \(\tau_1\) est le coefficient de première étape d'intérêt. Il désigne l'effet d'être incité sur l'intensité d'utilisation de la plateforme et nous permet de savoir dans quelle mesure envoyer 4 mails espacés d'environ 10 jours pendant 4 semaines modifie le comportement moyen des étudiants vis à vis de la plateforme.

\quad Nous rappelons qu'il existe deux conditions de validité de \(z_i\) qui sont l'exogénéité et l'existence de corrélation avec \(g_i\). La condition d'exogénéité est elle-même constituée de deux hypothèses. La première est que l'incitation doit être considérée comme au moins proche de l'aléatoire. Dans l'équation \eqref{eq:g20pe}, elle se traduit par l'absence de corrélation entre \(z_i\) et \(\psi_i\). Cette condition est aisément vérifiée dans notre cas puisque l'incitation a été explicitement assignée de manière aléatoire aux néo-bacheliers inscrits dans Parcoursup ayant donné leur avis favorable. Toutefois, au vu des différents phénomènes de sélection décrits dans les Sections \ref{g20instuniv} et \ref{g20data} entre les inscrits de Parcoursup et ceux ayant une note aux TD ou aux examens, il nous faut vérifier si cette assignation aléatoire se reflète sur nos deux échantillons d'estimation (colonnes 9 et 10 ; et 11 et 12 du Tableau \ref{tab:g20stats}). Pour ce faire, nous comparons les caractéristiques prédéterminées des participants en fonction de l'incitation dans chaque échantillon pertinent, c'est-à-dire à partir des néo-bacheliers de Parcoursup jusqu'à ceux qui se sont présentés aux évaluations de mathématiques. Le Tableau \ref{tab:g20compintz0z1} de l'Annexe \ref{g20compintinscz0z1} montre clairement que les caractéristiques prédéterminées sont comparables entre les individus non incités et incités chez les néo-bacheliers inscrits dans Parcoursup et \emph{a priori} intéressés par l'utilisation de la plateforme.

Les tests détectent qu'une légère sélection se produit entre les intéressés et les inscrits (Tableau \ref{tab:g20compintinsc} de l'Annexe \ref{g20compintinsc}) puisque ceux qui se sont effectivement inscrits sont significativement plus jeunes et ont une composition différente en ce qui concerne la série au bac. Nous pensons toutefois que les étudiants inscrits restent représentatifs des ceux intéressés par la plateforme au vu de la grande proportion d'inscrits (85\%) et du fait que les autres caractéristiques (notes au bac, mention au bac, sexe, etc.) sont comparables. De plus, comme nous faisons des tests sur beaucoup de variables, une différence significative peut apparaître par hasard.

Chez les néo-bacheliers inscrits, l'assignation aléatoire de l'instrument reste clairement reflétée par le fait que les caractéristiques des non incités sont comparables à celles des incités (Tableau \ref{tab:g20compinscz0z1} de l'Annexe \ref{g20compintinscz0z1}).

Nous en venons aux deux échantillons d'estimation qui sont ceux, parmi les néo-bacheliers intéressés et inscrits, qui disposent d'une note aux TD et qui disposent d'une note aux examens. Les comparaisons correspondantes en fonction de l'incitation sont montrées dans les Tableaux \ref{tab:g20compvenustdz0z1} et \ref{tab:g20compvenusctqcmz0z1}, respectivement\footnote{À ce niveau, nous n'analysons pas encore les variables dans la plateforme dans ces tableaux.}. A première vue, aucune différence statistiquement significative n'est détectée. Les différences marquantes dans le Tableau \ref{tab:g20compvenustdz0z1} concernent les proportions de séries professionnelles (13\% chez les non incités contre 6\% chez les incités) et technologiques (22\% chez les non incités contre 32\% chez les incités). Au vu de nos effectifs de plus en plus réduits, de telles différences peuvent survenir par hasard. Deux éléments nous rassurent. Le premier est que les différences sur les autres variables ne sont pas de grande ampleur sauf sur le département de l'établissement d'origine. Le second élément est que de telles différences, lorsqu'elles ne sont pas trop importantes, sont neutralisées par l'inclusion des variables de contrôle dans nos régressions. Dans le Tableau \ref{tab:g20compvenusctqcmz0z1}, nous n'observons pas de différence statistiquement significative ou d'ampleur considérable entre les incités et les non incités.

\newpage
\begingroup\fontsize{5}{7}\selectfont

\begin{ThreePartTable}
\begin{TableNotes}
\item \textit{Sources :} Base Parcoursup - L1 AES et Eco-Ge, plateforme '\textit{J'ai 20 en maths}', APOGEE (version extraite en mai 2021), calculs de l'auteur.
\item \textit{Notes :} Moyennes et proportions. Écart-types entre parenthèses. Les probabilités critiques correspondent à des tests de comparaison de moyennes pour les variables quantitatives et à des tests de comparaison de proportions pour les variables qualitatives. ES : Économique et social, S : Scientifique, AES : Administration Économique et Sociale. Eco-Ge : Économie-Gestion. TD : Travaux Dirigés.
\item Significativité : 10\% * 5\% ** 1\% ***.
\end{TableNotes}
\begin{longtable}[t]{llll}
\caption{\label{tab:g20compvenustdz0z1}Comparaison entre les étudiants non incités et incités (Ayant une note aux TD)}\\
\toprule
  & \makecell{\makecell{Non incités \\ \ } \\ (1) } & \makecell{\makecell{Incités \\ \ } \\ (2) } & \makecell{\makecell{(1) = (2) \\ probabilité critique} \\ (3) }\\
\midrule
\endfirsthead
\caption[]{\label{tab:g20compvenustdz0z1}Comparaison entre les étudiants non incités et incités (Ayant une note aux TD) (suite)}\\
\toprule
  & \makecell{\makecell{Non incités \\ \ } \\ (1) } & \makecell{\makecell{Incités \\ \ } \\ (2) } & \makecell{\makecell{(1) = (2) \\ probabilité critique} \\ (3) }\\
\midrule
\endhead

\endfoot
\bottomrule
\insertTableNotes
\endlastfoot
\addlinespace[0.3em]
\multicolumn{4}{l}{\textbf{Note au bac}}\\
\hspace{1em}Totale & 12.19 (1.58) & 12.01 (1.76) & 0.42\\
\hspace{1em}En français (écrits) & 10.25 (3.17) & 10.75 (3.61) & 0.31\\
\hspace{1em}En mathématiques & 11.45 (3.37) & 11.38 (3.22) & 0.87\\
\addlinespace[0.3em]
\multicolumn{4}{l}{\textbf{Série au bac}}\\
\hspace{1em}Pro & 0.13 & 0.06 & 0.13\\
\hspace{1em}Techno & 0.22 & 0.32 & \\
\hspace{1em}ES & 0.45 & 0.42 & \\
\hspace{1em}S & 0.18 & 0.14 & \\
\hspace{1em}Autre & 0.02 & 0.05 & \\
\addlinespace[0.3em]
\multicolumn{4}{l}{\textbf{Mention au bac}}\\
\hspace{1em}Aucune & 0.41 & 0.53 & 0.19\\
\hspace{1em}Assez bien & 0.42 & 0.3 & \\
\hspace{1em}Bien & 0.15 & 0.13 & \\
\hspace{1em}Très bien & 0.02 & 0.04 & \\
\addlinespace[0.3em]
\multicolumn{4}{l}{\textbf{ }}\\
\hspace{1em}Âge à la rentrée & 18.43 (0.64) & 18.48 (0.7) & 0.56\\
\addlinespace[0.3em]
\multicolumn{4}{l}{\textbf{Sexe}}\\
\hspace{1em}Fille & 0.66 & 0.75 & 0.23\\
\hspace{1em}Homme & 0.34 & 0.25 & \\
\addlinespace[0.3em]
\multicolumn{4}{l}{\textbf{Pays de naissance}}\\
\hspace{1em}À l'étranger & 0.09 & 0.09 & 1\\
\hspace{1em}En France & 0.91 & 0.91 & \\
\addlinespace[0.3em]
\multicolumn{4}{l}{\textbf{Nationalité}}\\
\hspace{1em}Hors union européenne & 0.03 & 0.05 & 0.73\\
\hspace{1em}Française & 0.97 & 0.95 & \\
\addlinespace[0.3em]
\multicolumn{4}{l}{\textbf{Statut de boursier}}\\
\hspace{1em}Non boursier & 0.34 & 0.38 & 0.56\\
\hspace{1em}Boursier du secondaire & 0.66 & 0.62 & \\
\addlinespace[0.3em]
\multicolumn{4}{l}{\textbf{Statut de l'établissement d'origine}}\\
\hspace{1em}Public & 0.93 & 0.96 & 0.55\\
\hspace{1em}Privé & 0.07 & 0.04 & \\
\addlinespace[0.3em]
\multicolumn{4}{l}{\textbf{Type de l'établissement d'origine}}\\
\hspace{1em}Sans postbac & 0.06 & 0.03 & 0.35\\
\hspace{1em}Avec postbac & 0.94 & 0.97 & \\
\addlinespace[0.3em]
\multicolumn{4}{l}{\textbf{Département de l'établissement d'origine}}\\
\hspace{1em}La Réunion & 0.83 & 0.76 & 0.23\\
\hspace{1em}France hors Réunion & 0.13 & 0.22 & \\
\hspace{1em}Étranger & 0.03 & 0.02 & \\
\addlinespace[0.3em]
\multicolumn{4}{l}{\textbf{Campus}}\\
\hspace{1em}Tampon & 0.4 & 0.44 & 0.64\\
\hspace{1em}Saint-Denis & 0.6 & 0.56 & \\
\addlinespace[0.3em]
\multicolumn{4}{l}{\textbf{Filière}}\\
\hspace{1em}AES & 0.71 & 0.78 & 0.36\\
\hspace{1em}Eco-Ge & 0.29 & 0.22 & \\
\addlinespace[0.3em]
\multicolumn{4}{l}{\textbf{Plateforme}}\\
\hspace{1em}Connecté & 0.75 (0.44) & 0.9 (0.3) & 0***\\
\hspace{1em}Vidéos (complètes) ou exercices $\geq 1$ & 0.63 (0.48) & 0.75 (0.44) & 0.06*\\
\hspace{1em}Vidéos (complètes) ou exercices & 10.82 (22.07) & 17.74 (36.97) & 0.09*\\
\hspace{1em}Vidéos ou exercices $\geq 1$ & 0.63 (0.48) & 0.76 (0.43) & 0.04**\\
\hspace{1em}Vidéos ou exercices & 15.52 (32.53) & 22.45 (42.36) & 0.17\\
\hspace{1em}Vidéos (complètes) $\geq 1$ & 0.1 (0.3) & 0.17 (0.38) & 0.13\\
\hspace{1em}Vidéos (complètes) & 0.35 (1.23) & 1.25 (4.35) & 0.03**\\
\hspace{1em}Vidéos $\geq 1$ & 0.32 (0.47) & 0.43 (0.5) & 0.08*\\
\hspace{1em}Vidéos & 5.05 (12.57) & 5.97 (12.78) & 0.59\\
\hspace{1em}Exercices $\geq 1$ & 0.63 (0.48) & 0.75 (0.44) & 0.06*\\
\hspace{1em}Exercices & 10.47 (21.66) & 16.48 (34.78) & 0.12\\
\addlinespace[0.3em]
\multicolumn{4}{l}{\textbf{Résultats}}\\
\hspace{1em}Note aux TD & 7.11 (6.63) & 8.47 (6.44) & 0.13\\
\hspace{1em}Ayant une note aux examens & 0.96 (0.2) & 0.94 (0.24) & 0.53\\
\hspace{1em}Note aux examens & 4.62 (4.89) & 4.32 (5.39) & 0.68\\
\addlinespace[0.3em]
\multicolumn{4}{l}{\textbf{ }}\\
\hspace{1em}Observations & 119 & 99 & \\*
\end{longtable}
\end{ThreePartTable}
\endgroup{}

\newpage
\begingroup\fontsize{5.75}{7.75}\selectfont

\begin{ThreePartTable}
\begin{TableNotes}
\item \textit{Sources :} Base Parcoursup - L1 AES et Eco-Ge, plateforme '\textit{J'ai 20 en maths}', APOGEE (version extraite en mai 2021), calculs de l'auteur.
\item \textit{Notes :} Moyennes et proportions. Écart-types entre parenthèses. Les probabilités critiques correspondent à des tests de comparaison de moyennes pour les variables quantitatives et à des tests de comparaison de proportions pour les variables qualitatives. ES : Économique et social, S : Scientifique, AES : Administration Économique et Sociale. Eco-Ge : Économie-Gestion. TD : Travaux Dirigés.
\item Significativité : 10\% * 5\% ** 1\% ***.
\end{TableNotes}
\begin{longtable}[t]{llll}
\caption{\label{tab:g20compvenusctqcmz0z1}Comparaison entre les étudiants non incités et incités (Ayant une note aux examens)}\\
\toprule
  & \makecell{\makecell{Non incités \\ \ } \\ (1) } & \makecell{\makecell{Incités \\ \ } \\ (2) } & \makecell{\makecell{(1) = (2) \\ probabilité critique} \\ (3) }\\
\midrule
\endfirsthead
\caption[]{\label{tab:g20compvenusctqcmz0z1}Comparaison entre les étudiants non incités et incités (Ayant une note aux examens) (suite)}\\
\toprule
  & \makecell{\makecell{Non incités \\ \ } \\ (1) } & \makecell{\makecell{Incités \\ \ } \\ (2) } & \makecell{\makecell{(1) = (2) \\ probabilité critique} \\ (3) }\\
\midrule
\endhead

\endfoot
\bottomrule
\insertTableNotes
\endlastfoot
\addlinespace[0.3em]
\multicolumn{4}{l}{\textbf{Note au bac}}\\
\hspace{1em}Totale & 12.12 (1.59) & 12.09 (1.73) & 0.86\\
\hspace{1em}En français (écrits) & 10.14 (3.25) & 10.56 (3.53) & 0.4\\
\hspace{1em}En mathématiques & 11.38 (3.36) & 11.4 (3.31) & 0.95\\
\addlinespace[0.3em]
\multicolumn{4}{l}{\textbf{Série au bac}}\\
\hspace{1em}Pro & 0.13 & 0.12 & 0.48\\
\hspace{1em}Techno & 0.24 & 0.27 & \\
\hspace{1em}ES & 0.44 & 0.42 & \\
\hspace{1em}S & 0.17 & 0.14 & \\
\hspace{1em}Autre & 0.02 & 0.06 & \\
\addlinespace[0.3em]
\multicolumn{4}{l}{\textbf{Mention au bac}}\\
\hspace{1em}Aucune & 0.43 & 0.5 & 0.44\\
\hspace{1em}Assez bien & 0.4 & 0.32 & \\
\hspace{1em}Bien & 0.15 & 0.14 & \\
\hspace{1em}Très bien & 0.02 & 0.04 & \\
\addlinespace[0.3em]
\multicolumn{4}{l}{\textbf{ }}\\
\hspace{1em}Âge à la rentrée & 18.48 (0.73) & 18.58 (1.04) & 0.4\\
\addlinespace[0.3em]
\multicolumn{4}{l}{\textbf{Sexe}}\\
\hspace{1em}Fille & 0.68 & 0.72 & 0.64\\
\hspace{1em}Homme & 0.32 & 0.28 & \\
\addlinespace[0.3em]
\multicolumn{4}{l}{\textbf{Pays de naissance}}\\
\hspace{1em}À l'étranger & 0.1 & 0.07 & 0.56\\
\hspace{1em}En France & 0.9 & 0.93 & \\
\addlinespace[0.3em]
\multicolumn{4}{l}{\textbf{Nationalité}}\\
\hspace{1em}Hors union européenne & 0.03 & 0.03 & 1\\
\hspace{1em}Française & 0.97 & 0.97 & \\
\addlinespace[0.3em]
\multicolumn{4}{l}{\textbf{Statut de boursier}}\\
\hspace{1em}Non boursier & 0.34 & 0.34 & 1\\
\hspace{1em}Boursier du secondaire & 0.66 & 0.66 & \\
\addlinespace[0.3em]
\multicolumn{4}{l}{\textbf{Statut de l'établissement d'origine}}\\
\hspace{1em}Public & 0.93 & 0.94 & 1\\
\hspace{1em}Privé & 0.07 & 0.06 & \\
\addlinespace[0.3em]
\multicolumn{4}{l}{\textbf{Type de l'établissement d'origine}}\\
\hspace{1em}Sans postbac & 0.07 & 0.05 & 0.78\\
\hspace{1em}Avec postbac & 0.93 & 0.95 & \\
\addlinespace[0.3em]
\multicolumn{4}{l}{\textbf{Département de l'établissement d'origine}}\\
\hspace{1em}La Réunion & 0.8 & 0.76 & 0.32\\
\hspace{1em}France hors Réunion & 0.16 & 0.22 & \\
\hspace{1em}Étranger & 0.04 & 0.02 & \\
\addlinespace[0.3em]
\multicolumn{4}{l}{\textbf{Campus}}\\
\hspace{1em}Tampon & 0.39 & 0.41 & 0.93\\
\hspace{1em}Saint-Denis & 0.61 & 0.59 & \\
\addlinespace[0.3em]
\multicolumn{4}{l}{\textbf{Filière}}\\
\hspace{1em}AES & 0.7 & 0.76 & 0.47\\
\hspace{1em}Eco-Ge & 0.3 & 0.24 & \\
\addlinespace[0.3em]
\multicolumn{4}{l}{\textbf{Plateforme}}\\
\hspace{1em}Connecté & 0.73 (0.45) & 0.9 (0.3) & 0***\\
\hspace{1em}Vidéos (complètes) ou exercices $\geq 1$ & 0.61 (0.49) & 0.76 (0.43) & 0.02**\\
\hspace{1em}Vidéos (complètes) ou exercices & 10.4 (21.82) & 17.34 (36.39) & 0.08*\\
\hspace{1em}Vidéos ou exercices $\geq 1$ & 0.61 (0.49) & 0.77 (0.42) & 0.01**\\
\hspace{1em}Vidéos ou exercices & 14.77 (32.04) & 22.03 (41.74) & 0.14\\
\hspace{1em}Vidéos (complètes) $\geq 1$ & 0.1 (0.3) & 0.18 (0.39) & 0.06*\\
\hspace{1em}Vidéos (complètes) & 0.34 (1.21) & 1.31 (4.33) & 0.02**\\
\hspace{1em}Vidéos $\geq 1$ & 0.3 (0.46) & 0.41 (0.49) & 0.1\\
\hspace{1em}Vidéos & 4.71 (12.3) & 6 (12.68) & 0.44\\
\hspace{1em}Exercices $\geq 1$ & 0.61 (0.49) & 0.76 (0.43) & 0.02**\\
\hspace{1em}Exercices & 10.06 (21.4) & 16.03 (34.23) & 0.11\\
\addlinespace[0.3em]
\multicolumn{4}{l}{\textbf{Résultats}}\\
\hspace{1em}Ayant une note aux TD & 0.93 (0.25) & 0.9 (0.3) & 0.39\\
\hspace{1em}Note aux examens & 4.36 (4.83) & 3.99 (5.27) & 0.58\\
\addlinespace[0.3em]
\multicolumn{4}{l}{\textbf{ }}\\
\hspace{1em}\hspace{1em}Observations & 122 & 103 & \\*
\end{longtable}
\end{ThreePartTable}
\endgroup{}

Tous ces éléments nous donnent suffisamment de raison de penser que sur nos deux échantillons d'estimation, nous pouvons considérer que les variables \(z_i\) et \(\psi_i\) ne sont pas corrélées dans l'équation de première étape (équation \ref{eq:g20pe}).

\quad La seconde hypothèse d'exogénéité est la restriction d'exclusion. Selon cette dernière, l'incitation ne doit affecter les résultats en mathématiques que via l'utilisation de la plateforme. La restriction d'exclusion ne serait pas vérifiée si l'incitation motive les étudiants à aller réviser les mathématiques sans passer par la plateforme, par un effet de prise de conscience personnelle, par exemple. Cette hypothèse n'est pas directement testable mais nous avons des éléments pour en discuter. Pour ce faire, nous montrons d'abord les mails envoyés aux incités dans la Figure \ref{fig:g20mails} de l'Annexe \ref{g20mails}. En plus du fait que ces mails ont été espacés de 10 jours chacun, leurs formulations nous laissent difficilement penser qu'ils auraient pu motiver les étudiants à travailler les mathématiques hors de la plateforme. Ensuite, nous pensons que les valeurs manquantes dans les notes de mathématiques peuvent donner une information sur le comportement des étudiants vis à vis des mathématiques. Plus précisément, s'il se trouve que l'incitation affecte la probabilité d'avoir une note aux TD (surtout) ou aux examens, nous pouvons ``craindre'' que cela soit le reflet d'un meilleur comportement des étudiants vis à vis des mathématiques (plus de motivation, typiquement). Après vérification avec des modèles de probabilité linéaire et probit (non reportés), l'incitation n'a clairement pas d'effet sur la probabilité d'avoir une note aux TD ou aux examens. Il est aussi possible que la probabilité d'avoir une note ait été influencée par l'incitation en passant par l'utilisation de la plateforme. Nous pouvons imaginer des étudiants en moyenne qui, après avoir utilisé la plateforme, ont une meilleure attitude vis à vis des mathématiques, notamment en se présentant aux évaluations. Dans ce cas, il ne s'agit pas d'un problème de restriction d'exclusion. Nous avons vérifié que l'estimation par variable instrumentale de l'utilisation de la plateforme sur la probabilité d'avoir une note aux TD ou aux examens et avons trouvé des coefficients non significatifs et largement négligeables.

À notre sens, une autre manière de discuter de la restriction d'exclusion est de regarder l'effet de l'incitation sur les résultats dans d'autres matières. Le Tableau \ref{tab:g20exclrestrmodels} montre les résultats d'estimation de la forme réduite (équation \ref{eq:g20rf} de la section suivante) sur les notes aux UE de gestion et d'économie. Nous voyons que nous pouvons craindre un effet direct de l'incitation sur la note en gestion (0.2 UET, colonne 1) uniquement parmi ceux ayant une note aux TD\footnote{Le coefficient de 8 points de pourcentage sur la probabilité d'avoir la moyenne en gestion est également à considérer même s'il est estimé de manière imprécise.}. Cela est possible si les étudiants incités ont voulu se rattraper dans la matière de gestion uniquement grâce à l'incitation. Toutefois, nous avons déjà justifié que la forme des mails envoyés ne donnait pas de crédit à cette possibilité. En plus, nous montrerons dans les résultats que cet effet de 0.2 UET détecté dans le Tableau \ref{tab:g20exclrestrmodels} est le reflet de l'effet de l'utilisation de la plateforme elle-même, ce qui ne constitue pas une violation de la restriction d'exclusion.

\begin{landscape}\begingroup\fontsize{7}{9}\selectfont

\begin{ThreePartTable}
\begin{TableNotes}
\item \textit{Sources :} Base Parcoursup - L1 AES et Eco-Ge, plateforme '\textit{J'ai 20 en maths}', APOGEE (version extraite en mai 2021), calculs de l'auteur.
\item \textit{Notes :} Moindres carrés ordinaires. Écart-types robustes entre parenthèses. 
    Les variables dépendantes sont des notes normalisées sauf pour les modèles de probabilité linéaire. Une colonne correspond à une régression. ES : Économique et social, S : Scientifique, AES : Administration Économique et Sociale. Eco-Ge : Économie-Gestion. TD : Travaux Dirigés.
\item Significativité : 10\% * 5\% ** 1\% ***.
\end{TableNotes}
\begin{longtable}[t]{lllllllll}
\caption{\label{tab:g20exclrestrmodels}Effets potentiels de l'incitation dans d'autres matières}\\
\toprule
\multicolumn{1}{c}{ } & \multicolumn{4}{c}{Ayant une note aux TD} & \multicolumn{4}{c}{Aux une note aux examens} \\
\cmidrule(l{3pt}r{3pt}){2-5} \cmidrule(l{3pt}r{3pt}){6-9}
\multicolumn{1}{c}{ } & \multicolumn{8}{c}{Variable dépendante : } \\
\cmidrule(l{3pt}r{3pt}){2-9}
  & \makecell{\makecell{Note en gestion \\ \ } \\ (1) } & \makecell{\makecell{Note en gestion \\ $\geq 10$} \\ (2) } & \makecell{\makecell{Note en économie \\ \ } \\ (3) } & \makecell{\makecell{Note en économie \\ $\geq 10$} \\ (4) } & \makecell{\makecell{Note en gestion \\ \ } \\ (5) } & \makecell{\makecell{Note en gestion \\ $\geq 10$} \\ (6) } & \makecell{\makecell{Note en économie \\ \ } \\ (7) } & \makecell{\makecell{Note en économie \\ $\geq 10$} \\ (8) }\\
\midrule
\endfirsthead
\caption[]{\label{tab:g20exclrestrmodels}Effets potentiels de l'incitation dans d'autres matières (suite)}\\
\toprule
  & \makecell{\makecell{Note en gestion \\ \ } \\ (1) } & \makecell{\makecell{Note en gestion \\ $\geq 10$} \\ (2) } & \makecell{\makecell{Note en économie \\ \ } \\ (3) } & \makecell{\makecell{Note en économie \\ $\geq 10$} \\ (4) } & \makecell{\makecell{Note en gestion \\ \ } \\ (5) } & \makecell{\makecell{Note en gestion \\ $\geq 10$} \\ (6) } & \makecell{\makecell{Note en économie \\ \ } \\ (7) } & \makecell{\makecell{Note en économie \\ $\geq 10$} \\ (8) }\\
\midrule
\endhead

\endfoot
\bottomrule
\insertTableNotes
\endlastfoot
Constante & $-$3.678$^{**}$ & $-$2.296$^{***}$ & $-$4.433$^{*}$ & $-$1.006$^{*}$ & $-$3.402$^{**}$ & $-$2.168$^{***}$ & $-$4.286$^{**}$ & $-$0.955$^{*}$\\
 & (1.683) & (0.745) & (2.364) & (0.581) & (1.619) & (0.64) & (2.122) & (0.528)\\
Incitation - Oui & 0.212$^{**}$ & 0.08 & $-$0.014 & $-$0.003 & 0.103 & 0.048 & $-$0.052 & $-$0.005\\
 & (0.107) & (0.053) & (0.145) & (0.04) & (0.102) & (0.051) & (0.134) & (0.037)\\
Note au bac & 0.224$^{***}$ & 0.12$^{***}$ & 0.239$^{***}$ & 0.05$^{***}$ & 0.229$^{***}$ & 0.118$^{***}$ & 0.232$^{***}$ & 0.048$^{***}$\\
 & (0.031) & (0.017) & (0.055) & (0.017) & (0.031) & (0.017) & (0.055) & (0.017)\\
Série au bac - Techno & 0.495$^{**}$ & 0.071 & 0.383$^{**}$ & 0.106$^{**}$ & 0.631$^{***}$ & 0.112 & 0.406$^{**}$ & 0.107$^{***}$\\
 & (0.227) & (0.077) & (0.188) & (0.042) & (0.207) & (0.069) & (0.173) & (0.04)\\
Série au bac - ES & 0.887$^{***}$ & 0.196$^{**}$ & 0.927$^{***}$ & 0.107$^{**}$ & 1.017$^{***}$ & 0.233$^{***}$ & 0.984$^{***}$ & 0.105$^{***}$\\
 & (0.219) & (0.081) & (0.205) & (0.046) & (0.193) & (0.07) & (0.18) & (0.04)\\
Série au bac - S & 1.123$^{***}$ & 0.323$^{***}$ & 1.584$^{***}$ & 0.242$^{***}$ & 1.348$^{***}$ & 0.386$^{***}$ & 1.736$^{***}$ & 0.256$^{***}$\\
 & (0.236) & (0.101) & (0.265) & (0.071) & (0.202) & (0.094) & (0.244) & (0.071)\\
Série au bac - Autre & 0.38 & $-$0.09 & $-$0.156 & 0.03 & 0.448$^{*}$ & $-$0.008 & $-$0.034 & 0.052\\
 & (0.268) & (0.099) & (0.197) & (0.053) & (0.264) & (0.089) & (0.177) & (0.049)\\
Campus - Saint-Denis & 0.18 & 0.031 & $-$0.183 & $-$0.115$^{***}$ & 0.172 & 0.025 & $-$0.199 & $-$0.114$^{***}$\\
 & (0.12) & (0.055) & (0.149) & (0.043) & (0.116) & (0.054) & (0.144) & (0.042)\\
Filière - Eco-Ge & $-$0.284$^{**}$ & $-$0.008 & $-$0.42$^{***}$ & 0.031 & $-$0.351$^{***}$ & $-$0.03 & $-$0.459$^{***}$ & 0.021\\
 & (0.127) & (0.065) & (0.155) & (0.05) & (0.125) & (0.064) & (0.15) & (0.049)\\
Âge à la rentrée & 0.023 & 0.038 & 0.072 & 0.023 & $-$0.001 & 0.032 & 0.062 & 0.022\\
 & (0.083) & (0.036) & (0.115) & (0.026) & (0.078) & (0.03) & (0.101) & (0.023)\\
Sexe - Homme & $-$0.096 & 0.005 & $-$0.037 & $-$0.024 & $-$0.09 & 0.009 & $-$0.035 & $-$0.026\\
 & (0.125) & (0.057) & (0.153) & (0.042) & (0.122) & (0.055) & (0.145) & (0.04)\\
Pays de naissance - En France & 0.071 & 0.227$^{***}$ & 0.127 & 0.048 & 0.151 & 0.228$^{***}$ & 0.225 & 0.04\\
 & (0.152) & (0.053) & (0.19) & (0.037) & (0.152) & (0.055) & (0.18) & (0.039)\\
Statut de boursier - Secondaire & $-$0.088 & $-$0.017 & $-$0.234 & $-$0.033 & $-$0.152 & $-$0.024 & $-$0.238 & $-$0.036\\
 & (0.122) & (0.059) & (0.158) & (0.043) & (0.109) & (0.058) & (0.15) & (0.042)\\
Établissement d'origine - Privé & 0.297 & 0.162 & 0.169 & 0.055 & 0.266 & 0.161 & 0.197 & 0.051\\
 & (0.214) & (0.108) & (0.362) & (0.099) & (0.205) & (0.108) & (0.362) & (0.099)\\
 &  &  &  &  &  &  &  & \\
Observations & 203 & 203 & 205 & 205 & 210 & 210 & 214 & 214\\
R$^2$ ajusté & 0.341 & 0.307 & 0.316 & 0.152 & 0.401 & 0.312 & 0.351 & 0.159\\*
\end{longtable}
\end{ThreePartTable}
\endgroup{}
\end{landscape}

La seconde condition de validité de \(z_i\) est qu'elle doit être corrélée avec \(g_i\). Plus précisément, les incités doivent plus utiliser la plateforme que les non incités. Ici, il est nécessaire de différencier les différentes mesures d'utilisation de la plateforme puisqu'il est raisonnable de penser qu'être incité ne ferait pas réagir tous les individus de la même manière, selon des facteurs tels que les préférences ou le niveau initial en mathématiques : certains iraient voir plus de vidéos alors que d'autres feraient plus d'exercices. Aussi, le temps passé à regarder des vidéos correspond à du temps en moins disponible pour regarder des exercices.

Cette condition est quant à elle directement testable en estimant l'équation \eqref{eq:g20pe} pour chaque mesure de l'utilisation de la plateforme. Les Tableaux \ref{tab:g20pemodelsvenustd} et \ref{tab:g20pemodelsvenusctqcm} montrent le résultat de cet exercice sur ceux qui ont une note aux TD et aux examens, respectivement. Particulièrement, nous pouvons assurément affirmer qu'inciter les étudiants par mail n'a pas d'effet sur le nombre de vidéos regardées au vu du coefficient de 1 devant l'incitation (colonnes 4 des deux tableaux) comparé au coefficient de 6 (colonnes 5 des deux tableaux) qui correspond à l'effet de l'incitation sur le nombre d'exercices vus. Par ailleurs, ce coefficient de 6 n'est que marginalement significatif au seuil de 10\% sur l'échantillon de ceux qui ont une note aux examens. Le seul coefficient significatif\footnote{De manière robuste selon les variables de contrôles utilisées.} est l'effet estimé de l'incitation sur les vidéos visionnées jusqu'à la fin (colonnes 3 des deux tableaux). L'ampleur de ce coefficient est d'un peu moins de 1. Nous n'avons pas d'élément de comparaison dans la littérature pour affirmer objectivement que regarder entièrement une vidéo de plus dans une période de 4 semaines est importante ou non. Cependant, comparé à la très faible moyenne de 0.8 environ (Tableau \ref{tab:g20stats}), cet effet est relativement important. Cela veut dire que même si les étudiants sont globalement très peu intéressés par la plateforme, il est possible d'influer sur l'intensité à laquelle ils l'utilisent. Cela peut être une raison d'espérer que, pour inciter les étudiants à plus utiliser la plateforme, des incitations sur des périodes plus longues seraient efficaces et largement plus intéressants que des investissements sur l'attractivité de la plateforme, évidemment beaucoup plus coûteux.\\
L'effet des vidéos visionnées entièrement est donc celui que nous analyserons en priorité.

\quad Ceci étant dit, l'instrument apparaît faible au vu des statistiques de Fisher associés largement inférieurs à 10 (\protect\hyperlink{ref-STA:STO:97}{Staiger \& Stock, 1997}). La spécification linéaire de l'équation de première étape peut en être la cause puisque les variables de mesure de l'utilisation de la plateforme sont des données de comptage. Nous avons essayé des spécifications potentiellement plus adaptées (régressions de Poisson ou binomiale négative\footnote{La régression binomiale négative est plus appropriée dans le sens où la variance et la moyenne des variables d'utilisation de la plateforme sont largement différentes, ce qui constitue un argument fort en défaveur des régressions de Poisson.}) et trouvons des coefficients de plus forte ampleur tout en étant plus significatifs.\\
Il est à noter que, sous réserve de la validité de l'instrument, la distribution de la variable dépendante de la première étape n'est pas une considération de premier ordre dans l'estimation de l'équation de structurelle par variable instrumentale (\protect\hyperlink{ref-WOO:02}{Wooldridge, 2002}, par exemple).

\quad Les signes et ampleurs des coefficients devant les variables de contrôle sont comparables à ceux des régressions effectuées sur les inscrits non incités discutées plus haut (Tableau \ref{tab:g20ovbmodels} de l'Annexe \ref{g20ovbmodels}).

\begin{landscape}\begingroup\fontsize{8}{10}\selectfont

\begin{ThreePartTable}
\begin{TableNotes}
\item \textit{Sources :} Base Parcoursup - L1 AES et Eco-Ge, plateforme '\textit{J'ai 20 en maths}', APOGEE (version extraite en mai 2021), calculs de l'auteur.
\item \textit{Notes :} Moindres carrés ordinaires. Écart-types robustes entre parenthèses. Une colonne correspond à une régression. ES : Économique et social, S : Scientifique, AES : Administration Économique et Sociale. Eco-Ge : Économie-Gestion.
\item Significativité : 10\% * 5\% ** 1\% ***.
\end{TableNotes}
\begin{longtable}[t]{llllll}
\caption{\label{tab:g20pemodelsvenustd}Résultats de première étape (Ayant une note aux TD)}\\
\toprule
\multicolumn{1}{c}{ } & \multicolumn{5}{c}{Variable dépendante : Nombre de } \\
\cmidrule(l{3pt}r{3pt}){2-6}
  & \makecell{Vidéos complètes et exercices \\ (1) } & \makecell{Vidéos et exercices \\ (2) } & \makecell{Vidéos complètes \\ (3) } & \makecell{Vidéos \\ (4) } & \makecell{Exercices \\ (5) }\\
\midrule
\endfirsthead
\caption[]{\label{tab:g20pemodelsvenustd}Résultats de première étape (Ayant une note aux TD) (suite)}\\
\toprule
  & \makecell{Vidéos complètes et exercices \\ (1) } & \makecell{Vidéos et exercices \\ (2) } & \makecell{Vidéos complètes \\ (3) } & \makecell{Vidéos \\ (4) } & \makecell{Exercices \\ (5) }\\
\midrule
\endhead

\endfoot
\bottomrule
\insertTableNotes
\endlastfoot
Constante & $-$3.34 & $-$11.753 & $-$2.168 & $-$10.581 & $-$1.171\\
 & (56.003) & (72.094) & (5.852) & (23.831) & (53.475)\\
Incitation - Oui & 7.123$^{*}$ & 7.344 & 0.824$^{**}$ & 1.045 & 6.299\\
 & (4.09) & (5.005) & (0.407) & (1.628) & (3.941)\\
Note au bac & 3.575$^{***}$ & 5.137$^{***}$ & 0.231$^{*}$ & 1.793$^{***}$ & 3.344$^{***}$\\
 & (1.133) & (1.449) & (0.137) & (0.542) & (1.101)\\
Série au bac - Techno & 6.45$^{*}$ & 8.378$^{*}$ & 0.634$^{*}$ & 2.561$^{*}$ & 5.816$^{*}$\\
 & (3.386) & (4.342) & (0.376) & (1.461) & (3.153)\\
Série au bac - ES & 9.682$^{**}$ & 15.424$^{***}$ & 0.792 & 6.534$^{***}$ & 8.889$^{**}$\\
 & (3.979) & (5.482) & (0.516) & (2.159) & (3.741)\\
Série au bac - S & 20.315$^{***}$ & 27.722$^{***}$ & $-$0.069 & 7.338$^{***}$ & 20.384$^{***}$\\
 & (7.288) & (8.952) & (0.356) & (2.37) & (7.189)\\
Série au bac - Autre & 4.421 & 6.082 & $-$0.521 & 1.14 & 4.943\\
 & (4.616) & (5.827) & (0.405) & (2.023) & (4.424)\\
Campus - Saint-Denis & 4.151 & 3.409 & $-$0.305 & $-$1.048 & 4.457\\
 & (3.845) & (4.704) & (0.458) & (1.639) & (3.621)\\
Filière - Eco-Ge & $-$3.549 & $-$4.115 & 0.113 & $-$0.453 & $-$3.662\\
 & (4.091) & (5.36) & (0.305) & (1.805) & (4.007)\\
Âge à la rentrée & $-$1.596 & $-$1.758 & $-$0.032 & $-$0.194 & $-$1.563\\
 & (2.266) & (2.897) & (0.239) & (0.951) & (2.14)\\
Sexe - Homme & $-$2.581 & $-$4.166 & $-$0.573$^{**}$ & $-$2.159 & $-$2.008\\
 & (4.24) & (5.171) & (0.27) & (1.397) & (4.146)\\
Pays de naissance - En France & $-$4.929 & $-$10.223 & 0.572$^{*}$ & $-$4.722 & $-$5.5\\
 & (8.147) & (12.321) & (0.299) & (4.856) & (8.039)\\
Statut de boursier - Secondaire & $-$7.406 & $-$9.357 & $-$0.488 & $-$2.438 & $-$6.918\\
 & (5.479) & (6.199) & (0.603) & (1.826) & (5.097)\\
Établissement d'origine - Privé & $-$13.084 & $-$15.179 & $-$0.351 & $-$2.446 & $-$12.732\\
 & (9.15) & (14.302) & (0.566) & (6.356) & (8.782)\\
 &  &  &  &  & \\
Observations & 218 & 218 & 218 & 218 & 218\\
R$^2$ ajusté & 0.073 & 0.098 & 0.012 & 0.09 & 0.075\\
F & 3.136 & 2.191 & 3.685 & 0.385 & 2.713\\*
\end{longtable}
\end{ThreePartTable}
\endgroup{}
\end{landscape}

\begin{landscape}\begingroup\fontsize{8}{10}\selectfont

\begin{ThreePartTable}
\begin{TableNotes}
\item \textit{Sources :} Base Parcoursup - L1 AES et Eco-Ge, plateforme '\textit{J'ai 20 en maths}', APOGEE (version extraite en mai 2021), calculs de l'auteur.
\item \textit{Notes :} Moindres carrés ordinaires. Écart-types robustes entre parenthèses. Une colonne correspond à une régression. ES : Économique et social, S : Scientifique, AES : Administration Économique et Sociale. Eco-Ge : Économie-Gestion.
\item Significativité : 10\% * 5\% ** 1\% ***.
\end{TableNotes}
\begin{longtable}[t]{llllll}
\caption{\label{tab:g20pemodelsvenusctqcm}Résultats de première étape (Ayant une note aux examens)}\\
\toprule
\multicolumn{1}{c}{ } & \multicolumn{5}{c}{Variable dépendante : Nombre de } \\
\cmidrule(l{3pt}r{3pt}){2-6}
  & \makecell{Vidéos complètes et exercices \\ (1) } & \makecell{Vidéos et exercices \\ (2) } & \makecell{Vidéos complètes \\ (3) } & \makecell{Vidéos \\ (4) } & \makecell{Exercices \\ (5) }\\
\midrule
\endfirsthead
\caption[]{\label{tab:g20pemodelsvenusctqcm}Résultats de première étape (Ayant une note aux examens) (suite)}\\
\toprule
  & \makecell{Vidéos complètes et exercices \\ (1) } & \makecell{Vidéos et exercices \\ (2) } & \makecell{Vidéos complètes \\ (3) } & \makecell{Vidéos \\ (4) } & \makecell{Exercices \\ (5) }\\
\midrule
\endhead

\endfoot
\bottomrule
\insertTableNotes
\endlastfoot
Constante & $-$1.148 & $-$7.165 & $-$0.112 & $-$6.13 & $-$1.036\\
 & (52.521) & (67.26) & (5.52) & (22.086) & (50.177)\\
Incitation - Oui & 7.214$^{*}$ & 7.832$^{*}$ & 0.923$^{**}$ & 1.542 & 6.29$^{*}$\\
 & (3.854) & (4.664) & (0.409) & (1.535) & (3.678)\\
Note au bac & 3.402$^{***}$ & 4.889$^{***}$ & 0.196 & 1.683$^{***}$ & 3.206$^{***}$\\
 & (1.148) & (1.471) & (0.137) & (0.545) & (1.112)\\
Série au bac - Techno & 8.381$^{**}$ & 10.697$^{**}$ & 0.91$^{**}$ & 3.226$^{**}$ & 7.471$^{**}$\\
 & (3.71) & (4.497) & (0.452) & (1.353) & (3.383)\\
Série au bac - ES & 11.144$^{***}$ & 17.088$^{***}$ & 1.012$^{**}$ & 6.956$^{***}$ & 10.132$^{***}$\\
 & (3.464) & (4.822) & (0.474) & (1.966) & (3.277)\\
Série au bac - S & 21.765$^{***}$ & 28.779$^{***}$ & 0.08 & 7.094$^{***}$ & 21.685$^{***}$\\
 & (7.156) & (8.771) & (0.314) & (2.229) & (7.078)\\
Série au bac - Autre & 8.256$^{*}$ & 10.607$^{*}$ & 0.002 & 2.353 & 8.254$^{**}$\\
 & (4.423) & (5.588) & (0.539) & (1.982) & (4.095)\\
Campus - Saint-Denis & 3.393 & 2.524 & $-$0.453 & $-$1.322 & 3.846\\
 & (3.566) & (4.418) & (0.435) & (1.563) & (3.375)\\
Filière - Eco-Ge & $-$3.564 & $-$3.767 & 0.21 & 0.008 & $-$3.775\\
 & (3.921) & (5.084) & (0.321) & (1.694) & (3.838)\\
Âge à la rentrée & $-$1.651 & $-$1.889 & $-$0.112 & $-$0.351 & $-$1.539\\
 & (2.035) & (2.569) & (0.224) & (0.838) & (1.918)\\
Sexe - Homme & $-$2.704 & $-$4.258 & $-$0.68$^{**}$ & $-$2.234 & $-$2.023\\
 & (4.155) & (5.069) & (0.288) & (1.382) & (4.054)\\
Pays de naissance - En France & $-$4.535 & $-$10.109 & 0.432 & $-$5.142 & $-$4.967\\
 & (8.218) & (12.497) & (0.274) & (4.941) & (8.126)\\
Statut de boursier - Secondaire & $-$7.957 & $-$10.109 & $-$0.688 & $-$2.839 & $-$7.27\\
 & (5.463) & (6.14) & (0.606) & (1.798) & (5.066)\\
Établissement d'origine - Privé & $-$13.144 & $-$15.321 & $-$0.553 & $-$2.729 & $-$12.592\\
 & (9.247) & (14.436) & (0.594) & (6.426) & (8.863)\\
 &  &  &  &  & \\
Observations & 223 & 223 & 223 & 223 & 223\\
R$^2$ ajusté & 0.077 & 0.1 & 0.028 & 0.095 & 0.078\\
F & 3.455 & 2.679 & 4.91 & 0.905 & 2.905\\*
\end{longtable}
\end{ThreePartTable}
\endgroup{}
\end{landscape}

Concernant l'interprétation précise du paramètre \(\sigma_1\), nous renvoyons à la Section \ref{agemethodscfh} du Chapitre \ref{age}. Dans le présent chapitre, il est \emph{a priori} improbable que l'hypothèse de monotonie soit violée. Nous imaginons mal pourquoi des étudiants utiliseraient moins la plateforme lorsqu'ils sont incités et l'utiliseraient plus lorsqu'ils ne le sont pas. De plus, la forme et le rythme d'envoi des mails d'incitation (Figure \ref{g20mails} de l'Annexe \ref{g20mails}) ne sont pas de nature à ``harceler'' les étudiants, ce qui serait de nature à favoriser un comportement contraire à l'hypothèse de monotonie.

La nature binaire de notre instrument nous permet de vérifier une implication empirique de l'hypothèse de monotonie : si elle est vérifiée, les fonctions de répartition empiriques des variables d'utilisation de la plateforme pour les deux valeurs de l'instrument ne devraient pas se couper (\protect\hyperlink{ref-ANG:eal:96}{Angrist et al., 1996, p. 435}). La Figure \ref{fig:g20ecdfz} montre ces fonctions de répartition empiriques. La seule variable de la plateforme pour laquelle nous pouvons craindre une violation de l'hypothèse de monotonie est le nombre de vidéos visionnées. Cela est très probablement dû au sous-échantillon des économistes du Tampon dans lequel les étudiants non incités ont très significativement regardé plus de vidéos et d'exercices que les étudiants incités (Tableau \ref{tab:g20cfztreat} de l'Annexe \ref{g20cfztreat}). Nous n'avons pas d'explication à ce phénomène. Une hypothèse plutôt crédible serait que les étudiants incités ont regardé moins de vidéos puisqu'ils ont pris le temps de regarder les vidéos jusqu'à la fin tandis que les non incités n'ont fait que survoler les vidéos. Le même phénomène s'est probablement produit pour les exercices. Les indicatrices de campus et de filière incluses dans les prochaines régressions devraient contribuer à limiter d'éventuels biais dus à ce sous-échantillon suggérant la présence d'individus se comportant à contre-sens du comportement attendu.

Au Tampon comme à Saint-Denis, ce problème n'est pas constaté lorsque nous nous focalisons sur les vidéos visionnées entièrement, ce qui constitue une raison de plus pour concentrer notre analyse sur cette variable.

\begin{figure}[H]

{\centering \includegraphics[width=1\linewidth]{000_files/figure-latex/g20ecdfz-1} 

}

\caption{Fonctions de distribution empirique des variables d'utilisation de la plateforme en fonction de l'incitation (inscrits)}\label{fig:g20ecdfz}
\end{figure}

Pour rappel, sous l'hypothèse de monotonie\footnote{En cohérence avec le Chapitre \ref{age}, nous supposons implicitement que l'effet de l'utilisation de la plateforme n'est pas le même pour tous les individus. Vu que le service proposé ne s'adapte pas au niveau de l'individu, nous avons de bonnes raisons de penser que son effet varie d'un individu à l'autre, notamment en fonction du niveau de ce dernier.}, \(\sigma_1\) correspond à une moyenne pondérée de l'effet de l'utilisation de la plateforme pour les individus ayant réagi à l'incitation.

\quad Pour finir, il nous faut considérer les formes des distributions des différentes notes des néo-bacheliers participants présentées dans la Figure \ref{fig:g20notes} de l'Annexe \ref{g20age26aoutnotescompl}. Le fait le plus remarquable est le pic de zéros aux examens de mathématiques\footnote{Et par extension, de l'UE de mathématiques.} et d'économie. Ces constats sont cohérents avec le faible niveau des étudiants (Tableau \ref{tab:g20stats}). Plus précisément, ces pics de zéros reflètent le fait que les examens, présentant naturellement un minimum de difficulté, n'ont pas pu entièrement capter la distribution du niveau en mathématiques et en économie des néo-bacheliers. Autrement dit, beaucoup parmi ces derniers ont un niveau trop faible pour obtenir le moindre point\footnote{Ceux qui ont eu zéro point représentent environ 32\% pour les mathématiques et 26\% pour l'économie.}.
Ce phénomène ne semble pas apparaître en ce qui concerne les notes aux TD de mathématiques et les notes en gestion. Il y a en plus un pic de notes maximale (20 sur 20) aux TD de mathématiques. Ces deux constats ne sont pas étonnants puisque les évaluations de TD de mathématiques sont censées être plus faciles que les examens et de brefs entretiens avec l'enseignant de gestion nous confirment qu'il est relativement plus facile d'avoir de bonnes notes en gestion (révision des examens des années passées, mini-évaluations régulières, etc.).

Les notes aux TD de mathématiques, aux examens de mathématiques et en économie sont alors censurées et nous mobilisons des modèles Tobit\footnote{Wooldridge (\protect\hyperlink{ref-WOO:02}{2002, p. 667}) et Henningsen (\protect\hyperlink{ref-HEN:10}{2010, p. 2}), par exemple.} pour prendre cette situation en compte. Les premières sont censurées à droite et les deux dernières à gauche.

La prise en compte des censures ne constitue pas l'essence de notre stratégie d'identification et nos conclusions sont qualitativement proches de celles issues des modèles linéaires.

\hypertarget{g20res}{%
\section{Résultats et discussions}\label{g20res}}

\hypertarget{g20resprinc}{%
\subsection{Résultats principaux}\label{g20resprinc}}

Avant la présentation des résultats d'estimation de l'équation structurelle (équation \ref{eq:g20ols}), nous proposons les résultats d'estimation de l'intention de traiter. Ils correspondent à l'estimation de l'équation :

\begin{equation}
\label{eq:g20rf}
y_i^{l1} = \gamma_0 + \gamma_1 z_i + x'_i \gamma_2 + \zeta_i, 
\end{equation}

le coefficient \(\gamma_1\) étant le paramètre d'intérêt. Il désigne l'effet d'être incité sur les notes, indépendamment des mécanismes sous-jacents.

\quad Estimer cette forme réduite (\protect\hyperlink{ref-ANG:PIS:08}{Angrist \& Pischke, 2009}) est intéressant à plusieurs titres. Le premier est un intérêt en termes de politiques publiques car envoyer des mails d'incitation à des étudiants est en pratique très peu coûteux. S'il s'avère que le simple fait d'inciter les étudiants à aller sur la plateforme suffit à avoir une augmentation significative de performance, ce résultat, s'il est suffisamment robuste à divers contextes, suffit à justifier des politiques basées uniquement sur des incitations\footnote{Si l'incitation possédait un effet direct sur la motivation ou sur l'attrait des étudiants pour les mathématiques, cela rendrait la condition de la restriction d'exclusion très peu crédible et nous n'aurions pas pu étudier l'effet de l'utilisation de la plateforme. Mais cela aurait été intéressant pour ceux qui veulent aider les étudiants de constater que de simples incitations fonctionnent, d'une manière ou d'une autre.}.\\
Le deuxième intérêt est que l'estimation de l'équation \eqref{eq:g20rf} se repose sur des hypothèses moins fortes puisque l'incitation est par construction exogène et une simple estimation par moindres carrés ordinaires de cette forme réduite suffit à produire des connaissances relativement fiables aussi bien sur l'effet de l'incitation que sur ceux des variables de contrôle.\\
Le troisième intérêt est que cette équation peut nous aider à anticiper l'existence ou non de l'effet d'intérêt (celui de l'utilisation de la plateforme) ou non puisque si aucun effet significatif n'est capté par cette forme réduite, alors il n'existe probablement pas d'effet causal de l'utilisation de la plateforme sur la note en mathématiques.

\quad Les résultats d'estimation de l'équation \eqref{eq:g20rf} et de ceux des modèles Tobit correspondant sont montrés dans le Tableau \ref{tab:g20rfmodels}. Même si les coefficients ne sont pas directement comparables entre les modèles linéaires et les modèles Tobit, les variables dépendantes de ce tableau sont des notes en points sur 20, à part celle des modèles de probabilité linéaire (colonnes 3, 6 et 9). Nous constatons que ces coefficients soutiennent globalement les mêmes propos dans ce tableau ou dans les résultats montrés plus bas.

D'un côté, l'incitation apparaît comme ayant un effet très fort et significatif sur les notes aux TD, d'un peu moins de 2 points sur 20 (colonnes 1 et 2). D'un autre côté, bien que nous trouvions un effet d'ampleur considérable de 6 points de pourcentage sur la probabilité d'avoir la moyenne en TD (colonne 3), le coefficient n'est pas significatif. Cela suggère que l'effet de l'incitation s'est manifesté pour des individus avec des notes aux TD éloignées de la moyenne (soit très faibles soit très forts). L'incitation n'a pas d'effet statistiquement significatif ou d'ampleur marquante sur la note aux examens de mathématiques ou sur la probabilité d'avoir la moyenne (colonnes 4 à 6).

Il est toujours intéressant de montrer les effets sur la note à l'UE puisque c'est cette dernière qui détermine si l'étudiant a validé la matière ou non. En tout, l'effet positif et de forte ampleur sur les notes aux TD n'aura pas pu être reflété sur les résultats à l'UE de mathématiques (colonnes 7 à 9).

\quad Les coefficients, significatifs ou non, sur les variables de contrôle sont assez cohérents. Par exemple, un point de plus au bac procure environ 1.5 points aux TD de mathématiques, à séries du bac égales. De la même manière, être titulaire d'un bac ES ou S constitue un très fort prédicteur des résultats aux TD de mathématiques. Excepté le fait que choisir la filière Eco-Ge est fortement positivement corrélé avec les résultats aux examens, les autres variables de contrôle ne sont ni statistiquement ni économiquement significatives de manière robuste.\\
De manière annexe, même s'il ne s'agit que d'estimations par moindre carrés ordinaires, est intéressant de constater la très faible ampleur de la corrélation entre l'âge et les notes de mathématiques en première année d'université. Cela fait suite à nos propos du Chapitre \ref{age} sur le fait que les effets de l'âge se dissipent avec le temps.

Le Tableau \ref{tab:g20rfmodelsnormpop} de l'Annexe \ref{g20rfmodelsnormpop} montre les résultats de régression homologues lorsque les notes sont normalisées par filière et par campus qui aboutissent aux mêmes conclusions.

\quad En tout, nous pouvons \emph{a priori} nous attendre à trouver un effet fort de l'utilisation de la plateforme sur les notes aux TD et au plus un effet significatif de faible ampleur sur les notes aux examens.

\begin{landscape}\begingroup\fontsize{7}{9}\selectfont

\begin{ThreePartTable}
\begin{TableNotes}
\item \textit{Sources :} Base Parcoursup - L1 AES et Eco-Ge, plateforme '\textit{J'ai 20 en maths}', APOGEE (version extraite en mai 2021), calculs de l'auteur.
\item \textit{Notes :} Écart-types robustes entre parenthèses. 
    Les variables dépendantes sont des notes sur 20 (pour comparer les modèles linéaires et Tobit) sauf pour les modèles de probabilité linéaire. Dans les modèles Tobit, les notes aux TD sont censurées à gauche et à droite et les autres notes à gauche uniquement. Une colonne correspond à une régression. ES : Économique et social, S : Scientifique, AES : Administration Économique et Sociale. Eco-Ge : Économie-Gestion. TD : Travaux Dirigés. UE : Unité d'Enseignement.
\item Significativité : 10\% * 5\% ** 1\% ***.
\end{TableNotes}
\begin{longtable}[t]{llllllllll}
\caption{\label{tab:g20rfmodels}Effets de l'incitation sur les notes de mathématiques (intention de traiter)}\\
\toprule
\multicolumn{1}{c}{ } & \multicolumn{9}{c}{Variable dépendante : Note } \\
\cmidrule(l{3pt}r{3pt}){2-10}
\multicolumn{1}{c}{ } & \multicolumn{2}{c}{\makecell{Aux TD \\  }} & \multicolumn{1}{c}{\makecell{Aux TD \\ $\geq 10$}} & \multicolumn{2}{c}{\makecell{Aux examens \\  }} & \multicolumn{1}{c}{\makecell{Aux examens \\ $\geq 10$}} & \multicolumn{2}{c}{\makecell{À l'UE \\  }} & \multicolumn{1}{c}{\makecell{À l'UE \\ $\geq 10$}} \\
\cmidrule(l{3pt}r{3pt}){2-3} \cmidrule(l{3pt}r{3pt}){4-4} \cmidrule(l{3pt}r{3pt}){5-6} \cmidrule(l{3pt}r{3pt}){7-7} \cmidrule(l{3pt}r{3pt}){8-9} \cmidrule(l{3pt}r{3pt}){10-10}
  & \makecell{MCO \\ (1) } & \makecell{Tobit \\ (2) } & \makecell{MCO \\ (3) } & \makecell{MCO \\ (4) } & \makecell{Tobit \\ (5) } & \makecell{MCO \\ (6) } & \makecell{MCO \\ (7) } & \makecell{Tobit \\ (8) } & \makecell{MCO \\ (9) }\\
\midrule
\endfirsthead
\caption[]{\label{tab:g20rfmodels}Effets de l'incitation sur les notes de mathématiques (intention de traiter) (suite)}\\
\toprule
  & \makecell{MCO \\ (1) } & \makecell{Tobit \\ (2) } & \makecell{MCO \\ (3) } & \makecell{MCO \\ (4) } & \makecell{Tobit \\ (5) } & \makecell{MCO \\ (6) } & \makecell{MCO \\ (7) } & \makecell{Tobit \\ (8) } & \makecell{MCO \\ (9) }\\
\midrule
\endhead

\endfoot
\bottomrule
\insertTableNotes
\endlastfoot
Constante & $-$15.902 & $-$19.015 & $-$0.61 & $-$15.79$^{***}$ & $-$21.221$^{**}$ & $-$1.802$^{***}$ & $-$14.136$^{**}$ & $-$16.796$^{**}$ & $-$1.701$^{***}$\\
 & (10.715) & (12.565) & (0.876) & (5.879) & (9.289) & (0.538) & (6.106) & (7.993) & (0.501)\\
Incitation - Oui & 1.853$^{***}$ & 2.098$^{***}$ & 0.059 & 0.072 & $-$0.145 & 0.042 & 0.554 & 0.602 & 0.018\\
 & (0.636) & (0.769) & (0.054) & (0.433) & (0.57) & (0.039) & (0.433) & (0.504) & (0.037)\\
Note au bac & 1.429$^{***}$ & 1.696$^{***}$ & 0.086$^{***}$ & 1.176$^{***}$ & 1.42$^{***}$ & 0.085$^{***}$ & 1.3$^{***}$ & 1.414$^{***}$ & 0.099$^{***}$\\
 & (0.224) & (0.257) & (0.018) & (0.178) & (0.193) & (0.017) & (0.168) & (0.174) & (0.015)\\
Série au bac - Techno & 3.162$^{***}$ & 3.215$^{**}$ & 0.205$^{***}$ & 2.017$^{***}$ & 4.958$^{***}$ & 0.096$^{***}$ & 2.648$^{***}$ & 3.692$^{***}$ & 0.099$^{***}$\\
 & (0.819) & (1.441) & (0.06) & (0.493) & (1.337) & (0.035) & (0.499) & (0.89) & (0.032)\\
Série au bac - ES & 6.672$^{***}$ & 7.16$^{***}$ & 0.446$^{***}$ & 4.398$^{***}$ & 8.563$^{***}$ & 0.161$^{***}$ & 5.26$^{***}$ & 6.639$^{***}$ & 0.232$^{***}$\\
 & (0.793) & (1.396) & (0.065) & (0.507) & (1.274) & (0.043) & (0.507) & (0.849) & (0.045)\\
Série au bac - S & 11.383$^{***}$ & 12.952$^{***}$ & 0.733$^{***}$ & 9.38$^{***}$ & 13.767$^{***}$ & 0.53$^{***}$ & 10.306$^{***}$ & 11.765$^{***}$ & 0.56$^{***}$\\
 & (1.059) & (1.606) & (0.077) & (0.832) & (1.382) & (0.08) & (0.861) & (1.002) & (0.073)\\
Série au bac - Autre & 4.739$^{*}$ & 4.896$^{**}$ & 0.339$^{*}$ & 1.969$^{***}$ & 5.343$^{***}$ & 0.043 & 3.302$^{***}$ & 4.335$^{***}$ & 0.078\\
 & (2.422) & (2.445) & (0.177) & (0.706) & (1.933) & (0.065) & (1.042) & (1.553) & (0.068)\\
Campus - Saint-Denis & $-$0.965 & $-$1.377$^{*}$ & $-$0.042 & $-$0.188 & 0.6 & $-$0.055 & $-$0.667 & $-$1.022$^{**}$ & $-$0.073$^{*}$\\
 & (0.651) & (0.784) & (0.055) & (0.433) & (0.601) & (0.039) & (0.435) & (0.518) & (0.039)\\
Filière - Eco-Ge & 0.875 & 1.382 & 0.024 & 1.67$^{***}$ & 2.172$^{***}$ & 0.112$^{**}$ & 0.835 & 0.929 & 0.081$^{*}$\\
 & (0.74) & (0.903) & (0.059) & (0.544) & (0.663) & (0.05) & (0.546) & (0.593) & (0.046)\\
Âge à la rentrée & 0.01 & $-$0.031 & $-$0.013 & 0.113 & $-$0.006 & 0.042$^{*}$ & $-$0.069 & $-$0.08 & 0.026\\
 & (0.494) & (0.598) & (0.041) & (0.247) & (0.438) & (0.023) & (0.272) & (0.378) & (0.022)\\
Sexe - Homme & 0.705 & 1.1 & 0.069 & 0.13 & $-$0.159 & $-$0.003 & 0.574 & 0.713 & 0.041\\
 & (0.734) & (0.849) & (0.06) & (0.487) & (0.641) & (0.045) & (0.476) & (0.56) & (0.042)\\
Pays de naissance - En France & $-$0.276 & $-$0.326 & $-$0.187$^{*}$ & $-$0.708 & $-$1.047 & $-$0.009 & $-$0.067 & $-$0.21 & $-$0.008\\
 & (1.21) & (1.371) & (0.099) & (0.626) & (1.086) & (0.072) & (0.74) & (0.89) & (0.061)\\
Statut de boursier - Secondaire & $-$0.174 & $-$0.075 & $-$0.067 & $-$0.128 & $-$0.212 & $-$0.015 & 0.237 & 0.407 & 0.021\\
 & (0.764) & (0.847) & (0.066) & (0.481) & (0.639) & (0.046) & (0.478) & (0.551) & (0.041)\\
Établissement d'origine - Privé & $-$0.487 & $-$0.087 & $-$0.156 & 0.598 & 0.194 & 0.181 & 1.02 & 1.175 & 0.045\\
 & (1.527) & (1.88) & (0.134) & (1.082) & (1.334) & (0.119) & (1.173) & (1.279) & (0.125)\\
 &  &  &  &  &  &  &  &  & \\
Observations & 218 & 218 & 218 & 223 & 223 & 223 & 258 & 258 & 258\\*
\end{longtable}
\end{ThreePartTable}
\endgroup{}
\end{landscape}

Venons-en désormais aux résultats principaux de notre étude. Les résultats d'estimation de l'équation structurelle (équation \ref{eq:g20ols}) par variable instrumentale en utilisant l'équation \eqref{eq:g20pe} comme première étape sont présentés par le Tableau \ref{tab:g20models}. Chaque panel correspond à une mesure de l'utilisation de la plateforme. Nous montrons en plus les résultats d'estimation par moindres carrés ordinaires (colonnes 1, 5 et 9). Que la variable dépendante soit les notes aux TD ou aux examens, les coefficients correspondant à ces derniers sont globalement positifs, significatifs et de faible ampleur (allant d'environ 0.02 à 0.08 points sur 20) sauf lorsque nous utilisons le nombre de vidéos visionnées entièrement comme mesure d'utilisation de l'intensité de la plateforme (panel C).
Le coefficient des moindres carrés ordinaires lorsque nous considérons les vidéos visionnées entièrement est de 0.3 point, de 0.135 point (significatifs au seuil de 5\%) et de 0.11 point (non significatif) lorsque la variable dépendante est la note au TD, la note aux examens et la note à l'UE, respectivement.

Comme nous l'avons justifié dans la Section \ref{g20methods}, ces coefficients sont vraisemblablement surestimés. Nous avons alors de bonnes raisons de penser que la plateforme, dans les conditions dans lesquelles elle a été mise à disposition des participants, n'a réellement d'effet perceptible que sur les notes aux TD. Cet effet serait d'une ampleur modeste de moins de 0.3 point. Un vrai effet de moins de 0.135 point (note aux examens) ne semble pas intéressant, dans l'absolu. Puisque l'intervention a été de courte durée, nous imaginons mal un effet négatif de la plateforme, via un effet de perte de temps des participants, par exemple.

Les colonnes restantes du Tableau \ref{tab:g20models} montrent les résultats d'estimation par variables instrumentales des modèles linéaires (colonnes 2,4, 6, 8, 10 et 12) et non linéaires (Tobit, colonnes 3, 7 et 11). Nous observons un phénomène inattendu : lorsque la variable dépendante est la note aux TD, l'ampleur des coefficients estimés par variable instrumentale dans les modèles linéaires est systématiquement supérieure à leurs homologues estimés par moindres carrés ordinaires. Plus particulièrement, une vidéo entièrement visionnée en plus procurerait un avantage d'un peu plus de 2 points sur 20 aux TD\footnote{Significatif à 10\% dans le modèle linéaire et à 5\% dans le modèle Tobit.}. Ce constat est robuste à plusieurs considérations (voir Section \ref{g20ressupp}). Il est également maintenu dans les modèles Tobit lorsque nous comparons les coefficients estimés avec ou sans variable instrumentale\footnote{Les résultats avec les modèles Tobit sans variable instrumentale ne sont pas reportés pour rendre le Tableau \ref{tab:g20models} plus lisible tout en mettant côte à côte les résultats d'estimation par variable instrumentale issus des modèles linéaires ou Tobit (colonnes 2 et 3, 6 et 7, et 10 et 11).}. Nous proposons par la suite des pistes d'explication à ce phénomène. Ces explications constituent une contribution supplémentaire de notre étude puisque nous n'avons pas trouvé de propositions similaires dans la littérature présentée en Section \ref{g20litt}.

\begin{landscape}\begingroup\fontsize{6}{8}\selectfont

\begin{ThreePartTable}
\begin{TableNotes}
\item \textit{Sources :} Base Parcoursup - L1 AES et Eco-Ge, plateforme '\textit{J'ai 20 en maths}', APOGEE (version extraite en mai 2021), calculs de l'auteur.
\item \textit{Notes :} Écart-types robustes entre parenthèses. 
    Les variables dépendantes sont des notes sur 20 (pour comparer les modèles linéaires et Tobit) sauf pour les modèles de probabilité linéaire. Dans les modèles Tobit, les notes aux TD sont censurées à gauche et à droite et les autres notes à gauche uniquement. Une colonne et un panel donnés correspondent à une régression. Les variables liées à la plateforme sont exprimées en comptages. ES : Économique et social, S : Scientifique, AES : Administration Économique et Sociale. Eco-Ge : Économie-Gestion. TD : Travaux Dirigés. UE : Unité d'Enseignement.
\item Significativité : 10\% * 5\% ** 1\% ***.
\end{TableNotes}
\begin{longtable}[t]{lllllllllllll}
\caption{\label{tab:g20models}Effets de l'utilisation de la plateforme sur les notes de mathématiques}\\
\toprule
\multicolumn{1}{c}{ } & \multicolumn{12}{c}{Variable dépendante : Note } \\
\cmidrule(l{3pt}r{3pt}){2-13}
\multicolumn{1}{c}{ } & \multicolumn{3}{c}{\makecell{Aux TD \\ \ }} & \multicolumn{1}{c}{\makecell{Aux TD \\ $\geq 10$}} & \multicolumn{3}{c}{\makecell{Aux examens \\ \ }} & \multicolumn{1}{c}{\makecell{Aux examens \\ $\geq 10$}} & \multicolumn{3}{c}{\makecell{À l'UE \\ \ }} & \multicolumn{1}{c}{\makecell{À l'UE \\ $\geq 10$}} \\
\cmidrule(l{3pt}r{3pt}){2-4} \cmidrule(l{3pt}r{3pt}){5-5} \cmidrule(l{3pt}r{3pt}){6-8} \cmidrule(l{3pt}r{3pt}){9-9} \cmidrule(l{3pt}r{3pt}){10-12} \cmidrule(l{3pt}r{3pt}){13-13}
  & \makecell{MCO \\ (1) } & \makecell{VI \\ (2) } & \makecell{VI-Tobit \\ (3) } & \makecell{VI \\ (4) } & \makecell{MCO \\ (5) } & \makecell{VI \\ (6) } & \makecell{VI-Tobit \\ (7) } & \makecell{VI \\ (8) } & \makecell{MCO \\ (9) } & \makecell{VI \\ (10) } & \makecell{VI-Tobit \\ (11) } & \makecell{VI \\ (12) }\\
\midrule
\endfirsthead
\caption[]{\label{tab:g20models}Effets de l'utilisation de la plateforme sur les notes de mathématiques (suite)}\\
\toprule
  & \makecell{MCO \\ (1) } & \makecell{VI \\ (2) } & \makecell{VI-Tobit \\ (3) } & \makecell{VI \\ (4) } & \makecell{MCO \\ (5) } & \makecell{VI \\ (6) } & \makecell{VI-Tobit \\ (7) } & \makecell{VI \\ (8) } & \makecell{MCO \\ (9) } & \makecell{VI \\ (10) } & \makecell{VI-Tobit \\ (11) } & \makecell{VI \\ (12) }\\
\midrule
\endhead

\endfoot
\bottomrule
\insertTableNotes
\endlastfoot
\addlinespace[0.3em]
\multicolumn{13}{l}{\textbf{Panel A : Vidéos (complètes) et exercices}}\\
\hline
\hspace{1em}Vidéos (complètes) et exercices & 0.047$^{***}$ & 0.26$^{*}$ & $-$0.057 & 0.008 & 0.009 & 0.01 & $-$0.02 & 0.006 & 0.025$^{***}$ & 0.074 & 0.08 & 0.002\\
\hspace{1em} & (0.012) & (0.154) & (0.104) & (0.008) & (0.008) & (0.06) & (0.079) & (0.006) & (0.008) & (0.061) & (0.066) & (0.005)\\
\hspace{1em} &  &  &  &  &  &  &  &  &  &  &  \vphantom{8} & \\
\hspace{1em}Contrôles & Oui & Oui & Oui & Oui & Oui & Oui & Oui & Oui & Oui & Oui & Oui & \vphantom{4} Oui\\
\hspace{1em}Observations & 218 & 218 & 218 & 218 & 223 & 223 & 223 & 223 & 258 & 258 & 258 & \vphantom{4} 258\\
\hspace{1em} &  &  &  &  &  &  &  &  &  &  &  \vphantom{7} & \\
\addlinespace[0.3em]
\multicolumn{13}{l}{\textbf{Panel B : Vidéos et exercices}}\\
\hline
\hspace{1em}Vidéos et exercices & 0.038$^{***}$ & 0.252 & 0.074 & 0.008 & 0.012$^{*}$ & 0.009 & $-$0.018 & 0.005 & 0.022$^{***}$ & 0.069 & 0.074 & 0.002\\
\hspace{1em} & (0.009) & (0.172) & (0.101) & (0.008) & (0.007) & (0.055) & (0.072) & (0.006) & (0.007) & (0.058) & (0.061) & (0.004)\\
\hspace{1em} &  &  &  &  &  &  &  &  &  &  &  \vphantom{6} & \\
\hspace{1em}Contrôles & Oui & Oui & Oui & Oui & Oui & Oui & Oui & Oui & Oui & Oui & Oui & \vphantom{3} Oui\\
\hspace{1em}Observations & 218 & 218 & 218 & 218 & 223 & 223 & 223 & 223 & 258 & 258 & 258 & \vphantom{3} 258\\
\hspace{1em} &  &  &  &  &  &  &  &  &  &  &  \vphantom{5} & \\
\addlinespace[0.3em]
\multicolumn{13}{l}{\textbf{Panel C : Vidéos (complètes)}}\\
\hline
\hspace{1em}Vidéos (complètes) & 0.309$^{**}$ & 2.247$^{*}$ & 2.255$^{**}$ & 0.071 & 0.135$^{**}$ & 0.078 & $-$0.169 & 0.046 & 0.11 & 0.538 & 0.583 & 0.017\\
\hspace{1em} & (0.133) & (1.249) & (0.918) & (0.067) & (0.061) & (0.464) & (0.612) & (0.044) & (0.099) & (0.457) & (0.487) & (0.036)\\
\hspace{1em} &  &  &  &  &  &  &  &  &  &  &  \vphantom{4} & \\
\hspace{1em}Contrôles & Oui & Oui & Oui & Oui & Oui & Oui & Oui & Oui & Oui & Oui & Oui & \vphantom{2} Oui\\
\hspace{1em}Observations & 218 & 218 & 218 & 218 & 223 & 223 & 223 & 223 & 258 & 258 & 258 & \vphantom{2} 258\\
\hspace{1em} &  &  &  &  &  &  &  &  &  &  &  \vphantom{3} & \\
\addlinespace[0.3em]
\multicolumn{13}{l}{\textbf{Panel D : Vidéos}}\\
\hline
\hspace{1em}Vidéos & 0.081$^{***}$ & 1.773 & 1.743$^{**}$ & 0.056 & 0.059$^{**}$ & 0.047 & $-$0.088 & 0.028 & 0.056$^{**}$ & 0.344 & 0.375 & 0.011\\
\hspace{1em} & (0.026) & (2.71) & (0.72) & (0.092) & (0.023) & (0.271) & (0.361) & (0.032) & (0.022) & (0.367) & (0.308) & (0.022)\\
\hspace{1em} &  &  &  &  &  &  &  &  &  &  &  \vphantom{2} & \\
\hspace{1em}Contrôles & Oui & Oui & Oui & Oui & Oui & Oui & Oui & Oui & Oui & Oui & Oui & \vphantom{1} Oui\\
\hspace{1em}Observations & 218 & 218 & 218 & 218 & 223 & 223 & 223 & 223 & 258 & 258 & 258 & \vphantom{1} 258\\
\hspace{1em} &  &  &  &  &  &  &  &  &  &  &  \vphantom{1} & \\
\addlinespace[0.3em]
\multicolumn{13}{l}{\textbf{Panel E : Exercices}}\\
\hline
\hspace{1em}Exercices & 0.048$^{***}$ & 0.294 & 0.152 & 0.009 & 0.008 & 0.011 & $-$0.023 & 0.007 & 0.026$^{***}$ & 0.086 & 0.093 & 0.003\\
\hspace{1em} & (0.013) & (0.188) & (0.118) & (0.01) & (0.009) & (0.068) & (0.09) & (0.007) & (0.009) & (0.073) & (0.076) & (0.006)\\
\hspace{1em} &  &  &  &  &  &  &  &  &  &  &  & \\
\hspace{1em}Contrôles & Oui & Oui & Oui & Oui & Oui & Oui & Oui & Oui & Oui & Oui & Oui & Oui\\
\hspace{1em}Observations & 218 & 218 & 218 & 218 & 223 & 223 & 223 & 223 & 258 & 258 & 258 & 258\\*
\end{longtable}
\end{ThreePartTable}
\endgroup{}
\end{landscape}

À notre sens, les raisons possibles pour lesquelles les coefficients des estimations par variable instrumentale sont supérieurs à ceux estimés sans sont au nombre de trois. La première est que nous avons peut-être supposé à tort que ce sont les plus forts \emph{a priori} qui utilisent le plus la plateforme. Théoriquement, il n'est pas irréaliste de penser que ce sont les plus faibles \emph{a priori} qui utilisent plus la plateforme s'ils ressentent plus de sentiment d'efficacité de cette dernière par rapport aux étudiants qui ont en moins besoin (ceux forts \emph{a priori}). Notre discussion dans la Section \ref{g20methods} (Tableau \ref{tab:g20exclrestrmodels}) rejette cette première raison. Nous avons en effet montré que dans un monde sans le protocole, il est plus raisonnable de penser que les plus forts \emph{a priori} utilisent plus la plateforme.\\
La deuxième raison est une violation de la restriction d'exclusion\footnote{Il est connu qu'une telle violation entraîne la supériorité de l'ampleur du coefficient estimé par moindres carrés ordinaires.}. Nous avons débuté les discussions sur cette possibilité dans la Section \ref{g20methods} et fourni des arguments contre. Pour rappel, nous avons montré que la forme de l'incitation ne donne pas de crédit à l'existence d'un effet direct de celle-ci sur les résultats de mathématiques. Nous avons également montré que l'incitation n'affecte pas la probabilité d'avoir une note dans d'autres matières majeures que sont la gestion et l'économie. Pour augmenter notre assurance, le Tableau \ref{tab:g20modelsnotesgestion} montre que les effets que nous craignions directs de l'incitation sur les notes en gestion pour ceux ayant une note aux TD de mathématiques sont en réalité des effets de l'utilisation de la plateforme. Cela peut s'expliquer par un comportement de compensation des faibles notes en mathématiques par les notes en gestion de la part des étudiants. Ce dernier propos est d'autant plus crédible lorsque nous avons regardé les sujets d'examen en gestion des concernés et constaté qu'ils ne contiennent pas de mathématiques. L'utilisation de la plateforme n'aurait alors pu faire augmenter la note en gestion que via ce comportement de compensation\footnote{Nous n'avons pas pu retrouver l'emploi du temps des examens concernés mais si les examens de gestion s'étaient déroulés après ceux des mathématiques, cela donnerait encore plus de crédit à l'explication du comportement de compensation.}. De plus, en cohérence avec l'absence d'effet de l'incitation sur les résultats en économie de la section précédente, nous ne détectons aucun potentiel effet de l'utilisation de la plateforme sur les notes en économie (Tableau \ref{tab:g20modelsnoteseconomie}).

\begin{landscape}\begingroup\fontsize{6}{8}\selectfont

\begin{ThreePartTable}
\begin{TableNotes}
\item \textit{Sources :} Base Parcoursup - L1 AES et Eco-Ge, plateforme '\textit{J'ai 20 en maths}', APOGEE (version extraite en mai 2021), calculs de l'auteur.
\item \textit{Notes :} Écart-types robustes entre parenthèses. 
    Les variables dépendantes sont des notes sur 20 (pour comparer les modèles linéaires et Tobit) sauf pour les modèles de probabilité linéaire. Dans les modèles Tobit, les notes aux TD sont censurées à gauche et à droite et les autres notes à gauche uniquement. Une colonne et un panel donnés correspondent à une régression. Les variables liées à la plateforme sont exprimées en comptages. ES : Économique et social, S : Scientifique, AES : Administration Économique et Sociale. Eco-Ge : Économie-Gestion. TD : Travaux Dirigés. UE : Unité d'Enseignement.
\item Significativité : 10\% * 5\% ** 1\% ***.
\end{TableNotes}
\begin{longtable}[t]{lllllllll}
\caption{\label{tab:g20modelsnotesgestion}Effets de l'utilisation de la plateforme sur les notes en gestion}\\
\toprule
\multicolumn{1}{c}{ } & \multicolumn{4}{c}{Ayant une note aux TD} & \multicolumn{4}{c}{Ayant une note aux examens} \\
\cmidrule(l{3pt}r{3pt}){2-5} \cmidrule(l{3pt}r{3pt}){6-9}
\multicolumn{1}{c}{ } & \multicolumn{8}{c}{Variable dépendante : Note à l'UE} \\
\cmidrule(l{3pt}r{3pt}){2-9}
\multicolumn{1}{c}{ } & \multicolumn{3}{c}{\makecell{En gestion \\ \ }} & \multicolumn{1}{c}{\makecell{En gestion \\ $\geq 10$}} & \multicolumn{3}{c}{\makecell{En gestion \\ \ }} & \multicolumn{1}{c}{\makecell{En gestion \\ $\geq 10$}} \\
\cmidrule(l{3pt}r{3pt}){2-4} \cmidrule(l{3pt}r{3pt}){5-5} \cmidrule(l{3pt}r{3pt}){6-8} \cmidrule(l{3pt}r{3pt}){9-9}
  & \makecell{MCO \\ (1) } & \makecell{VI \\ (2) } & \makecell{VI-Tobit \\ (3) } & \makecell{VI \\ (4) } & \makecell{MCO \\ (5) } & \makecell{VI \\ (6) } & \makecell{VI-Tobit \\ (7) } & \makecell{VI \\ (8) }\\
\midrule
\endfirsthead
\caption[]{\label{tab:g20modelsnotesgestion}Effets de l'utilisation de la plateforme sur les notes en gestion (suite)}\\
\toprule
  & \makecell{MCO \\ (1) } & \makecell{VI \\ (2) } & \makecell{VI-Tobit \\ (3) } & \makecell{VI \\ (4) } & \makecell{MCO \\ (5) } & \makecell{VI \\ (6) } & \makecell{VI-Tobit \\ (7) } & \makecell{VI \\ (8) }\\
\midrule
\endhead

\endfoot
\bottomrule
\insertTableNotes
\endlastfoot
\addlinespace[0.3em]
\multicolumn{9}{l}{\textbf{Panel A : Vidéos (complètes) et exercices}}\\
\hline
\hspace{1em}Vidéos (complètes) et exercices & 0.018$^{***}$ & 0.108 & 0.114$^{**}$ & 0.011 & 0.017$^{***}$ & 0.054 & 0.058 & 0.007\\
\hspace{1em} & (0.005) & (0.074) & (0.058) & (0.009) & (0.005) & (0.056) & (0.055) & (0.007)\\
\hspace{1em} &  &  &  &  &  &  &  \vphantom{8} & \\
\hspace{1em}Contrôles & Oui & Oui & Oui & Oui & Oui & Oui & Oui & \vphantom{4} Oui\\
\hspace{1em}Observations & 203 & 203 & 203 & 203 & 210 & 210 & 210 & \vphantom{4} 210\\
\hspace{1em} &  &  &  &  &  &  &  \vphantom{7} & \\
\addlinespace[0.3em]
\multicolumn{9}{l}{\textbf{Panel B : Vidéos et exercices}}\\
\hline
\hspace{1em}Vidéos et exercices & 0.014$^{***}$ & 0.104 & 0.111$^{**}$ & 0.01 & 0.013$^{***}$ & 0.05 & 0.054 & 0.006\\
\hspace{1em} & (0.004) & (0.081) & (0.056) & (0.009) & (0.004) & (0.054) & (0.052) & (0.007)\\
\hspace{1em} &  &  &  &  &  &  &  \vphantom{6} & \\
\hspace{1em}Contrôles & Oui & Oui & Oui & Oui & Oui & Oui & Oui & \vphantom{3} Oui\\
\hspace{1em}Observations & 203 & 203 & 203 & 203 & 210 & 210 & 210 & \vphantom{3} 210\\
\hspace{1em} &  &  &  &  &  &  &  \vphantom{5} & \\
\addlinespace[0.3em]
\multicolumn{9}{l}{\textbf{Panel C : Vidéos (complètes)}}\\
\hline
\hspace{1em}Vidéos (complètes) & 0.087 & 0.885 & 0.941$^{**}$ & 0.087 & 0.068 & 0.409 & 0.443 & 0.049\\
\hspace{1em} & (0.056) & (0.587) & (0.478) & (0.072) & (0.052) & (0.429) & (0.424) & (0.057)\\
\hspace{1em} &  &  &  &  &  &  &  \vphantom{4} & \\
\hspace{1em}Contrôles & Oui & Oui & Oui & Oui & Oui & Oui & Oui & \vphantom{2} Oui\\
\hspace{1em}Observations & 203 & 203 & 203 & 203 & 210 & 210 & 210 & \vphantom{2} 210\\
\hspace{1em} &  &  &  &  &  &  &  \vphantom{3} & \\
\addlinespace[0.3em]
\multicolumn{9}{l}{\textbf{Panel D : Vidéos}}\\
\hline
\hspace{1em}Vidéos & 0.022 & 0.711 & 0.756$^{**}$ & 0.07 & 0.026$^{*}$ & 0.259 & 0.281 & 0.031\\
\hspace{1em} & (0.016) & (1.104) & (0.384) & (0.112) & (0.015) & (0.354) & (0.268) & (0.045)\\
\hspace{1em} &  &  &  &  &  &  &  \vphantom{2} & \\
\hspace{1em}Contrôles & Oui & Oui & Oui & Oui & Oui & Oui & Oui & \vphantom{1} Oui\\
\hspace{1em}Observations & 203 & 203 & 203 & 203 & 210 & 210 & 210 & \vphantom{1} 210\\
\hspace{1em} &  &  &  &  &  &  &  \vphantom{1} & \\
\addlinespace[0.3em]
\multicolumn{9}{l}{\textbf{Panel E : Exercices}}\\
\hline
\hspace{1em}Exercices & 0.019$^{***}$ & 0.123 & 0.13$^{**}$ & 0.012 & 0.017$^{***}$ & 0.062 & 0.067 & 0.008\\
\hspace{1em} & (0.005) & (0.089) & (0.066) & (0.01) & (0.005) & (0.066) & (0.064) & (0.009)\\
\hspace{1em} &  &  &  &  &  &  &  & \\
\hspace{1em}Contrôles & Oui & Oui & Oui & Oui & Oui & Oui & Oui & Oui\\
\hspace{1em}Observations & 203 & 203 & 203 & 203 & 210 & 210 & 210 & 210\\*
\end{longtable}
\end{ThreePartTable}
\endgroup{}
\end{landscape}

\begin{landscape}\begingroup\fontsize{6}{8}\selectfont

\begin{ThreePartTable}
\begin{TableNotes}
\item \textit{Sources :} Base Parcoursup - L1 AES et Eco-Ge, plateforme '\textit{J'ai 20 en maths}', APOGEE (version extraite en mai 2021), calculs de l'auteur.
\item \textit{Notes :} Écart-types robustes entre parenthèses. 
    Les variables dépendantes sont des notes sur 20 (pour comparer les modèles linéaires et Tobit) sauf pour les modèles de probabilité linéaire. Dans les modèles Tobit, les notes aux TD sont censurées à gauche et à droite et les autres notes à gauche uniquement. Une colonne et un panel donnés correspondent à une régression. Les variables liées à la plateforme sont exprimées en comptages. ES : Économique et social, S : Scientifique, AES : Administration Économique et Sociale. Eco-Ge : Économie-Gestion. TD : Travaux Dirigés. UE : Unité d'Enseignement.
\item Significativité : 10\% * 5\% ** 1\% ***.
\end{TableNotes}
\begin{longtable}[t]{lllllllll}
\caption{\label{tab:g20modelsnoteseconomie}Effets de l'utilisation de la plateforme sur les notes en économie}\\
\toprule
\multicolumn{1}{c}{ } & \multicolumn{4}{c}{Ayant une note aux TD} & \multicolumn{4}{c}{Ayant une note aux examens} \\
\cmidrule(l{3pt}r{3pt}){2-5} \cmidrule(l{3pt}r{3pt}){6-9}
\multicolumn{1}{c}{ } & \multicolumn{8}{c}{Variable dépendante : Note à l'UE} \\
\cmidrule(l{3pt}r{3pt}){2-9}
\multicolumn{1}{c}{ } & \multicolumn{3}{c}{\makecell{En économie \\ \ }} & \multicolumn{1}{c}{\makecell{En économie \\ $\geq 10$}} & \multicolumn{3}{c}{\makecell{En économie \\ \ }} & \multicolumn{1}{c}{\makecell{En économie \\ $\geq 10$}} \\
\cmidrule(l{3pt}r{3pt}){2-4} \cmidrule(l{3pt}r{3pt}){5-5} \cmidrule(l{3pt}r{3pt}){6-8} \cmidrule(l{3pt}r{3pt}){9-9}
  & \makecell{MCO \\ (1) } & \makecell{VI \\ (2) } & \makecell{VI-Tobit \\ (3) } & \makecell{VI \\ (4) } & \makecell{MCO \\ (5) } & \makecell{VI \\ (6) } & \makecell{VI-Tobit \\ (7) } & \makecell{VI \\ (8) }\\
\midrule
\endfirsthead
\caption[]{\label{tab:g20modelsnoteseconomie}Effets de l'utilisation de la plateforme sur les notes en économie (suite)}\\
\toprule
  & \makecell{MCO \\ (1) } & \makecell{VI \\ (2) } & \makecell{VI-Tobit \\ (3) } & \makecell{VI \\ (4) } & \makecell{MCO \\ (5) } & \makecell{VI \\ (6) } & \makecell{VI-Tobit \\ (7) } & \makecell{VI \\ (8) }\\
\midrule
\endhead

\endfoot
\bottomrule
\insertTableNotes
\endlastfoot
\addlinespace[0.3em]
\multicolumn{9}{l}{\textbf{Panel A : Vidéos (complètes) et exercices}}\\
\hline
\hspace{1em}Vidéos (complètes) et exercices & 0.009 & $-$0.004 & $-$0.018 & 0 & 0.009 & $-$0.018 & $-$0.046 & $-$0.001\\
\hspace{1em} & (0.007) & (0.065) & (0.078) & (0.006) & (0.007) & (0.062) & (0.075) & (0.005)\\
\hspace{1em} &  &  &  &  &  &  &  \vphantom{8} & \\
\hspace{1em}Contrôles & Oui & Oui & Oui & Oui & Oui & Oui & Oui & \vphantom{4} Oui\\
\hspace{1em}Observations & 205 & 205 & 205 & 205 & 214 & 214 & 214 & \vphantom{4} 214\\
\hspace{1em} &  &  &  &  &  &  &  \vphantom{7} & \\
\addlinespace[0.3em]
\multicolumn{9}{l}{\textbf{Panel B : Vidéos et exercices}}\\
\hline
\hspace{1em}Vidéos et exercices & 0.006 & $-$0.003 & $-$0.018 & 0 & 0.007 & $-$0.017 & $-$0.043 & $-$0.001\\
\hspace{1em} & (0.006) & (0.063) & (0.077) & (0.005) & (0.006) & (0.057) & (0.069) & (0.005)\\
\hspace{1em} &  &  &  &  &  &  &  \vphantom{6} & \\
\hspace{1em}Contrôles & Oui & Oui & Oui & Oui & Oui & Oui & Oui & \vphantom{3} Oui\\
\hspace{1em}Observations & 205 & 205 & 205 & 205 & 214 & 214 & 214 & \vphantom{3} 214\\
\hspace{1em} &  &  &  &  &  &  &  \vphantom{5} & \\
\addlinespace[0.3em]
\multicolumn{9}{l}{\textbf{Panel C : Vidéos (complètes)}}\\
\hline
\hspace{1em}Vidéos (complètes) & $-$0.003 & $-$0.029 & $-$0.148 & $-$0.003 & 0.003 & $-$0.139 & $-$0.351 & $-$0.006\\
\hspace{1em} & (0.091) & (0.53) & (0.647) & (0.045) & (0.088) & (0.461) & (0.572) & (0.039)\\
\hspace{1em} &  &  &  &  &  &  &  \vphantom{4} & \\
\hspace{1em}Contrôles & Oui & Oui & Oui & Oui & Oui & Oui & Oui & \vphantom{2} Oui\\
\hspace{1em}Observations & 205 & 205 & 205 & 205 & 214 & 214 & 214 & \vphantom{2} 214\\
\hspace{1em} &  &  &  &  &  &  &  \vphantom{3} & \\
\addlinespace[0.3em]
\multicolumn{9}{l}{\textbf{Panel D : Vidéos}}\\
\hline
\hspace{1em}Vidéos & 0.001 & $-$0.025 & $-$0.125 & $-$0.003 & 0.008 & $-$0.087 & $-$0.217 & $-$0.004\\
\hspace{1em} & (0.023) & (0.456) & (0.555) & (0.039) & (0.022) & (0.3) & (0.357) & (0.024)\\
\hspace{1em} &  &  &  &  &  &  &  \vphantom{2} & \\
\hspace{1em}Contrôles & Oui & Oui & Oui & Oui & Oui & Oui & Oui & \vphantom{1} Oui\\
\hspace{1em}Observations & 205 & 205 & 205 & 205 & 214 & 214 & 214 & \vphantom{1} 214\\
\hspace{1em} &  &  &  &  &  &  &  \vphantom{1} & \\
\addlinespace[0.3em]
\multicolumn{9}{l}{\textbf{Panel E : Exercices}}\\
\hline
\hspace{1em}Exercices & 0.01 & $-$0.004 & $-$0.021 & 0 & 0.01 & $-$0.021 & $-$0.053 & $-$0.001\\
\hspace{1em} & (0.008) & (0.074) & (0.089) & (0.006) & (0.008) & (0.071) & (0.086) & (0.006)\\
\hspace{1em} &  &  &  &  &  &  &  & \\
\hspace{1em}Contrôles & Oui & Oui & Oui & Oui & Oui & Oui & Oui & Oui\\
\hspace{1em}Observations & 205 & 205 & 205 & 205 & 214 & 214 & 214 & 214\\*
\end{longtable}
\end{ThreePartTable}
\endgroup{}
\end{landscape}

Notons que l'instrument potentiellement faible n'est pas une raison crédible de la supériorité d'un coefficient estimé par variable instrumentale par rapport à celui estimé par MCO. Un instrument faible a comme conséquence le fait que les coefficients estimés par variable instrumentale tendent vers ceux estimés par moindres carrés ordinaires (\protect\hyperlink{ref-PIS:18}{Pischke, 2018}), ce qui ne correspond pas à notre cas.

La raison restante à notre connaissance est alors le postulat selon lequel ceux qui réagissent à l'incitation \(-\) les \emph{compliers}, selon la terminologie utilisée par Imbens \& Angrist (\protect\hyperlink{ref-IMB:ANG:94}{1994}), par exemple\(-\) sont extrêmement différents de la population et que l'effet de l'utilisation de la plateforme (plus précisément, l'effet de regarder plus de vidéos entièrement) est très élevé pour cette sous-population. Quelques caractéristiques de la plateforme et de son contenu nous amènent à penser qu'elle ne pourrait être efficace que pour une sous-population d'étudiants forts et/ou motivés \emph{a priori}. Notamment, les vidéos et les exercices proposés ne s'adaptent pas au niveau initial de l'utilisateur (voir Section \ref{g20instproto}). Et après avoir effectivement visualisé le contenu de la plateforme longtemps après la fin du protocole, nous pensons que le niveau en mathématiques du contenu proposé apparaît relativement élevé, par rapport au niveau d'un étudiant moyen de notre étude. Un étudiant initialement faible aura alors plus de mal à profiter de la plateforme qu'un étudiant initialement fort. Aussi, la présentation des contenus n'est pas ludifiée, ce qui implique potentiellement une baisse d'intérêt pour les plus faibles par rapport à une présentation ludifiée (\protect\hyperlink{ref-PUT:eal:20}{Putz et al., 2020}). Ce seraient alors les plus forts/motivés qui auraient manifesté le plus d'intérêt pour la plateforme et par conséquent qui en auraient profité le plus pour avoir de meilleures notes aux TD. La remise en contexte des résultats du Tableau \ref{tab:g20models} et des évaluations de TD nous permettent de donner plus de crédit à cette possibilité. Les évaluations de TD ont pris la forme de QCM en ligne effectuées à distance (hors de l'université) annoncées à l'avance et les évaluations finales des QCM en présentiel. Les évaluations de TD sont de plus censées être plus faciles que les examens. L'effet positif de grande ampleur des vidéos visionnées entièrement est alors très probablement un effet d'une stratégie de révision à court terme dans le but d'avoir de très bonnes notes aux TD de la part des plus motivés/ambitieux et des plus forts vu le niveau en mathématiques relativement élevé du contenu.

Il n'est pas possible d'identifier individuellement les compliers mais il est possible de connaître la part qu'ils représentent dans les échantillons d'estimation et il existe un moyen pour les caractériser par rapport aux observables (\protect\hyperlink{ref-ANG:PIS:08}{Angrist \& Pischke, 2009, p. 171}). Il nécessite de considérer une version binarisée des variables d'utilisation de la plateforme. Par exemple, au lieu de considérer le nombre de vidéos visionnées entièrement par un individu, nous nous limitons à une indicatrice égale à 1 si cet individu a visionné au moins une vidéo entièrement, et égale à 0 sinon. La part de compliers correspond simplement au coefficient de première étape (sans variables de contrôle). Quant à la caractérisation, elle consiste concrètement à calculer le rapport entre la proportion d'un attribut particulier (la proportion de femmes, par exemple) chez les compliers et celle dans tout l'échantillon. Il est connu qu'en utilisant des propriétés statistiques de l'espérance conditionnelle, cette quantité est égale au rapport entre la proportion de compliers parmi ceux qui présentent l'attribut et celle dans tout l'échantillon. Ce second rapport est quant à lui calculable puisqu'il s'agit du rapport entre le coefficient de première étape (sans variables de contrôle) chez les individus qui présentent l'attribut considéré et celui dans tout l'échantillon.\\
Très simplement, cette comparaison de proportions devrait nous permettre de dire si, chez les compliers, il existe relativement plus d'étudiants issus de telle ou telle série du bac, par exemple ; et cela pour chaque caractéristique prédéterminée pertinente.

Nous rappelons que les compliers peuvent être différents d'une mesure d'utilisation de la plateforme à une autre, en fonction du niveau ou des préférences, par exemple. Nous proposons alors dans le Tableau \ref{tab:g20compliersvideoscomptot} ainsi que les Tableaux \ref{tab:g20compliersvideosviewstot} et \ref{tab:g20complierssheetsviewstot} de l'Annexe \ref{g20compliers} les résultats de la caractérisation décrite ci-dessus pour trois mesures d'utilisation de la plateforme (vidéos visionnées entièrement, vidéos et exercices). Les rapports de part de compliers sont les quantités d'intérêt. À titre d'exemple, le chiffre de 1.22 du Tableau \ref{tab:g20compliersvideoscomptot} correspondant à la série technologique pour ceux ayant une note aux TD signifie qu'il y a 1.22 fois plus d'étudiants issus d'un bac technologique chez les compliers que dans l'échantillon global. Un rapport égal à 1 veut dire que l'attribut est représentatif de l'échantillon parmi les compliers. Un chiffre inférieur (respectivement supérieur) à 1 veut dire que l'attribut est moins (respectivement plus) retrouvé chez les compliers. Un chiffre négatif n'a pas spécialement de sens et nous nous abstenons de les interpréter. Les petits effectifs constituent une limite directe des informations contenues dans ce type de tableau.

Les compliers des vidéos visionnées entièrement représentent environ 7\%. Ce chiffre est égal à la différence entre la part de ceux qui ont regardé au moins une vidéo entière parmi les incités et cette même part chez les non incités (Tableau \ref{tab:g20stats}). Ce sont plutôt des étudiants en série technologique et ES, en filière AES, assez dispersés en termes de note au bac et d'âge à la rentrée\footnote{Les divisions en quartiles de la note au bac et de l'âge à la rentrée permettent de répartir les effectifs à travers les modalités (les quartiles). D'autres spécifications (avec ou sans mention pour la note au bac et à l'heure, en retard ou en avance pour l'âge à la rentrée, par exemple) amènent aux mêmes conclusions.}, des femmes, du campus de Saint-Denis, des étudiants nés à l'étranger, des non boursiers et des étudiants provenant de lycées privés (Tableau \ref{tab:g20compliersvideoscomptot}).

\newpage
\begingroup\fontsize{8}{10}\selectfont

\begin{ThreePartTable}
\begin{TableNotes}
\item \textit{Sources :} Base Parcoursup - L1 AES et Eco-Ge, plateforme '\textit{J'ai 20 en maths}', APOGEE (version extraite en mai 2021)
\item \textit{Notes :} Le traitement est binarisé (au moins un) par souci de simplification. Rapport de parts de compliers = différence de moyennes du traitement en fonction de l'incitation divisée par la différence de moyennes du traitement en fonction de l'incitation sur l'échantillon considéré (ayant une note aux TD ou ayant une note aux examens) - Angrist et Pischke (2008, p. 171). UE : Unité d'enseignement, ES : Économique et social, S : Scientifique, AES : Administration Économique et Sociale, Eco-Ge : Économie-Gestion. TD : Travaux Dirigés.
\end{TableNotes}
\begin{longtable}[t]{llrrrr}
\caption{\label{tab:g20compliersvideoscomptot}Caractérisation des compliers - Vidéos (complètes)}\\
\toprule
\multicolumn{2}{c}{ } & \multicolumn{2}{c}{Ayant une note aux TD} & \multicolumn{2}{c}{Ayant une note aux examens} \\
\cmidrule(l{3pt}r{3pt}){3-4} \cmidrule(l{3pt}r{3pt}){5-6}
Variable & Modalité & Effectif & \makecell{Rapport de parts \\ de compliers} & Effectif & \makecell{Rapport de parts \\ de compliers}\\
\midrule
\endfirsthead
\caption[]{\label{tab:g20compliersvideoscomptot}Caractérisation des compliers - Vidéos (complètes) (suite)}\\
\toprule
Variable & Modalité & Effectif & \makecell{Rapport de parts \\ de compliers} & Effectif & \makecell{Rapport de parts \\ de compliers}\\
\midrule
\endhead

\endfoot
\bottomrule
\insertTableNotes
\endlastfoot
\addlinespace[0.3em]
\multicolumn{6}{l}{\textbf{ }}\\
Série au bac & Pro & 21 & 0.00 & 28 & 0.00\\
 & Techno & 58 & 1.22 & 57 & 1.26\\
 & ES & 96 & 1.60 & 97 & 1.52\\
 & S & 36 & 0.09 & 35 & 0.00\\
 & Autre & 7 & 0.00 & 8 & 1.94\\
\addlinespace[0.3em]
\multicolumn{6}{l}{\textbf{ }}\\
Filière & AES & 161 & 1.11 & 164 & 1.03\\
 & Eco-Ge & 57 & 0.80 & 61 & 1.02\\
\addlinespace[0.3em]
\multicolumn{6}{l}{\textbf{ }}\\
Quintile de la note au bac & Q1 & 45 & 1.28 & 42 & 1.22\\
 & Q2 & 36 & 1.57 & 40 & 1.94\\
 & Q3 & 43 & 0.66 & 45 & 0.93\\
 & Q4 & 48 & 2.31 & 50 & 1.53\\
 & Q5 & 46 & -0.27 & 46 & -0.22\\
\addlinespace[0.3em]
\multicolumn{6}{l}{\textbf{ }}\\
Quintile de l'âge à la rentrée & Q1 & 44 & 0.92 & 46 & 1.15\\
 & Q2 & 44 & 1.65 & 45 & 1.28\\
 & Q3 & 45 & 0.52 & 44 & 1.27\\
 & Q4 & 45 & 2.08 & 45 & 1.94\\
 & Q5 & 40 & 0.67 & 45 & 0.48\\
\addlinespace[0.3em]
\multicolumn{6}{l}{\textbf{ }}\\
Sexe & Fille & 153 & 0.90 & 157 & 1.13\\
 & Homme & 65 & 0.78 & 68 & 0.50\\
\addlinespace[0.3em]
\multicolumn{6}{l}{\textbf{ }}\\
Campus & Tampon & 92 & 0.19 & 90 & 0.52\\
 & Saint-Denis & 126 & 1.57 & 135 & 1.31\\
\addlinespace[0.3em]
\multicolumn{6}{l}{\textbf{ }}\\
Statut de boursier & Non boursier & 78 & 2.30 & 77 & 2.54\\
 & Boursier du secondaire & 140 & 0.19 & 148 & 0.20\\
\addlinespace[0.3em]
\multicolumn{6}{l}{\textbf{ }}\\
Pays de naissance & À l'étranger & 20 & 1.57 & 19 & 1.66\\
 & En France & 198 & 0.94 & 206 & 0.91\\
\addlinespace[0.3em]
\multicolumn{6}{l}{\textbf{ }}\\
Statut d'établissement d'origine & Public & 206 & 0.83 & 211 & 0.91\\
 & Privé & 12 & 5.29 & 14 & 2.42\\*
\end{longtable}
\end{ThreePartTable}
\endgroup{}
\newpage

En ce qui concerne les vidéos visionnées, les compliers représentent environ 11\% des inscrits (Tableau \ref{tab:g20stats}). Sur leur caractérisation, la seule différence qualitative par rapport à celle pour les vidéos visionnées entièrement réside dans la filière et l'âge à la rentrée puisque les compliers correspondants sont plutôt des étudiants d'Eco-Ge et plutôt âgés (Tableau \ref{tab:g20compliersvideosviewstot}, Annexe \ref{g20compliers}). Nous rappelons que l'instrument est faible lorsque l'utilisation de la plateforme est mesurée avec le nombre de vidéos visionnées.

Pour les exercices, il y a environ 12\% de compliers (Tableau \ref{tab:g20stats}). Le Tableau \ref{tab:g20complierssheetsviewstot} (Annexe \ref{g20compliers}) qui les caractérise montre des informations plus chaotiques mais nous pouvons distinguer qu'ils sont plutôt ceux en série pro et techno, en Eco-Ge, plutôt âgés, du campus de Saint-Denis, des non boursiers et issus d'établissements privés.

En somme, d'après nos raisonnements \emph{a priori}, nous nous attendions à ce que les compliers des vidéos visionnées entièrement soient des individus clairement plus forts en mathématiques. Cela ne semble pas être le cas lorsque nous observons les rapports pour la note au bac. Nous pouvons alors penser que ce sont juste les individus les plus motivés. Ce qui est d'ailleurs cohérent avec l'effet sur les notes aux TD.

\quad En ce qui concerne l'effet de l'utilisation de la plateforme sur la note aux examens, les résultats sont conformes aux attentes. Dans les modèles linéaires comme dans les modèles Tobit, les coefficients estimés d'intérêt sont généralement inférieurs sans variable instrumentale par rapport à leurs homologues estimés par variable instrumentale\footnote{La mesure par le nombre d'exercices vus (Panel E, colonnes 5 et 6) constitue une exception mais la différence (0.011 - 0.008) est négligeable.}.

\quad L'équivalent du Tableau \ref{tab:g20models} lorsque les notes sont exprimées en unité d'écart-type est le Tableau \ref{tab:g20modelsnormpop} de l'Annexe \ref{g20modelsnormpop}.

\hypertarget{g20reshetero}{%
\subsection{Effets potentiellement hétérogènes}\label{g20reshetero}}

Les résultats de la Section \ref{g20resprinc} valent pour les compliers, une sous-population que nous ne pouvons pas individuellement identifier même si nous avons pu la quantifier et la caractériser. Cette caractérisation repose toutefois sur la transformation en variable binaire des mesures d'utilisation de la plateforme. La présente sous-section vient en complément de la précédente en tentant de mettre explicitement en évidence des effets potentiellement hétérogènes en fonction des observables qui varient au niveau individuel comme la note et la série au bac, l'âge à la rentrée, le sexe, le pays de naissance et le statut de boursier sans avoir à transformer les mesures de l'utilisation de la plateforme en variables binaires. Au vu de l'utilisation très faible de la plateforme, nous nous limitons aux effets de l'incitation sur les notes (intention de traiter).

Concrètement, nous restons cohérent avec les spécifications des modèles de régression des Chapitres \ref{age} et \ref{pe} et ajoutons des interactions entre l'observable en question et l'incitation (sections \ref{agemodelsheterores} et \ref{peresfrhnolinq5}). De plus, dans ce chapitre, la taille de nos échantillons d'estimation nous incite à éviter les estimations sur échantillons séparées.\\
Les résultats de ces régressions sont présentés dans le Tableau \ref{tab:g20rfheteromodels}. En majorité, les coefficients devant les interactions sont estimés de manière très imprécise mais nous pouvons en tirer quelques leçons. Si un coefficient est de grande ampleur et que cette ampleur est robuste à travers les spécifications, la faible précision de l'estimation n'est très probablement due qu'à la faible taille de notre échantillon.\\
Après avoir regardé systématiquement la robustesse de ces coefficients en fonction des variables de contrôle utilisées, nous trouvons qu'ils sont robustes sauf dans le cas des interactions avec la mention au bac (Panel A) et avec le sexe (Panel D).

En observant plus précisément ensuite le Tableau \ref{tab:g20rfheteromodels}, hors panels A et D, nous trouvons que l'effet de l'incitation sur la note aux TD est d'ampleur négligeable et non significatif pour les titulaires d'un bac professionnel (valeur de référence). L'ampleur de l'effet supplémentaire pour les titulaires d'un bac technologique est considérable (2.5 points sur 20, panel B) même s'il est non significatif. Nous soupçonnons alors qu'il existe un réel effet pour cette sous-population que nous n'avons pas pu détecter statistiquement. De manière plus probante, l'effet supplémentaire pour les titulaires d'un bac S est deux fois plus grand et significatif au seuil de 5\%. Les colonnes (2) et (3) du panel B qui montrent les estimations Tobit et en modèle de probabilité linéaire. Les résultats sont qualitativement les mêmes. Nous avons assez d'éléments qui indiquent que l'effet de l'incitation est le plus fort pour ceux en série S au bac. Les mêmes coefficients lorsque la variable dépendante est la note aux examens (colonnes 4 à 6 du Panel B) sont clairement plus mitigés mais nous soupçonnons qu'il reste un effet considérable pour ceux en série S vu l'ampleur des coefficients (environ 2 points sur 20, 17 points de pourcentage dans le modèle de probabilité linéaire de la colonne 6).

Les résultats d'estimation des effets potentiellement hétérogènes de l'incitation en fonction de la classe d'âge (panel C) ne nous apprennent pas grand-chose. Les coefficients devant les interactions entre l'incitation et les classes d'âge ne vont pas dans un sens précis (un effet qui augmente clairement avec l'âge, par exemple). Les effets trouvés sur la note aux examens dans le modèle de probabilité linéaire (colonne 6) sont différents de ceux obtenus dans les modèles linéaires ou Tobit. Et les coefficients pour la valeur de référence ne sont pas significatifs, malgré des ampleurs considérables (1.4 en colonne 1, par exemple). En résumé, nous préférons rester prudent et constater que nous n'en savons pas plus sur d'éventuels effets différenciés de l'incitation en fonction de l'âge. Le fait que l'âge ne présente que peu de variation dans notre étude\footnote{En guise d'illustration, le premier quartile vaut 18 ans et le troisième 18.7 ans.} peut être la raison pour laquelle nous ne détectons pas d'effet hétérogène en fonction de l'âge, s'il en existait.

D'après les panels E et F du Tableau \ref{tab:g20rfheteromodels}, l'incitation n'a d'effet que sur la note aux TD de ceux à l'étranger et les non boursiers. Aucun effet d'ampleur intéressante n'est détecté dans les régressions des notes aux examens (colonnes 3 à 6).

\quad En résumé, par rapport aux observables, l'effet considérable sur la note aux TD ne semble valoir que pour une petite partie des néo-bacheliers (série S, pas nés en France, non boursiers). Même en autorisant les effets de l'incitation à être hétérogène par rapport aux observables, nous ne détectons aucun effet notable de l'incitation sur la note aux examens, ce qui implique qu'il n'y a très probablement pas d'effet de l'utilisation de la plateforme.

\begin{landscape}\begingroup\fontsize{5.75}{7.75}\selectfont

\begin{ThreePartTable}
\begin{TableNotes}
\item \textit{Sources :} Base Parcoursup - L1 AES et Eco-Ge, plateforme '\textit{J'ai 20 en maths}', APOGEE (version extraite en mai 2021), calculs de l'auteur.
\item \textit{Notes :} Écart-types robustes entre parenthèses. 
    Les variables dépendantes sont des notes sur 20 (pour comparer les modèles linéaires et Tobit) sauf pour les modèles de probabilité linéaire. Dans les modèles Tobit, les notes aux TD sont censurées à gauche et à droite et les autres notes à gauche uniquement. Une colonne et un panel donnés correspondent à une régression. ES : Économique et social, S : Scientifique, AES : Administration Économique et Sociale. Eco-Ge : Économie-Gestion. TD : Travaux Dirigés. UE : Unité d'Enseignement.
\item Significativité : 10\% * 5\% ** 1\% ***.
\end{TableNotes}
\begin{longtable}[t]{llllllllll}
\caption{\label{tab:g20rfheteromodels}Effets potentiellement hétérogènes de l'incitation sur les notes de mathématiques (intention de traiter)}\\
\toprule
\multicolumn{1}{c}{ } & \multicolumn{9}{c}{Variable dépendante : Note } \\
\cmidrule(l{3pt}r{3pt}){2-10}
\multicolumn{1}{c}{ } & \multicolumn{2}{c}{\makecell{Aux TD \\  }} & \multicolumn{1}{c}{\makecell{Aux TD \\ $\geq 10$}} & \multicolumn{2}{c}{\makecell{Aux examens \\  }} & \multicolumn{1}{c}{\makecell{Aux examens \\ $\geq 10$}} & \multicolumn{2}{c}{\makecell{À l'UE \\  }} & \multicolumn{1}{c}{\makecell{À l'UE \\ $\geq 10$}} \\
\cmidrule(l{3pt}r{3pt}){2-3} \cmidrule(l{3pt}r{3pt}){4-4} \cmidrule(l{3pt}r{3pt}){5-6} \cmidrule(l{3pt}r{3pt}){7-7} \cmidrule(l{3pt}r{3pt}){8-9} \cmidrule(l{3pt}r{3pt}){10-10}
  & \makecell{MCO \\ (1) } & \makecell{Tobit \\ (2) } & \makecell{MCO \\ (3) } & \makecell{MCO \\ (4) } & \makecell{Tobit \\ (5) } & \makecell{MCO \\ (6) } & \makecell{MCO \\ (7) } & \makecell{Tobit \\ (8) } & \makecell{MCO \\ (9) }\\
\midrule
\endfirsthead
\caption[]{\label{tab:g20rfheteromodels}Effets potentiellement hétérogènes de l'incitation sur les notes de mathématiques (intention de traiter) (suite)}\\
\toprule
  & \makecell{MCO \\ (1) } & \makecell{Tobit \\ (2) } & \makecell{MCO \\ (3) } & \makecell{MCO \\ (4) } & \makecell{Tobit \\ (5) } & \makecell{MCO \\ (6) } & \makecell{MCO \\ (7) } & \makecell{Tobit \\ (8) } & \makecell{MCO \\ (9) }\\
\midrule
\endhead

\endfoot
\bottomrule
\insertTableNotes
\endlastfoot
\addlinespace[0.3em]
\multicolumn{10}{l}{\textbf{Panel A : Hétérogénéité en fonction de la mention au bac}}\\
\hline
\hspace{1em}Incitation - Oui & 2.41$^{**}$ & 2.818$^{**}$ & 0.101 & 0.041 & 0.102 & 0.022 & 0.815 & 1.11 & $-$0.012\\
\hspace{1em} & (1.007) & (1.156) & (0.083) & (0.503) & (0.896) & (0.041) & (0.567) & (0.735) & (0.039)\\
\hspace{1em}Mention au bac - Assez bien & 2.346$^{**}$ & 2.457$^{**}$ & 0.178$^{**}$ & 1.563$^{***}$ & 2.466$^{***}$ & 0.088 & 1.925$^{***}$ & 2.324$^{***}$ & 0.131$^{**}$\\
\hspace{1em} & (0.978) & (1.171) & (0.081) & (0.581) & (0.921) & (0.055) & (0.616) & (0.782) & (0.059)\\
\hspace{1em}Mention au bac - Bien & 5.399$^{***}$ & 6.881$^{***}$ & 0.267$^{**}$ & 4.389$^{***}$ & 5.258$^{***}$ & 0.341$^{***}$ & 5.19$^{***}$ & 5.755$^{***}$ & 0.357$^{***}$\\
\hspace{1em} & (1.563) & (1.618) & (0.116) & (0.913) & (1.162) & (0.103) & (0.991) & (1.077) & (0.095)\\
\hspace{1em}Mention au bac - Très bien & 11.669$^{***}$ & 44.075 & 0.655$^{***}$ & 8.624$^{*}$ & 8.96$^{***}$ & 0.618$^{***}$ & 10.063$^{**}$ & 10.421$^{***}$ & 0.669$^{***}$\\
\hspace{1em} & (2.407) & (5454.028) & (0.136) & (4.589) & (2.94) & (0.232) & (3.977) & (2.881) & (0.178)\\
\hspace{1em}Incitation - Oui $\times$ Mention au bac - Assez bien & $-$1.149 & $-$1.154 & $-$0.144 & 0.032 & $-$0.314 & 0.03 & $-$0.631 & $-$1.142 & 0.023\\
\hspace{1em} & (1.425) & (1.747) & (0.128) & (0.929) & (1.33) & (0.081) & (0.944) & (1.139) & (0.085)\\
\hspace{1em}Incitation - Oui $\times$ Mention au bac - Bien & $-$2.008 & $-$3.367 & $-$0.025 & $-$0.497 & $-$0.947 & 0.05 & $-$1.38 & $-$1.831 & 0.024\\
\hspace{1em} & (2.153) & (2.397) & (0.176) & (1.69) & (1.72) & (0.156) & (1.571) & (1.574) & (0.145)\\
\hspace{1em}Incitation - Oui $\times$ Mention au bac - Très bien & $-$0.473 & $-$30.615 & 0.086 & 0.832 & 1.575 & $-$0.182 & 0.736 & 0.671 & 0.29\\
\hspace{1em} & (2.847) & (5454.029) & (0.149) & (4.786) & (3.529) & (0.316) & (4.202) & (3.462) & (0.178)\\
\hspace{1em} &  &  &  &  &  &  &  &  \vphantom{10} & \\
\hspace{1em}Contrôles & Oui & Oui & Oui & Oui & Oui & Oui & Oui & Oui & \vphantom{5} Oui\\
\hspace{1em}Observations & 218 & 218 & 218 & 225 & 225 & 225 & 260 & 260 & \vphantom{5} 260\\
\hspace{1em} &  &  &  &  &  &  &  &  \vphantom{9} & \\
\addlinespace[0.3em]
\multicolumn{10}{l}{\textbf{Panel B : Hétérogénéité en fonction de la série au bac}}\\
\hline
\hspace{1em}Incitation - Oui & 0.175 & 0.773 & $-$0.039 & $-$0.606 & $-$1.152 & $-$0.054 & $-$0.824 & $-$1.112 & $-$0.059\\
\hspace{1em} & (0.911) & (2.636) & (0.073) & (0.586) & (2.225) & (0.047) & (0.575) & (1.469) & (0.046)\\
\hspace{1em}Série au bac - Techno & 1.576$^{*}$ & 1.614 & 0.097 & 1.268$^{**}$ & 3.806$^{**}$ & 0.037 & 1.155$^{**}$ & 1.985$^{*}$ & 0.025\\
\hspace{1em} & (0.952) & (1.801) & (0.071) & (0.564) & (1.606) & (0.038) & (0.481) & (1.122) & (0.032)\\
\hspace{1em}Série au bac - ES & 6.424$^{***}$ & 7.023$^{***}$ & 0.435$^{***}$ & 3.998$^{***}$ & 7.373$^{***}$ & 0.133$^{**}$ & 4.819$^{***}$ & 6.008$^{***}$ & 0.21$^{***}$\\
\hspace{1em} & (0.974) & (1.651) & (0.077) & (0.619) & (1.508) & (0.052) & (0.582) & (1.038) & (0.056)\\
\hspace{1em}Série au bac - S & 9.885$^{***}$ & 11.479$^{***}$ & 0.655$^{***}$ & 8.414$^{***}$ & 12.136$^{***}$ & 0.439$^{***}$ & 8.973$^{***}$ & 10.344$^{***}$ & 0.464$^{***}$\\
\hspace{1em} & (1.458) & (1.938) & (0.107) & (0.951) & (1.66) & (0.105) & (1.058) & (1.242) & (0.094)\\
\hspace{1em}Série au bac - Autre & 6.678 & 6.708 & 0.422 & 0.577 & 3.892 & $-$0.229$^{**}$ & 2.087 & 2.411 & $-$0.119\\
\hspace{1em} & (5.949) & (4.163) & (0.393) & (1.419) & (3.146) & (0.094) & (2.293) & (2.537) & (0.095)\\
\hspace{1em}Incitation - Oui $\times$ Série au bac - Techno & 2.538 & 2.172 & 0.174 & 0.565 & 0.154 & 0.108$^{*}$ & 2.156$^{**}$ & 2.537 & 0.089$^{*}$\\
\hspace{1em} & (1.545) & (3.003) & (0.118) & (0.767) & (2.527) & (0.056) & (0.882) & (1.765) & (0.054)\\
\hspace{1em}Incitation - Oui $\times$ Série au bac - ES & 0.29 & $-$0.321 & 0.005 & 0.29 & 1.01 & 0.056 & 0.468 & 0.811 & 0.007\\
\hspace{1em} & (1.287) & (2.849) & (0.119) & (0.969) & (2.384) & (0.087) & (0.944) & (1.655) & (0.086)\\
\hspace{1em}Incitation - Oui $\times$ Série au bac - S & 4.093$^{**}$ & 3.835 & 0.219$^{*}$ & 1.841 & 2.44 & 0.177 & 2.827$^{*}$ & 2.951 & 0.193\\
\hspace{1em} & (1.794) & (3.312) & (0.132) & (1.661) & (2.613) & (0.153) & (1.625) & (1.954) & (0.139)\\
\hspace{1em}Incitation - Oui $\times$ Série au bac - Autre & $-$2.034 & $-$2.244 & $-$0.073 & 1.653 & 1.382 & 0.382$^{***}$ & 1.655 & 2.839 & 0.278$^{**}$\\
\hspace{1em} & (6.358) & (5.357) & (0.438) & (1.639) & (4.022) & (0.111) & (2.559) & (3.242) & (0.111)\\
\hspace{1em} &  &  &  &  &  &  &  &  \vphantom{8} & \\
\hspace{1em}Contrôles & Oui & Oui & Oui & Oui & Oui & Oui & Oui & Oui & \vphantom{4} Oui\\
\hspace{1em}Observations & 218 & 218 & 218 & 225 & 225 & 225 & 260 & 260 & \vphantom{4} 260\\
\hspace{1em} &  &  &  &  &  &  &  &  \vphantom{7} & \\
\addlinespace[0.3em]
\multicolumn{10}{l}{\textbf{Panel C : Hétérogénéité en fonction du quintile de l'âge à la rentrée}}\\
\hline
\hspace{1em}Incitation - Oui & 1.402 & 1.78 & 0.02 & 0.924 & 1.355 & $-$0.029 & 1.111 & 1.5 & $-$0.067\\
\hspace{1em} & (1.373) & (1.719) & (0.129) & (0.809) & (1.252) & (0.062) & (0.859) & (1.137) & (0.078)\\
\hspace{1em}Quintile de l'âge à la rentrée - Q2 & 1.628 & 1.677 & 0.122 & 0.897 & 0.962 & $-$0.02 & 0.981 & 1.111 & 0.057\\
\hspace{1em} & (1.455) & (1.716) & (0.11) & (0.804) & (1.228) & (0.079) & (0.893) & (1.088) & (0.083)\\
\hspace{1em}Quintile de l'âge à la rentrée - Q3 & $-$0.663 & $-$0.56 & 0.003 & 0.266 & 0.078 & 0.081 & $-$0.363 & $-$0.215 & $-$0.095\\
\hspace{1em} & (1.487) & (1.793) & (0.124) & (0.853) & (1.2) & (0.084) & (0.893) & (1.077) & (0.082)\\
\hspace{1em}Quintile de l'âge à la rentrée - Q4 & $-$0.569 & $-$0.999 & $-$0.051 & 0.386 & 0.357 & 0.018 & 0.009 & 0.256 & $-$0.043\\
\hspace{1em} & (1.673) & (2.09) & (0.14) & (0.826) & (1.291) & (0.081) & (0.818) & (1.163) & (0.084)\\
\hspace{1em}Quintile de l'âge à la rentrée - Q5 & 1.249 & 0.908 & 0.172 & 0.552 & $-$0.706 & 0.077 & 0.474 & 0.858 & 0.005\\
\hspace{1em} & (2.615) & (3.272) & (0.221) & (1.122) & (1.862) & (0.101) & (1.153) & (1.603) & (0.1)\\
\hspace{1em}Incitation - Oui $\times$ Quintile de l'âge à la rentrée - Q2 & $-$1.382 & $-$1.85 & $-$0.056 & $-$2.369$^{**}$ & $-$3.079$^{*}$ & 0.048 & $-$1.747 & $-$1.96 & $-$0.014\\
\hspace{1em} & (2.072) & (2.389) & (0.177) & (1.187) & (1.747) & (0.108) & (1.288) & (1.57) & (0.112)\\
\hspace{1em}Incitation - Oui $\times$ Quintile de l'âge à la rentrée - Q3 & 0.693 & 0.229 & $-$0.03 & $-$1.578 & $-$2.075 & $-$0.016 & $-$0.68 & $-$0.964 & 0.13\\
\hspace{1em} & (1.917) & (2.408) & (0.176) & (1.446) & (1.793) & (0.12) & (1.369) & (1.607) & (0.119)\\
\hspace{1em}Incitation - Oui $\times$ Quintile de l'âge à la rentrée - Q4 & 1.364 & 1.232 & 0.161 & 0.065 & $-$0.637 & 0.237$^{**}$ & 0.33 & $-$0.036 & 0.185\\
\hspace{1em} & (1.939) & (2.422) & (0.172) & (1.294) & (1.81) & (0.112) & (1.256) & (1.606) & (0.114)\\
\hspace{1em}Incitation - Oui $\times$ Quintile de l'âge à la rentrée - Q5 & 0.041 & $-$0.259 & 0.006 & $-$0.701 & $-$1.543 & 0.095 & $-$1.435 & $-$2.36 & 0.042\\
\hspace{1em} & (1.997) & (2.5) & (0.174) & (1.116) & (1.913) & (0.097) & (1.254) & (1.617) & (0.104)\\
\hspace{1em} &  &  &  &  &  &  &  &  \vphantom{6} & \\
\hspace{1em}Contrôles & Oui & Oui & Oui & Oui & Oui & Oui & Oui & Oui & \vphantom{3} Oui\\
\hspace{1em}Observations & 218 & 218 & 218 & 225 & 225 & 225 & 260 & 260 & \vphantom{3} 260\\
\hspace{1em} &  &  &  &  &  &  &  &  \vphantom{5} & \\
\addlinespace[0.3em]
\multicolumn{10}{l}{\textbf{Panel D : Hétérogénéité en fonction du sexe}}\\
\hline
\hspace{1em}Incitation - Oui & 1.764$^{**}$ & 1.766$^{*}$ & 0.061 & $-$0.394 & $-$0.549 & $-$0.003 & 0.322 & 0.512 & $-$0.01\\
\hspace{1em} & (0.768) & (0.944) & (0.065) & (0.49) & (0.679) & (0.045) & (0.508) & (0.601) & (0.043)\\
\hspace{1em}Sexe - Homme & 0.917 & 1.188 & 0.091 & $-$0.456 & $-$0.886 & $-$0.057 & 0.501 & 0.829 & 0.026\\
\hspace{1em} & (0.986) & (1.114) & (0.082) & (0.543) & (0.838) & (0.057) & (0.573) & (0.743) & (0.058)\\
\hspace{1em}Incitation - Oui $\times$ Sexe - Homme & $-$0.328 & 0.02 & $-$0.045 & 1.308 & 1.441 & 0.129 & 0.335 & $-$0.188 & 0.053\\
\hspace{1em} & (1.551) & (1.766) & (0.126) & (1.04) & (1.286) & (0.089) & (0.996) & (1.134) & (0.084)\\
\hspace{1em} &  &  &  &  &  &  &  &  \vphantom{4} & \\
\hspace{1em}Contrôles & Oui & Oui & Oui & Oui & Oui & Oui & Oui & Oui & \vphantom{2} Oui\\
\hspace{1em}Observations & 218 & 218 & 218 & 225 & 225 & 225 & 260 & 260 & \vphantom{2} 260\\
\hspace{1em} &  &  &  &  &  &  &  &  \vphantom{3} & \\
\addlinespace[0.3em]
\multicolumn{10}{l}{\textbf{Panel E : Hétérogénéité en fonction du pays de naissance}}\\
\hline
\hspace{1em}Incitation - Oui & 3.792$^{**}$ & 4.627$^{*}$ & 0.168 & 0.617 & 0.592 & 0.231$^{*}$ & 0.693 & 0.996 & 0.103\\
\hspace{1em} & (1.85) & (2.492) & (0.169) & (1.262) & (2.029) & (0.136) & (1.408) & (1.605) & (0.114)\\
\hspace{1em}Pays de naissance - En France & 0.519 & 0.873 & $-$0.143 & $-$0.687 & $-$0.856 & 0.043 & 0.026 & 0.018 & 0.029\\
\hspace{1em} & (1.618) & (1.811) & (0.136) & (0.71) & (1.331) & (0.077) & (0.802) & (1.169) & (0.072)\\
\hspace{1em}Incitation - Oui $\times$ Pays de naissance - En France & $-$2.344 & $-$3.151 & $-$0.133 & $-$0.673 & $-$0.784 & $-$0.213 & $-$0.302 & $-$0.597 & $-$0.108\\
\hspace{1em} & (1.966) & (2.612) & (0.18) & (1.349) & (2.119) & (0.142) & (1.471) & (1.69) & (0.12)\\
\hspace{1em} &  &  &  &  &  &  &  &  \vphantom{2} & \\
\hspace{1em}Contrôles & Oui & Oui & Oui & Oui & Oui & Oui & Oui & Oui & \vphantom{1} Oui\\
\hspace{1em}Observations & 218 & 218 & 218 & 225 & 225 & 225 & 260 & 260 & \vphantom{1} 260\\
\hspace{1em} &  &  &  &  &  &  &  &  \vphantom{1} & \\
\addlinespace[0.3em]
\multicolumn{10}{l}{\textbf{Panel F : Hétérogénéité en fonction du statut de boursier}}\\
\hline
\hspace{1em}Incitation - Oui & 2.559$^{**}$ & 2.804$^{**}$ & 0.137 & 0.296 & 0.051 & 0.073 & 1.002 & 1.116 & 0.052\\
\hspace{1em} & (1.155) & (1.325) & (0.099) & (0.78) & (0.972) & (0.076) & (0.744) & (0.84) & (0.067)\\
\hspace{1em}Statut de boursier - Secondaire & 0.329 & 0.475 & $-$0.01 & $-$0.028 & $-$0.17 & $-$0.001 & 0.528 & 0.761 & 0.041\\
\hspace{1em} & (1.072) & (1.184) & (0.088) & (0.612) & (0.857) & (0.065) & (0.627) & (0.753) & (0.059)\\
\hspace{1em}Incitation - Oui $\times$ Statut de boursier - Secondaire & $-$1.385 & $-$1.588 & $-$0.139 & $-$0.452 & $-$0.282 & $-$0.057 & $-$0.917 & $-$1.04 & $-$0.074\\
\hspace{1em} & (1.402) & (1.649) & (0.119) & (0.917) & (1.237) & (0.087) & (0.911) & (1.064) & (0.077)\\
\hspace{1em} &  &  &  &  &  &  &  &  & \\
\hspace{1em}Contrôles & Oui & Oui & Oui & Oui & Oui & Oui & Oui & Oui & Oui\\
\hspace{1em}Observations & 218 & 218 & 218 & 225 & 225 & 225 & 260 & 260 & 260\\*
\end{longtable}
\end{ThreePartTable}
\endgroup{}
\end{landscape}

\hypertarget{g20ressupp}{%
\subsection{Robustesse des résultats}\label{g20ressupp}}

Dans cette sous-section, nous nous limitons à l'analyse de la robustesse de l'effet considérable des vidéos visionnées entièrement sur la note aux TD et de l'absence de leur effet sur les notes aux examens. Ce cadre délimite nos résultats les plus importants puisque lorsque nous considérons les vidéos et les exercices comme mesure d'utilisation de la plateforme, il est possible que le participant survole rapidement ces vidéos ou ces exercices sans les avoir réellement exploités. Visionner une vidéo entièrement est un signal, bien qu'imparfait, que le participant concerné a effectivement exploité la plateforme. Il aurait été intéressant de pouvoir distinguer les exercices effectués des exercices visionnés. Mais même si nous avions pu les distinguer, nous n'aurions très probablement pas trouvé d'effet considérable vu que les effets estimés des exercices\footnote{Surtout sur les notes de TD mais aussi sur les notes aux examens.} (Panel E du Tableau \ref{tab:g20models}, colonnes 2 et 3) sont très inférieurs à ceux des vidéos (Panel D du Tableau \ref{tab:g20models}, colonnes 2 et 3).

\quad Premièrement, pour se prémunir du biais de spécification, nous montrons dans la Figure \ref{fig:g20modelsrobcovsgraph}\footnote{L'équivalent pour les notes normalisées est la Figure \ref{fig:g20modelsrobcovsgraphnormpop} de l'Annexe \ref{g20modelsrobcovsgraphnormpop}.} qu'en fonction des variables de contrôle utilisées, l'effet de regarder une vidéo entière de plus sur la note aux TD varie légèrement en dessous de 2.5 points sur 20 tandis que son homologue sur la note aux examens est systématiquement proche de 0. Les effets sur la note aux TD ne sont que marginalement significatifs dans les modèles Tobit\footnote{Et marginalement non significatifs dans les modèles linéaires.} mais les ampleurs trouvées nous incitent à parier sur l'existence d'un réel effet positif. Il est d'ailleurs plus raisonnable de se fier aux résultats des modèles Tobit puisque nous prenons en compte les censures des variables dépendantes.\\
Que cela soit dans les modèles linéaires ou Tobit, l'absence d'effet sur la note aux examens est robuste aux variables de contrôles utilisées.

\begin{figure}[H]

{\centering \includegraphics[width=1\linewidth]{000_files/figure-latex/g20modelsrobcovsgraph-1} 

}

\caption{Robustesse des résultats principaux - Vidéos visionnées entièrement}\label{fig:g20modelsrobcovsgraph}
\end{figure}

\quad Deuxièmement, nous pouvons craindre que nos résultats sur les notes aux TD soient uniquement dus aux pics de 20 (Figure \ref{fig:g20notes} de l'Annexe \ref{g20age26aoutnotescompl}). Nous prouvons que cela n'est pas le cas d'abord en constatant que les étudiants ayant obtenus ces notes maximales sont tous non incités et que les résultats sont qualitativement maintenus si nous excluons les étudiants qui ont 20 sur 20 (Panel A du Tableau \ref{tab:g20modelsrob2} de l'Annexe \ref{g20modelsrob2}).

\quad Troisièmement et sans prétendre à l'exhaustivité, nous avons mis à zéro les différentes mesures d'utilisation de la plateforme pour ceux qui n'ont pas activé leur compte (voir Section \ref{g20instproto}) pour ne pas perdre les observations correspondantes. Cela peut biaiser nos résultats vers le bas dans le cas où ces étudiants auraient beaucoup utilisé la plateforme s'ils avaient activé leur compte. Cette possibilité est assez improbable mais reste envisageable dans la mesure où l'étudiant n'a plus consulté ses mails après les étapes importantes de l'inscription via Parcoursup, par exemple. Excepté l'effet légèrement négatif sur la note aux examens (-0.251 point sur 20), nous retrouvons qualitativement les mêmes résultats principaux pour les étudiants connectés uniquement (Panel B du Tableau \ref{tab:g20modelsrob2} de l'Annexe \ref{g20modelsrob2}).

\hypertarget{g20concl}{%
\section{Conclusion}\label{g20concl}}

Dans ce chapitre, nous analysons l'effet de la révision des mathématiques du lycée via une plateforme de cours en ligne sur les performances en mathématiques des étudiants néo-bacheliers en première année de Licence d'Économie-Gestion et d'AES de l'Université de La Réunion. Les performances sont mesurées par les notes aux évaluations de travaux dirigées (TD) et les notes aux examens de mathématiques. Dans notre étude, l'effet d'intérêt sur les notes aux TD peut refléter un effet plus immédiat et pour des épreuves relativement plus faciles que les examens. L'effet sur les notes aux examens reflète un effet relativement à plus long terme (par rapport aux TD) et pour des épreuves dont la difficulté satisfait les standards français de la filière. Parmi celles disponibles, nous pensons que le nombre de vidéos visionnées entièrement est la mesure pertinente de l'intensité d'utilisation de la plateforme. La plateforme a été mise à disposition des participants pendant 4 semaines. Elle n'est accessible que par ordinateur et avec une connexion internet mais les participants peuvent y avoir accès même hors des établissements de l'université.\\
En plus de faire partie des rares évaluations d'impact des interventions en mathématiques dans le supérieur, une de nos principales contributions est la mobilisation d'un protocole d'encouragement pour mesurer les effets susmentionnés. Nous avons en effet constaté dans la littérature un manque d'utilisation des méthodes d'évaluation d'impact se reposant sur la constitution d'un ou plusieurs groupes de contrôle. Le protocole d'encouragement en question consiste à constituer de manière explicitement aléatoire un groupe incité et un groupe non incité. Les incitations sont des mails envoyés à intervalle régulier censés encourager leurs récepteurs à utiliser la plateforme.

\quad Nous constatons d'abord que les participants ont peu utilisé la plateforme proposée. Toutefois, ceux incités ont bien utilisé la plateforme plus que les non incités. Nous justifions également qu'il est peu probable que l'incitation ait eu un effet direct important sur les performances en mathématiques. Ces deux derniers propos nous permettent de s'assurer que l'effet mesuré est bien un effet causal de l'utilisation de la plateforme, même s'il ne concerne que ceux qui ont effectivement réagi à l'incitation.
De manière robuste, regarder plus de vidéos de la plateforme dans leur entièreté a fait grandement augmenter les résultats aux évaluations de TD des étudiants sans avoir aucun effet sur les résultats aux examens de mathématiques. Face à un sens inattendu de la différence entre les coefficients estimés par variable instrumentale et ceux estimés sans pour le cas des TD, nous éliminons rationnellement les raisons possibles derrière ce constat et concluons que c'est lié au fait les effets estimés par variable instrumentale ne valent que pour une petite proportion d'individus qui sont de plus vraisemblablement très motivés mais pas forcément plus forts en mathématiques par rapport au reste des participants. Poussés par leur motivation, ces individus ont alors pu exploiter la plateforme pour obtenir de très bonnes notes aux TD. Vu que l'accès à la plateforme a été retirée des participants quelques mois avant la période des examens et l'absence totale d'effet de l'utilisation de la plateforme sur les notes aux examens, nous affirmons assez assurément que la plateforme, dans les conditions dans lesquelles elle a été mise à disposition des étudiants, n'a pas affecté le capital de connaissances et de compétences en mathématiques des néo-bacheliers de l'Université de La Réunion. Au vu des modèles que nous mobilisons qui prennent en compte le fait que les notes observées aux examens ne reflètent pas tout le spectre du niveau réel des participants, l'absence d'effet sur les résultats aux examens de mathématiques n'est pas dû au niveau de difficulté des examens vraisemblablement perçus difficiles pour les étudiants.

\quad Dans l'objectif d'améliorer les acquis en mathématiques des étudiants en Licence d'Économie-Gestion et d'AES de l'Université de La Réunion, ces résultats impliquent qu'il n'est pas intéressant de proposer la plateforme dans les conditions décrites dans la présente étude. Toutefois, indépendamment de l'évolution future du contenu de la plateforme, il est potentiellement intéressant de réitérer l'expérimentation pour ceux inscrits en première année de Licence Économie-Gestion pendant au moins une dizaine de semaine, durée typique dans la littérature (Section \ref{g20litt}). Avec une telle durée, il est probable d'avoir une proportion plus importante d'individus qui réagissent aux incitations et il y a plus de chance que ces derniers soient représentatifs de l'ensemble des participants. Même si les étudiants n'utilisent la plateforme que dans le but d'obtenir de bonnes notes, cela semble avoir eu un effet sur les notes aux TD. On peut alors espérer un effet de l'utilisation de la plateforme sur les notes aux examens si cette dernière est mise à disposition pendant tout le semestre. Un tel effet, même s'il n'est motivé que par l'ambition de bonnes notes, reste ultimement un apprentissage pour les étudiants et probablement une nouvelle source de motivation.

\hypertarget{conclusion-guxe9nuxe9rale}{%
\chapter*{Conclusion générale}\label{conclusion-guxe9nuxe9rale}}
\addcontentsline{toc}{chapter}{Conclusion générale}

Dans cette thèse, nous avons proposé l'estimation pour La Réunion de trois spécifications de la fonction de production d'éducation pour trois niveaux différents du système éducatif français. La Réunion est à cet égard un objet d'étude intéressant du fait de ses spécificités marquées par rapport à la métropole. En particulier, pour une grande proportion d'élèves, la langue maternelle n'est pas le Français, ce qui entraîne des difficultés d'apprentissage spécifiques, surtout pour les jeunes enfants.

Ces difficultés peuvent amener à s'interroger sur le bien-fondé de la règle unique d'entrée à l'école qui implique que des enfants avec des différences d'âge significatives reçoivent les mêmes enseignements et sont évalués uniformément. Les plus jeunes pourraient alors être fortement désavantagés uniquement à cause de leur âge, que ce soit à l'entrée à l'école, par rapport à leurs camarades de classe ou au moment des évaluations. Nous abordons dans un premier temps le lien entre l'âge et les résultats scolaires à la fin de l'école primaire. Ce niveau est crucial dans la mesure où, à partir du collège, en fonction de leurs résultats en fin de primaire, les élèves peuvent être placés dans des classes assez différentes, même si le programme et les évaluations sont uniformes. Nous analysons ensuite comment ces différences de résultats scolaires dues à des différences d'âge évoluent au cours de la scolarité au collège.

En mobilisant les méthodes qui permettent d'éviter les biais d'estimation dus à l'endogénéité de l'âge, nous trouvons, sur plusieurs cohortes d'élèves réunionnais, qu'être plus âgé au moment des examens est en moyenne un avantage important en termes de résultats scolaires en fin de primaire. L'origine de ce résultat est vraisemblablement le manque de préparation des élèves les plus jeunes par rapport à leurs camarades plus âgés à l'entrée à l'école. En revanche, être plus jeune dans la classe (et non dans la cohorte) se révèle être un avantage pour les élèves. L'effet de l'âge s'atténue, mais demeure, à la fin du collège.

\quad Les collèges étant des établissements de plus grande taille que les écoles primaires, avec en particulier davantage de classes de même niveau, la question de la composition des classes est cruciale, d'autant plus que, par rapport au primaire, le potentiel d'interactions des élèves entre eux et avec les enseignants est plus important. Ces éléments nous ont amené à nous intéresser, dans le second chapitre de la thèse, aux effets de pair sur les résultats aux examens nationaux en fin de collège.

L'estimation de l'effet causal du niveau d'entrée et du comportement des pairs sur les résultats scolaires est compliquée par le fait que les élèves ne sont pas affectés de manière aléatoire dans les différentes établissements et les différentes classes. En mobilisant les techniques permettant de prendre en compte l'endogénéité de la composition des pairs, nous trouvons que la présence de pairs de niveaux élevé est bénéfique pour les élèves plus faibles, sauf pour les élèves en très grande difficulté scolaire. De manière symétrique, la présence d'élèves en grande difficulté scolaire a un effet négatif sur les autres élèves, sauf pour ceux ayant un niveau très élevé. Ce premier type d'effet de pairs est particulièrement accentué en mathématiques. Les effets du comportement des pairs que nous trouvons sont positifs et importants. Ces effets de pairs sont particulièrement importants pour les établissements classés en éducation prioritaire, ce qui semble indiquer que les élèves socialement moins favorisés sont davantage dépendants du contexte scolaire pour leur apprentissage. Enfin, ces effets de pairs semblent, au cours de la scolarité au collège, s'atténuer pour le français et augmenter pour les mathématiques.

\quad L'analyse de l'effet de l'âge et des pairs a montré un problème spécifique au niveau de l'apprentissage des mathématiques, qui peut provoquer des inégalités importantes à plus long terme entre les élèves, du fait de la différentiation des parcours à partir du lycée. Plus généralement, l'insuffisance du niveau en mathématique des étudiants à l'entrée à l'université est un problème bien connu. Cela nous a amené à tenter d'évaluer dans le troisième chapitre l'efficacité d'une plateforme permettant aux élèves de renforcer leur niveau en mathématiques au cours des vacances précédant leur entrée à l'université. La démarche de participation à cette plateforme étant volontaire, la mesure de l'effet causal de l'utilisation de la plateforme sur les résultats ultérieurs à l'université est compliquée par un problème de biais de sélection. Nous traitons ce problème d'endogénéité en donnant à une partie des étudiants sélectionnés de manière aléatoire une incitation spécifique à utiliser la plateforme. Nos résultats montrent que très peu d'élèves ont utilisé la plateforme (des étudiants motivés, mais d'un niveau pas particulièrement élevé en mathématiques) et que l'utilisation de la plateforme n'a pas eu d'effets notables sur leurs résultats ultérieurs dans les examens de mathématiques.

Nos résultats suggèrent qu'une mesure d'avancement de la date seuil pour entrer en primaire pourrait être profitable pour les élèves réunionnais puisque cela impliquerait une augmentation de l'âge moyenne d'entrée à l'école (des élèves en moyenne plus prêts) et de celui des camarades de classe (ce apparaît comme avantageux). Cependant, une mesure de ce type se décide au niveau national. Or, les résultats rapportés par Grenet (\protect\hyperlink{ref-GRE:10}{2010}), des effets plus forts chez les élèves désavantagés pour les résultats scolaires au tout début du scolaire ne plaident pas pour une mesure de ce type à l'échelle de la France. Dans ces conditions, une politique plus modeste consisterait à informer
davantage les parents et les enseignants de l'importance de l'âge d'entrée à l'école primaire des enfants. Les résultats obtenus sur les effets de pairs conduirait à préconiser une répartition uniforme des élèves les plus forts dans les différentes classes et un regroupement des élèves en grande difficulté dans une même classe. Cette dernière préconisation doit bien entendu être considérée avec une grande prudence, dans la mesure où elle est susceptible de générer davantage de ségrégation sociale. Enfin, les résultats décevants obtenus concernant l'utilisation d'une plateforme de révision en mathématiques nous conduisent à préconiser une intervention de plus grande ampleur, en particulier sur une période plus importante.

\quad Comme tout travail de recherche, cette thèse présente des limites, dont nous nous servons pour identifier plusieurs perspectives de recherches ultérieures.

En particulier, il serait intéressant d'étendre l'analyse de l'effet de l'âge sur le parcours scolaire (redoublement, filière choisie au lycée et dans le supérieur) et sur les compétences non cognitives avec les mêmes méthodes et les mêmes spécifications. Nous pensons en effet que le maintien de l'effet de l'âge au collège peut sûrement s'expliquer par des phénomènes de démotivation provoqués par une perte de confiance, un rejet de l'école, etc.

Concernant la mesure des effets de pair, il aurait été intéressant d'étendre l'analyse au primaire. Cela nécessiterait de disposer de résultats à des évaluations antérieures au CM2 et l'exercice est compliqué par le fait qu'une grande partie des écoles primaires n'ont qu'une classe pour chaque niveau. Notre analyse des effets de pair mériterait également d'être étendue aux compétences non-cognitive des élèves.

Enfin, concernant l'intérêt des plateforme de révision à l'entrée à l'université, les résultats décevants que nous trouvons mériteraient d'être confirmés ou infirmés dans le cadre d'un échantillon de plus grande taille et en étendant l'analyse à d'autres disciplines que les mathématiques, et en particulier au français. De plus, la connaissance que nous avons des étudiants ayant participé à l'expérience nous montrent que des effets de pairs mériteraient également d'être pris en compte au niveau de l'utilisation de la plateforme.

\quad Au total, avant cette thèse, peu d'études formelles mobilisant les outils modernes permettant une mesure robuste de l'effet causal avaient été menées concernant la fonction de production d'éducation à La Réunion, alors que nous avons montré que ce territoire présente des caractéristiques qui rendent ce type d'études particulièrement intéressantes. Nos résultats contribuent modestement à la connaissance des déterminants de la réussite scolaire à La Réunion. Beaucoup de points mériteraient une analyse plus approfondie afin de pouvoir proposer des politiques publiques susceptibles d'améliorer les performances du système éducatif à La Réunion, et plus généralement dans les territoires français d'outre-mer.

\hypertarget{appendix-annexes}{%
\appendix}


\newpage

\newcommand{\smark}{\sectionmark}
\renewcommand\smark[1]{\markboth{#1}{}}
\markboth{}{}

\centering \LARGE

\pagebreak
\hspace{0pt}
\vfill

\textbf{Annexes}
\vfill
\hspace{0pt}
\pagebreak

\raggedright \normalsize

\hypertarget{annexes-au-chapitre-refage}{%
\chapter*{Annexes au Chapitre \ref{age}}\label{annexes-au-chapitre-refage}}
\addcontentsline{toc}{chapter}{Annexes au Chapitre \ref{age}}

\renewcommand*{\theHchapter}{\thechapter}
\renewcommand*{\thesection}{\Alph{section}}
\renewcommand*{\theHsection}{Appendix.\thechapter\thesection}

\renewcommand*{\thetable}{\Alph{section}.\arabic{table}}
\renewcommand*{\theHtable}{Appendix.\thetable}

\renewcommand*{\thefigure}{\Alph{section}.\arabic{figure}}
\renewcommand*{\theHfigure}{Appendix.\thefigure}

\setcounter{section}{0}

\setcounter{table}{0}
\setcounter{figure}{0}

\hypertarget{ageconstats}{%
\section{Les fichiers constats et récupération d'informations correspondantes}\label{ageconstats}}

Les fichiers constat sont des fichiers établis en début d'année scolaire et qui couvrent tous les établissements publics et privés sous contrat de La Réunion. Elles contiennent des informations sur l'établissement, la classe et l'individu. On n'y retrouve pas de variables de performances scolaires.
Nous disposons de deux catégories de fichiers constat : ceux de 6\textsuperscript{ème} et ceux de 3\textsuperscript{ème}. Les fichiers constat en 6\textsuperscript{ème} à notre disposition couvrent les années scolaires allant de 2010-2011 à 2012-2013 tandis que ceux en 3\textsuperscript{ème} couvrent les années scolaires allant de 2013-2014 à 2016-2017. Les débuts de ces années correspondent aux fins des années des fichiers d'évaluation de CM2 et de DNB, respectivement.

\quad Les informations au niveau individuel disponibles dans les fichiers constat sont l'INE (Identifiant National de l'Élève), la date de naissance exacte, le sexe, la catégorie sociale en deux chiffres, la commune d'origine, le régime scolaire, le régime de transport de transport, le statut de boursier, des identifiants et libellés de niveau-option (exemple : 6\textsuperscript{ème} internationale) et les différentes options de langue vivante choisies par l'élève.

\quad Au niveau de la classe, seuls les identifiants de la classe (en début d'année, au collège, pour le niveau considéré).

\quad Au niveau de l'école, nous avons les identifiants d'école et le statut (public ou privé).

\quad Au CM2, nous récupérons la CSP au sein des fichiers constat de 6\textsuperscript{ème} via l'INE. Tous les INE des élèves de CM2 ne sont pas retrouvés dans les fichiers constat de 6\textsuperscript{ème}. Une exception à cette récupération est l'année 2011-2012 : les correspondances entre l'INE et les CSP au sein des fichiers constat de 6\textsuperscript{ème} 2011-2012 apparaissent douteuses. Pour le savoir, nous avons régressé (non reporté), par année scolaire au CM2, les notes en fonction des CSP récupérées au sein des fichiers constat de 6\textsuperscript{ème}. Nous trouvons que les coefficients en 2011 sont très différents des coefficients des deux autres années, tant en ampleur qu'en significativité.\\
Pour l'année de 2011, nous récupérons alors les CSP au sein des fichiers constat de 3\textsuperscript{ème}. Cela suppose que nous possédons une erreur de mesure de la CSP pour l'année de 2011. Cette erreur de mesure est très probablement négligeable car il est difficile de penser qu'un changement structurel des CSP puisse se produire sur le territoire entre le CM2 et la 3\textsuperscript{ème}.

Il n'y a pas eu de problème sur la récupération de la CSP des candidats du DNB.

\quad Le Tableau \ref{tab:agerecuppcs} illustre, pour chaque année au CM2 et au DNB des fichiers d'examens la récupération des informations sur la CSP.

\begingroup\fontsize{8}{10}\selectfont

\begin{ThreePartTable}
\begin{TableNotes}
\item \textit{Sources :} Fichiers CM2 (2009 à 2012), Fichiers DNB (2014 à 2016), Fichiers CONSTAT 6\textsuperscript{ème} et 3\textsuperscript{ème}, calculs de l'auteur.
\item \textit{Notes :} CM2 : Cours Moyen 2\textsuperscript{ème} année. DNB : Diplôme National du Brevet. CSP : Catégorie Socio-Professionnelle. INE : Identifiant National Élève.
\end{TableNotes}
\begin{longtable}[t]{lrlr}
\caption{\label{tab:agerecuppcs}Récupération des CSP}\\
\toprule
Année & Proportion d'INE manquants & Source & \makecell[l]{Proportion de CSP manquants \\ (Parmi les INE non manquants)}\\
\midrule
\endfirsthead
\caption[]{\label{tab:agerecuppcs}Récupération des CSP (suite)}\\
\toprule
Année & Proportion d'INE manquants & Source & \makecell[l]{Proportion de CSP manquants \\ (Parmi les INE non manquants)}\\
\midrule
\endhead

\endfoot
\bottomrule
\insertTableNotes
\endlastfoot
\addlinespace[0.3em]
\multicolumn{4}{l}{\textbf{CM2}}\\
\hspace{1em}2009 & 1.000 & - & -\\
\hspace{1em}2010 & 0.165 & Constat de 6e & 0\\
\hspace{1em}2011 & 0.139 & Constat de 3e & 0.036\\
\hspace{1em}2012 & 0.135 & Constat de 6e & 0\\
\addlinespace[0.3em]
\multicolumn{4}{l}{\textbf{DNB}}\\
\hspace{1em}2014 & 0.000 & Constat de 3e & 0\\
\hspace{1em}2015 & 0.000 & Constat de 3e & 0\\
\hspace{1em}2016 & 0.000 & Constat de 3e & 0\\*
\end{longtable}
\end{ThreePartTable}
\endgroup{}

\newpage  
\setcounter{table}{0}
\setcounter{figure}{0}

\hypertarget{agecorrespcsp}{%
\section{Correspondances entre la classification en 4 postes de la catégorie socio-professionnelle et la version détaillée}\label{agecorrespcsp}}

\begingroup\fontsize{8}{10}\selectfont

\begin{ThreePartTable}
\begin{TableNotes}
\item Source : Métayer et al. 2017, p. 124
\end{TableNotes}
\begin{longtable}[t]{ll}
\caption{\label{tab:agecorrespcsp}Correspondances entre la CSP en 4 postes et sa version détaillée}\\
\toprule
Version groupée & Version détaillée\\
\midrule
\endfirsthead
\caption[]{\label{tab:agecorrespcsp}Correspondances entre la CSP en 4 postes et sa version détaillée (suite)}\\
\toprule
Version groupée & Version détaillée\\
\midrule
\endhead

\endfoot
\bottomrule
\insertTableNotes
\endlastfoot
\addlinespace[0.3em]
\multicolumn{2}{l}{\textbf{}}\\
Défavorisée & Ouvriers qualifiés\\
 & Ouvriers non qualifiés\\
 & Ouvriers agricoles\\
 & Anciens employés et ouvriers\\
 & Chercheurs n'ayant jamais travaillé\\
 & Inactifs divers (autres que retraités)\\
\addlinespace[0.3em]
\multicolumn{2}{l}{\textbf{}}\\
Moyenne & Agriculteurs exploitants\\
 & Artisans\\
 & Commerçants et assimilés\\
 & \makecell[tl]{Employés civils et agents de service de la fonction publique}\\
 & Agents de surveillance\\
 & Employés administratifs d'entreprise\\
 & Employés de commerce\\
 & Personnels des services directs aux particuliers\\
 & Anciens agriculteurs exploitants\\
 & Anciens artisans, commerçants, chefs d'entreprise\\
\addlinespace[0.3em]
\multicolumn{2}{l}{\textbf{}}\\
Favorisée & \makecell[tl]{Professions intermédiaires de la santé et du travail social}\\
 & Clergé, religieux\\
 & \makecell[tl]{Professions intermédiaires administratives de la fonction publique}\\
 & \makecell[tl]{Professions intermédiaires administratives \\ et commerciales des entreprises}\\
 & Techniciens\\
 & Contremaîtres, agents de maîtrise\\
 & Anciens cadres et professions intermédiaires\\
\addlinespace[0.3em]
\multicolumn{2}{l}{\textbf{}}\\
Très favorisée & Chefs d'entreprise de 10 salariés ou plus\\
 & Professions libérales et assimilés\\
 & \makecell[tl]{Cadres de la fonction publique}\\
 & \makecell[tl]{Professeurs, professions scientifiques}\\
 & \makecell[tl]{Professions de l'information, des arts et des spectacles}\\
 & \makecell[tl]{Cadres administratifs et commerciaux d'entreprises}\\
 & \makecell[tl]{Ingénieurs et cadres techniques d'entreprises}\\
 & \makecell[tl]{Professeurs des écoles, instituteurs et professions assimilées}\\*
\end{longtable}
\end{ThreePartTable}
\endgroup{}

\newpage
\setcounter{table}{0}
\setcounter{figure}{0}

\hypertarget{agechisqsupp}{%
\section{\texorpdfstring{Tests de \(\chi^2\) supplémentaires}{Tests de \textbackslash chi\^{}2 supplémentaires}}\label{agechisqsupp}}

\begingroup\fontsize{8}{10}\selectfont

\begin{ThreePartTable}
\begin{TableNotes}
\item \textit{Source :} Fichiers CM2 (2010 à 2012), calculs de l'auteur.
\end{TableNotes}
\begin{longtable}[t]{lrr}
\caption{\label{tab:agechisqsupp}Tests supplémentaires d'indépendance entre le mois de naissance et la catégorie sociale
        }\\
\toprule
Année & Statistique de $\chi^2$ & Probabilité critique\\
\midrule
\endfirsthead
\caption[]{\label{tab:agechisqsupp}Tests supplémentaires d'indépendance entre le mois de naissance et la catégorie soci (suite)}\\
\toprule
Année & Statistique de $\chi^2$ & Probabilité critique\\
\midrule
\endhead

\endfoot
\bottomrule
\insertTableNotes
\endlastfoot
\addlinespace[0.3em]
\multicolumn{3}{l}{\textbf{Spécification de la CSP : Autres et valeurs manquantes comme modalité}}\\
\hspace{1em}2010 & 63.75 & 0.20\\
\hspace{1em}2011 & 46.01 & 0.80\\
\hspace{1em}2012 & 43.53 & 0.87\\
\addlinespace[0.3em]
\multicolumn{3}{l}{\textbf{Spécification de la CSP : sans Autres et valeurs manquantes comme modalité}}\\
\hspace{1em}2010 & 42.93 & 0.12\\
\hspace{1em}2011 & 31.62 & 0.54\\
\hspace{1em}2012 & 26.86 & 0.77\\*
\end{longtable}
\end{ThreePartTable}
\endgroup{}

\setcounter{table}{0}
\setcounter{figure}{0}

\hypertarget{agewbootdesc}{%
\section{\texorpdfstring{Description du calcul d'écart-type par \emph{bootstrap}}{Description du calcul d'écart-type par bootstrap}}\label{agewbootdesc}}

Considérons l'équation \eqref{eq:agecfh}. \(\hat{\nu}_{i}\) est une variable issue d'une première étape. Il est de connaissance commune qu'une inférence standard n'est pas valide dans ce type de cas (\protect\hyperlink{ref-WOO:15}{Wooldridge, 2015}). Ainsi,nous calculons les écart-types des coefficients estimés par fonction de contrôle par \emph{bootstrap}. Plus particulièrement, nous suivons une procédure bien précise (parmi plusieurs) proposée par Davidson \& Flachaire (\protect\hyperlink{ref-DAV:FLA:08}{2008}) : le \emph{wild bootstrap}, qui est robuste à l'hétéroscédasticité de forme inconnue. De manière générale, il s'agit de réestimer un grand nombre de fois l'équation \eqref{eq:agecfh} à chaque fois avec les mêmes variables explicatives mais différents vecteurs de variables dépendantes. Ces vecteurs sont eux-mêmes obtenus à partir des valeurs prédites et résidus transformés issus d'une première estimation standard de l'équation \eqref{eq:agecfh}. Au bout, nous obtenons une distribution empirique de paramètres estimés de laquelle nous pouvons calculer l'écart-type pour en faire une inférence.

\quad Formellement, considérons une première estimation par moindre carrés ordinaires de l'équation \eqref{eq:agecfh}. Considérons ensuite le vecteur de résidus associé : \(\hat{a}_{i}\). Désignons ensuite par \(r()\) la fonction de masse de la loi discrète de \emph{Rademacher} :

\[
r(x) = 
\begin{cases} 
x = - 1 \text{ avec probabilité } \frac{1}{2} \\
x = 1 \text{ avec probabilité } \frac{1}{2} 
\end{cases}
\]

Définissons ensuite un nombre de réplications \(B\). Nous considérons \(B = 1001\). À chaque réplication \(b = \{1, 2, ..., B\}\), un résidu transformé \(\hat{a}^b_i = \hat{a}_i r(x), \forall i\) est calculé. Ce résidu transformé est utilisé pour obtenir le nouveau vecteur de variable dépendante composée de \(\hat{y}^b_i = y_i + \hat{a}^b_i, \forall i\) où \(y_i\) est la valeur observée de la variable indépendante pour l'individu \(i\). Une nouvelle estimation de l'équation \eqref{eq:agecfh} est ensuite effectuée en utilisant la variable dépendante composée de \(\hat{y}^b_i\) et les mêmes variables explicatives et données que dans l'estimation principale.\\
En tout, on obtient alors \(B\) valeurs de \(\hat{\alpha}_1^b, b = \{1, 2, ..., B\}\). Pour effectuer les tests d'hypothèse, L'écart-type estimé du paramètre estimé \(\hat{\alpha_1}\) retenu est alors l'écart-type de la distribution empirique de \(\hat{\alpha}_1^b, b = \{1, 2, ..., B\}\).

\newpage
\setcounter{table}{0}
\setcounter{figure}{0}

\hypertarget{ageheterofac}{%
\section{Hétérogénéité des effets de l'âge aux examens par rapport aux facultés}\label{ageheterofac}}

À partir de l'équation estimable de l'approche par fonction de contrôle (équation \ref{eq:agecfh}), nous pouvons écrire

\begin{equation}
\label{eq:agecfhest}
  y_i = \hat{\alpha_0} + \hat{\alpha_1} a_i + x'_i \hat{\alpha_2} + \hat{c_1} \hat{\nu_i} + \hat{c_2} \hat{\nu_i} a_i + \hat{\epsilon_i}.
\end{equation}

Dans notre étude, nous trouvons \(\hat{c_1} < 0\) et \(\hat{c_2} < 0\) (Tableau \ref{tab:agemodels}, colonne 5). Il est important de comprendre que cela pourrait en être autrement dans d'autres applications empiriques sur les effets de l'âge.

D'une part, selon l'équation \eqref{eq:agecfhest}, la variation de la note en fonction de \(\hat{\nu_i}\) est

\[
\frac{\partial y_i}{\partial \hat{\nu_i}} = \hat{c_1} + \hat{c_2} a_i.
\]

Puisque \(\hat{c_1} < 0\) et \(\hat{c_2} < 0\), combiné au fait que \(a_i > 0\), nous en déduisons que

\begin{equation}
\label{eq:agesignefac}
\frac{\partial y_i}{\partial \hat{\nu_i}} < 0.
\end{equation}

L'équation \eqref{eq:agesignefac} nous permet d'interpréter \(\hat{\nu_i}\) comme une variable d'\emph{anti-faculté} ou des difficultés scolaires, pour le dire plus simplement. Cela implique que \(- \nu_i\) peut être interprétée pratiquement comme une variable de faculté.

\quad D'autre part, l'équation \eqref{eq:agecfhest} nous permet également d'écrire l'effet de l'âge aux examens sur la note totale comme suit :

\[
\frac{\partial y_i}{\partial a_i} = \hat{\alpha_1} + \hat{c_2} \hat{\nu_i} = \hat{\alpha_1} - \hat{c_2}(- \hat{\nu_i}). 
\]
La variation de cet effet en fonction des facultés est alors

\[
\frac{\partial ^ 2 y_i}{\partial (\hat{-\nu_i}) \partial a_i} = - \hat{c_2}.
\]
Comme \(\hat{c_2} < 0\),

\[
\frac{\partial ^ 2 y_i}{\partial (-\hat{\nu_i}) \partial a_i} > 0.
\]
Autrement dit, nos résultats indiquent que l'effet de l'âge aux examens augmente avec les facultés.

\blandscape

\setcounter{table}{0}
\setcounter{figure}{0}

\hypertarget{agemodelsssitems}{%
\section{Résultats sur les notes en sous-composantes de français et de mathématiques}\label{agemodelsssitems}}

\hypertarget{effets-homoguxe8nes-par-rapport-aux-observables}{%
\subsection{Effets homogènes par rapport aux observables}\label{effets-homoguxe8nes-par-rapport-aux-observables}}

\begingroup\fontsize{8}{10}\selectfont

\begin{ThreePartTable}
\begin{TableNotes}
\item \textit{Sources :} Fichiers CM2 (2009 à 2012), calculs de l'auteur.
\item \textit{Notes :} Une colonne correspond à une régression. La note est normalisée sur l'année scolaire. Écart-types entre parenthèses. Les écart-types des estimations par fonction de contrôle sont calculés par wild bootstrap avec 1001 réplications. La variable $\hat{\nu}$ est le résidu de la première étape. Les contrôles utilisés sont le sexe, la CSP et l'année scolaire.
\item VI : Variable Instrumentale. FCH : Fonction de Contrôle avec prise en compte de l'Hétérogénéité de l'effet de l'âge.
\item Significativité : 10\% * 5\% ** 1\% ***.
\end{TableNotes}
\begin{longtable}[t]{lllllllllll}
\caption{\label{tab:agemodelsssitemsfrench}Résultats principaux, sous-items de français}\\
\toprule
\multicolumn{1}{c}{} & \multicolumn{10}{c}{Variable dépendante : Note en } \\
\cmidrule(l{3pt}r{3pt}){2-11}
\multicolumn{1}{c}{} & \multicolumn{2}{c}{Écriture} & \multicolumn{2}{c}{Grammaire} & \multicolumn{2}{c}{Lecture} & \multicolumn{2}{c}{Orthographe} & \multicolumn{2}{c}{Vocabulaire} \\
\cmidrule(l{3pt}r{3pt}){2-3} \cmidrule(l{3pt}r{3pt}){4-5} \cmidrule(l{3pt}r{3pt}){6-7} \cmidrule(l{3pt}r{3pt}){8-9} \cmidrule(l{3pt}r{3pt}){10-11}
 & \makecell{VI \\ (1) } & \makecell{FCH \\ (2) } & \makecell{VI \\ (3) } & \makecell{FCH \\ (4) } & \makecell{VI \\ (5) } & \makecell{FCH \\ (6) } & \makecell{VI \\ (7) } & \makecell{FCH \\ (8) } & \makecell{VI \\ (9) } & \makecell{FCH \\ (10) }\\
\midrule
\endfirsthead
\caption[]{\label{tab:agemodelsssitemsfrench}Résultats principaux, sous-items de français (suite)}\\
\toprule
 & \makecell{VI \\ (1) } & \makecell{FCH \\ (2) } & \makecell{VI \\ (3) } & \makecell{FCH \\ (4) } & \makecell{VI \\ (5) } & \makecell{FCH \\ (6) } & \makecell{VI \\ (7) } & \makecell{FCH \\ (8) } & \makecell{VI \\ (9) } & \makecell{FCH \\ (10) }\\
\midrule
\endhead

\endfoot
\bottomrule
\insertTableNotes
\endlastfoot
Âge aux examens & 0.256$^{***}$ & 0.253$^{***}$ & 0.257$^{***}$ & 0.256$^{***}$ & 0.28$^{***}$ & 0.276$^{***}$ & 0.239$^{***}$ & 0.237$^{***}$ & 0.256$^{***}$ & 0.252$^{***}$\\
 & (0.018) & (0.017) & (0.018) & (0.017) & (0.018) & (0.017) & (0.018) & (0.017) & (0.018) & (0.017)\\
$\hat{\nu}$ & - & $-$0.208 & - & $-$0.716$^{***}$ & - & $-$0.125 & - & $-$0.516$^{***}$ & - & $-$0.05\\
 & - & (0.139) & - & (0.148) & - & (0.15) & - & (0.14) & - & (0.152)\\
Âge aux examens $\times$ $\hat{\nu}$ & - & $-$0.066$^{***}$ & - & $-$0.021 & - & $-$0.078$^{***}$ & - & $-$0.046$^{***}$ & - & $-$0.083$^{***}$\\
 & - & (0.012) & - & (0.013) & - & (0.013) & - & (0.012) & - & (0.013)\\
 &  &  &  &  &  &  &  &  &  & \\
Contrôles & Oui & Oui & Oui & Oui & Oui & Oui & Oui & Oui & Oui & Oui\\
Observations & 54341 & 54341 & 54341 & 54341 & 54341 & 54341 & 54341 & 54341 & 54341 & 54341\\
R$^2$ ajusté & - & 0.18 & - & 0.163 & - & 0.176 & - & 0.206 & - & 0.174\\*
\end{longtable}
\end{ThreePartTable}
\endgroup{}

\newpage

\begingroup\fontsize{8}{10}\selectfont

\begin{ThreePartTable}
\begin{TableNotes}
\item \textit{Sources :} Fichiers CM2 (2009 à 2012), calculs de l'auteur.
\item \textit{Notes :} Une colonne correspond à une régression. La note est normalisée sur l'année scolaire. Écart-types entre parenthèses. Les écart-types des estimations par fonction de contrôle sont calculés par wild bootstrap avec 1001 réplications. La variable $\hat{\nu}$ est le résidu de la première étape. Les contrôles utilisés sont le sexe, la CSP et l'année scolaire.
\item VI : Variable Instrumentale. FCH : Fonction de Contrôle avec prise en compte de l'Hétérogénéité de l'effet de l'âge.
\item Significativité : 10\% * 5\% ** 1\% ***.
\end{TableNotes}
\begin{longtable}[t]{lllllllllll}
\caption{\label{tab:agemodelsssitemsmaths}Résultats principaux, sous-items de mathématiques}\\
\toprule
\multicolumn{1}{c}{} & \multicolumn{10}{c}{Variable dépendante : Note en } \\
\cmidrule(l{3pt}r{3pt}){2-11}
\multicolumn{1}{c}{} & \multicolumn{2}{c}{Calcul} & \multicolumn{2}{c}{Géométrie} & \multicolumn{2}{c}{\makecell{Grandeurs et \\ mesures}} & \multicolumn{2}{c}{Nombre} & \multicolumn{2}{c}{\makecell{Organisation et \\ gestion de données}} \\
\cmidrule(l{3pt}r{3pt}){2-3} \cmidrule(l{3pt}r{3pt}){4-5} \cmidrule(l{3pt}r{3pt}){6-7} \cmidrule(l{3pt}r{3pt}){8-9} \cmidrule(l{3pt}r{3pt}){10-11}
 & \makecell{VI \\ (1) } & \makecell{FCH \\ (2) } & \makecell{VI \\ (3) } & \makecell{FCH \\ (4) } & \makecell{VI \\ (5) } & \makecell{FCH \\ (6) } & \makecell{VI \\ (7) } & \makecell{FCH \\ (8) } & \makecell{VI \\ (9) } & \makecell{FCH \\ (10) }\\
\midrule
\endfirsthead
\caption[]{\label{tab:agemodelsssitemsmaths}Résultats principaux, sous-items de mathématiques (suite)}\\
\toprule
 & \makecell{VI \\ (1) } & \makecell{FCH \\ (2) } & \makecell{VI \\ (3) } & \makecell{FCH \\ (4) } & \makecell{VI \\ (5) } & \makecell{FCH \\ (6) } & \makecell{VI \\ (7) } & \makecell{FCH \\ (8) } & \makecell{VI \\ (9) } & \makecell{FCH \\ (10) }\\
\midrule
\endhead

\endfoot
\bottomrule
\insertTableNotes
\endlastfoot
Âge aux examens & 0.262$^{***}$ & 0.259$^{***}$ & 0.281$^{***}$ & 0.279$^{***}$ & 0.315$^{***}$ & 0.314$^{***}$ & 0.228$^{***}$ & 0.226$^{***}$ & 0.277$^{***}$ & 0.278$^{***}$\\
 & (0.019) & (0.017) & (0.019) & (0.018) & (0.019) & (0.018) & (0.019) & (0.018) & (0.019) & (0.017)\\
$\hat{\nu}$ & - & $-$0.418$^{***}$ & - & $-$0.156 & - & $-$0.805$^{***}$ & - & $-$0.459$^{***}$ & - & $-$0.946$^{***}$\\
 & - & (0.147) & - & (0.143) & - & (0.14) & - & (0.143) & - & (0.151)\\
Âge aux examens $\times$ $\hat{\nu}$ & - & $-$0.046$^{***}$ & - & $-$0.052$^{***}$ & - & $-$0.007 & - & $-$0.032$^{**}$ & - & 0.006\\
 & - & (0.013) & - & (0.013) & - & (0.012) & - & (0.013) & - & (0.013)\\
 &  &  &  &  &  &  &  &  &  & \\
Contrôles & Oui & Oui & Oui & Oui & Oui & Oui & Oui & Oui & Oui & Oui\\
Observations & 54341 & 54341 & 54341 & 54341 & 54341 & 54341 & 54341 & 54341 & 54341 & 54341\\
R$^2$ ajusté & - & 0.133 & - & 0.072 & - & 0.114 & - & 0.106 & - & 0.123\\*
\end{longtable}
\end{ThreePartTable}
\endgroup{}

\elandscape

\blandscape

\hypertarget{agemodelssexessitems}{%
\subsection{Effets hétérogènes selon le sexe}\label{agemodelssexessitems}}

\begingroup\fontsize{8}{10}\selectfont

\begin{ThreePartTable}
\begin{TableNotes}
\item \textit{Sources :} Fichiers CM2 (2009 à 2012), calculs de l'auteur.
\item \textit{Notes :} Une colonne correspond à une régression. La note est normalisée sur l'année scolaire. Écart-types entre parenthèses. Les écart-types des estimations par fonction de contrôle sont calculés par wild bootstrap avec 1001 réplications. Les contrôles utilisés sont le sexe, la CSP et l'année scolaire.
\item VI : Variable Instrumentale. FCH : Fonction de Contrôle avec prise en compte de l'Hétérogénéité de l'effet de l'âge.
\item Significativité : 10\% * 5\% ** 1\% ***.
\end{TableNotes}
\begin{longtable}[t]{lllllllllll}
\caption{\label{tab:agemodelssexessitemsfrench}Résultats avec effets hétérogènes selon le sexe, sous-items de français}\\
\toprule
\multicolumn{1}{c}{} & \multicolumn{10}{c}{Variable dépendante : Note en } \\
\cmidrule(l{3pt}r{3pt}){2-11}
\multicolumn{1}{c}{} & \multicolumn{2}{c}{Écriture} & \multicolumn{2}{c}{Grammaire} & \multicolumn{2}{c}{Lecture} & \multicolumn{2}{c}{Orthographe} & \multicolumn{2}{c}{Vocabulaire} \\
\cmidrule(l{3pt}r{3pt}){2-3} \cmidrule(l{3pt}r{3pt}){4-5} \cmidrule(l{3pt}r{3pt}){6-7} \cmidrule(l{3pt}r{3pt}){8-9} \cmidrule(l{3pt}r{3pt}){10-11}
 & \makecell{VI \\ (1) } & \makecell{FCH \\ (2) } & \makecell{VI \\ (3) } & \makecell{FCH \\ (4) } & \makecell{VI \\ (5) } & \makecell{FCH \\ (6) } & \makecell{VI \\ (7) } & \makecell{FCH \\ (8) } & \makecell{VI \\ (9) } & \makecell{FCH \\ (10) }\\
\midrule
\endfirsthead
\caption[]{\label{tab:agemodelssexessitemsfrench}Résultats avec effets hétérogènes selon le sexe, sous-items de français (suite)}\\
\toprule
\multicolumn{1}{c}{} & \multicolumn{10}{c}{Variable dépendante : Note en } \\
\cmidrule(l{3pt}r{3pt}){2-11}
\multicolumn{1}{c}{} & \multicolumn{2}{c}{Écriture} & \multicolumn{2}{c}{Grammaire} & \multicolumn{2}{c}{Lecture} & \multicolumn{2}{c}{Orthographe} & \multicolumn{2}{c}{Vocabulaire} \\
\cmidrule(l{3pt}r{3pt}){2-3} \cmidrule(l{3pt}r{3pt}){4-5} \cmidrule(l{3pt}r{3pt}){6-7} \cmidrule(l{3pt}r{3pt}){8-9} \cmidrule(l{3pt}r{3pt}){10-11}
 & \makecell{VI \\ (1) } & \makecell{FCH \\ (2) } & \makecell{VI \\ (3) } & \makecell{FCH \\ (4) } & \makecell{VI \\ (5) } & \makecell{FCH \\ (6) } & \makecell{VI \\ (7) } & \makecell{FCH \\ (8) } & \makecell{VI \\ (9) } & \makecell{FCH \\ (10) }\\
\midrule
\endhead

\endfoot
\bottomrule
\insertTableNotes
\endlastfoot
Âge aux examens & 0.235$^{***}$ & 0.249$^{***}$ & 0.26$^{***}$ & 0.29$^{***}$ & 0.281$^{***}$ & 0.314$^{***}$ & 0.248$^{***}$ & 0.274$^{***}$ & 0.25$^{***}$ & 0.259$^{***}$\\
 & (0.024) & (0.02) & (0.026) & (0.022) & (0.025) & (0.022) & (0.025) & (0.022) & (0.025) & (0.022)\\
Âge aux examens $\times$ Sexe - Garçon & 0.044 & 0.008 & $-$0.006 & $-$0.066$^{**}$ & $-$0.003 & $-$0.075$^{***}$ & $-$0.018 & $-$0.074$^{***}$ & 0.012 & $-$0.013\\
 & (0.037) & (0.027) & (0.037) & (0.028) & (0.037) & (0.027) & (0.036) & (0.027) & (0.037) & (0.027)\\
 &  &  &  &  &  &  &  &  &  & \\
Contrôles & Oui & Oui & Oui & Oui & Oui & Oui & Oui & Oui & Oui & Oui\\
Observations & 54341 & 54341 & 54341 & 54341 & 54341 & 54341 & 54341 & 54341 & 54341 & 54341\\
R$^2$ ajusté & - & 0.18 & - & 0.164 & - & 0.176 & - & 0.206 & - & 0.174\\*
\end{longtable}
\end{ThreePartTable}
\endgroup{}

\newpage

\begingroup\fontsize{8}{10}\selectfont

\begin{ThreePartTable}
\begin{TableNotes}
\item \textit{Sources :} Fichiers CM2 (2009 à 2012), calculs de l'auteur.
\item \textit{Notes :} Une colonne correspond à une régression. La note est normalisée sur l'année scolaire. Écart-types entre parenthèses. Les écart-types des estimations par fonction de contrôle sont calculés par wild bootstrap avec 1001 réplications. Les contrôles utilisés sont le sexe, la CSP et l'année scolaire.
\item VI : Variable Instrumentale. FCH : Fonction de Contrôle avec prise en compte de l'Hétérogénéité de l'effet de l'âge.
\item Significativité : 10\% * 5\% ** 1\% ***.
\end{TableNotes}
\begin{longtable}[t]{lllllllllll}
\caption{\label{tab:agemodelssexessitemsmaths}Résultats avec effets hétérogènes selon le sexe, sous-items de mathématiques}\\
\toprule
\multicolumn{1}{c}{} & \multicolumn{10}{c}{Variable dépendante : Note en } \\
\cmidrule(l{3pt}r{3pt}){2-11}
\multicolumn{1}{c}{} & \multicolumn{2}{c}{Calcul} & \multicolumn{2}{c}{Géométrie} & \multicolumn{2}{c}{\makecell{Grandeurs et \\ mesures}} & \multicolumn{2}{c}{Nombre} & \multicolumn{2}{c}{\makecell{Organisation et \\ gestion de données}} \\
\cmidrule(l{3pt}r{3pt}){2-3} \cmidrule(l{3pt}r{3pt}){4-5} \cmidrule(l{3pt}r{3pt}){6-7} \cmidrule(l{3pt}r{3pt}){8-9} \cmidrule(l{3pt}r{3pt}){10-11}
 & \makecell{VI \\ (1) } & \makecell{FCH \\ (2) } & \makecell{VI \\ (3) } & \makecell{FCH \\ (4) } & \makecell{VI \\ (5) } & \makecell{FCH \\ (6) } & \makecell{VI \\ (7) } & \makecell{FCH \\ (8) } & \makecell{VI \\ (9) } & \makecell{FCH \\ (10) }\\
\midrule
\endfirsthead
\caption[]{\label{tab:agemodelssexessitemsmaths}Résultats avec effets hétérogènes selon le sexe, sous-items de mathématiques (suite)}\\
\toprule
 & \makecell{VI \\ (1) } & \makecell{FCH \\ (2) } & \makecell{VI \\ (3) } & \makecell{FCH \\ (4) } & \makecell{VI \\ (5) } & \makecell{FCH \\ (6) } & \makecell{VI \\ (7) } & \makecell{FCH \\ (8) } & \makecell{VI \\ (9) } & \makecell{FCH \\ (10) }\\
\midrule
\endhead

\endfoot
\bottomrule
\insertTableNotes
\endlastfoot
Âge aux examens & 0.269$^{***}$ & 0.307$^{***}$ & 0.273$^{***}$ & 0.288$^{***}$ & 0.308$^{***}$ & 0.351$^{***}$ & 0.205$^{***}$ & 0.257$^{***}$ & 0.258$^{***}$ & 0.291$^{***}$\\
 & (0.025) & (0.022) & (0.026) & (0.024) & (0.026) & (0.023) & (0.026) & (0.023) & (0.026) & (0.023)\\
Âge aux examens $\times$ Sexe - Garçon & $-$0.015 & $-$0.095$^{***}$ & 0.018 & $-$0.019 & 0.014 & $-$0.073$^{**}$ & 0.046 & $-$0.062$^{**}$ & 0.039 & $-$0.026\\
 & (0.037) & (0.028) & (0.037) & (0.032) & (0.037) & (0.03) & (0.037) & (0.028) & (0.037) & (0.029)\\
 &  &  &  &  &  &  &  &  &  & \\
Contrôles & Oui & Oui & Oui & Oui & Oui & Oui & Oui & Oui & Oui & Oui\\
Observations & 54341 & 54341 & 54341 & 54341 & 54341 & 54341 & 54341 & 54341 & 54341 & 54341\\
R$^2$ ajusté & - & 0.133 & - & 0.072 & - & 0.114 & - & 0.106 & - & 0.123\\*
\end{longtable}
\end{ThreePartTable}
\endgroup{}

\elandscape

\blandscape

\hypertarget{agemodelspcsg2ssitems}{%
\subsection{Effets hétérogènes selon la catégorie sociale}\label{agemodelspcsg2ssitems}}

\begingroup\fontsize{8}{10}\selectfont

\begin{ThreePartTable}
\begin{TableNotes}
\item \textit{Sources :} Fichiers CM2 (2009 à 2012), calculs de l'auteur.
\item \textit{Notes :} Une colonne correspond à une régression. La note est normalisée sur l'année scolaire. Écart-types entre parenthèses. Les écart-types des estimations par fonction de contrôle sont calculés par wild bootstrap avec 1001 réplications. Les contrôles utilisés sont le sexe, la CSP et l'année scolaire.
\item VI : Variable Instrumentale. FCH : Fonction de Contrôle avec prise en compte de l'Hétérogénéité de l'effet de l'âge. CSP : Catégorie Socio-Professionnelle.
\item Significativité : 10\% * 5\% ** 1\% ***.
\end{TableNotes}
\begin{longtable}[t]{lllllllllll}
\caption{\label{tab:agemodelspcsg2ssitemsfrench}Résultats avec effets hétérogènes selon la catégorie sociale, sous-items de français}\\
\toprule
\multicolumn{1}{c}{} & \multicolumn{10}{c}{Variable dépendante : Note en } \\
\cmidrule(l{3pt}r{3pt}){2-11}
\multicolumn{1}{c}{} & \multicolumn{2}{c}{Écriture} & \multicolumn{2}{c}{Grammaire} & \multicolumn{2}{c}{Lecture} & \multicolumn{2}{c}{Orthographe} & \multicolumn{2}{c}{Vocabulaire} \\
\cmidrule(l{3pt}r{3pt}){2-3} \cmidrule(l{3pt}r{3pt}){4-5} \cmidrule(l{3pt}r{3pt}){6-7} \cmidrule(l{3pt}r{3pt}){8-9} \cmidrule(l{3pt}r{3pt}){10-11}
 & \makecell{VI \\ (1) } & \makecell{FCH \\ (2) } & \makecell{VI \\ (3) } & \makecell{FCH \\ (4) } & \makecell{VI \\ (5) } & \makecell{FCH \\ (6) } & \makecell{VI \\ (7) } & \makecell{FCH \\ (8) } & \makecell{VI \\ (9) } & \makecell{FCH \\ (10) }\\
\midrule
\endfirsthead
\caption[]{\label{tab:agemodelspcsg2ssitemsfrench}Résultats avec effets hétérogènes selon la catégorie sociale, sous-items de français (suite)}\\
\toprule
\multicolumn{1}{c}{} & \multicolumn{10}{c}{Variable dépendante : Note en } \\
\cmidrule(l{3pt}r{3pt}){2-11}
\multicolumn{1}{c}{} & \multicolumn{2}{c}{Écriture} & \multicolumn{2}{c}{Grammaire} & \multicolumn{2}{c}{Lecture} & \multicolumn{2}{c}{Orthographe} & \multicolumn{2}{c}{Vocabulaire} \\
\cmidrule(l{3pt}r{3pt}){2-3} \cmidrule(l{3pt}r{3pt}){4-5} \cmidrule(l{3pt}r{3pt}){6-7} \cmidrule(l{3pt}r{3pt}){8-9} \cmidrule(l{3pt}r{3pt}){10-11}
 & \makecell{VI \\ (1) } & \makecell{FCH \\ (2) } & \makecell{VI \\ (3) } & \makecell{FCH \\ (4) } & \makecell{VI \\ (5) } & \makecell{FCH \\ (6) } & \makecell{VI \\ (7) } & \makecell{FCH \\ (8) } & \makecell{VI \\ (9) } & \makecell{FCH \\ (10) }\\
\midrule
\endhead

\endfoot
\bottomrule
\insertTableNotes
\endlastfoot
Âge aux examens & 0.188$^{***}$ & 0.179$^{***}$ & 0.217$^{***}$ & 0.25$^{***}$ & 0.21$^{***}$ & 0.217$^{***}$ & 0.201$^{***}$ & 0.22$^{***}$ & 0.234$^{***}$ & 0.212$^{***}$\\
 & (0.03) & (0.025) & (0.03) & (0.025) & (0.03) & (0.025) & (0.029) & (0.025) & (0.03) & (0.025)\\
Âge aux examens $\times$ CSP - Moyenne & 0.051 & 0.084$^{**}$ & 0.073 & 0.039 & 0.166$^{***}$ & 0.143$^{***}$ & 0.039 & 0.028 & 0.044 & 0.09$^{**}$\\
 & (0.051) & (0.04) & (0.052) & (0.04) & (0.051) & (0.039) & (0.052) & (0.042) & (0.051) & (0.04)\\
Âge aux examens $\times$ CSP - Favorisée & 0.003 & 0.141$^{**}$ & $-$0.092 & $-$0.058 & $-$0.093 & 0.048 & $-$0.035 & 0.069 & $-$0.089 & 0.061\\
 & (0.075) & (0.06) & (0.078) & (0.062) & (0.076) & (0.062) & (0.077) & (0.063) & (0.073) & (0.06)\\
Âge aux examens $\times$ CSP - Très favorisée & 0.182$^{***}$ & 0.191$^{***}$ & 0.099 & 0.027 & 0.096 & 0.114$^{**}$ & 0.125$^{*}$ & 0.058 & 0.079 & 0.188$^{***}$\\
 & (0.067) & (0.052) & (0.07) & (0.051) & (0.065) & (0.048) & (0.07) & (0.051) & (0.064) & (0.048)\\
Âge aux examens $\times$ CSP - Autre & 0.176 & 0.303 & 0.296 & 0.416$^{*}$ & 0.228 & 0.3 & 0.043 & 0.064 & 0.149 & 0.27\\
 & (0.235) & (0.226) & (0.235) & (0.233) & (0.231) & (0.222) & (0.219) & (0.229) & (0.221) & (0.212)\\
 &  &  &  &  &  &  &  &  &  & \\
Contrôles & Oui & Oui & Oui & Oui & Oui & Oui & Oui & Oui & Oui & Oui\\
Observations & 54341 & 54341 & 54341 & 54341 & 54341 & 54341 & 54341 & 54341 & 54341 & 54341\\
R$^2$ ajusté & - & 0.182 & - & 0.165 & - & 0.179 & - & 0.209 & - & 0.177\\*
\end{longtable}
\end{ThreePartTable}
\endgroup{}

\newpage

\begingroup\fontsize{8}{10}\selectfont

\begin{ThreePartTable}
\begin{TableNotes}
\item \textit{Sources :} Fichiers CM2 (2009 à 2012), calculs de l'auteur.
\item \textit{Notes :} Une colonne correspond à une régression. La note est normalisée sur l'année scolaire. Écart-types entre parenthèses. Les écart-types des estimations par fonction de contrôle sont calculés par wild bootstrap avec 1001 réplications. Les contrôles utilisés sont le sexe, la CSP et l'année scolaire.
\item VI : Variable Instrumentale. FCH : Fonction de Contrôle avec prise en compte de l'Hétérogénéité de l'effet de l'âge. CSP : Catégorie Socio-Professionnelle.
\item Significativité : 10\% * 5\% ** 1\% ***.
\end{TableNotes}
\begin{longtable}[t]{lllllllllll}
\caption{\label{tab:agemodelspcsg2ssitemsmaths}Résultats avec effets hétérogènes da la catégorie sociale, sous-items de mathématiques}\\
\toprule
\multicolumn{1}{c}{} & \multicolumn{10}{c}{Variable dépendante : Note en } \\
\cmidrule(l{3pt}r{3pt}){2-11}
\multicolumn{1}{c}{} & \multicolumn{2}{c}{Calcul} & \multicolumn{2}{c}{Géométrie} & \multicolumn{2}{c}{\makecell{Grandeurs et \\ mesures}} & \multicolumn{2}{c}{Nombre} & \multicolumn{2}{c}{\makecell{Organisation et \\ gestion de données}} \\
\cmidrule(l{3pt}r{3pt}){2-3} \cmidrule(l{3pt}r{3pt}){4-5} \cmidrule(l{3pt}r{3pt}){6-7} \cmidrule(l{3pt}r{3pt}){8-9} \cmidrule(l{3pt}r{3pt}){10-11}
 & \makecell{VI \\ (1) } & \makecell{FCH \\ (2) } & \makecell{VI \\ (3) } & \makecell{FCH \\ (4) } & \makecell{VI \\ (5) } & \makecell{FCH \\ (6) } & \makecell{VI \\ (7) } & \makecell{FCH \\ (8) } & \makecell{VI \\ (9) } & \makecell{FCH \\ (10) }\\
\midrule
\endfirsthead
\caption[]{\label{tab:agemodelspcsg2ssitemsmaths}Résultats avec effets hétérogènes da la catégorie sociale, sous-items de mathématiques (suite)}\\
\toprule
 & \makecell{VI \\ (1) } & \makecell{FCH \\ (2) } & \makecell{VI \\ (3) } & \makecell{FCH \\ (4) } & \makecell{VI \\ (5) } & \makecell{FCH \\ (6) } & \makecell{VI \\ (7) } & \makecell{FCH \\ (8) } & \makecell{VI \\ (9) } & \makecell{FCH \\ (10) }\\
\midrule
\endhead

\endfoot
\bottomrule
\insertTableNotes
\endlastfoot
Âge aux examens & 0.216$^{***}$ & 0.241$^{***}$ & 0.277$^{***}$ & 0.293$^{***}$ & 0.263$^{***}$ & 0.273$^{***}$ & 0.175$^{***}$ & 0.217$^{***}$ & 0.228$^{***}$ & 0.274$^{***}$\\
 & (0.03) & (0.026) & (0.031) & (0.028) & (0.03) & (0.026) & (0.03) & (0.026) & (0.029) & (0.025)\\
Âge aux examens $\times$ CSP - Moyenne & 0.089$^{*}$ & 0.032 & $-$0.031 & $-$0.02 & 0.085 & 0.102$^{**}$ & 0.043 & $-$0.002 & 0.121$^{**}$ & 0.078$^{*}$\\
 & (0.052) & (0.043) & (0.053) & (0.043) & (0.053) & (0.043) & (0.053) & (0.044) & (0.053) & (0.044)\\
Âge aux examens $\times$ CSP - Favorisée & $-$0.04 & 0.048 & $-$0.153$^{*}$ & $-$0.16$^{**}$ & $-$0.034 & 0.04 & 0.004 & 0.006 & $-$0.059 & $-$0.041\\
 & (0.079) & (0.064) & (0.081) & (0.07) & (0.081) & (0.069) & (0.08) & (0.067) & (0.083) & (0.069)\\
Âge aux examens $\times$ CSP - Très favorisée & 0.074 & 0.068 & 0.032 & $-$0.069 & 0.188$^{**}$ & 0.126$^{**}$ & 0.183$^{**}$ & 0.013 & 0.173$^{**}$ & $-$0.023\\
 & (0.07) & (0.053) & (0.072) & (0.056) & (0.076) & (0.054) & (0.073) & (0.053) & (0.078) & (0.056)\\
Âge aux examens $\times$ CSP - Autre & 0.057 & 0.151 & $-$0.157 & 0.165 & $-$0.118 & 0.09 & 0.006 & 0.105 & $-$0.108 & 0.081\\
 & (0.219) & (0.213) & (0.22) & (0.232) & (0.225) & (0.232) & (0.216) & (0.214) & (0.207) & (0.208)\\
 &  &  &  &  &  &  &  &  &  & \\
Contrôles & Oui & Oui & Oui & Oui & Oui & Oui & Oui & Oui & Oui & Oui\\
Observations & 54341 & 54341 & 54341 & 54341 & 54341 & 54341 & 54341 & 54341 & 54341 & 54341\\
R$^2$ ajusté & - & 0.135 & - & 0.074 & - & 0.115 & - & 0.108 & - & 0.124\\*
\end{longtable}
\end{ThreePartTable}
\endgroup{}

\elandscape

\blandscape

\setcounter{table}{0}
\setcounter{figure}{0}

\hypertarget{agemodelssexereseaubinsep}{%
\section{Résultats sur les notes en sous-composantes de français et de mathématiques avec effets hétérogènes selon la catégorie sociale, par éducation prioritaire}\label{agemodelssexereseaubinsep}}

\begingroup\fontsize{7}{9}\selectfont

\begin{ThreePartTable}
\begin{TableNotes}
\item \textit{Sources :} Fichiers CM2 (2009 à 2012), calculs de l'auteur.
\item \textit{Notes :} Une colonne correspond à une régression. La note est normalisée sur l'année scolaire. Écart-types entre parenthèses. Les écart-types des estimations par fonction de contrôle sont calculés par wild bootstrap avec 1001 réplications. Les contrôles utilisés sont le sexe, la CSP et l'année scolaire.
\item FCH : Fonction de Contrôle avec prise en compte de l'Hétérogénéité de l'effet de l'âge. CSP : Catégorie Socio-Professionnelle.
\item Significativité : 10\% * 5\% ** 1\% ***.
\end{TableNotes}
\begin{longtable}[t]{lllllllllll}
\caption{\label{tab:agemodelspcsg2reseaubinsepssitemsfrench}Résultats avec effets hétérogènes da la catégorie sociale, sous-items de français, par éducation prioritaire}\\
\toprule
\multicolumn{1}{c}{} & \multicolumn{10}{c}{Variable dépendante : Note en } \\
\cmidrule(l{3pt}r{3pt}){2-11}
\multicolumn{1}{c}{} & \multicolumn{2}{c}{Écriture} & \multicolumn{2}{c}{Grammaire} & \multicolumn{2}{c}{Lecture} & \multicolumn{2}{c}{Orthographe} & \multicolumn{2}{c}{Vocabulaire} \\
\cmidrule(l{3pt}r{3pt}){2-3} \cmidrule(l{3pt}r{3pt}){4-5} \cmidrule(l{3pt}r{3pt}){6-7} \cmidrule(l{3pt}r{3pt}){8-9} \cmidrule(l{3pt}r{3pt}){10-11}
 & \makecell{FCH, Hors EP \\ (1) } & \makecell{FCH, EP \\ (2) } & \makecell{FCH, Hors EP \\ (3) } & \makecell{FCH, EP \\ (4) } & \makecell{FCH, Hors EP \\ (5) } & \makecell{FCH, EP \\ (6) } & \makecell{FCH, Hors EP \\ (7) } & \makecell{FCH, EP \\ (8) } & \makecell{FCH, Hors EP \\ (9) } & \makecell{FCH, EP \\ (10) }\\
\midrule
\endfirsthead
\caption[]{\label{tab:agemodelspcsg2reseaubinsepssitemsfrench}Résultats avec effets hétérogènes da la catégorie sociale, sous-items de français, par éducation prioritaire (suite)}\\
\toprule
 & \makecell{FCH, Hors EP \\ (1) } & \makecell{FCH, EP \\ (2) } & \makecell{FCH, Hors EP \\ (3) } & \makecell{FCH, EP \\ (4) } & \makecell{FCH, Hors EP \\ (5) } & \makecell{FCH, EP \\ (6) } & \makecell{FCH, Hors EP \\ (7) } & \makecell{FCH, EP \\ (8) } & \makecell{FCH, Hors EP \\ (9) } & \makecell{FCH, EP \\ (10) }\\
\midrule
\endhead

\endfoot
\bottomrule
\insertTableNotes
\endlastfoot
Âge aux examens & 0.159$^{***}$ & 0.225$^{***}$ & 0.211$^{***}$ & 0.246$^{***}$ & 0.261$^{***}$ & 0.241$^{***}$ & 0.251$^{***}$ & 0.196$^{***}$ & 0.171$^{***}$ & 0.237$^{***}$\\
 & (0.036) & (0.036) & (0.037) & (0.035) & (0.038) & (0.034) & (0.035) & (0.033) & (0.037) & (0.035)\\
Âge aux examens $\times$ CSP - Moyenne & 0.156$^{***}$ & $-$0.005 & 0.17$^{***}$ & 0.102$^{*}$ & 0.075 & $-$0.004 & 0.075 & $-$0.046 & 0.167$^{***}$ & 0.002\\
 & (0.057) & (0.06) & (0.057) & (0.057) & (0.058) & (0.057) & (0.056) & (0.058) & (0.055) & (0.057)\\
Âge aux examens $\times$ CSP - Favorisée & 0.13$^{*}$ & 0.187$^{*}$ & 0.015 & 0.12 & $-$0.132 & 0.106 & 0.034 & 0.123 & 0.12 & $-$0.006\\
 & (0.079) & (0.1) & (0.078) & (0.111) & (0.082) & (0.111) & (0.077) & (0.108) & (0.077) & (0.105)\\
Âge aux examens $\times$ CSP - Très favorisée & 0.187$^{***}$ & 0.233$^{**}$ & 0.156$^{**}$ & 0.024 & 0.04 & 0.017 & 0.056 & 0.032 & 0.251$^{***}$ & 0.094\\
 & (0.062) & (0.095) & (0.061) & (0.098) & (0.068) & (0.096) & (0.066) & (0.1) & (0.06) & (0.098)\\
Âge aux examens $\times$ CSP - Autre & 0.212 & 0.42 & 0.105 & 0.512 & 0.285 & 0.544 & 0.194 & $-$0.114 & $-$0.009 & 0.551$^{*}$\\
 & (0.298) & (0.347) & (0.291) & (0.352) & (0.303) & (0.349) & (0.294) & (0.313) & (0.253) & (0.321)\\
 &  &  &  &  &  &  &  &  &  & \\
Contrôles & Oui & Oui & Oui & Oui & Oui & Oui & Oui & Oui & Oui & Oui\\
Observations & 28506 & 25835 & 28506 & 25835 & 28506 & 25835 & 28506 & 25835 & 28506 & 25835\\
R$^2$ ajusté & 0.169 & 0.188 & 0.172 & 0.171 & 0.16 & 0.157 & 0.2 & 0.205 & 0.17 & 0.17\\*
\end{longtable}
\end{ThreePartTable}
\endgroup{}

\newpage

\begingroup\fontsize{7}{9}\selectfont

\begin{ThreePartTable}
\begin{TableNotes}
\item \textit{Sources :} Fichiers CM2 (2009 à 2012), calculs de l'auteur.
\item \textit{Notes :} Une colonne correspond à une régression. La note est normalisée sur l'année scolaire. Écart-types entre parenthèses. Les écart-types des estimations par fonction de contrôle sont calculés par wild bootstrap avec 1001 réplications. Les contrôles utilisés sont le sexe, la CSP et l'année scolaire.
\item FCH : Fonction de Contrôle avec prise en compte de l'Hétérogénéité de l'effet de l'âge. CSP : Catégorie Socio-Professionnelle.
\item Significativité : 10\% * 5\% ** 1\% ***.
\end{TableNotes}
\begin{longtable}[t]{lllllllllll}
\caption{\label{tab:agemodelspcsg2reseaubinsepssitemsmaths}Résultats avec effets hétérogènes da la catégorie sociale, sous-items de mathématiques, par éducation prioritaire}\\
\toprule
\multicolumn{1}{c}{} & \multicolumn{10}{c}{Variable dépendante : Note en } \\
\cmidrule(l{3pt}r{3pt}){2-11}
\multicolumn{1}{c}{} & \multicolumn{2}{c}{Calcul} & \multicolumn{2}{c}{Géométrie} & \multicolumn{2}{c}{\makecell{Grandeurs et \\ mesures}} & \multicolumn{2}{c}{Nombre} & \multicolumn{2}{c}{\makecell{Organisation et \\ gestion de données}} \\
\cmidrule(l{3pt}r{3pt}){2-3} \cmidrule(l{3pt}r{3pt}){4-5} \cmidrule(l{3pt}r{3pt}){6-7} \cmidrule(l{3pt}r{3pt}){8-9} \cmidrule(l{3pt}r{3pt}){10-11}
 & \makecell{FCH, Hors EP \\ (1) } & \makecell{FCH, EP \\ (2) } & \makecell{FCH, Hors EP \\ (3) } & \makecell{FCH, EP \\ (4) } & \makecell{FCH, Hors EP \\ (5) } & \makecell{FCH, EP \\ (6) } & \makecell{FCH, Hors EP \\ (7) } & \makecell{FCH, EP \\ (8) } & \makecell{FCH, Hors EP \\ (9) } & \makecell{FCH, EP \\ (10) }\\
\midrule
\endfirsthead
\caption[]{\label{tab:agemodelspcsg2reseaubinsepssitemsmaths}Résultats avec effets hétérogènes da la catégorie sociale, sous-items de mathématiques, par éducation prioritaire (suite)}\\
\toprule
 & \makecell{FCH, Hors EP \\ (1) } & \makecell{FCH, EP \\ (2) } & \makecell{FCH, Hors EP \\ (3) } & \makecell{FCH, EP \\ (4) } & \makecell{FCH, Hors EP \\ (5) } & \makecell{FCH, EP \\ (6) } & \makecell{FCH, Hors EP \\ (7) } & \makecell{FCH, EP \\ (8) } & \makecell{FCH, Hors EP \\ (9) } & \makecell{FCH, EP \\ (10) }\\
\midrule
\endhead

\endfoot
\bottomrule
\insertTableNotes
\endlastfoot
Âge aux examens & 0.26$^{***}$ & 0.22$^{***}$ & 0.345$^{***}$ & 0.261$^{***}$ & 0.289$^{***}$ & 0.292$^{***}$ & 0.248$^{***}$ & 0.191$^{***}$ & 0.31$^{***}$ & 0.271$^{***}$\\
 & (0.039) & (0.036) & (0.039) & (0.038) & (0.038) & (0.034) & (0.039) & (0.037) & (0.037) & (0.035)\\
Âge aux examens $\times$ CSP - Moyenne & 0.084 & $-$0.036 & $-$0.027 & $-$0.031 & 0.095 & 0.098 & 0.004 & $-$0.016 & 0.084 & 0.062\\
 & (0.058) & (0.064) & (0.061) & (0.064) & (0.058) & (0.061) & (0.059) & (0.062) & (0.061) & (0.063)\\
Âge aux examens $\times$ CSP - Favorisée & 0.037 & 0.082 & $-$0.285$^{***}$ & 0.038 & $-$0.021 & 0.141 & $-$0.068 & 0.153 & $-$0.144 & 0.16\\
 & (0.081) & (0.114) & (0.085) & (0.119) & (0.083) & (0.126) & (0.086) & (0.114) & (0.088) & (0.127)\\
Âge aux examens $\times$ CSP - Très favorisée & 0.049 & 0.13 & $-$0.135$^{**}$ & 0.02 & 0.09 & 0.184$^{*}$ & $-$0.017 & 0.094 & $-$0.046 & 0.041\\
 & (0.069) & (0.102) & (0.068) & (0.107) & (0.069) & (0.102) & (0.071) & (0.103) & (0.071) & (0.108)\\
Âge aux examens $\times$ CSP - Autre & 0.096 & 0.166 & 0.418 & $-$0.091 & 0.369 & $-$0.23 & $-$0.035 & 0.268 & 0.095 & 0.08\\
 & (0.299) & (0.308) & (0.314) & (0.337) & (0.291) & (0.366) & (0.308) & (0.309) & (0.292) & (0.299)\\
 &  &  &  &  &  &  &  &  &  & \\
Contrôles & Oui & Oui & Oui & Oui & Oui & Oui & Oui & Oui & Oui & Oui\\
Observations & 28506 & 25835 & 28506 & 25835 & 28506 & 25835 & 28506 & 25835 & 28506 & 25835\\
R$^2$ ajusté & 0.13 & 0.13 & 0.07 & 0.073 & 0.113 & 0.104 & 0.106 & 0.1 & 0.123 & 0.113\\*
\end{longtable}
\end{ThreePartTable}
\endgroup{}

\elandscape

\blandscape

\setcounter{table}{0}
\setcounter{figure}{0}

\hypertarget{agefrdcfhmodelsssitems0}{%
\section{Résultats de régressions sur une discontinuité sur les sous-composantes de français et de mathématiques}\label{agefrdcfhmodelsssitems0}}

\hypertarget{agefrdcfhmodelsssitems}{%
\subsection{Effets homogènes selon les observables}\label{agefrdcfhmodelsssitems}}

\begingroup\fontsize{8}{10}\selectfont

\begin{ThreePartTable}
\begin{TableNotes}
\item \textit{Sources :} Fichiers CM2 (2009 à 2012), calculs de l'auteur.
\item \textit{Notes :} Une colonne correspond à une régression. La note est normalisée sur l'année scolaire. Écart-types entre parenthèses. Les estimations se font par fonction de contrôle avec prise en compte de l'hétérogénéité de l'effet de l'âge. Les écart-types des estimations par fonction de contrôle sont calculés par wild bootstrap avec 1001 réplications. Les contrôles utilisés sont le sexe, la CSP et l'année scolaire.
\item FRD : \textit{Fuzzy Regression Discontinuity}. La variable p désigne degré de polynômes de la date de naissance.
\item Significativité : 10\% * 5\% ** 1\% ***.
\end{TableNotes}
\begin{longtable}[t]{lllllllllll}
\caption{\label{tab:agefrdcfhmodelsssitemsfrench}Résultats avec la régression sur une discontinuité, sous-items de français}\\
\toprule
\multicolumn{1}{c}{} & \multicolumn{10}{c}{Variable dépendante : } \\
\cmidrule(l{3pt}r{3pt}){2-11}
\multicolumn{1}{c}{} & \multicolumn{2}{c}{Écriture} & \multicolumn{2}{c}{Grammaire} & \multicolumn{2}{c}{Lecture} & \multicolumn{2}{c}{Orthographe} & \multicolumn{2}{c}{Vocabulaire} \\
\cmidrule(l{3pt}r{3pt}){2-3} \cmidrule(l{3pt}r{3pt}){4-5} \cmidrule(l{3pt}r{3pt}){6-7} \cmidrule(l{3pt}r{3pt}){8-9} \cmidrule(l{3pt}r{3pt}){10-11}
 & \makecell{FRD, p = 1 \\ (1) } & \makecell{FRD, p = 2 \\ (2) } & \makecell{FRD, p = 1 \\ (3) } & \makecell{FRD, p = 2 \\ (4) } & \makecell{FRD, p = 1 \\ (5) } & \makecell{FRD, p = 2 \\ (6) } & \makecell{FRD, p = 1 \\ (7) } & \makecell{FRD, p = 2 \\ (8) } & \makecell{FRD, p = 1 \\ (9) } & \makecell{FRD, p = 2 \\ (10) }\\
\midrule
\endfirsthead
\caption[]{\label{tab:agefrdcfhmodelsssitemsfrench}Résultats avec la régression sur une discontinuité, sous-items de français (suite)}\\
\toprule
\multicolumn{1}{c}{} & \multicolumn{10}{c}{Variable dépendante : } \\
\cmidrule(l{3pt}r{3pt}){2-11}
\multicolumn{1}{c}{} & \multicolumn{2}{c}{Écriture} & \multicolumn{2}{c}{Grammaire} & \multicolumn{2}{c}{Lecture} & \multicolumn{2}{c}{Orthographe} & \multicolumn{2}{c}{Vocabulaire} \\
\cmidrule(l{3pt}r{3pt}){2-3} \cmidrule(l{3pt}r{3pt}){4-5} \cmidrule(l{3pt}r{3pt}){6-7} \cmidrule(l{3pt}r{3pt}){8-9} \cmidrule(l{3pt}r{3pt}){10-11}
 & \makecell{FRD, p = 1 \\ (1) } & \makecell{FRD, p = 2 \\ (2) } & \makecell{FRD, p = 1 \\ (3) } & \makecell{FRD, p = 2 \\ (4) } & \makecell{FRD, p = 1 \\ (5) } & \makecell{FRD, p = 2 \\ (6) } & \makecell{FRD, p = 1 \\ (7) } & \makecell{FRD, p = 2 \\ (8) } & \makecell{FRD, p = 1 \\ (9) } & \makecell{FRD, p = 2 \\ (10) }\\
\midrule
\endhead

\endfoot
\bottomrule
\insertTableNotes
\endlastfoot
Âge aux examens & 0.384$^{***}$ & 0.245 & 0.494$^{***}$ & 0.6$^{***}$ & 0.552$^{***}$ & 0.551$^{***}$ & 0.4$^{***}$ & 0.623$^{***}$ & 0.526$^{***}$ & 0.563$^{***}$\\
 & (0.102) & (0.152) & (0.098) & (0.145) & (0.102) & (0.152) & (0.1) & (0.148) & (0.099) & (0.145)\\
dist & 0 & $-$0.007 & $-$0.004 & $-$0.024$^{**}$ & $-$0.002 & $-$0.014 & $-$0.001 & $-$0.037$^{***}$ & $-$0.003 & $-$0.026$^{**}$\\
 & (0.003) & (0.013) & (0.003) & (0.012) & (0.003) & (0.012) & (0.003) & (0.012) & (0.003) & (0.013)\\
old $\times$ dist & $-$0.004 & 0.033$^{*}$ & 0.002 & 0.03$^{*}$ & $-$0.003 & 0.023 & $-$0.002 & 0.041$^{**}$ & $-$0.002 & 0.043$^{***}$\\
 & (0.004) & (0.017) & (0.004) & (0.016) & (0.004) & (0.017) & (0.004) & (0.017) & (0.004) & (0.016)\\
dist$^2$ & - & 0 & - & $-$0.001$^{*}$ & - & 0 & - & $-$0.001$^{***}$ & - & $-$0.001$^{*}$\\
 & - & (0) & - & (0) & - & (0) & - & (0) & - & (0)\\
old $\times$ dist$^2$ & - & $-$0.001 & - & 0 & - & 0 & - & 0.001$^{*}$ & - & 0\\
 & - & (0.001) & - & (0.001) & - & (0.001) & - & (0.001) & - & (0.001)\\
 &  &  &  &  &  &  &  &  &  & \\
Contrôles & Oui & Oui & Oui & Oui & Oui & Oui & Oui & Oui & Oui & Oui\\
Observations & 2271 & 2271 & 2271 & 2271 & 2271 & 2271 & 2271 & 2271 & 2271 & 2271\\
R$^2$ ajusté & 0.171 & 0.172 & 0.185 & 0.185 & 0.18 & 0.18 & 0.215 & 0.217 & 0.175 & 0.176\\*
\end{longtable}
\end{ThreePartTable}
\endgroup{}

\newpage

\begingroup\fontsize{8}{10}\selectfont

\begin{ThreePartTable}
\begin{TableNotes}
\item \textit{Sources :} Fichiers CM2 (2009 à 2012), calculs de l'auteur.
\item \textit{Notes :} Une colonne correspond à une régression. La note est normalisée sur l'année scolaire. Écart-types entre parenthèses. Les estimations se font par fonction de contrôle avec prise en compte de l'hétérogénéité de l'effet de l'âge. Les écart-types des estimations par fonction de contrôle sont calculés par wild bootstrap avec 1001 réplications. Les contrôles utilisés sont le sexe, la CSP et l'année scolaire.
\item FRD : \textit{Fuzzy Regression Discontinuity}. La variable p désigne degré de polynômes de la date de naissance.
\item Significativité : 10\% * 5\% ** 1\% ***.
\end{TableNotes}
\begin{longtable}[t]{lllllllllll}
\caption{\label{tab:agefrdcfhmodelsssitemsmaths}Résultats avec la régression sur une discontinuité, sous-items de mathématiques}\\
\toprule
\multicolumn{1}{c}{} & \multicolumn{10}{c}{Variable dépendante : } \\
\cmidrule(l{3pt}r{3pt}){2-11}
\multicolumn{1}{c}{} & \multicolumn{2}{c}{Calcul} & \multicolumn{2}{c}{Géométrie} & \multicolumn{2}{c}{Grandeurs et mesures} & \multicolumn{2}{c}{Nombre} & \multicolumn{2}{c}{\makecell{Organisation et \\
                     gestion des données}} \\
\cmidrule(l{3pt}r{3pt}){2-3} \cmidrule(l{3pt}r{3pt}){4-5} \cmidrule(l{3pt}r{3pt}){6-7} \cmidrule(l{3pt}r{3pt}){8-9} \cmidrule(l{3pt}r{3pt}){10-11}
 & \makecell{FRD, p = 1 \\ (1) } & \makecell{FRD, p = 2 \\ (2) } & \makecell{FRD, p = 1 \\ (3) } & \makecell{FRD, p = 2 \\ (4) } & \makecell{FRD, p = 1 \\ (5) } & \makecell{FRD, p = 2 \\ (6) } & \makecell{FRD, p = 1 \\ (7) } & \makecell{FRD, p = 2 \\ (8) } & \makecell{FRD, p = 1 \\ (9) } & \makecell{FRD, p = 2 \\ (10) }\\
\midrule
\endfirsthead
\caption[]{\label{tab:agefrdcfhmodelsssitemsmaths}Résultats avec la régression sur une discontinuité, sous-items de mathématiques (suite)}\\
\toprule
\multicolumn{1}{c}{} & \multicolumn{10}{c}{Variable dépendante : } \\
\cmidrule(l{3pt}r{3pt}){2-11}
\multicolumn{1}{c}{} & \multicolumn{2}{c}{Calcul} & \multicolumn{2}{c}{Géométrie} & \multicolumn{2}{c}{Grandeurs et mesures} & \multicolumn{2}{c}{Nombre} & \multicolumn{2}{c}{\makecell{Organisation et \\
                     gestion des données}} \\
\cmidrule(l{3pt}r{3pt}){2-3} \cmidrule(l{3pt}r{3pt}){4-5} \cmidrule(l{3pt}r{3pt}){6-7} \cmidrule(l{3pt}r{3pt}){8-9} \cmidrule(l{3pt}r{3pt}){10-11}
 & \makecell{FRD, p = 1 \\ (1) } & \makecell{FRD, p = 2 \\ (2) } & \makecell{FRD, p = 1 \\ (3) } & \makecell{FRD, p = 2 \\ (4) } & \makecell{FRD, p = 1 \\ (5) } & \makecell{FRD, p = 2 \\ (6) } & \makecell{FRD, p = 1 \\ (7) } & \makecell{FRD, p = 2 \\ (8) } & \makecell{FRD, p = 1 \\ (9) } & \makecell{FRD, p = 2 \\ (10) }\\
\midrule
\endhead

\endfoot
\bottomrule
\insertTableNotes
\endlastfoot
Âge aux examens & 0.6$^{***}$ & 0.62$^{***}$ & 0.688$^{***}$ & 0.884$^{***}$ & 0.627$^{***}$ & 0.613$^{***}$ & 0.489$^{***}$ & 0.482$^{***}$ & 0.577$^{***}$ & 0.462$^{***}$\\
 & (0.104) & (0.155) & (0.102) & (0.148) & (0.103) & (0.147) & (0.105) & (0.146) & (0.103) & (0.146)\\
dist & $-$0.003 & $-$0.022$^{*}$ & $-$0.001 & $-$0.025$^{*}$ & $-$0.003 & $-$0.017 & $-$0.004 & $-$0.017 & $-$0.001 & $-$0.008\\
 & (0.003) & (0.013) & (0.003) & (0.013) & (0.003) & (0.012) & (0.003) & (0.012) & (0.003) & (0.011)\\
old $\times$ dist & 0.001 & 0.037$^{**}$ & $-$0.006 & 0.016 & 0.002 & 0.033$^{*}$ & 0.001 & 0.03$^{*}$ & $-$0.003 & 0.031$^{*}$\\
 & (0.004) & (0.017) & (0.005) & (0.017) & (0.004) & (0.017) & (0.004) & (0.017) & (0.004) & (0.017)\\
dist$^2$ & - & $-$0.001 & - & $-$0.001$^{*}$ & - & 0 & - & 0 & - & 0\\
 & - & (0) & - & (0) & - & (0) & - & (0) & - & (0)\\
old $\times$ dist$^2$ & - & 0 & - & 0.001 & - & 0 & - & 0 & - & $-$0.001\\
 & - & (0.001) & - & (0.001) & - & (0.001) & - & (0.001) & - & (0.001)\\
 &  &  &  &  &  &  &  &  &  & \\
Contrôles & Oui & Oui & Oui & Oui & Oui & Oui & Oui & Oui & Oui & Oui\\
Observations & 2271 & 2271 & 2271 & 2271 & 2271 & 2271 & 2271 & 2271 & 2271 & 2271\\
R$^2$ ajusté & 0.163 & 0.164 & 0.088 & 0.088 & 0.138 & 0.138 & 0.109 & 0.109 & 0.153 & 0.154\\*
\end{longtable}
\end{ThreePartTable}
\endgroup{}

\elandscape

\blandscape

\hypertarget{agefrdcfhmodelssexessitems}{%
\subsection{Effets hétérogènes selon le sexe}\label{agefrdcfhmodelssexessitems}}

\begingroup\fontsize{7}{9}\selectfont

\begin{ThreePartTable}
\begin{TableNotes}
\item \textit{Sources :} Fichiers CM2 (2009 à 2012), calculs de l'auteur.
\item \textit{Notes :} Une colonne correspond à une régression. La note est normalisée sur l'année scolaire. Écart-types entre parenthèses. Les estimations se font par fonction de contrôle avec prise en compte de l'hétérogénéité de l'effet de l'âge. Les écart-types des estimations par fonction de contrôle sont calculés par wild bootstrap avec 1001 réplications. Les contrôles utilisés sont le sexe, la CSP et l'année scolaire.
\item FRD : \textit{Fuzzy Regression Discontinuity}. La variable p désigne degré de polynômes de la date de naissance.
\item Significativité : 10\% * 5\% ** 1\% ***.
\end{TableNotes}
\begin{longtable}[t]{lllllllllll}
\caption{\label{tab:agefrdcfhmodelssexessitemsfrench}Résultats de régressions sur une discontinuité, effets hétérogènes selon le sexe sur les sous-composantes de français}\\
\toprule
\multicolumn{1}{c}{} & \multicolumn{10}{c}{Variable dépendante : } \\
\cmidrule(l{3pt}r{3pt}){2-11}
\multicolumn{1}{c}{} & \multicolumn{2}{c}{Écriture} & \multicolumn{2}{c}{Grammaire} & \multicolumn{2}{c}{Lecture} & \multicolumn{2}{c}{Orthographe} & \multicolumn{2}{c}{Vocabulaire} \\
\cmidrule(l{3pt}r{3pt}){2-3} \cmidrule(l{3pt}r{3pt}){4-5} \cmidrule(l{3pt}r{3pt}){6-7} \cmidrule(l{3pt}r{3pt}){8-9} \cmidrule(l{3pt}r{3pt}){10-11}
 & \makecell{FRD, p = 1 \\ (1) } & \makecell{FRD, p = 2 \\ (2) } & \makecell{FRD, p = 1 \\ (3) } & \makecell{FRD, p = 2 \\ (4) } & \makecell{FRD, p = 1 \\ (5) } & \makecell{FRD, p = 2 \\ (6) } & \makecell{FRD, p = 1 \\ (7) } & \makecell{FRD, p = 2 \\ (8) } & \makecell{FRD, p = 1 \\ (9) } & \makecell{FRD, p = 2 \\ (10) }\\
\midrule
\endfirsthead
\caption[]{\label{tab:agefrdcfhmodelssexessitemsfrench}Résultats de régressions sur une discontinuité, effets hétérogènes selon le sexe sur les sous-composantes de français (suite)}\\
\toprule
 & \makecell{FRD, p = 1 \\ (1) } & \makecell{FRD, p = 2 \\ (2) } & \makecell{FRD, p = 1 \\ (3) } & \makecell{FRD, p = 2 \\ (4) } & \makecell{FRD, p = 1 \\ (5) } & \makecell{FRD, p = 2 \\ (6) } & \makecell{FRD, p = 1 \\ (7) } & \makecell{FRD, p = 2 \\ (8) } & \makecell{FRD, p = 1 \\ (9) } & \makecell{FRD, p = 2 \\ (10) }\\
\midrule
\endhead

\endfoot
\bottomrule
\insertTableNotes
\endlastfoot
Âge aux examens & 0.415$^{***}$ & 0.278$^{*}$ & 0.624$^{***}$ & 0.735$^{***}$ & 0.677$^{***}$ & 0.682$^{***}$ & 0.371$^{***}$ & 0.599$^{***}$ & 0.588$^{***}$ & 0.628$^{***}$\\
 & (0.113) & (0.162) & (0.115) & (0.158) & (0.118) & (0.163) & (0.116) & (0.161) & (0.114) & (0.153)\\
Âge aux examens $\times$ Sexe - Garçon & $-$0.057 & $-$0.059 & $-$0.256$^{**}$ & $-$0.261$^{**}$ & $-$0.241$^{**}$ & $-$0.245$^{**}$ & 0.066 & 0.063 & $-$0.115 & $-$0.117\\
 & (0.114) & (0.113) & (0.113) & (0.113) & (0.111) & (0.111) & (0.107) & (0.106) & (0.109) & (0.109)\\
dist & 0 & $-$0.007 & $-$0.004 & $-$0.025$^{**}$ & $-$0.003 & $-$0.015 & $-$0.001 & $-$0.037$^{***}$ & $-$0.003 & $-$0.027$^{**}$\\
 & (0.003) & (0.013) & (0.003) & (0.012) & (0.003) & (0.012) & (0.003) & (0.012) & (0.003) & (0.013)\\
old $\times$ dist & $-$0.004 & 0.033$^{*}$ & 0.002 & 0.03$^{*}$ & $-$0.002 & 0.024 & $-$0.001 & 0.041$^{**}$ & $-$0.002 & 0.044$^{***}$\\
 & (0.004) & (0.017) & (0.004) & (0.016) & (0.004) & (0.017) & (0.004) & (0.017) & (0.004) & (0.016)\\
dist$^2$ & - & 0 & - & $-$0.001$^{*}$ & - & 0 & - & $-$0.001$^{***}$ & - & $-$0.001$^{*}$\\
 & - & (0) & - & (0) & - & (0) & - & (0) & - & (0)\\
old $\times$ dist$^2$ & - & $-$0.001 & - & 0 & - & 0 & - & 0.001$^{*}$ & - & 0\\
 & - & (0.001) & - & (0.001) & - & (0.001) & - & (0.001) & - & (0.001)\\
 &  &  &  &  &  &  &  &  &  & \\
Contrôles & Oui & Oui & Oui & Oui & Oui & Oui & Oui & Oui & Oui & Oui\\
Observations & 2271 & 2271 & 2271 & 2271 & 2271 & 2271 & 2271 & 2271 & 2271 & 2271\\
R$^2$ ajusté & 0.17 & 0.171 & 0.186 & 0.186 & 0.182 & 0.182 & 0.215 & 0.217 & 0.176 & 0.177\\*
\end{longtable}
\end{ThreePartTable}
\endgroup{}

\newpage

\begingroup\fontsize{7}{9}\selectfont

\begin{ThreePartTable}
\begin{TableNotes}
\item \textit{Sources :} Fichiers CM2 (2009 à 2012), calculs de l'auteur.
\item \textit{Notes :} Une colonne correspond à une régression. La note est normalisée sur l'année scolaire. Écart-types entre parenthèses. Les estimations se font par fonction de contrôle avec prise en compte de l'hétérogénéité de l'effet de l'âge. Les écart-types des estimations par fonction de contrôle sont calculés par wild bootstrap avec 1001 réplications. Les contrôles utilisés sont le sexe, la CSP et l'année scolaire.
\item FRD : \textit{Fuzzy Regression Discontinuity}. La variable p désigne degré de polynômes de la date de naissance.
\item Significativité : 10\% * 5\% ** 1\% ***.
\end{TableNotes}
\begin{longtable}[t]{lllllllllll}
\caption{\label{tab:agefrdcfhmodelssexessitemsmaths}Résultats de régressions sur une discontinuité, effets hétérogènes selon le sexe sur les sous-composantes de mathématiques}\\
\toprule
\multicolumn{1}{c}{} & \multicolumn{10}{c}{Variable dépendante : Note en } \\
\cmidrule(l{3pt}r{3pt}){2-11}
\multicolumn{1}{c}{} & \multicolumn{2}{c}{Calcul} & \multicolumn{2}{c}{Géométrie} & \multicolumn{2}{c}{\makecell{Grandeurs et \\ mesures}} & \multicolumn{2}{c}{Nombre} & \multicolumn{2}{c}{\makecell{Organisation et \\ gestion de données}} \\
\cmidrule(l{3pt}r{3pt}){2-3} \cmidrule(l{3pt}r{3pt}){4-5} \cmidrule(l{3pt}r{3pt}){6-7} \cmidrule(l{3pt}r{3pt}){8-9} \cmidrule(l{3pt}r{3pt}){10-11}
 & \makecell{FRD, p = 1 \\ (1) } & \makecell{FRD, p = 2 \\ (2) } & \makecell{FRD, p = 1 \\ (3) } & \makecell{FRD, p = 2 \\ (4) } & \makecell{FRD, p = 1 \\ (5) } & \makecell{FRD, p = 2 \\ (6) } & \makecell{FRD, p = 1 \\ (7) } & \makecell{FRD, p = 2 \\ (8) } & \makecell{FRD, p = 1 \\ (9) } & \makecell{FRD, p = 2 \\ (10) }\\
\midrule
\endfirsthead
\caption[]{\label{tab:agefrdcfhmodelssexessitemsmaths}Résultats de régressions sur une discontinuité, effets hétérogènes selon le sexe sur les sous-composantes de mathématiques (suite)}\\
\toprule
 & \makecell{FRD, p = 1 \\ (1) } & \makecell{FRD, p = 2 \\ (2) } & \makecell{FRD, p = 1 \\ (3) } & \makecell{FRD, p = 2 \\ (4) } & \makecell{FRD, p = 1 \\ (5) } & \makecell{FRD, p = 2 \\ (6) } & \makecell{FRD, p = 1 \\ (7) } & \makecell{FRD, p = 2 \\ (8) } & \makecell{FRD, p = 1 \\ (9) } & \makecell{FRD, p = 2 \\ (10) }\\
\midrule
\endhead

\endfoot
\bottomrule
\insertTableNotes
\endlastfoot
Âge aux examens & 0.67$^{***}$ & 0.695$^{***}$ & 0.779$^{***}$ & 0.98$^{***}$ & 0.694$^{***}$ & 0.683$^{***}$ & 0.5$^{***}$ & 0.497$^{***}$ & 0.607$^{***}$ & 0.497$^{***}$\\
 & (0.117) & (0.166) & (0.12) & (0.163) & (0.117) & (0.159) & (0.119) & (0.157) & (0.116) & (0.158)\\
Âge aux examens $\times$ Sexe - Garçon & $-$0.13 & $-$0.134 & $-$0.173 & $-$0.176 & $-$0.126 & $-$0.128 & $-$0.015 & $-$0.02 & $-$0.046 & $-$0.051\\
 & (0.111) & (0.111) & (0.116) & (0.116) & (0.112) & (0.112) & (0.113) & (0.113) & (0.111) & (0.111)\\
dist & $-$0.004 & $-$0.023$^{*}$ & $-$0.001 & $-$0.026$^{**}$ & $-$0.003 & $-$0.017 & $-$0.004 & $-$0.017 & $-$0.001 & $-$0.01\\
 & (0.003) & (0.013) & (0.003) & (0.013) & (0.003) & (0.012) & (0.003) & (0.012) & (0.003) & (0.012)\\
old $\times$ dist & 0.001 & 0.038$^{**}$ & $-$0.006 & 0.017 & 0.002 & 0.034$^{**}$ & 0.001 & 0.03$^{*}$ & $-$0.003 & 0.033$^{*}$\\
 & (0.004) & (0.017) & (0.005) & (0.017) & (0.004) & (0.017) & (0.004) & (0.017) & (0.004) & (0.017)\\
dist$^2$ & - & $-$0.001 & - & $-$0.001$^{*}$ & - & 0 & - & 0 & - & 0\\
 & - & (0) & - & (0) & - & (0) & - & (0) & - & (0)\\
old $\times$ dist$^2$ & - & 0 & - & 0.001 & - & 0 & - & 0 & - & $-$0.001\\
 & - & (0.001) & - & (0.001) & - & (0.001) & - & (0.001) & - & (0.001)\\
 &  &  &  &  &  &  &  &  &  & \\
Contrôles & Oui & Oui & Oui & Oui & Oui & Oui & Oui & Oui & Oui & Oui\\
Observations & 2271 & 2271 & 2271 & 2271 & 2271 & 2271 & 2271 & 2271 & 2271 & 2271\\
R$^2$ ajusté & 0.164 & 0.164 & 0.088 & 0.088 & 0.139 & 0.139 & 0.108 & 0.108 & 0.155 & 0.156\\*
\end{longtable}
\end{ThreePartTable}
\endgroup{}

\elandscape

\setcounter{table}{0}
\setcounter{figure}{0}

\hypertarget{agemodelsmtssmoy0}{%
\section{Résultats à moyen terme en utilisant des variables dépendantes alternatives}\label{agemodelsmtssmoy0}}

\hypertarget{agemodelsmtssmoy}{%
\subsection{Effets homogènes par rapport aux observables}\label{agemodelsmtssmoy}}

\begingroup\fontsize{8}{10}\selectfont

\begin{ThreePartTable}
\begin{TableNotes}
\item \textit{Sources :} Fichiers CM2 (2009 à 2012), calculs de l'auteur.
\item \textit{Notes :} Une colonne correspond à une régression. La note est normalisée sur l'année scolaire. Écart-types entre parenthèses. Les écart-types des estimations par fonction de contrôle sont calculés par wild bootstrap avec 1001 réplications. La variable $\hat{\nu}$ est le résidu de la première étape. Les contrôles utilisés sont le sexe, la CSP et l'année scolaire.
\item VI : Variable Instrumentale. FCH : Fonction de Contrôle avec prise en compte de l'Hétérogénéité de l'effet de l'âge.
\item Significativité : 10\% * 5\% ** 1\% ***.
\end{TableNotes}
\begin{longtable}[t]{lllllll}
\caption{\label{tab:agemodelsmtssmoy}Résultats principaux à moyen terme, variables dépendantes alternatives}\\
\toprule
\multicolumn{1}{c}{} & \multicolumn{6}{c}{Variable dépendante : Note en} \\
\cmidrule(l{3pt}r{3pt}){2-7}
\multicolumn{1}{c}{} & \multicolumn{2}{c}{Histoire-et-géographie} & \multicolumn{2}{c}{Dictée} & \multicolumn{2}{c}{Rédaction} \\
\cmidrule(l{3pt}r{3pt}){2-3} \cmidrule(l{3pt}r{3pt}){4-5} \cmidrule(l{3pt}r{3pt}){6-7}
 & \makecell{\makecell{VI \\ \ } \\ (1) } & \makecell{\makecell{FCH \\ \ } \\ (2) } & \makecell{\makecell{VI \\ \ } \\ (3) } & \makecell{\makecell{FCH \\ \ } \\ (4) } & \makecell{\makecell{VI \\ \ } \\ (5) } & \makecell{\makecell{FCH \\ \ } \\ (6) }\\
\midrule
\endfirsthead
\caption[]{\label{tab:agemodelsmtssmoy}Résultats principaux à moyen terme, variables dépendantes alternatives (suite)}\\
\toprule
\multicolumn{1}{c}{} & \multicolumn{6}{c}{Variable dépendante : Note en} \\
\cmidrule(l{3pt}r{3pt}){2-7}
\multicolumn{1}{c}{} & \multicolumn{2}{c}{Histoire-et-géographie} & \multicolumn{2}{c}{Dictée} & \multicolumn{2}{c}{Rédaction} \\
\cmidrule(l{3pt}r{3pt}){2-3} \cmidrule(l{3pt}r{3pt}){4-5} \cmidrule(l{3pt}r{3pt}){6-7}
 & \makecell{\makecell{VI \\ \ } \\ (1) } & \makecell{\makecell{FCH \\ \ } \\ (2) } & \makecell{\makecell{VI \\ \ } \\ (3) } & \makecell{\makecell{FCH \\ \ } \\ (4) } & \makecell{\makecell{VI \\ \ } \\ (5) } & \makecell{\makecell{FCH \\ \ } \\ (6) }\\
\midrule
\endhead

\endfoot
\bottomrule
\insertTableNotes
\endlastfoot
Âge aux examens & 0.171$^{***}$ & 0.177$^{***}$ & 0.165$^{***}$ & 0.167$^{***}$ & 0.147$^{***}$ & 0.148$^{***}$\\
 & (0.021) & (0.02) & (0.02) & (0.019) & (0.021) & (0.021)\\
$\hat{\nu}$ & - & $-$1.643$^{***}$ & - & $-$1.054$^{***}$ & - & $-$0.608$^{***}$\\
 & - & (0.202) & - & (0.209) & - & (0.208)\\
Âge aux examens $\times$ $\hat{\nu}$ & - & 0.062$^{***}$ & - & 0.019 & - & 0.004\\
 & - & (0.013) & - & (0.013) & - & (0.013)\\
 &  &  &  &  &  & \\
Contrôles & Oui & Oui & Oui & Oui & Oui & Oui\\
Observations & 41464 & 41464 & 41479 & 41479 & 41445 & 41445\\
R$^2$ ajusté & - & 0.183 & - & 0.242 & - & 0.12\\*
\end{longtable}
\end{ThreePartTable}
\endgroup{}

\newpage

\hypertarget{agemodelsmtssmoysexemod}{%
\subsection{Effets hétérogènes selon le sexe}\label{agemodelsmtssmoysexemod}}

\begingroup\fontsize{8}{10}\selectfont

\begin{ThreePartTable}
\begin{TableNotes}
\item \textit{Sources :} Fichiers CM2 (2009 à 2012), calculs de l'auteur.
\item \textit{Notes :} Une colonne correspond à une régression. La note est normalisée sur l'année scolaire. Écart-types entre parenthèses. Les écart-types des estimations par fonction de contrôle sont calculés par wild bootstrap avec 1001 réplications. Les contrôles utilisés sont le sexe, la CSP et l'année scolaire.
\item VI : Variable Instrumentale. FCH : Fonction de Contrôle avec prise en compte de l'Hétérogénéité de l'effet de l'âge.
\item Significativité : 10\% * 5\% ** 1\% ***.
\end{TableNotes}
\begin{longtable}[t]{lllllll}
\caption{\label{tab:agemodelsmtsexemodssmoy}Résultats principaux hétérogènes selon le sexe, à moyen terme, variables dépendantes alternatives}\\
\toprule
\multicolumn{1}{c}{} & \multicolumn{6}{c}{Variable dépendante : Note en} \\
\cmidrule(l{3pt}r{3pt}){2-7}
\multicolumn{1}{c}{} & \multicolumn{2}{c}{Histoire-et-géographie} & \multicolumn{2}{c}{Dictée} & \multicolumn{2}{c}{Rédaction} \\
\cmidrule(l{3pt}r{3pt}){2-3} \cmidrule(l{3pt}r{3pt}){4-5} \cmidrule(l{3pt}r{3pt}){6-7}
 & \makecell{\makecell{VI \\ \ } \\ (1) } & \makecell{\makecell{FCH \\ \ } \\ (2) } & \makecell{\makecell{VI \\ \ } \\ (3) } & \makecell{\makecell{FCH \\ \ } \\ (4) } & \makecell{\makecell{VI \\ \ } \\ (5) } & \makecell{\makecell{FCH \\ \ } \\ (6) }\\
\midrule
\endfirsthead
\caption[]{\label{tab:agemodelsmtsexemodssmoy}Résultats principaux hétérogènes selon le sexe, à moyen terme, variables dépendantes alternatives (suite)}\\
\toprule
\multicolumn{1}{c}{} & \multicolumn{6}{c}{Variable dépendante : Note en} \\
\cmidrule(l{3pt}r{3pt}){2-7}
\multicolumn{1}{c}{} & \multicolumn{2}{c}{Histoire-et-géographie} & \multicolumn{2}{c}{Dictée} & \multicolumn{2}{c}{Rédaction} \\
\cmidrule(l{3pt}r{3pt}){2-3} \cmidrule(l{3pt}r{3pt}){4-5} \cmidrule(l{3pt}r{3pt}){6-7}
 & \makecell{\makecell{VI \\ \ } \\ (1) } & \makecell{\makecell{FCH \\ \ } \\ (2) } & \makecell{\makecell{VI \\ \ } \\ (3) } & \makecell{\makecell{FCH \\ \ } \\ (4) } & \makecell{\makecell{VI \\ \ } \\ (5) } & \makecell{\makecell{FCH \\ \ } \\ (6) }\\
\midrule
\endhead

\endfoot
\bottomrule
\insertTableNotes
\endlastfoot
Âge aux examens & 0.162$^{***}$ & 0.151$^{***}$ & 0.164$^{***}$ & 0.157$^{***}$ & 0.144$^{***}$ & 0.159$^{***}$\\
 & (0.028) & (0.026) & (0.028) & (0.025) & (0.028) & (0.026)\\
Âge aux examens $\times$ Sexe - Garçon & 0.018 & 0.055 & 0.002 & 0.022 & 0.006 & $-$0.021\\
 & (0.041) & (0.036) & (0.04) & (0.034) & (0.042) & (0.037)\\
 &  &  &  &  &  & \\
Contrôles & Oui & Oui & Oui & Oui & Oui & Oui\\
Observations & 41464 & 41464 & 41479 & 41479 & 41445 & 41445\\
R$^2$ ajusté & - & 0.183 & - & 0.242 & - & 0.12\\*
\end{longtable}
\end{ThreePartTable}
\endgroup{}

\hypertarget{agemodelsmtssmoypcsregmod}{%
\subsection{Effets hétérogènes selon la catégorie sociale}\label{agemodelsmtssmoypcsregmod}}

\begingroup\fontsize{8}{10}\selectfont

\begin{ThreePartTable}
\begin{TableNotes}
\item \textit{Sources :} Fichiers CM2 (2009 à 2012), calculs de l'auteur.
\item \textit{Notes :} Une colonne correspond à une régression. La note est normalisée sur l'année scolaire. Écart-types entre parenthèses. Les écart-types des estimations par fonction de contrôle sont calculés par wild bootstrap avec 1001 réplications. Les contrôles utilisés sont le sexe, la CSP et l'année scolaire.
\item VI : Variable instrumentale. FCH : Fonction de Contrôle avec prise en compte de l'Hétérogénéité de l'effet de l'âge. CSP : Catégorie Socio-Professionnelle.
\item Significativité : 10\% * 5\% ** 1\% ***.
\end{TableNotes}
\begin{longtable}[t]{lllllll}
\caption{\label{tab:agemodelsmtpcsregmodssmoy}Résultats principaux hétérogènes selon la catégorie sociale, à moyen terme, variables dépendantes alternatives}\\
\toprule
\multicolumn{1}{c}{} & \multicolumn{6}{c}{Variable dépendante : Note en} \\
\cmidrule(l{3pt}r{3pt}){2-7}
\multicolumn{1}{c}{} & \multicolumn{2}{c}{Histoire-et-géographie} & \multicolumn{2}{c}{Dictée} & \multicolumn{2}{c}{Rédaction} \\
\cmidrule(l{3pt}r{3pt}){2-3} \cmidrule(l{3pt}r{3pt}){4-5} \cmidrule(l{3pt}r{3pt}){6-7}
 & \makecell{\makecell{VI \\ \ } \\ (1) } & \makecell{\makecell{FCH \\ \ } \\ (2) } & \makecell{\makecell{VI \\ \ } \\ (3) } & \makecell{\makecell{FCH \\ \ } \\ (4) } & \makecell{\makecell{VI \\ \ } \\ (5) } & \makecell{\makecell{FCH \\ \ } \\ (6) }\\
\midrule
\endfirsthead
\caption[]{\label{tab:agemodelsmtpcsregmodssmoy}Résultats principaux hétérogènes selon la catégorie sociale, à moyen terme, variables dépendantes alternatives (suite)}\\
\toprule
\multicolumn{1}{c}{} & \multicolumn{6}{c}{Variable dépendante : Note en} \\
\cmidrule(l{3pt}r{3pt}){2-7}
\multicolumn{1}{c}{} & \multicolumn{2}{c}{Histoire-et-géographie} & \multicolumn{2}{c}{Dictée} & \multicolumn{2}{c}{Rédaction} \\
\cmidrule(l{3pt}r{3pt}){2-3} \cmidrule(l{3pt}r{3pt}){4-5} \cmidrule(l{3pt}r{3pt}){6-7}
 & \makecell{\makecell{VI \\ \ } \\ (1) } & \makecell{\makecell{FCH \\ \ } \\ (2) } & \makecell{\makecell{VI \\ \ } \\ (3) } & \makecell{\makecell{FCH \\ \ } \\ (4) } & \makecell{\makecell{VI \\ \ } \\ (5) } & \makecell{\makecell{FCH \\ \ } \\ (6) }\\
\midrule
\endhead

\endfoot
\bottomrule
\insertTableNotes
\endlastfoot
Âge aux examens & 0.141$^{***}$ & 0.158$^{***}$ & 0.152$^{***}$ & 0.156$^{***}$ & 0.147$^{***}$ & 0.139$^{***}$\\
 & (0.028) & (0.025) & (0.028) & (0.026) & (0.03) & (0.028)\\
Âge aux examens $\times$ CSP - Moyenne & 0.082$^{*}$ & 0.098$^{**}$ & 0.057 & 0.09$^{**}$ & 0.024 & 0.065\\
 & (0.05) & (0.045) & (0.049) & (0.045) & (0.05) & (0.048)\\
Âge aux examens $\times$ CSP - Favorisée & $-$0.069 & $-$0.064 & $-$0.071 & $-$0.044 & $-$0.023 & 0.013\\
 & (0.076) & (0.076) & (0.073) & (0.071) & (0.075) & (0.074)\\
Âge aux examens $\times$ CSP - Très favorisée & 0.136$^{**}$ & 0.102$^{*}$ & 0.062 & 0.056 & $-$0.011 & 0.018\\
 & (0.068) & (0.059) & (0.063) & (0.057) & (0.066) & (0.061)\\
Âge aux examens $\times$ CSP - Autres & $-$0.092 & 0.092 & $-$0.274 & $-$0.156 & $-$0.356 & $-$0.294\\
 & (0.238) & (0.228) & (0.213) & (0.216) & (0.252) & (0.27)\\
 &  &  &  &  &  & \\
Contrôles & Oui & Oui & Oui & Oui & Oui & Oui\\
Observations & 41464 & 41464 & 41479 & 41479 & 41445 & 41445\\
R$^2$ ajusté & - & 0.185 & - & 0.244 & - & 0.12\\*
\end{longtable}
\end{ThreePartTable}
\endgroup{}

\setcounter{table}{0}
\setcounter{figure}{0}
\blandscape

\hypertarget{agemodelssuppssitems0}{%
\section{Résultats des modèles d'extension en utilisant les sous-composantes des notes au CM2 comme variables dépendantes}\label{agemodelssuppssitems0}}

\hypertarget{agemodelssuppssitems}{%
\subsection{Effets homogènes par rapport aux observables}\label{agemodelssuppssitems}}

\begingroup\fontsize{5}{7}\selectfont

\begin{ThreePartTable}
\begin{TableNotes}
\item \textit{Sources :} Fichiers CM2 (2009 à 2012), calculs de l'auteur.
\item \textit{Notes :} Une colonne correspond à une régression. La note est normalisée sur l'année scolaire. Écart-types entre parenthèses. Les modèles labellisés ABS sont estimés par fonction de contrôle. Les autres modèles sont estimés par variables instrumentales. Les écart-types des estimations par fonction de contrôle sont calculés par wild bootstrap avec 1001 réplications. Les contrôles utilisés sont le sexe, la CSP et l'année scolaire.
\item ABS, REL et ABSREL sont décrits dans le corps du texte.
\item Significativité : 10\% * 5\% ** 1\% ***.
\end{TableNotes}
\begin{longtable}[t]{llllllllllllllll}
\caption{\label{tab:agemodelsrelssitemsfrench}Effets de l'âge absolu et de l'âge relatif (CM2), sous-items de français}\\
\toprule
\multicolumn{1}{c}{} & \multicolumn{15}{c}{Variable dépendante : Note en } \\
\cmidrule(l{3pt}r{3pt}){2-16}
\multicolumn{1}{c}{} & \multicolumn{3}{c}{Écriture} & \multicolumn{3}{c}{Grammaire} & \multicolumn{3}{c}{Lecture} & \multicolumn{3}{c}{Orthographe} & \multicolumn{3}{c}{Vocabulaire} \\
\cmidrule(l{3pt}r{3pt}){2-4} \cmidrule(l{3pt}r{3pt}){5-7} \cmidrule(l{3pt}r{3pt}){8-10} \cmidrule(l{3pt}r{3pt}){11-13} \cmidrule(l{3pt}r{3pt}){14-16}
 & \makecell{ABS \\ (1) } & \makecell{REL \\ (2) } & \makecell{ABSREL \\ (3) } & \makecell{ABS \\ (4) } & \makecell{REL \\ (5) } & \makecell{ABSREL \\ (6) } & \makecell{ABS \\ (7) } & \makecell{REL \\ (8) } & \makecell{ABSREL \\ (9) } & \makecell{ABS \\ (10) } & \makecell{REL \\ (11) } & \makecell{ABSREL \\ (12) } & \makecell{ABS \\ (13) } & \makecell{REL \\ (14) } & \makecell{ABSREL \\ (15) }\\
\midrule
\endfirsthead
\caption[]{\label{tab:agemodelsrelssitemsfrench}Effets de l'âge absolu et de l'âge relatif (CM2), sous-items de français (suite)}\\
\toprule
\multicolumn{1}{c}{} & \multicolumn{15}{c}{Variable dépendante : Note en } \\
\cmidrule(l{3pt}r{3pt}){2-16}
\multicolumn{1}{c}{} & \multicolumn{3}{c}{Écriture} & \multicolumn{3}{c}{Grammaire} & \multicolumn{3}{c}{Lecture} & \multicolumn{3}{c}{Orthographe} & \multicolumn{3}{c}{Vocabulaire} \\
\cmidrule(l{3pt}r{3pt}){2-4} \cmidrule(l{3pt}r{3pt}){5-7} \cmidrule(l{3pt}r{3pt}){8-10} \cmidrule(l{3pt}r{3pt}){11-13} \cmidrule(l{3pt}r{3pt}){14-16}
 & \makecell{ABS \\ (1) } & \makecell{REL \\ (2) } & \makecell{ABSREL \\ (3) } & \makecell{ABS \\ (4) } & \makecell{REL \\ (5) } & \makecell{ABSREL \\ (6) } & \makecell{ABS \\ (7) } & \makecell{REL \\ (8) } & \makecell{ABSREL \\ (9) } & \makecell{ABS \\ (10) } & \makecell{REL \\ (11) } & \makecell{ABSREL \\ (12) } & \makecell{ABS \\ (13) } & \makecell{REL \\ (14) } & \makecell{ABSREL \\ (15) }\\
\midrule
\endhead

\endfoot
\bottomrule
\insertTableNotes
\endlastfoot
Âge aux examens & 0.275$^{***}$ & 0.311$^{***}$ & 0.355$^{***}$ & 0.27$^{***}$ & 0.485$^{***}$ & 0.493$^{***}$ & 0.294$^{***}$ & 0.494$^{***}$ & 0.523$^{***}$ & 0.253$^{***}$ & 0.37$^{***}$ & 0.399$^{***}$ & 0.244$^{***}$ & 0.481$^{***}$ & 0.465$^{***}$\\
 & (0.035) & (0.076) & (0.084) & (0.036) & (0.078) & (0.086) & (0.035) & (0.079) & (0.087) & (0.035) & (0.077) & (0.085) & (0.036) & (0.079) & (0.087)\\
Âge relatif (classe) & - & $-$0.055 & $-$0.054 & - & $-$0.228$^{***}$ & $-$0.228$^{***}$ & - & $-$0.215$^{***}$ & $-$0.215$^{***}$ & - & $-$0.131$^{*}$ & $-$0.131$^{*}$ & - & $-$0.225$^{***}$ & $-$0.225$^{***}$\\
 & - & (0.075) & (0.075) & - & (0.076) & (0.076) & - & (0.077) & (0.077) & - & (0.075) & (0.076) & - & (0.077) & (0.077)\\
Âge aux examens $\times$ Cohorte - 2010 & 0.01 & - & $-$0.056 & 0.02 & - & $-$0.023 & 0.04 & - & $-$0.012 & 0.019 & - & $-$0.032 & 0.07 & - & 0.051\\
 & (0.048) & - & (0.054) & (0.047) & - & (0.055) & (0.047) & - & (0.055) & (0.047) & - & (0.054) & (0.047) & - & (0.055)\\
Âge aux examens $\times$ Cohorte - 2011 & $-$0.061 & - & $-$0.073 & $-$0.055 & - & 0 & $-$0.053 & - & $-$0.052 & $-$0.056 & - & $-$0.041 & $-$0.041 & - & $-$0.015\\
 & (0.045) & - & (0.053) & (0.047) & - & (0.054) & (0.047) & - & (0.054) & (0.047) & - & (0.054) & (0.046) & - & (0.054)\\
Âge aux examens $\times$ Cohorte - 2012 & $-$0.027 & - & $-$0.043 & $-$0.012 & - & $-$0.01 & $-$0.047 & - & $-$0.048 & $-$0.022 & - & $-$0.043 & 0.011 & - & 0.028\\
 & (0.047) & - & (0.054) & (0.049) & - & (0.055) & (0.047) & - & (0.055) & (0.047) & - & (0.054) & (0.048) & - & (0.055)\\
 &  &  &  &  &  &  &  &  &  &  &  &  &  &  & \\
Contrôles & Oui & Oui & Oui & Oui & Oui & Oui & Oui & Oui & Oui & Oui & Oui & Oui & Oui & Oui & Oui\\
Observations & 54341 & 54319 & 54319 & 54341 & 54319 & 54319 & 54341 & 54319 & 54319 & 54341 & 54319 & 54319 & 54341 & 54319 & 54319\\
R$^2$ ajusté & 0.18 & - & - & 0.164 & - & - & 0.176 & - & - & 0.207 & - & - & 0.174 & - & -\\*
\end{longtable}
\end{ThreePartTable}
\endgroup{}

\newpage

\begingroup\fontsize{5}{7}\selectfont

\begin{ThreePartTable}
\begin{TableNotes}
\item \textit{Sources :} Fichiers CM2 (2009 à 2012), calculs de l'auteur.
\item \textit{Notes :} Une colonne correspond à une régression. La note est normalisée sur l'année scolaire. Écart-types entre parenthèses. Les modèles labellisés ABS sont estimés par fonction de contrôle. Les autres modèles sont estimés par variables instrumentales. Les écart-types des estimations par fonction de contrôle sont calculés par wild bootstrap avec 1001 réplications. Les contrôles utilisés sont le sexe, la CSP et l'année scolaire.
\item ABS, REL et ABSREL sont décrits dans le corps du texte.
\item Significativité : 10\% * 5\% ** 1\% ***.
\end{TableNotes}
\begin{longtable}[t]{llllllllllllllll}
\caption{\label{tab:agemodelsrelssitemsmaths}Effets de l'âge absolu et de l'âge relatif (CM2), sous-items de mathématiques}\\
\toprule
\multicolumn{1}{c}{} & \multicolumn{15}{c}{Variable dépendante : Note en } \\
\cmidrule(l{3pt}r{3pt}){2-16}
\multicolumn{1}{c}{} & \multicolumn{3}{c}{Calcul} & \multicolumn{3}{c}{Géométrie} & \multicolumn{3}{c}{\makecell{Grandeurs et \\ mesures}} & \multicolumn{3}{c}{Nombre} & \multicolumn{3}{c}{\makecell{Organisation et \\ gestion de données}} \\
\cmidrule(l{3pt}r{3pt}){2-4} \cmidrule(l{3pt}r{3pt}){5-7} \cmidrule(l{3pt}r{3pt}){8-10} \cmidrule(l{3pt}r{3pt}){11-13} \cmidrule(l{3pt}r{3pt}){14-16}
 & \makecell{ABS \\ (1) } & \makecell{REL \\ (2) } & \makecell{ABSREL \\ (3) } & \makecell{ABS \\ (4) } & \makecell{REL \\ (5) } & \makecell{ABSREL \\ (6) } & \makecell{ABS \\ (7) } & \makecell{REL \\ (8) } & \makecell{ABSREL \\ (9) } & \makecell{ABS \\ (10) } & \makecell{REL \\ (11) } & \makecell{ABSREL \\ (12) } & \makecell{ABS \\ (13) } & \makecell{REL \\ (14) } & \makecell{ABSREL \\ (15) }\\
\midrule
\endfirsthead
\caption[]{\label{tab:agemodelsrelssitemsmaths}Effets de l'âge absolu et de l'âge relatif (CM2), sous-items de mathématiques (suite)}\\
\toprule
\multicolumn{1}{c}{} & \multicolumn{15}{c}{Variable dépendante : Note en } \\
\cmidrule(l{3pt}r{3pt}){2-16}
\multicolumn{1}{c}{} & \multicolumn{3}{c}{Calcul} & \multicolumn{3}{c}{Géométrie} & \multicolumn{3}{c}{\makecell{Grandeurs et \\ mesures}} & \multicolumn{3}{c}{Nombre} & \multicolumn{3}{c}{\makecell{Organisation et \\ gestion de données}} \\
\cmidrule(l{3pt}r{3pt}){2-4} \cmidrule(l{3pt}r{3pt}){5-7} \cmidrule(l{3pt}r{3pt}){8-10} \cmidrule(l{3pt}r{3pt}){11-13} \cmidrule(l{3pt}r{3pt}){14-16}
 & \makecell{ABS \\ (1) } & \makecell{REL \\ (2) } & \makecell{ABSREL \\ (3) } & \makecell{ABS \\ (4) } & \makecell{REL \\ (5) } & \makecell{ABSREL \\ (6) } & \makecell{ABS \\ (7) } & \makecell{REL \\ (8) } & \makecell{ABSREL \\ (9) } & \makecell{ABS \\ (10) } & \makecell{REL \\ (11) } & \makecell{ABSREL \\ (12) } & \makecell{ABS \\ (13) } & \makecell{REL \\ (14) } & \makecell{ABSREL \\ (15) }\\
\midrule
\endhead

\endfoot
\bottomrule
\insertTableNotes
\endlastfoot
Âge aux examens & 0.259$^{***}$ & 0.754$^{***}$ & 0.76$^{***}$ & 0.332$^{***}$ & 0.598$^{***}$ & 0.659$^{***}$ & 0.347$^{***}$ & 0.727$^{***}$ & 0.761$^{***}$ & 0.251$^{***}$ & 0.449$^{***}$ & 0.478$^{***}$ & 0.347$^{***}$ & 0.727$^{***}$ & 0.761$^{***}$\\
 & (0.035) & (0.079) & (0.087) & (0.038) & (0.08) & (0.088) & (0.037) & (0.079) & (0.087) & (0.035) & (0.078) & (0.086) & (0.037) & (0.079) & (0.087)\\
Âge relatif (classe) & - & $-$0.494$^{***}$ & $-$0.494$^{***}$ & - & $-$0.318$^{***}$ & $-$0.317$^{***}$ & - & $-$0.414$^{***}$ & $-$0.414$^{***}$ & - & $-$0.221$^{***}$ & $-$0.22$^{***}$ & - & $-$0.414$^{***}$ & $-$0.414$^{***}$\\
 & - & (0.077) & (0.077) & - & (0.078) & (0.078) & - & (0.077) & (0.077) & - & (0.076) & (0.076) & - & (0.077) & (0.077)\\
Âge aux examens $\times$ Cohorte - 2010 & 0.061 & - & 0 & $-$0.017 & - & $-$0.047 & 0.044 & - & $-$0.037 & 0.025 & - & 0.006 & 0.044 & - & $-$0.037\\
 & (0.048) & - & (0.055) & (0.051) & - & (0.056) & (0.05) & - & (0.056) & (0.049) & - & (0.055) & (0.05) & - & (0.056)\\
Âge aux examens $\times$ Cohorte - 2011 & $-$0.045 & - & $-$0.001 & $-$0.103$^{**}$ & - & $-$0.072 & $-$0.082$^{*}$ & - & $-$0.02 & $-$0.091$^{*}$ & - & $-$0.07 & $-$0.082$^{*}$ & - & $-$0.02\\
 & (0.046) & - & (0.055) & (0.049) & - & (0.055) & (0.048) & - & (0.055) & (0.047) & - & (0.054) & (0.048) & - & (0.055)\\
Âge aux examens $\times$ Cohorte - 2012 & $-$0.01 & - & $-$0.025 & $-$0.082 & - & $-$0.123$^{**}$ & $-$0.085$^{*}$ & - & $-$0.079 & $-$0.022 & - & $-$0.05 & $-$0.085$^{*}$ & - & $-$0.079\\
 & (0.048) & - & (0.055) & (0.052) & - & (0.056) & (0.05) & - & (0.056) & (0.051) & - & (0.055) & (0.05) & - & (0.056)\\
 &  &  &  &  &  &  &  &  &  &  &  &  &  &  & \\
Contrôles & Oui & Oui & Oui & Oui & Oui & Oui & Oui & Oui & Oui & Oui & Oui & Oui & Oui & Oui & Oui\\
Observations & 54341 & 54319 & 54319 & 54341 & 54319 & 54319 & 54341 & 54319 & 54319 & 54341 & 54319 & 54319 & 54341 & 54319 & 54319\\
R$^2$ ajusté & 0.133 & - & - & 0.073 & - & - & 0.115 & - & - & 0.107 & - & - & 0.115 & - & -\\*
\end{longtable}
\end{ThreePartTable}
\endgroup{}
\elandscape

\blandscape

\hypertarget{agemodelssuppsexessitems}{%
\subsection{Effets hétérogènes selon le sexe}\label{agemodelssuppsexessitems}}

\begingroup\fontsize{4}{6}\selectfont

\begin{ThreePartTable}
\begin{TableNotes}
\item \textit{Sources :} Fichiers CM2 (2009 à 2012), calculs de l'auteur.
\item \textit{Notes :} Une colonne correspond à une régression. La note est normalisée sur l'année scolaire. Écart-types entre parenthèses. Les modèles labellisés ABS sont estimés par fonction de contrôle. Les autres modèles sont estimés par variables instrumentales. Les écart-types des estimations par fonction de contrôle sont calculés par wild bootstrap avec 1001 réplications. Les contrôles utilisés sont le sexe, la CSP et l'année scolaire.
\item ABS, REL et ABSREL sont décrits dans le corps du texte.
\item Significativité : 10\% * 5\% ** 1\% ***.
\end{TableNotes}
\begin{longtable}[t]{llllllllllllllll}
\caption{\label{tab:agemodelsrelsexessitemsfrench}Effets de l'âge absolu et de l'âge relatif hétérogènes selon le sexe (CM2), sous-items de français}\\
\toprule
\multicolumn{1}{c}{} & \multicolumn{15}{c}{Variable dépendante : Note en } \\
\cmidrule(l{3pt}r{3pt}){2-16}
\multicolumn{1}{c}{} & \multicolumn{3}{c}{Écriture} & \multicolumn{3}{c}{Grammaire} & \multicolumn{3}{c}{Lecture} & \multicolumn{3}{c}{Orthographe} & \multicolumn{3}{c}{Vocabulaire} \\
\cmidrule(l{3pt}r{3pt}){2-4} \cmidrule(l{3pt}r{3pt}){5-7} \cmidrule(l{3pt}r{3pt}){8-10} \cmidrule(l{3pt}r{3pt}){11-13} \cmidrule(l{3pt}r{3pt}){14-16}
 & \makecell{ABS \\ (1) } & \makecell{REL \\ (2) } & \makecell{ABSREL \\ (3) } & \makecell{ABS \\ (4) } & \makecell{REL \\ (5) } & \makecell{ABSREL \\ (6) } & \makecell{ABS \\ (7) } & \makecell{REL \\ (8) } & \makecell{ABSREL \\ (9) } & \makecell{ABS \\ (10) } & \makecell{REL \\ (11) } & \makecell{ABSREL \\ (12) } & \makecell{ABS \\ (13) } & \makecell{REL \\ (14) } & \makecell{ABSREL \\ (15) }\\
\midrule
\endfirsthead
\caption[]{\label{tab:agemodelsrelsexessitemsfrench}Effets de l'âge absolu et de l'âge relatif hétérogènes selon le sexe (CM2), sous-items de français (suite)}\\
\toprule
\multicolumn{1}{c}{} & \multicolumn{15}{c}{Variable dépendante : Note en } \\
\cmidrule(l{3pt}r{3pt}){2-16}
\multicolumn{1}{c}{} & \multicolumn{3}{c}{Écriture} & \multicolumn{3}{c}{Grammaire} & \multicolumn{3}{c}{Lecture} & \multicolumn{3}{c}{Orthographe} & \multicolumn{3}{c}{Vocabulaire} \\
\cmidrule(l{3pt}r{3pt}){2-4} \cmidrule(l{3pt}r{3pt}){5-7} \cmidrule(l{3pt}r{3pt}){8-10} \cmidrule(l{3pt}r{3pt}){11-13} \cmidrule(l{3pt}r{3pt}){14-16}
 & \makecell{ABS \\ (1) } & \makecell{REL \\ (2) } & \makecell{ABSREL \\ (3) } & \makecell{ABS \\ (4) } & \makecell{REL \\ (5) } & \makecell{ABSREL \\ (6) } & \makecell{ABS \\ (7) } & \makecell{REL \\ (8) } & \makecell{ABSREL \\ (9) } & \makecell{ABS \\ (10) } & \makecell{REL \\ (11) } & \makecell{ABSREL \\ (12) } & \makecell{ABS \\ (13) } & \makecell{REL \\ (14) } & \makecell{ABSREL \\ (15) }\\
\midrule
\endhead

\endfoot
\bottomrule
\insertTableNotes
\endlastfoot
Âge aux examens & 0.24$^{***}$ & 0.14 & 0.184$^{*}$ & 0.218$^{***}$ & 0.357$^{***}$ & 0.357$^{***}$ & 0.259$^{***}$ & 0.413$^{***}$ & 0.438$^{***}$ & 0.227$^{***}$ & 0.284$^{***}$ & 0.3$^{***}$ & 0.219$^{***}$ & 0.325$^{***}$ & 0.298$^{***}$\\
 & (0.046) & (0.101) & (0.109) & (0.05) & (0.109) & (0.118) & (0.048) & (0.107) & (0.116) & (0.046) & (0.105) & (0.113) & (0.049) & (0.106) & (0.114)\\
Âge aux examens $\times$ Sexe - Garçon & 0.089 & 0.337$^{**}$ & 0.337$^{**}$ & 0.081 & 0.253$^{*}$ & 0.273$^{*}$ & 0.072 & 0.161 & 0.167 & 0.055 & 0.17 & 0.196 & 0.019 & 0.307$^{**}$ & 0.334$^{**}$\\
 & (0.072) & (0.145) & (0.152) & (0.072) & (0.148) & (0.156) & (0.074) & (0.15) & (0.158) & (0.069) & (0.147) & (0.154) & (0.073) & (0.15) & (0.158)\\
Âge relatif (classe) & - & 0.095 & 0.096 & - & $-$0.096 & $-$0.09 & - & $-$0.132 & $-$0.13 & - & $-$0.035 & $-$0.023 & - & $-$0.075 & $-$0.066\\
 & - & (0.099) & (0.101) & - & (0.107) & (0.11) & - & (0.106) & (0.108) & - & (0.103) & (0.105) & - & (0.104) & (0.106)\\
Âge relatif (classe) $\times$ Sexe - Garçon & - & $-$0.295$^{**}$ & $-$0.294$^{**}$ & - & $-$0.261$^{*}$ & $-$0.273$^{*}$ & - & $-$0.164 & $-$0.167 & - & $-$0.19 & $-$0.212 & - & $-$0.296$^{**}$ & $-$0.316$^{**}$\\
 & - & (0.143) & (0.149) & - & (0.146) & (0.152) & - & (0.148) & (0.154) & - & (0.144) & (0.151) & - & (0.147) & (0.154)\\
Âge aux examens $\times$ Cohorte - 2010 & 0.086 & - & $-$0.056 & 0.137$^{**}$ & - & $-$0.022 & 0.142$^{**}$ & - & $-$0.011 & 0.088 & - & $-$0.03 & 0.172$^{***}$ & - & 0.055\\
 & (0.064) & - & (0.054) & (0.065) & - & (0.055) & (0.066) & - & (0.056) & (0.062) & - & (0.054) & (0.065) & - & (0.055)\\
Âge aux examens $\times$ Cohorte - 2011 & 0.002 & - & $-$0.073 & 0.045 & - & 0.003 & 0.031 & - & $-$0.051 & $-$0.004 & - & $-$0.042 & 0.06 & - & $-$0.013\\
 & (0.061) & - & (0.054) & (0.068) & - & (0.054) & (0.063) & - & (0.055) & (0.064) & - & (0.054) & (0.063) & - & (0.054)\\
Âge aux examens $\times$ Cohorte - 2012 & $-$0.005 & - & $-$0.042 & 0.03 & - & $-$0.005 & 0.001 & - & $-$0.045 & 0.045 & - & $-$0.039 & 0.003 & - & 0.031\\
 & (0.064) & - & (0.054) & (0.07) & - & (0.055) & (0.064) & - & (0.055) & (0.065) & - & (0.055) & (0.067) & - & (0.055)\\
Âge aux examens $\times$ Cohorte - 2010 $\times$ Sexe - Garçon & $-$0.185$^{*}$ & - & 0.002 & $-$0.222$^{**}$ & - & $-$0.005$^{*}$ & $-$0.225$^{**}$ & - & $-$0.001 & $-$0.14 & - & $-$0.003 & $-$0.136 & - & $-$0.011$^{***}$\\
 & (0.095) & - & (0.003) & (0.093) & - & (0.003) & (0.099) & - & (0.003) & (0.093) & - & (0.003) & (0.095) & - & (0.003)\\
Âge aux examens $\times$ Cohorte - 2011 $\times$ Sexe - Garçon & $-$0.137 & - & 0.001 & $-$0.168$^{*}$ & - & $-$0.007$^{**}$ & $-$0.173$^{*}$ & - & $-$0.002 & $-$0.128 & - & 0.002 & $-$0.164$^{*}$ & - & $-$0.006$^{**}$\\
 & (0.095) & - & (0.003) & (0.094) & - & (0.003) & (0.093) & - & (0.003) & (0.094) & - & (0.003) & (0.094) & - & (0.003)\\
Âge aux examens $\times$ Cohorte - 2012 $\times$ Sexe - Garçon & $-$0.081 & - & $-$0.004 & $-$0.03 & - & $-$0.015$^{***}$ & $-$0.083 & - & $-$0.008$^{*}$ & $-$0.12 & - & $-$0.009$^{**}$ & 0.038 & - & $-$0.011$^{**}$\\
 & (0.096) & - & (0.004) & (0.099) & - & (0.005) & (0.096) & - & (0.005) & (0.096) & - & (0.004) & (0.095) & - & (0.005)\\
 &  &  &  &  &  &  &  &  &  &  &  &  &  &  & \\
Contrôles & Oui & Oui & Oui & Oui & Oui & Oui & Oui & Oui & Oui & Oui & Oui & Oui & Oui & Oui & Oui\\
Observations & 54341 & 54319 & 54319 & 54341 & 54319 & 54319 & 54341 & 54319 & 54319 & 54341 & 54319 & 54319 & 54341 & 54319 & 54319\\
R$^2$ ajusté & 0.18 & - & - & 0.164 & - & - & 0.176 & - & - & 0.207 & - & - & 0.175 & - & -\\*
\end{longtable}
\end{ThreePartTable}
\endgroup{}

\newpage

\begingroup\fontsize{4}{6}\selectfont

\begin{ThreePartTable}
\begin{TableNotes}
\item \textit{Sources :} Fichiers CM2 (2009 à 2012), calculs de l'auteur.
\item \textit{Notes :} Une colonne correspond à une régression. La note est normalisée sur l'année scolaire. Écart-types entre parenthèses. Les modèles labellisés ABS sont estimés par fonction de contrôle. Les autres modèles sont estimés par variables instrumentales. Les écart-types des estimations par fonction de contrôle sont calculés par wild bootstrap avec 1001 réplications. Les contrôles utilisés sont le sexe, la CSP et l'année scolaire.
\item ABS, REL et ABSREL sont décrits dans le corps du texte.
\item Significativité : 10\% * 5\% ** 1\% ***.
\end{TableNotes}
\begin{longtable}[t]{llllllllllllllll}
\caption{\label{tab:agemodelsrelsexessitemsmaths}Effets de l'âge absolu et de l'âge relatif hétérogènes selon le sexe (CM2), sous-items de mathématiques}\\
\toprule
\multicolumn{1}{c}{} & \multicolumn{15}{c}{Variable dépendante : Note en } \\
\cmidrule(l{3pt}r{3pt}){2-16}
\multicolumn{1}{c}{} & \multicolumn{3}{c}{Calcul} & \multicolumn{3}{c}{Géométrie} & \multicolumn{3}{c}{\makecell{Grandeurs et \\ mesures}} & \multicolumn{3}{c}{Nombre} & \multicolumn{3}{c}{\makecell{Organisation et \\ gestion de données}} \\
\cmidrule(l{3pt}r{3pt}){2-4} \cmidrule(l{3pt}r{3pt}){5-7} \cmidrule(l{3pt}r{3pt}){8-10} \cmidrule(l{3pt}r{3pt}){11-13} \cmidrule(l{3pt}r{3pt}){14-16}
 & \makecell{ABS \\ (1) } & \makecell{REL \\ (2) } & \makecell{ABSREL \\ (3) } & \makecell{ABS \\ (4) } & \makecell{REL \\ (5) } & \makecell{ABSREL \\ (6) } & \makecell{ABS \\ (7) } & \makecell{REL \\ (8) } & \makecell{ABSREL \\ (9) } & \makecell{ABS \\ (10) } & \makecell{REL \\ (11) } & \makecell{ABSREL \\ (12) } & \makecell{ABS \\ (13) } & \makecell{REL \\ (14) } & \makecell{ABSREL \\ (15) }\\
\midrule
\endfirsthead
\caption[]{\label{tab:agemodelsrelsexessitemsmaths}Effets de l'âge absolu et de l'âge relatif hétérogènes selon le sexe (CM2), sous-items de mathématiques (suite)}\\
\toprule
\multicolumn{1}{c}{} & \multicolumn{15}{c}{Variable dépendante : Note en } \\
\cmidrule(l{3pt}r{3pt}){2-16}
\multicolumn{1}{c}{} & \multicolumn{3}{c}{Calcul} & \multicolumn{3}{c}{Géométrie} & \multicolumn{3}{c}{\makecell{Grandeurs et \\ mesures}} & \multicolumn{3}{c}{Nombre} & \multicolumn{3}{c}{\makecell{Organisation et \\ gestion de données}} \\
\cmidrule(l{3pt}r{3pt}){2-4} \cmidrule(l{3pt}r{3pt}){5-7} \cmidrule(l{3pt}r{3pt}){8-10} \cmidrule(l{3pt}r{3pt}){11-13} \cmidrule(l{3pt}r{3pt}){14-16}
 & \makecell{ABS \\ (1) } & \makecell{REL \\ (2) } & \makecell{ABSREL \\ (3) } & \makecell{ABS \\ (4) } & \makecell{REL \\ (5) } & \makecell{ABSREL \\ (6) } & \makecell{ABS \\ (7) } & \makecell{REL \\ (8) } & \makecell{ABSREL \\ (9) } & \makecell{ABS \\ (10) } & \makecell{REL \\ (11) } & \makecell{ABSREL \\ (12) } & \makecell{ABS \\ (13) } & \makecell{REL \\ (14) } & \makecell{ABSREL \\ (15) }\\
\midrule
\endhead

\endfoot
\bottomrule
\insertTableNotes
\endlastfoot
Âge aux examens & 0.245$^{***}$ & 0.607$^{***}$ & 0.601$^{***}$ & 0.333$^{***}$ & 0.508$^{***}$ & 0.563$^{***}$ & 0.34$^{***}$ & 0.608$^{***}$ & 0.634$^{***}$ & 0.237$^{***}$ & 0.368$^{***}$ & 0.388$^{***}$ & 0.258$^{***}$ & 0.501$^{***}$ & 0.519$^{***}$\\
 & (0.049) & (0.108) & (0.116) & (0.053) & (0.111) & (0.12) & (0.05) & (0.108) & (0.117) & (0.049) & (0.108) & (0.117) & (0.05) & (0.107) & (0.115)\\
Âge aux examens $\times$ Sexe - Garçon & 0.022 & 0.288$^{*}$ & 0.313$^{**}$ & $-$0.004 & 0.178 & 0.19 & 0.001 & 0.236 & 0.253 & 0.016 & 0.161 & 0.179 & 0.061 & 0.292$^{**}$ & 0.296$^{*}$\\
 & (0.073) & (0.151) & (0.158) & (0.078) & (0.152) & (0.16) & (0.073) & (0.15) & (0.158) & (0.076) & (0.149) & (0.156) & (0.073) & (0.149) & (0.156)\\
Âge relatif (classe) & - & $-$0.339$^{***}$ & $-$0.329$^{***}$ & - & $-$0.236$^{**}$ & $-$0.23$^{**}$ & - & $-$0.301$^{***}$ & $-$0.294$^{***}$ & - & $-$0.163 & $-$0.154 & - & $-$0.243$^{**}$ & $-$0.242$^{**}$\\
 & - & (0.106) & (0.108) & - & (0.109) & (0.112) & - & (0.106) & (0.109) & - & (0.106) & (0.109) & - & (0.105) & (0.107)\\
Âge relatif (classe) $\times$ Sexe - Garçon & - & $-$0.305$^{**}$ & $-$0.325$^{**}$ & - & $-$0.161 & $-$0.17 & - & $-$0.224 & $-$0.236 & - & $-$0.115 & $-$0.129 & - & $-$0.255$^{*}$ & $-$0.257$^{*}$\\
 & - & (0.148) & (0.155) & - & (0.15) & (0.156) & - & (0.148) & (0.154) & - & (0.146) & (0.153) & - & (0.146) & (0.152)\\
Âge aux examens $\times$ Cohorte - 2010 & 0.136$^{**}$ & - & 0.001 & $-$0.012 & - & $-$0.046 & 0.09 & - & $-$0.037 & 0.08 & - & 0.006 & 0.081 & - & $-$0.02\\
 & (0.068) & - & (0.056) & (0.07) & - & (0.056) & (0.068) & - & (0.056) & (0.068) & - & (0.055) & (0.067) & - & (0.055)\\
Âge aux examens $\times$ Cohorte - 2011 & 0.019 & - & 0 & $-$0.082 & - & $-$0.072 & $-$0.044 & - & $-$0.019 & $-$0.09 & - & $-$0.071 & $-$0.019 & - & $-$0.024\\
 & (0.064) & - & (0.055) & (0.068) & - & (0.055) & (0.065) & - & (0.055) & (0.065) & - & (0.054) & (0.066) & - & (0.055)\\
Âge aux examens $\times$ Cohorte - 2012 & 0.044 & - & $-$0.02 & $-$0.109 & - & $-$0.121$^{**}$ & $-$0.05 & - & $-$0.075 & $-$0.025 & - & $-$0.046 & 0.066 & - & $-$0.028\\
 & (0.067) & - & (0.056) & (0.072) & - & (0.056) & (0.068) & - & (0.056) & (0.07) & - & (0.055) & (0.068) & - & (0.056)\\
Âge aux examens $\times$ Cohorte - 2010 $\times$ Sexe - Garçon & $-$0.15 & - & $-$0.001 & $-$0.009 & - & $-$0.001 & $-$0.091 & - & $-$0.001 & $-$0.117 & - & $-$0.001 & $-$0.117 & - & 0.001\\
 & (0.098) & - & (0.003) & (0.1) & - & (0.003) & (0.1) & - & (0.003) & (0.1) & - & (0.003) & (0.098) & - & (0.003)\\
Âge aux examens $\times$ Cohorte - 2011 $\times$ Sexe - Garçon & $-$0.13 & - & $-$0.001 & $-$0.046 & - & 0 & $-$0.058 & - & $-$0.003 & 0.001 & - & 0 & $-$0.103 & - & $-$0.002\\
 & (0.095) & - & (0.003) & (0.1) & - & (0.003) & (0.094) & - & (0.003) & (0.097) & - & (0.003) & (0.097) & - & (0.003)\\
Âge aux examens $\times$ Cohorte - 2012 $\times$ Sexe - Garçon & $-$0.077 & - & $-$0.013$^{***}$ & 0.063 & - & $-$0.006 & $-$0.039 & - & $-$0.011$^{**}$ & 0.055 & - & $-$0.011$^{**}$ & $-$0.111 & - & $-$0.009$^{*}$\\
 & (0.098) & - & (0.005) & (0.1) & - & (0.005) & (0.097) & - & (0.005) & (0.103) & - & (0.005) & (0.098) & - & (0.005)\\
 &  &  &  &  &  &  &  &  &  &  &  &  &  &  & \\
Contrôles & Oui & Oui & Oui & Oui & Oui & Oui & Oui & Oui & Oui & Oui & Oui & Oui & Oui & Oui & Oui\\
Observations & 54341 & 54319 & 54319 & 54341 & 54319 & 54319 & 54341 & 54319 & 54319 & 54341 & 54319 & 54319 & 54341 & 54319 & 54319\\
R$^2$ ajusté & 0.133 & - & - & 0.073 & - & - & 0.115 & - & - & 0.107 & - & - & 0.123 & - & -\\*
\end{longtable}
\end{ThreePartTable}
\endgroup{}

\elandscape

\blandscape

\hypertarget{agemodelssupppcsssitems}{%
\subsection{Effets hétérogènes selon la catégorie sociale}\label{agemodelssupppcsssitems}}

\begingroup\fontsize{4}{6}\selectfont

\begin{ThreePartTable}
\begin{TableNotes}
\item \textit{Sources :} Fichiers CM2 (2009 à 2012), calculs de l'auteur.
\item \textit{Notes :} Une colonne correspond à une régression. La note est normalisée sur l'année scolaire. Écart-types entre parenthèses. Les modèles labellisés ABS sont estimés par fonction de contrôle. Les autres modèles sont estimés par variables instrumentales. Les écart-types des estimations par fonction de contrôle sont calculés par wild bootstrap avec 1001 réplications. Les contrôles utilisés sont le sexe, la CSP et l'année scolaire.
\item ABS, REL et ABSREL sont décrits dans le corps du texte.
\item CSP : Catégorie Socio-Professionnelle.
\item Significativité : 10\% * 5\% ** 1\% ***.
\end{TableNotes}
\begin{longtable}[t]{llllllllllllllll}
\caption{\label{tab:agemodelsrelpcsssitemsfrench}Effets de l'âge absolu et de l'âge relatif hétérogènes selon la catégorie sociale (CM2), sous-items de français}\\
\toprule
\multicolumn{1}{c}{} & \multicolumn{15}{c}{Variable dépendante : Note en } \\
\cmidrule(l{3pt}r{3pt}){2-16}
\multicolumn{1}{c}{} & \multicolumn{3}{c}{Écriture} & \multicolumn{3}{c}{Grammaire} & \multicolumn{3}{c}{Lecture} & \multicolumn{3}{c}{Orthographe} & \multicolumn{3}{c}{Vocabulaire} \\
\cmidrule(l{3pt}r{3pt}){2-4} \cmidrule(l{3pt}r{3pt}){5-7} \cmidrule(l{3pt}r{3pt}){8-10} \cmidrule(l{3pt}r{3pt}){11-13} \cmidrule(l{3pt}r{3pt}){14-16}
 & \makecell{ABS \\ (1) } & \makecell{REL \\ (2) } & \makecell{ABSREL \\ (3) } & \makecell{ABS \\ (4) } & \makecell{REL \\ (5) } & \makecell{ABSREL \\ (6) } & \makecell{ABS \\ (7) } & \makecell{REL \\ (8) } & \makecell{ABSREL \\ (9) } & \makecell{ABS \\ (10) } & \makecell{REL \\ (11) } & \makecell{ABSREL \\ (12) } & \makecell{ABS \\ (13) } & \makecell{REL \\ (14) } & \makecell{ABSREL \\ (15) }\\
\midrule
\endfirsthead
\caption[]{\label{tab:agemodelsrelpcsssitemsfrench}Effets de l'âge absolu et de l'âge relatif hétérogènes selon la catégorie sociale (CM2), sous-items de français (suite)}\\
\toprule
\multicolumn{1}{c}{} & \multicolumn{15}{c}{Variable dépendante : Note en } \\
\cmidrule(l{3pt}r{3pt}){2-16}
\multicolumn{1}{c}{} & \multicolumn{3}{c}{Écriture} & \multicolumn{3}{c}{Grammaire} & \multicolumn{3}{c}{Lecture} & \multicolumn{3}{c}{Orthographe} & \multicolumn{3}{c}{Vocabulaire} \\
\cmidrule(l{3pt}r{3pt}){2-4} \cmidrule(l{3pt}r{3pt}){5-7} \cmidrule(l{3pt}r{3pt}){8-10} \cmidrule(l{3pt}r{3pt}){11-13} \cmidrule(l{3pt}r{3pt}){14-16}
 & \makecell{ABS \\ (1) } & \makecell{REL \\ (2) } & \makecell{ABSREL \\ (3) } & \makecell{ABS \\ (4) } & \makecell{REL \\ (5) } & \makecell{ABSREL \\ (6) } & \makecell{ABS \\ (7) } & \makecell{REL \\ (8) } & \makecell{ABSREL \\ (9) } & \makecell{ABS \\ (10) } & \makecell{REL \\ (11) } & \makecell{ABSREL \\ (12) } & \makecell{ABS \\ (13) } & \makecell{REL \\ (14) } & \makecell{ABSREL \\ (15) }\\
\midrule
\endhead

\endfoot
\bottomrule
\insertTableNotes
\endlastfoot
Âge aux examens & 0.22$^{***}$ & 0.2 & 0.192 & 0.261$^{***}$ & 0.346$^{***}$ & 0.309$^{**}$ & 0.282$^{***}$ & 0.387$^{***}$ & 0.394$^{***}$ & 0.283$^{***}$ & 0.365$^{***}$ & 0.353$^{***}$ & 0.306$^{***}$ & 0.489$^{***}$ & 0.507$^{***}$\\
 & (0.048) & (0.124) & (0.131) & (0.047) & (0.123) & (0.131) & (0.049) & (0.126) & (0.134) & (0.048) & (0.123) & (0.131) & (0.048) & (0.125) & (0.133)\\
Âge aux examens $\times$ CSP - Moyenne & 0.005 & $-$0.069 & $-$0.058 & 0.021 & $-$0.114 & $-$0.095 & 0.204$^{**}$ & $-$0.064 & $-$0.057 & $-$0.099 & $-$0.242 & $-$0.216 & $-$0.012 & $-$0.226 & $-$0.229\\
 & (0.085) & (0.209) & (0.219) & (0.09) & (0.215) & (0.225) & (0.088) & (0.216) & (0.226) & (0.088) & (0.213) & (0.223) & (0.087) & (0.213) & (0.223)\\
Âge aux examens $\times$ CSP - Favorisée & 0.017 & $-$0.24 & $-$0.262 & $-$0.086 & $-$0.306 & $-$0.282 & $-$0.201 & $-$0.429 & $-$0.446 & $-$0.128 & $-$0.505 & $-$0.508 & $-$0.178 & $-$0.219 & $-$0.214\\
 & (0.143) & (0.329) & (0.353) & (0.151) & (0.356) & (0.385) & (0.141) & (0.354) & (0.381) & (0.146) & (0.352) & (0.379) & (0.139) & (0.341) & (0.367)\\
Âge aux examens $\times$ CSP - Très favorisée & 0.261$^{**}$ & 0.315 & 0.336 & 0.1 & $-$0.134 & $-$0.098 & 0.207$^{**}$ & 0.064 & 0.075 & 0.025 & $-$0.272 & $-$0.239 & 0.209$^{**}$ & $-$0.027 & $-$0.031\\
 & (0.114) & (0.261) & (0.287) & (0.113) & (0.284) & (0.313) & (0.104) & (0.262) & (0.288) & (0.106) & (0.28) & (0.308) & (0.1) & (0.257) & (0.282)\\
Âge aux examens $\times$ CSP - Autre & 0.294 & $-$0.044 & $-$0.077 & 0.521 & 0.051 & 0.023 & 0.332 & 0.088 & 0.108 & 0.234 & $-$0.141 & $-$0.206 & 0.055 & 0.396 & 0.422\\
 & (0.315) & (0.978) & (1.148) & (0.357) & (0.975) & (1.147) & (0.339) & (0.91) & (1.066) & (0.377) & (0.908) & (1.07) & (0.336) & (0.976) & (1.149)\\
Âge aux examens $\times$ CSP - Manquante & 0.185$^{*}$ & 0.302$^{*}$ & 0.352 & 0.095 & 0.483$^{***}$ & 0.271 & 0.028 & 0.361$^{*}$ & 0.346 & 0.046 & 0.228 & 0.07 & $-$0.03 & 0.101 & 0.373\\
 & (0.1) & (0.18) & (0.288) & (0.101) & (0.182) & (0.283) & (0.105) & (0.186) & (0.295) & (0.094) & (0.18) & (0.277) & (0.104) & (0.187) & (0.298)\\
Âge relatif (classe) & - & $-$0.013 & $-$0.008 & - & $-$0.129 & $-$0.109 & - & $-$0.178 & $-$0.165 & - & $-$0.165 & $-$0.147 & - & $-$0.256$^{**}$ & $-$0.248$^{**}$\\
 & - & (0.122) & (0.125) & - & (0.121) & (0.124) & - & (0.123) & (0.127) & - & (0.121) & (0.124) & - & (0.123) & (0.126)\\
Âge relatif (classe) $\times$ CSP - Moyenne & - & 0.121 & 0.106 & - & 0.189 & 0.165 & - & 0.232 & 0.221 & - & 0.282 & 0.25 & - & 0.272 & 0.268\\
 & - & (0.204) & (0.211) & - & (0.209) & (0.217) & - & (0.211) & (0.219) & - & (0.208) & (0.215) & - & (0.208) & (0.216)\\
Âge relatif (classe) $\times$ CSP - Favorisée & - & 0.251 & 0.261 & - & 0.219 & 0.188 & - & 0.344 & 0.347 & - & 0.482 & 0.47 & - & 0.131 & 0.117\\
 & - & (0.335) & (0.351) & - & (0.361) & (0.381) & - & (0.357) & (0.375) & - & (0.357) & (0.376) & - & (0.345) & (0.364)\\
Âge relatif (classe) $\times$ CSP - Très favorisée & - & $-$0.136 & $-$0.156 & - & 0.232 & 0.195 & - & 0.03 & 0.018 & - & 0.4 & 0.363 & - & 0.105 & 0.104\\
 & - & (0.258) & (0.281) & - & (0.281) & (0.306) & - & (0.259) & (0.282) & - & (0.276) & (0.301) & - & (0.252) & (0.274)\\
Âge relatif (classe) $\times$ CSP - Autre & - & 0.211 & 0.254 & - & 0.239 & 0.276 & - & 0.144 & 0.14 & - & 0.185 & 0.256 & - & $-$0.221 & $-$0.236\\
 & - & (0.943) & (1.092) & - & (0.954) & (1.106) & - & (0.891) & (1.027) & - & (0.875) & (1.015) & - & (0.941) & (1.095)\\
Âge relatif (classe) $\times$ CSP - Manquante & - & $-$0.169 & $-$0.174 & - & $-$0.417$^{**}$ & $-$0.16 & - & $-$0.25 & $-$0.209 & - & $-$0.159 & 0.001 & - & $-$0.064 & $-$0.281\\
 & - & (0.176) & (0.276) & - & (0.178) & (0.271) & - & (0.182) & (0.282) & - & (0.176) & (0.265) & - & (0.182) & (0.285)\\
Âge aux examens $\times$ Cohorte - 2011 & $-$0.055 & - & $-$0.014 & $-$0.035 & - & 0.029 & $-$0.101 & - & $-$0.033 & $-$0.114$^{*}$ & - & $-$0.008 & $-$0.152$^{**}$ & - & $-$0.06\\
 & (0.067) & - & (0.05) & (0.065) & - & (0.05) & (0.065) & - & (0.051) & (0.063) & - & (0.05) & (0.066) & - & (0.051)\\
Âge aux examens $\times$ Cohorte - 2012 & $-$0.078 & - & 0.022 & $-$0.1 & - & 0.021 & $-$0.134$^{**}$ & - & $-$0.026 & $-$0.151$^{**}$ & - & $-$0.011 & $-$0.094 & - & $-$0.017\\
 & (0.067) & - & (0.051) & (0.065) & - & (0.051) & (0.066) & - & (0.051) & (0.066) & - & (0.051) & (0.066) & - & (0.051)\\
Âge aux examens $\times$ Cohorte - 2011 $\times$ CSP - Moyenne & 0.072 & - & 0.006$^{*}$ & 0.041 & - & 0.008$^{**}$ & $-$0.052 & - & 0.004 & 0.246$^{**}$ & - & 0.011$^{***}$ & 0.145 & - & 0.005$^{*}$\\
 & (0.115) & - & (0.003) & (0.119) & - & (0.003) & (0.119) & - & (0.003) & (0.121) & - & (0.003) & (0.113) & - & (0.003)\\
Âge aux examens $\times$ Cohorte - 2012 $\times$ CSP - Moyenne & 0.204$^{*}$ & - & 0.009 & 0.213$^{*}$ & - & 0.006 & 0.072 & - & 0.005 & 0.283$^{**}$ & - & 0.012$^{*}$ & 0.161 & - & 0.014$^{*}$\\
 & (0.121) & - & (0.007) & (0.122) & - & (0.007) & (0.122) & - & (0.007) & (0.121) & - & (0.007) & (0.119) & - & (0.007)\\
Âge aux examens $\times$ Cohorte - 2011 $\times$ CSP - Favorisée & 0.066 & - & 0.003 & 0.154 & - & 0.009$^{*}$ & 0.31 & - & 0.006 & 0.33$^{*}$ & - & 0.01$^{*}$ & 0.304$^{*}$ & - & 0.008\\
 & (0.188) & - & (0.005) & (0.197) & - & (0.005) & (0.192) & - & (0.005) & (0.189) & - & (0.005) & (0.178) & - & (0.005)\\
Âge aux examens $\times$ Cohorte - 2012 $\times$ CSP - Favorisée & 0.177 & - & 0.016 & 0.071 & - & 0.005 & 0.247 & - & 0.019 & 0.209 & - & 0.02$^{*}$ & 0.244 & - & 0.014\\
 & (0.189) & - & (0.011) & (0.199) & - & (0.012) & (0.184) & - & (0.012) & (0.196) & - & (0.012) & (0.183) & - & (0.012)\\
Âge aux examens $\times$ Cohorte - 2011 $\times$ CSP - Très favorisée & $-$0.141 & - & 0 & $-$0.069 & - & 0.013$^{***}$ & $-$0.215 & - & 0.005 & 0.101 & - & 0.017$^{***}$ & $-$0.177 & - & 0.008$^{**}$\\
 & (0.154) & - & (0.004) & (0.154) & - & (0.004) & (0.142) & - & (0.004) & (0.146) & - & (0.004) & (0.14) & - & (0.004)\\
Âge aux examens $\times$ Cohorte - 2012 $\times$ CSP - Très favorisée & 0.008 & - & $-$0.006 & 0.162 & - & 0.005 & 0.007 & - & 0.002 & 0.313$^{**}$ & - & 0.012 & $-$0.025 & - & 0.014\\
 & (0.15) & - & (0.009) & (0.154) & - & (0.01) & (0.145) & - & (0.009) & (0.147) & - & (0.01) & (0.143) & - & (0.009)\\
Âge aux examens $\times$ Cohorte - 2011 $\times$ CSP - Autre & 0.414 & - & $-$0.01 & 0.113 & - & $-$0.011 & 0.495 & - & $-$0.018 & $-$0.456 & - & $-$0.013 & 0.82 & - & $-$0.014\\
 & (0.601) & - & (0.015) & (0.612) & - & (0.016) & (0.615) & - & (0.016) & (0.602) & - & (0.015) & (0.546) & - & (0.016)\\
Âge aux examens $\times$ Cohorte - 2012 $\times$ CSP - Autre & 0.081 & - & $-$0.002 & 0.226 & - & $-$0.006 & 0.153 & - & $-$0.02 & 0.154 & - & 0.002 & 0.41 & - & $-$0.017\\
 & (0.635) & - & (0.036) & (0.626) & - & (0.037) & (0.62) & - & (0.034) & (0.692) & - & (0.033) & (0.583) & - & (0.036)\\
Âge aux examens $\times$ Cohorte - 2011 $\times$ CSP - Manquante & $-$0.179 & - & 0.005 & $-$0.061 & - & 0.01$^{**}$ & 0.07 & - & 0.015$^{***}$ & $-$0.012 & - & 0.015$^{***}$ & 0.104 & - & 0.015$^{***}$\\
 & (0.14) & - & (0.004) & (0.137) & - & (0.004) & (0.14) & - & (0.004) & (0.129) & - & (0.004) & (0.144) & - & (0.004)\\
Âge aux examens $\times$ Cohorte - 2012 $\times$ CSP - Manquante & 0.055 & - & $-$0.008 & $-$0.029 & - & $-$0.009 & 0.093 & - & $-$0.003 & $-$0.008 & - & 0.004 & 0.128 & - & 0.001\\
 & (0.145) & - & (0.009) & (0.144) & - & (0.009) & (0.151) & - & (0.009) & (0.141) & - & (0.008) & (0.147) & - & (0.009)\\
 &  &  &  &  &  &  &  &  &  &  &  &  &  &  & \\
Contrôles & Oui & Oui & Oui & Oui & Oui & Oui & Oui & Oui & Oui & Oui & Oui & Oui & Oui & Oui & Oui\\
Observations & 42030 & 54319 & 42015 & 42030 & 54319 & 42015 & 42030 & 54319 & 42015 & 42030 & 54319 & 42015 & 42030 & 54319 & 42015\\
R$^2$ ajusté & 0.187 & - & - & 0.185 & - & - & 0.191 & - & - & 0.213 & - & - & 0.192 & - & -\\*
\end{longtable}
\end{ThreePartTable}
\endgroup{}

\begingroup\fontsize{4}{6}\selectfont

\begin{ThreePartTable}
\begin{TableNotes}
\item \textit{Sources :} Fichiers CM2 (2009 à 2012), calculs de l'auteur.
\item \textit{Notes :} Une colonne correspond à une régression. La note est normalisée sur l'année scolaire. Écart-types entre parenthèses. Les modèles labellisés ABS sont estimés par fonction de contrôle. Les autres modèles sont estimés par variables instrumentales. Les écart-types des estimations par fonction de contrôle sont calculés par wild bootstrap avec 1001 réplications. Les contrôles utilisés sont le sexe, la CSP et l'année scolaire.
\item ABS, REL et ABSREL sont décrits dans le corps du texte.
\item CSP : Catégorie Socio-Professionnelle.
\item Significativité : 10\% * 5\% ** 1\% ***.
\end{TableNotes}
\begin{longtable}[t]{llllllllllllllll}
\caption{\label{tab:agemodelsrelpcsssitemsmaths}Effets de l'âge absolu et de l'âge relatif hétérogènes selon la catégorie sociale (CM2), sous-items de mathématiques}\\
\toprule
\multicolumn{1}{c}{} & \multicolumn{15}{c}{Variable dépendante : Note en } \\
\cmidrule(l{3pt}r{3pt}){2-16}
\multicolumn{1}{c}{} & \multicolumn{3}{c}{Calcul} & \multicolumn{3}{c}{Géométrie} & \multicolumn{3}{c}{\makecell{Grandeurs et \\ mesures}} & \multicolumn{3}{c}{Nombre} & \multicolumn{3}{c}{\makecell{Organisation et \\ gestion de données}} \\
\cmidrule(l{3pt}r{3pt}){2-4} \cmidrule(l{3pt}r{3pt}){5-7} \cmidrule(l{3pt}r{3pt}){8-10} \cmidrule(l{3pt}r{3pt}){11-13} \cmidrule(l{3pt}r{3pt}){14-16}
 & \makecell{ABS \\ (1) } & \makecell{REL \\ (2) } & \makecell{ABSREL \\ (3) } & \makecell{ABS \\ (4) } & \makecell{REL \\ (5) } & \makecell{ABSREL \\ (6) } & \makecell{ABS \\ (7) } & \makecell{REL \\ (8) } & \makecell{ABSREL \\ (9) } & \makecell{ABS \\ (10) } & \makecell{REL \\ (11) } & \makecell{ABSREL \\ (12) } & \makecell{ABS \\ (13) } & \makecell{REL \\ (14) } & \makecell{ABSREL \\ (15) }\\
\midrule
\endfirsthead
\caption[]{\label{tab:agemodelsrelpcsssitemsmaths}Effets de l'âge absolu et de l'âge relatif hétérogènes selon la catégorie sociale (CM2), sous-items de mathématiques (suite)}\\
\toprule
\multicolumn{1}{c}{} & \multicolumn{15}{c}{Variable dépendante : Note en } \\
\cmidrule(l{3pt}r{3pt}){2-16}
\multicolumn{1}{c}{} & \multicolumn{3}{c}{Calcul} & \multicolumn{3}{c}{Géométrie} & \multicolumn{3}{c}{\makecell{Grandeurs et \\ mesures}} & \multicolumn{3}{c}{Nombre} & \multicolumn{3}{c}{\makecell{Organisation et \\ gestion de données}} \\
\cmidrule(l{3pt}r{3pt}){2-4} \cmidrule(l{3pt}r{3pt}){5-7} \cmidrule(l{3pt}r{3pt}){8-10} \cmidrule(l{3pt}r{3pt}){11-13} \cmidrule(l{3pt}r{3pt}){14-16}
 & \makecell{ABS \\ (1) } & \makecell{REL \\ (2) } & \makecell{ABSREL \\ (3) } & \makecell{ABS \\ (4) } & \makecell{REL \\ (5) } & \makecell{ABSREL \\ (6) } & \makecell{ABS \\ (7) } & \makecell{REL \\ (8) } & \makecell{ABSREL \\ (9) } & \makecell{ABS \\ (10) } & \makecell{REL \\ (11) } & \makecell{ABSREL \\ (12) } & \makecell{ABS \\ (13) } & \makecell{REL \\ (14) } & \makecell{ABSREL \\ (15) }\\
\midrule
\endhead

\endfoot
\bottomrule
\insertTableNotes
\endlastfoot
Âge aux examens & 0.286$^{***}$ & 0.771$^{***}$ & 0.758$^{***}$ & 0.326$^{***}$ & 0.634$^{***}$ & 0.65$^{***}$ & 0.288$^{***}$ & 0.665$^{***}$ & 0.651$^{***}$ & 0.241$^{***}$ & 0.442$^{***}$ & 0.479$^{***}$ & 0.237$^{***}$ & 0.502$^{***}$ & 0.476$^{***}$\\
 & (0.052) & (0.127) & (0.134) & (0.05) & (0.13) & (0.138) & (0.048) & (0.122) & (0.13) & (0.05) & (0.125) & (0.132) & (0.047) & (0.12) & (0.128)\\
Âge aux examens $\times$ CSP - Moyenne & 0.062 & $-$0.149 & $-$0.121 & 0.005 & $-$0.147 & $-$0.133 & 0.122 & $-$0.002 & 0.037 & 0.092 & $-$0.35 & $-$0.365 & 0.155$^{*}$ & 0.318 & 0.364\\
 & (0.092) & (0.219) & (0.229) & (0.094) & (0.221) & (0.231) & (0.095) & (0.223) & (0.233) & (0.094) & (0.219) & (0.229) & (0.093) & (0.222) & (0.233)\\
Âge aux examens $\times$ CSP - Favorisée & $-$0.026 & $-$0.282 & $-$0.241 & $-$0.024 & $-$0.796$^{**}$ & $-$0.788$^{**}$ & $-$0.088 & $-$0.419 & $-$0.381 & 0.105 & $-$0.488 & $-$0.511 & $-$0.017 & $-$0.138 & $-$0.07\\
 & (0.16) & (0.352) & (0.379) & (0.157) & (0.364) & (0.391) & (0.165) & (0.363) & (0.39) & (0.159) & (0.363) & (0.392) & (0.168) & (0.359) & (0.387)\\
Âge aux examens $\times$ CSP - Très favorisée & 0.117 & $-$0.359 & $-$0.35 & 0.066 & $-$0.025 & 0.034 & 0.337$^{**}$ & 0.095 & 0.169 & 0.201$^{*}$ & $-$0.212 & $-$0.178 & 0.386$^{***}$ & $-$0.015 & 0.063\\
 & (0.113) & (0.279) & (0.306) & (0.12) & (0.284) & (0.313) & (0.135) & (0.314) & (0.347) & (0.117) & (0.289) & (0.319) & (0.133) & (0.314) & (0.346)\\
Âge aux examens $\times$ CSP - Autre & 0.108 & $-$0.967 & $-$1.111 & 0.397 & 0.274 & 0.368 & 0.228 & 0.828 & 0.862 & 0.129 & $-$0.358 & $-$0.454 & 0.345 & $-$1.091 & $-$1.44\\
 & (0.374) & (0.931) & (1.095) & (0.409) & (1.006) & (1.191) & (0.444) & (1.051) & (1.235) & (0.394) & (0.961) & (1.14) & (0.301) & (0.892) & (1.036)\\
Âge aux examens $\times$ CSP - Manquante & $-$0.023 & 0.137 & 0.106 & $-$0.076 & 0.071 & 0.048 & 0.046 & 0.185 & $-$0.046 & 0.107 & 0.268 & 0.241 & 0.027 & 0.295$^{*}$ & $-$0.116\\
 & (0.102) & (0.185) & (0.293) & (0.098) & (0.187) & (0.288) & (0.093) & (0.18) & (0.273) & (0.1) & (0.181) & (0.276) & (0.095) & (0.177) & (0.264)\\
Âge relatif (classe) & - & $-$0.558$^{***}$ & $-$0.54$^{***}$ & - & $-$0.36$^{***}$ & $-$0.342$^{***}$ & - & $-$0.405$^{***}$ & $-$0.383$^{***}$ & - & $-$0.268$^{**}$ & $-$0.263$^{**}$ & - & $-$0.275$^{**}$ & $-$0.246$^{**}$\\
 & - & (0.125) & (0.128) & - & (0.128) & (0.131) & - & (0.12) & (0.123) & - & (0.122) & (0.125) & - & (0.118) & (0.121)\\
Âge relatif (classe) $\times$ CSP - Moyenne & - & 0.243 & 0.212 & - & 0.12 & 0.102 & - & 0.091 & 0.048 & - & 0.395$^{*}$ & 0.406$^{*}$ & - & $-$0.194 & $-$0.239\\
 & - & (0.214) & (0.222) & - & (0.216) & (0.224) & - & (0.218) & (0.225) & - & (0.214) & (0.222) & - & (0.216) & (0.224)\\
Âge relatif (classe) $\times$ CSP - Favorisée & - & 0.241 & 0.19 & - & 0.659$^{*}$ & 0.643$^{*}$ & - & 0.393 & 0.341 & - & 0.503 & 0.524 & - & 0.079 & 0.006\\
 & - & (0.357) & (0.377) & - & (0.367) & (0.386) & - & (0.367) & (0.386) & - & (0.368) & (0.388) & - & (0.361) & (0.381)\\
Âge relatif (classe) $\times$ CSP - Très favorisée & - & 0.434 & 0.42 & - & 0.056 & $-$0.002 & - & 0.092 & 0.019 & - & 0.396 & 0.365 & - & 0.189 & 0.114\\
 & - & (0.277) & (0.301) & - & (0.282) & (0.306) & - & (0.311) & (0.339) & - & (0.286) & (0.312) & - & (0.31) & (0.338)\\
Âge relatif (classe) $\times$ CSP - Autre & - & 1.004 & 1.142 & - & $-$0.388 & $-$0.504 & - & $-$0.875 & $-$0.897 & - & 0.36 & 0.45 & - & 0.95 & 1.286\\
 & - & (0.922) & (1.069) & - & (0.971) & (1.138) & - & (1.021) & (1.186) & - & (0.933) & (1.094) & - & (0.866) & (0.988)\\
Âge relatif (classe) $\times$ CSP - Manquante & - & $-$0.056 & 0.027 & - & $-$0.02 & $-$0.037 & - & $-$0.108 & 0.104 & - & $-$0.174 & $-$0.123 & - & $-$0.235 & 0.16\\
 & - & (0.181) & (0.281) & - & (0.183) & (0.276) & - & (0.176) & (0.261) & - & (0.177) & (0.264) & - & (0.174) & (0.252)\\
Âge aux examens $\times$ Cohorte - 2011 & $-$0.105 & - & 0.003 & $-$0.078 & - & $-$0.024 & $-$0.011 & - & 0.017 & $-$0.097 & - & $-$0.071 & $-$0.02 & - & $-$0.004\\
 & (0.069) & - & (0.051) & (0.069) & - & (0.051) & (0.066) & - & (0.051) & (0.068) & - & (0.051) & (0.064) & - & (0.051)\\
Âge aux examens $\times$ Cohorte - 2012 & $-$0.134$^{**}$ & - & $-$0.021 & $-$0.086 & - & $-$0.075 & $-$0.077 & - & $-$0.041 & $-$0.113 & - & $-$0.052 & $-$0.038 & - & $-$0.008\\
 & (0.068) & - & (0.052) & (0.069) & - & (0.052) & (0.069) & - & (0.052) & (0.07) & - & (0.052) & (0.066) & - & (0.052)\\
Âge aux examens $\times$ Cohorte - 2011 $\times$ CSP - Moyenne & 0.07 & - & 0.009$^{***}$ & 0.005 & - & 0.005 & 0.015 & - & 0.012$^{***}$ & $-$0.079 & - & $-$0.002 & $-$0.023 & - & 0.009$^{**}$\\
 & (0.122) & - & (0.003) & (0.123) & - & (0.003) & (0.129) & - & (0.003) & (0.127) & - & (0.003) & (0.127) & - & (0.004)\\
Âge aux examens $\times$ Cohorte - 2012 $\times$ CSP - Moyenne & 0.138 & - & 0.004 & $-$0.052 & - & 0.004 & 0.014 & - & 0.006 & 0.006 & - & 0.005 & 0.08 & - & $-$0.009\\
 & (0.122) & - & (0.007) & (0.132) & - & (0.007) & (0.129) & - & (0.008) & (0.132) & - & (0.007) & (0.128) & - & (0.007)\\
Âge aux examens $\times$ Cohorte - 2011 $\times$ CSP - Favorisée & 0.162 & - & 0.015$^{***}$ & 0.091 & - & 0.008 & 0.276 & - & 0.017$^{***}$ & 0.014 & - & $-$0.002 & 0.188 & - & 0.016$^{***}$\\
 & (0.202) & - & (0.005) & (0.206) & - & (0.006) & (0.214) & - & (0.006) & (0.213) & - & (0.006) & (0.22) & - & (0.006)\\
Âge aux examens $\times$ Cohorte - 2012 $\times$ CSP - Favorisée & 0.119 & - & 0.013 & $-$0.213 & - & 0.012 & 0.11 & - & 0.017 & $-$0.014 & - & 0.005 & $-$0.01 & - & 0.001\\
 & (0.211) & - & (0.012) & (0.207) & - & (0.013) & (0.21) & - & (0.013) & (0.209) & - & (0.013) & (0.212) & - & (0.013)\\
Âge aux examens $\times$ Cohorte - 2011 $\times$ CSP - Très favorisée & $-$0.05 & - & 0.014$^{***}$ & $-$0.114 & - & 0.012$^{***}$ & $-$0.186 & - & 0.019$^{***}$ & $-$0.041 & - & 0.004 & $-$0.358$^{*}$ & - & 0.011$^{**}$\\
 & (0.154) & - & (0.004) & (0.163) & - & (0.004) & (0.173) & - & (0.005) & (0.163) & - & (0.005) & (0.183) & - & (0.005)\\
Âge aux examens $\times$ Cohorte - 2012 $\times$ CSP - Très favorisée & 0.122 & - & 0.015 & 0.067 & - & $-$0.006 & $-$0.136 & - & $-$0.001 & 0.055 & - & $-$0.006 & $-$0.115 & - & $-$0.014\\
 & (0.161) & - & (0.01) & (0.167) & - & (0.01) & (0.179) & - & (0.011) & (0.166) & - & (0.01) & (0.174) & - & (0.011)\\
Âge aux examens $\times$ Cohorte - 2011 $\times$ CSP - Autre & 0.635 & - & $-$0.009 & 0.026 & - & 0.023 & $-$0.383 & - & $-$0.015 & 0.417 & - & $-$0.011 & $-$0.482 & - & $-$0.021\\
 & (0.597) & - & (0.016) & (0.604) & - & (0.016) & (0.666) & - & (0.015) & (0.595) & - & (0.015) & (0.511) & - & (0.013)\\
Âge aux examens $\times$ Cohorte - 2012 $\times$ CSP - Autre & $-$0.114 & - & 0.017 & $-$1.447$^{**}$ & - & 0.001 & $-$0.224 & - & $-$0.02 & 0.028 & - & 0.003 & $-$0.547 & - & 0.037\\
 & (0.601) & - & (0.034) & (0.63) & - & (0.038) & (0.653) & - & (0.04) & (0.625) & - & (0.035) & (0.552) & - & (0.033)\\
Âge aux examens $\times$ Cohorte - 2011 $\times$ CSP - Manquante & 0.204 & - & 0.013$^{***}$ & 0.113 & - & 0.013$^{***}$ & $-$0.13 & - & 0.005 & $-$0.031 & - & 0.007$^{*}$ & $-$0.028 & - & 0.011$^{***}$\\
 & (0.136) & - & (0.004) & (0.134) & - & (0.004) & (0.13) & - & (0.004) & (0.134) & - & (0.004) & (0.13) & - & (0.004)\\
Âge aux examens $\times$ Cohorte - 2012 $\times$ CSP - Manquante & 0.1 & - & 0.002 & 0.04 & - & 0.001 & $-$0.033 & - & $-$0.002 & $-$0.066 & - & $-$0.009 & $-$0.066 & - & $-$0.003\\
 & (0.147) & - & (0.009) & (0.147) & - & (0.009) & (0.145) & - & (0.008) & (0.152) & - & (0.008) & (0.143) & - & (0.008)\\
 &  &  &  &  &  &  &  &  &  &  &  &  &  &  & \\
Contrôles & Oui & Oui & Oui & Oui & Oui & Oui & Oui & Oui & Oui & Oui & Oui & Oui & Oui & Oui & Oui\\
Observations & 42030 & 54319 & 42015 & 42030 & 54319 & 42015 & 42030 & 54319 & 42015 & 42030 & 54319 & 42015 & 42030 & 54319 & 42015\\
R$^2$ ajusté & 0.145 & - & - & 0.08 & - & - & 0.131 & - & - & 0.115 & - & - & 0.138 & - & -\\*
\end{longtable}
\end{ThreePartTable}
\endgroup{}

\elandscape

\setcounter{table}{0}
\setcounter{figure}{0}

\hypertarget{agemodelmtsuppssmoy0}{%
\section{Résultats des modèles d'extension à moyen terme utilisant des variables dépendantes alternatives}\label{agemodelmtsuppssmoy0}}

\hypertarget{agemodelsmtsuppssmoy}{%
\subsection{Effets homogènes par rapport aux observables}\label{agemodelsmtsuppssmoy}}

\begingroup\fontsize{7}{9}\selectfont

\begin{ThreePartTable}
\begin{TableNotes}
\item \textit{Sources :} Fichiers DNB (2014 à 2016), Fichiers constats (2013 à 2016), calculs de l'auteur.
\item \textit{Notes :} Une colonne correspond à une régression. La note est normalisée sur l'année scolaire. Écart-types entre parenthèses. Les modèles labellisés ABS sont estimés par fonction de contrôle. Les autres modèles sont estimés par variables instrumentales. Les écart-types des estimations par fonction de contrôle sont calculés par wild bootstrap avec 1001 réplications. Les contrôles utilisés sont le sexe, la CSP et l'année scolaire.
\item ABS, REL et ABSREL sont décrits dans le corps du texte.
\item Significativité : 10\% * 5\% ** 1\% ***.
\end{TableNotes}
\begin{longtable}[t]{llll}
\caption{\label{tab:agemodelsmtrelssmoy}Effets de l'âge relatif à moyen terme, variables dépendantes alternatives}\\
\toprule
\multicolumn{1}{c}{} & \multicolumn{3}{c}{Variable dépendante :} \\
\cmidrule(l{3pt}r{3pt}){2-4}
  & \makecell{Histoire-et-géographie \\ (1) } & \makecell{Dictée \\ (2) } & \makecell{Rédaction \\ (3) }\\
\midrule
\endfirsthead
\caption[]{\label{tab:agemodelsmtrelssmoy}Effets de l'âge relatif à moyen terme, variables dépendantes alternatives (suite)}\\
\toprule
\multicolumn{1}{c}{} & \multicolumn{3}{c}{Variable dépendante :} \\
\cmidrule(l{3pt}r{3pt}){2-4}
  & \makecell{Histoire-et-géographie \\ (1) } & \makecell{Dictée \\ (2) } & \makecell{Rédaction \\ (3) }\\
\midrule
\endhead

\endfoot
\bottomrule
\insertTableNotes
\endlastfoot
Âge aux examens & 1.231$^{***}$ & 1.122$^{***}$ & 0.908$^{***}$\\
 & (0.134) & (0.129) & (0.13)\\
Âge relatif (classe) & $-$1.026$^{***}$ & $-$0.931$^{***}$ & $-$0.744$^{***}$\\
 & (0.13) & (0.126) & (0.126)\\
 &  &  & \\
Contrôles & Oui & Oui & Oui\\
Observations & 34938 & 34948 & 34912\\*
\end{longtable}
\end{ThreePartTable}
\endgroup{}

\hypertarget{agemodelsmtsuppsexemodssmoy}{%
\subsection{Effets hétérogènes selon le sexe}\label{agemodelsmtsuppsexemodssmoy}}

\begingroup\fontsize{7}{9}\selectfont

\begin{ThreePartTable}
\begin{TableNotes}
\item \textit{Sources :} Fichiers DNB (2014 à 2016), Fichiers constats (2013 à 2016), calculs de l'auteur.
\item \textit{Notes :} Une colonne correspond à une régression. La note est normalisée sur l'année scolaire. Écart-types entre parenthèses. Les modèles labellisés ABS sont estimés par fonction de contrôle. Les autres modèles sont estimés par variables instrumentales. Les écart-types des estimations par fonction de contrôle sont calculés par wild bootstrap avec 1001 réplications. Les contrôles utilisés sont le sexe, la CSP et l'année scolaire.
\item ABS, REL et ABSREL sont décrits dans le corps du texte.
\item Significativité : 10\% * 5\% ** 1\% ***.
\end{TableNotes}
\begin{longtable}[t]{llll}
\caption{\label{tab:agemodelsmtrelsexemodssmoy}Effets de l'âge relatif hétérogènes selon le sexe à moyen terme, variables dépendantes alternatives}\\
\toprule
\multicolumn{1}{c}{} & \multicolumn{3}{c}{Variable dépendante :} \\
\cmidrule(l{3pt}r{3pt}){2-4}
 & \makecell{Histoire-et-géographie \\ (1) } & \makecell{Dictée \\ (2) } & \makecell{Rédaction \\ (3) }\\
\midrule
\endfirsthead
\caption[]{\label{tab:agemodelsmtrelsexemodssmoy}Effets de l'âge relatif hétérogènes selon le sexe à moyen terme, variables dépendantes alternatives (suite)}\\
\toprule
\multicolumn{1}{c}{} & \multicolumn{3}{c}{Variable dépendante :} \\
\cmidrule(l{3pt}r{3pt}){2-4}
 & \makecell{Histoire-et-géographie \\ (1) } & \makecell{Dictée \\ (2) } & \makecell{Rédaction \\ (3) }\\
\midrule
\endhead

\endfoot
\bottomrule
\insertTableNotes
\endlastfoot
Âge aux examens & 1.243$^{***}$ & 1.221$^{***}$ & 1.048$^{***}$\\
 & (0.187) & (0.183) & (0.181)\\
Âge aux examens $\times$ Sexe - Garçon & $-$0.024 & $-$0.198 & $-$0.281\\
 & (0.27) & (0.261) & (0.261)\\
Âge relatif (classe) & $-$1.038$^{***}$ & $-$1.019$^{***}$ & $-$0.871$^{***}$\\
 & (0.181) & (0.177) & (0.175)\\
Âge relatif (classe) $\times$ Sexe - Garçon & 0.024 & 0.176 & 0.255\\
 & (0.263) & (0.254) & (0.254)\\
 &  &  & \\
Contrôles & Oui & Oui & Oui\\
Observations & 34938 & 34948 & 34912\\*
\end{longtable}
\end{ThreePartTable}
\endgroup{}

\newpage

\hypertarget{agemodelsmtsupppcsregmodssmoy}{%
\subsection{Effets hétérogènes selon la catégorie sociale}\label{agemodelsmtsupppcsregmodssmoy}}

\begingroup\fontsize{8}{10}\selectfont

\begin{ThreePartTable}
\begin{TableNotes}
\item \textit{Sources :} Fichiers DNB (2014 à 2016), Fichiers constats (2013 à 2016), calculs de l'auteur.
\item \textit{Notes :} Une colonne correspond à une régression. La note est normalisée sur l'année scolaire. Écart-types entre parenthèses. Les modèles labellisés ABS sont estimés par fonction de contrôle. Les autres modèles sont estimés par variables instrumentales. Les écart-types des estimations par fonction de contrôle sont calculés par wild bootstrap avec 1001 réplications. Les contrôles utilisés sont le sexe, la CSP et l'année scolaire.
\item ABS, REL et ABSREL sont décrits dans le corps du texte. CSP : Catégorie socio-professionnelle.
\item Significativité : 10\% * 5\% ** 1\% ***.
\end{TableNotes}
\begin{longtable}[t]{llll}
\caption{\label{tab:agemodelsmtrelpcsregmodssmoy}Effets de l'âge relatif à moyen terme hétérogènes selon la catégorie sociale, variables dépendantes alternatives}\\
\toprule
\multicolumn{1}{c}{} & \multicolumn{3}{c}{Variable dépendante :} \\
\cmidrule(l{3pt}r{3pt}){2-4}
  & \makecell{Histoire-et-géographie \\ (1) } & \makecell{Dictée \\ (2) } & \makecell{Rédaction \\ (3) }\\
\midrule
\endfirsthead
\caption[]{\label{tab:agemodelsmtrelpcsregmodssmoy}Effets de l'âge relatif à moyen terme hétérogènes selon la catégorie sociale, variables dépendantes alternatives (suite)}\\
\toprule
\multicolumn{1}{c}{} & \multicolumn{3}{c}{Variable dépendante :} \\
\cmidrule(l{3pt}r{3pt}){2-4}
  & \makecell{Histoire-et-géographie \\ (1) } & \makecell{Dictée \\ (2) } & \makecell{Rédaction \\ (3) }\\
\midrule
\endhead

\endfoot
\bottomrule
\insertTableNotes
\endlastfoot
Âge aux examens & 1.385$^{***}$ & 1.386$^{***}$ & 1.054$^{***}$\\
 & (0.197) & (0.196) & (0.194)\\
Âge aux examens $\times$ CSP - Moyenne & $-$0.498 & $-$0.487 & $-$0.532$^{*}$\\
 & (0.325) & (0.313) & (0.311)\\
Âge aux examens $\times$ CSP - Favorisée & $-$0.158 & $-$0.363 & 0.15\\
 & (0.435) & (0.418) & (0.42)\\
Âge aux examens $\times$ CSP - Très favorisée & $-$0.191 & $-$0.818$^{**}$ & $-$0.069\\
 & (0.373) & (0.345) & (0.359)\\
Âge aux examens $\times$ CSP - Autres & 5.578 & 3.438 & $-$4.891\\
 & (8.039) & (6.123) & (4.478)\\
Âge relatif (classe) & $-$1.205$^{***}$ & $-$1.206$^{***}$ & $-$0.896$^{***}$\\
 & (0.19) & (0.19) & (0.187)\\
Âge relatif (classe) $\times$ CSP - Moyenne & 0.569$^{*}$ & 0.542$^{*}$ & 0.56$^{*}$\\
 & (0.316) & (0.304) & (0.303)\\
Âge relatif (classe) $\times$ CSP - Favorisée & 0.032 & 0.241 & $-$0.177\\
 & (0.429) & (0.412) & (0.414)\\
Âge relatif (classe) $\times$ CSP - Très favorisée & 0.334 & 0.899$^{***}$ & 0.072\\
 & (0.36) & (0.331) & (0.348)\\
Âge relatif (classe) $\times$ CSP - Autres & $-$5.432 & $-$3.631 & 4.722\\
 & (7.748) & (5.891) & (4.305)\\
 &  &  & \\
Contrôles & Oui & Oui & Oui\\
Observations & 34938 & 34948 & 34912\\*
\end{longtable}
\end{ThreePartTable}
\endgroup{}

\hypertarget{annexes-au-chapitre-refpe}{%
\chapter*{Annexes au Chapitre \ref{pe}}\label{annexes-au-chapitre-refpe}}
\addcontentsline{toc}{chapter}{Annexes au Chapitre \ref{pe}}

\setcounter{section}{0}

\renewcommand*{\theHchapter}{\thechapter}
\renewcommand*{\thesection}{\Alph{section}}
\renewcommand*{\theHsection}{Appendix.\thechapter\thesection}

\renewcommand*{\thetable}{\Alph{section}.\arabic{table}}
\renewcommand*{\theHtable}{Appendix.\thetable}

\renewcommand*{\thefigure}{\Alph{section}.\arabic{figure}}
\renewcommand*{\theHfigure}{Appendix.\thefigure}

\setcounter{table}{0}
\setcounter{figure}{0}

\hypertarget{pebealwithin}{%
\section{Forme réduite en écart par rapport à la moyenne de la classe de l'équation structurelle}\label{pebealwithin}}

À partir de l'équation \eqref{eq:pebeal} et selon la formule du calcul de la moyenne chez les pairs, nous avons :

\[
\begin{aligned}
y^{dnb}_{ice} &= \beta_{ce} + \beta_1 (\frac{\displaystyle\sum_{i \in ce} y^{dnb}_{ice} - y^{dnb}_{ice}}{t_{ce} - 1}) + x_{ice}' \beta_2 + (\frac{\displaystyle\sum_{i \in ce} x_{ice}' - x_{ice}'}{t_{ce} - 1}) \beta_3 + \psi_{ice} \\
&= \beta_{ce} + \beta_1 (\frac{t_{ce} \bar{y}^{dnb}_{ce} - y^{dnb}_{ice}}{t_{ce} - 1}) + x_{ice}' \beta_2 + (\frac{\bar{x}_{ce}'t_{ce} - x_{ice}'}{t_{ce} - 1}) \beta_3 + \psi_{ice} \\
&= \beta_{ce} + \beta_1 \frac{t_{ce}}{t_{ce} - 1}  \bar{y}^{dnb}_{ce} - \beta_1 \frac{1}{t_{ce} - 1}y^{dnb}_{ice} + x_{ice}' \beta_2 + \bar{x}_{ce}' \frac{t_{ce}}{t_{ce} - 1} \beta_3 - x_{ice}' \frac{1}{t_{ce} - 1} \beta_3 + \psi_{ice} \\
&= \beta_{ce} + \beta_1 \frac{t_{ce}}{t_{ce} - 1} \bar{y}^{dnb}_{ce} - \beta_1 \frac{1}{t_{ce} - 1} y^{dnb}_{ice} + x_{ice}' (\beta_2 - \frac{1}{t_{ce} - 1} \beta_3) + \bar{x}_{ce}' \frac{t_{ce}}{t_{ce} - 1} + \psi_{ice}.
\end{aligned}
\]
En ramenant le terme \(- \beta_1 \frac{1}{t_{ce} - 1} y^{dnb}_{ice}\) à gauche, nous obtenons :

\begin{equation}
\label{eq:pebealwithin1}
y^{dnb}_{ice} (1 + \frac{\beta_1}{t_{ce} - 1}) = \beta_{ce} + \beta_1 \frac{t_{ce}}{t_{ce} - 1}  \bar{y}^{dnb}_{ce} + x_{ice}' \beta_2 + \bar{x}_{ce}' \frac{t_{ce}}{t_{ce} - 1} \beta_3 - x_{ice}' \frac{1}{t_{ce} - 1} \beta_3 + \psi_{ice}.
\end{equation}

Sachant que la moyenne sur la classe des variables ne dépendant que de la classe (\(t_{ce}, \bar{y}^{dnb}_{ce} \text{ et } \bar{x}_{ce}\)) sont égales à elles-mêmes, la moyenne sur la classe de l'équation \eqref{eq:pebealwithin1} s'écrit :

\begin{equation}
\label{eq:pebealwithin2}
\bar{y}^{dnb}_{ce} (1 + \frac{\beta_1}{t_{ce} - 1}) = \beta_{ce} + \beta_1 \frac{t_{ce}}{t_{ce} - 1} \bar{y}^{dnb}_{ce} + \bar{x}_{ce}' (\beta_2 - \frac{\beta_3}{t_{ce} - 1}) + \bar{x}_{ce}' \frac{t_{ce}}{t_{ce} - 1} \beta_3 + \bar{\psi_{ce}}.
\end{equation}

La différence entre l'équation \eqref{eq:pebealwithin1} et \eqref{eq:pebealwithin2} est alors :

\[
\begin{aligned}
(y_{ice}^{dnb} - \bar{y}_{ce}^{dnb})(1 + \frac{\beta_1}{t_{ce} - 1}) &= (x_{ice}' - \bar{x}_{ce}')(\beta_2 - \frac{\beta_3}{t_{ce} - 1}) + (\psi_{ice} - \bar{\psi}_{ce}) \\ 
\Rightarrow (y_{ice}^{dnb} - \bar{y}_{ce}^{dnb}) &= (x_{ice}' - \bar{x}_{ce}') \frac{(\beta_2 - \frac{\beta_3}{t_{ce} - 1})}{(1 + \frac{\beta_1}{t_{ce} - 1})} + \frac{1}{(1 + \frac{\beta_1}{t_{ce} - 1})} (\psi_{ice} - \bar{\psi}_{ce}), 
\end{aligned}
\]

ce qui correspond à l'équation \eqref{eq:pebealwithin} dans le corps du texte.

\setcounter{table}{0}
\setcounter{figure}{0}
\blandscape

\hypertarget{pemodels0compl}{%
\section{Résultats principaux sur les effets de pairs linéaires en moyenne - Version complète}\label{pemodels0compl}}

\begingroup\fontsize{8}{10}\selectfont

\begin{ThreePartTable}
\begin{TableNotes}
\item \textit{Sources :} Fichiers DNB (2014 à 2016), fichiers CM2 (2010 à 2012), fichiers CONSTAT (2013 à 2016), calculs de l'auteur.
\item \textit{Notes :} Une colonne correspond à une régression. Les notes au CM2 et au DNB sont normalisées sur l'année scolaire de CM2 et de DNB, respectivement. Écart-types robustes entre parenthèses. Des effets fixes collèges-année sont utilisés. CSP : catégorie socio-professionnelle. CM2 : Cours Moyen 2\textsuperscript{ème} année. DNB : Diplôme National du Brevet.
\item Significativité : 10\% * 5\% ** 1\% ***.
\end{TableNotes}
\begin{longtable}[t]{llllllllll}
\caption{\label{tab:pemodels0compl}Résultats principaux sur les effets de pairs linéaires en moyenne - version complète}\\
\toprule
\multicolumn{1}{c}{} & \multicolumn{9}{c}{\makecell{Variable dépendante : \\ Note DNB totale (écrits)}} \\
\cmidrule(l{3pt}r{3pt}){2-10}
\multicolumn{1}{c}{} & \multicolumn{2}{c}{\makecell{Régression simple, sans \\ var.endo.}} & \multicolumn{4}{c}{\makecell{Sojourner (2013), sans \\ var.endo.}} & \multicolumn{3}{c}{\makecell{Sojourner (2013), avec \\ var.endo.}} \\
\cmidrule(l{3pt}r{3pt}){2-3} \cmidrule(l{3pt}r{3pt}){4-7} \cmidrule(l{3pt}r{3pt}){8-10}
 & \makecell{\makecell{Sans \\ effets fixes \\ \ } \\ (1) } & \makecell{\makecell{Avec \\ effets fixes \\ \ } \\ (2) } & \makecell{\makecell{Sans \\ effets fixes \\ \ } \\ (3) } & \makecell{\makecell{Avec \\ effets fixes \\ \ } \\ (4) } & \makecell{\makecell{Avec \\ effets fixes, \\ + sec.av.} \\ (5) } & \makecell{\makecell{Avec \\ effets fixes \\ + sec.désav.} \\ (6) } & \makecell{\makecell{Avec \\ effets fixes \\ \ } \\ (7) } & \makecell{\makecell{Avec \\ effets fixes, \\ + sec.av.} \\ (8) } & \makecell{\makecell{Avec \\ effets fixes \\ + sec.désav.} \\ (9) }\\
\midrule
\endfirsthead
\caption[]{\label{tab:pemodels0compl}Résultats principaux sur les effets de pairs linéaires en moyenne - version complète (suite)}\\
\toprule
 & \makecell{\makecell{Sans \\ effets fixes \\ \ } \\ (1) } & \makecell{\makecell{Avec \\ effets fixes \\ \ } \\ (2) } & \makecell{\makecell{Sans \\ effets fixes \\ \ } \\ (3) } & \makecell{\makecell{Avec \\ effets fixes \\ \ } \\ (4) } & \makecell{\makecell{Avec \\ effets fixes, \\ + sec.av.} \\ (5) } & \makecell{\makecell{Avec \\ effets fixes \\ + sec.désav.} \\ (6) } & \makecell{\makecell{Avec \\ effets fixes \\ \ } \\ (7) } & \makecell{\makecell{Avec \\ effets fixes, \\ + sec.av.} \\ (8) } & \makecell{\makecell{Avec \\ effets fixes \\ + sec.désav.} \\ (9) }\\
\midrule
\endhead

\endfoot
\bottomrule
\insertTableNotes
\endlastfoot
\addlinespace[0.3em]
\multicolumn{10}{l}{\textbf{ }}\\
\hspace{1em}Note au CM2 des pairs & 0.069$^{***}$ & 0.216$^{***}$ & - & - & - & - & - & - & -\\
\hspace{1em} & (0.014) & (0.026) & - & - & - & - & - & - & -\\
\hspace{1em}Note au CM2 des pairs $\times \ p$ & - & - & 0.076$^{***}$ & 0.264$^{***}$ & 0.266$^{***}$ & $-$0.095$^{**}$ & 0.195$^{***}$ & 0.198$^{***}$ & $-$0.017\\
\hspace{1em} & - & - & (0.017) & (0.034) & (0.03) & (0.037) & (0.037) & (0.033) & (0.042)\\
\addlinespace[0.3em]
\multicolumn{10}{l}{\textbf{ }}\\
\hspace{1em}Note au CM2 & 0.665$^{***}$ & 0.674$^{***}$ & 0.666$^{***}$ & 0.674$^{***}$ & 0.671$^{***}$ & 0.633$^{***}$ & 0.638$^{***}$ & 0.635$^{***}$ & 0.608$^{***}$\\
\hspace{1em} & (0.005) & (0.009) & (0.005) & (0.009) & (0.009) & (0.008) & (0.009) & (0.009) & (0.008)\\
\hspace{1em}p & - & - & 0.073$^{*}$ & 0.147$^{*}$ & 0.133$^{*}$ & 0.302$^{***}$ & 0.144$^{*}$ & 0.126 & 0.413$^{***}$\\
\hspace{1em} & - & - & (0.037) & (0.082) & (0.08) & (0.104) & (0.079) & (0.077) & (0.102)\\
\hspace{1em}Âge au DNB & - & - & - & - & - & - & $-$0.073$^{***}$ & $-$0.08$^{***}$ & $-$0.054$^{***}$\\
\hspace{1em} & - & - & - & - & - & - & (0.016) & (0.015) & (0.014)\\
\hspace{1em}Position - À l'heure & - & - & - & - & - & - & 0.213$^{***}$ & 0.208$^{***}$ & 0.159$^{***}$\\
\hspace{1em} & - & - & - & - & - & - & (0.022) & (0.022) & (0.02)\\
\hspace{1em}Position - En avance & - & - & - & - & - & - & 0.509$^{***}$ & 0.493$^{***}$ & 0.498$^{***}$\\
\hspace{1em} & - & - & - & - & - & - & (0.04) & (0.04) & (0.039)\\
\hspace{1em}Sexe - Garçon & $-$0.149$^{***}$ & $-$0.14$^{***}$ & $-$0.149$^{***}$ & $-$0.14$^{***}$ & $-$0.139$^{***}$ & $-$0.11$^{***}$ & $-$0.129$^{***}$ & $-$0.128$^{***}$ & $-$0.106$^{***}$\\
\hspace{1em} & (0.009) & (0.01) & (0.009) & (0.01) & (0.01) & (0.01) & (0.01) & (0.01) & (0.01)\\
\hspace{1em}CSP - Moyenne & 0.16$^{***}$ & 0.149$^{***}$ & 0.16$^{***}$ & 0.148$^{***}$ & 0.148$^{***}$ & 0.123$^{***}$ & 0.134$^{***}$ & 0.134$^{***}$ & 0.114$^{***}$\\
\hspace{1em} & (0.011) & (0.012) & (0.011) & (0.012) & (0.011) & (0.011) & (0.011) & (0.011) & (0.011)\\
\hspace{1em}CSP - Favorisée & 0.239$^{***}$ & 0.235$^{***}$ & 0.239$^{***}$ & 0.234$^{***}$ & 0.233$^{***}$ & 0.203$^{***}$ & 0.215$^{***}$ & 0.214$^{***}$ & 0.189$^{***}$\\
\hspace{1em} & (0.017) & (0.018) & (0.017) & (0.018) & (0.017) & (0.017) & (0.018) & (0.017) & (0.017)\\
\hspace{1em}CSP - Très favorisée & 0.428$^{***}$ & 0.409$^{***}$ & 0.429$^{***}$ & 0.408$^{***}$ & 0.405$^{***}$ & 0.396$^{***}$ & 0.378$^{***}$ & 0.374$^{***}$ & 0.368$^{***}$\\
\hspace{1em} & (0.015) & (0.018) & (0.015) & (0.018) & (0.018) & (0.018) & (0.018) & (0.018) & (0.018)\\
\hspace{1em}CSP - Autres & $-$0.135$^{**}$ & $-$0.119$^{**}$ & $-$0.135$^{**}$ & $-$0.119$^{**}$ & $-$0.113$^{*}$ & $-$0.149$^{***}$ & $-$0.072 & $-$0.068 & $-$0.116$^{**}$\\
\hspace{1em} & (0.062) & (0.06) & (0.062) & (0.06) & (0.059) & (0.055) & (0.056) & (0.056) & (0.052)\\
\hspace{1em}Régime scolaire - Interne & $-$0.272$^{**}$ & $-$0.297$^{***}$ & $-$0.271$^{**}$ & $-$0.295$^{***}$ & $-$0.294$^{***}$ & $-$0.261$^{***}$ & $-$0.268$^{***}$ & $-$0.267$^{***}$ & $-$0.235$^{***}$\\
\hspace{1em} & (0.131) & (0.06) & (0.131) & (0.058) & (0.058) & (0.057) & (0.075) & (0.075) & (0.069)\\
\hspace{1em}Régime scolaire - Externe & $-$0.084$^{***}$ & $-$0.088$^{***}$ & $-$0.083$^{***}$ & $-$0.088$^{***}$ & $-$0.087$^{***}$ & $-$0.086$^{***}$ & $-$0.079$^{***}$ & $-$0.078$^{***}$ & $-$0.08$^{***}$\\
\hspace{1em} & (0.01) & (0.011) & (0.01) & (0.011) & (0.011) & (0.012) & (0.011) & (0.011) & (0.012)\\
\hspace{1em}Taille de classe & - & - & - & - & - & - & 0.018$^{***}$ & 0.017$^{***}$ & 0.015$^{***}$\\
\hspace{1em} & - & - & - & - & - & - & (0.004) & (0.004) & (0.005)\\
\addlinespace[0.3em]
\multicolumn{10}{l}{\textbf{Moyennes chez les pairs}}\\
\hspace{1em}Âge au DNB & - & - & - & - & - & - & 0.21$^{**}$ & 0.193$^{**}$ & 0.628$^{***}$\\
\hspace{1em} & - & - & - & - & - & - & (0.088) & (0.087) & (0.126)\\
\hspace{1em}Position - À l'heure & - & - & - & - & - & - & 0.232$^{*}$ & 0.24$^{*}$ & 0.008\\
\hspace{1em} & - & - & - & - & - & - & (0.125) & (0.123) & (0.169)\\
\hspace{1em}Position - En avance & - & - & - & - & - & - & 0.7$^{***}$ & 0.662$^{**}$ & 1.09$^{***}$\\
\hspace{1em} & - & - & - & - & - & - & (0.269) & (0.258) & (0.364)\\
\hspace{1em}Sexe - Garçon & $-$0.177$^{***}$ & $-$0.001 & $-$0.179$^{***}$ & 0.001 & $-$0.031 & 0.203$^{**}$ & 0.039 & 0.009 & 0.123\\
\hspace{1em} & (0.044) & (0.077) & (0.044) & (0.078) & (0.076) & (0.098) & (0.073) & (0.072) & (0.097)\\
\hspace{1em}CSP - Moyenne & 0.313$^{***}$ & 0.073 & 0.315$^{***}$ & 0.063 & 0.095 & $-$0.249$^{**}$ & 0.045 & 0.074 & $-$0.18$^{*}$\\
\hspace{1em} & (0.043) & (0.074) & (0.043) & (0.074) & (0.071) & (0.101) & (0.07) & (0.067) & (0.098)\\
\hspace{1em}CSP - Favorisée & 0.297$^{***}$ & 0.24$^{**}$ & 0.301$^{***}$ & 0.231$^{**}$ & 0.233$^{**}$ & $-$0.327$^{**}$ & 0.149 & 0.163 & $-$0.316$^{**}$\\
\hspace{1em} & (0.063) & (0.109) & (0.063) & (0.109) & (0.106) & (0.149) & (0.109) & (0.105) & (0.144)\\
\hspace{1em}CSP - Très favorisée & 0.676$^{***}$ & 0.353$^{***}$ & 0.704$^{***}$ & 0.363$^{***}$ & 0.374$^{***}$ & 0.168 & 0.291$^{***}$ & 0.3$^{***}$ & 0.149\\
\hspace{1em} & (0.043) & (0.094) & (0.043) & (0.095) & (0.093) & (0.121) & (0.091) & (0.09) & (0.107)\\
\hspace{1em}CSP - Autres & $-$0.039 & 0.245 & $-$0.033 & 0.262$^{*}$ & 0.261$^{*}$ & 0.006 & 0.312$^{**}$ & 0.315$^{**}$ & $-$0.026\\
\hspace{1em} & (0.165) & (0.153) & (0.164) & (0.152) & (0.152) & (0.193) & (0.151) & (0.151) & (0.204)\\
\hspace{1em}Régime scolaire - Interne & 0.459$^{***}$ & 0.259 & 0.464$^{***}$ & 0.28 & 0.316 & 0.27 & 0.113 & 0.143 & 0.237\\
\hspace{1em} & (0.158) & (0.229) & (0.158) & (0.215) & (0.214) & (0.211) & (0.207) & (0.202) & (0.223)\\
\hspace{1em}Régime scolaire - Externe & $-$0.009 & $-$0.114$^{**}$ & $-$0.007 & $-$0.105$^{*}$ & $-$0.061 & $-$0.077 & $-$0.101$^{*}$ & $-$0.053 & $-$0.118\\
\hspace{1em} & (0.021) & (0.058) & (0.021) & (0.058) & (0.058) & (0.083) & (0.054) & (0.053) & (0.084)\\
 &  &  &  &  &  &  &  &  & \\
Observations & 25387 & 25387 & 25387 & 25387 & 26735 & 29109 & 25387 & 26735 & 29109\\
R$^2$ ajusté & 0.546 & 0.505 & 0.546 & 0.506 & 0.512 & 0.431 & 0.518 & 0.525 & 0.445\\*
\end{longtable}
\end{ThreePartTable}
\endgroup{}
\elandscape

\setcounter{table}{0}
\setcounter{figure}{0}

\hypertarget{petestsbiaisasym}{%
\section{Pourquoi les effets de pairs sont fortement différents lorsque l'échantillon est rajouté des élèves des classes atypiques désavantagées mais pas avantagées ?}\label{petestsbiaisasym}}

\begin{figure}[H]

{\centering \includegraphics[width=1\linewidth]{000_files/figure-latex/petestsbiaisasymeff-1} 

}

\caption{Effets de pairs estimés lorsque moins de classes atypiques inférieures sont rajoutées à l'échantillon d'estimation}\label{fig:petestsbiaisasymeff}
\end{figure}

\setcounter{table}{0}
\setcounter{figure}{0}

\hypertarget{peecdfscore}{%
\section{\texorpdfstring{Fonctions de répartition empiriques des notes au CM2 selon le type de classe en 3\textsuperscript{ème}}{Fonctions de répartition empiriques des notes au CM2 selon le type de classe en 3ème}}\label{peecdfscore}}

\begin{figure}[H]

{\centering \includegraphics[width=1\linewidth]{000_files/figure-latex/peecdfscore-1} 

}

\caption{Fonctions de répartition empiriques des notes au CM2 selon le type de classe en 3ème}\label{fig:peecdfscore}
\end{figure}

\setcounter{table}{0}
\setcounter{figure}{0}
\blandscape

\hypertarget{pemodelsssmoy0}{%
\section{Résultats d'estimation des effets de pairs, notamment sur les autres matières au DNB}\label{pemodelsssmoy0}}

\hypertarget{pemodelsssmoy}{%
\subsection{Spécification d'effets de pairs linéaires en moyennes}\label{pemodelsssmoy}}

\begingroup\fontsize{8}{10}\selectfont

\begin{ThreePartTable}
\begin{TableNotes}
\item \textit{Sources :} Fichiers DNB (2014 à 2016), fichiers CM2 (2010 à 2012), fichiers CONSTAT (2013 à 2016), calculs de l'auteur.
\item \textit{Notes :} Une colonne correspond à une régression. Les notes au CM2 et au DNB sont normalisées sur l'année scolaire de CM2 et de DNB, respectivement. Écart-types robustes entre parenthèses. Des effets fixes collèges-année sont utilisés. Contrôles potentiellement endogènes : âge, position et taille de classe au DNB. Autres contrôles : sexe, catégorie socio-professionnelle et régime scolaire. CM2 : Cours Moyen 2\textsuperscript{ème} année.
\item Significativité : 10\% * 5\% ** 1\% ***.
\end{TableNotes}
\begin{longtable}[t]{lllllll}
\caption{\label{tab:pemodelsssmoy}Effets de pairs linéaires en moyenne sur les autres matières}\\
\toprule
\multicolumn{1}{c}{} & \multicolumn{6}{c}{Variable dépendante : Note en} \\
\cmidrule(l{3pt}r{3pt}){2-7}
\multicolumn{1}{c}{} & \multicolumn{2}{c}{Histoire-Géographie (écrits)} & \multicolumn{2}{c}{Dictée (écrits)} & \multicolumn{2}{c}{Rédaction (écrits)} \\
\cmidrule(l{3pt}r{3pt}){2-3} \cmidrule(l{3pt}r{3pt}){4-5} \cmidrule(l{3pt}r{3pt}){6-7}
 & \makecell{Sans var.endo. \\ (1) } & \makecell{Avec var.endo. \\ (2) } & \makecell{Sans var.endo. \\ (3) } & \makecell{Avec var.endo. \\ (4) } & \makecell{Sans var.endo. \\ (5) } & \makecell{Avec var.endo. \\ (6) }\\
\midrule
\endfirsthead
\caption[]{\label{tab:pemodelsssmoy}Effets de pairs linéaires en moyenne sur les autres matières (suite)}\\
\toprule
 & \makecell{Sans var.endo. \\ (1) } & \makecell{Avec var.endo. \\ (2) } & \makecell{Sans var.endo. \\ (3) } & \makecell{Avec var.endo. \\ (4) } & \makecell{Sans var.endo. \\ (5) } & \makecell{Avec var.endo. \\ (6) }\\
\midrule
\endhead

\endfoot
\bottomrule
\insertTableNotes
\endlastfoot
Note au CM2 des pairs $\times \ p$ & 0.223$^{***}$ & 0.158$^{***}$ & 0.248$^{***}$ & 0.176$^{***}$ & 0.169$^{***}$ & 0.154$^{***}$\\
 & (0.037) & (0.043) & (0.03) & (0.031) & (0.036) & (0.036)\\
Note au CM2 & 0.592$^{***}$ & 0.562$^{***}$ & 0.604$^{***}$ & 0.57$^{***}$ & 0.425$^{***}$ & 0.401$^{***}$\\
 & (0.009) & (0.008) & (0.008) & (0.008) & (0.009) & (0.008)\\
p & 0.077 & 0.076 & 0.179$^{**}$ & 0.17$^{**}$ & 0.233$^{***}$ & 0.245$^{***}$\\
 & (0.097) & (0.096) & (0.072) & (0.068) & (0.081) & (0.083)\\
 &  &  &  &  &  & \\
Contrôles individuels & Oui & Oui & Oui & Oui & Oui & Oui\\
Contrôles chez les pairs & Oui & Oui & Oui & Oui & Oui & Oui\\
Contrôles potentiellement endogènes & Non & Oui & Non & Oui & Non & Oui\\
Observations & 25339 & 25339 & 25340 & 25340 & 25317 & 25317\\
R$^2$ ajusté & 0.377 & 0.386 & 0.435 & 0.446 & 0.218 & 0.223\\*
\end{longtable}
\end{ThreePartTable}
\endgroup{}

\elandscape

\blandscape

\hypertarget{pemodelssexemodssmoy}{%
\subsection{Spécification d'effets de pairs linéaires en moyennes, différents selon le sexe}\label{pemodelssexemodssmoy}}

\begingroup\fontsize{8}{10}\selectfont

\begin{ThreePartTable}
\begin{TableNotes}
\item \textit{Sources :} Fichiers DNB (2014 à 2016), fichiers CM2 (2010 à 2012), fichiers CONSTAT (2013 à 2016), calculs de l'auteur.
\item \textit{Notes :} Une colonne correspond à une régression. Les notes au CM2 et au DNB sont normalisées sur l'année scolaire de CM2 et de DNB, respectivement. Écart-types robustes entre parenthèses. Des effets fixes collèges-année sont utilisés. Contrôles potentiellement endogènes : âge, position et taille de classe au DNB. Autres contrôles : sexe, catégorie socio-professionnelle et régime scolaire. CM2 : Cours Moyen 2\textsuperscript{ème} année.
\item Significativité : 10\% * 5\% ** 1\% ***.
\end{TableNotes}
\begin{longtable}[t]{lllllll}
\caption{\label{tab:pemodelssexemodssmoy}Effets de pairs linéaires en moyenne sur les autres matières, hétérogènes selon le sexe}\\
\toprule
\multicolumn{1}{c}{} & \multicolumn{6}{c}{Variable dépendante : Note} \\
\cmidrule(l{3pt}r{3pt}){2-7}
\multicolumn{1}{c}{} & \multicolumn{2}{c}{Histoire-Géographie (écrits)} & \multicolumn{2}{c}{Dictée (écrits)} & \multicolumn{2}{c}{Rédaction (écrits)} \\
\cmidrule(l{3pt}r{3pt}){2-3} \cmidrule(l{3pt}r{3pt}){4-5} \cmidrule(l{3pt}r{3pt}){6-7}
 & \makecell{Sans var.endo. \\ (1) } & \makecell{Avec var.endo. \\ (2) } & \makecell{Sans var.endo. \\ (3) } & \makecell{Avec var.endo. \\ (4) } & \makecell{Sans var.endo. \\ (5) } & \makecell{Avec var.endo. \\ (6) }\\
\midrule
\endfirsthead
\caption[]{\label{tab:pemodelssexemodssmoy}Effets de pairs linéaires en moyenne sur les autres matières, hétérogènes selon le sexe (suite)}\\
\toprule
 & \makecell{Sans var.endo. \\ (1) } & \makecell{Avec var.endo. \\ (2) } & \makecell{Sans var.endo. \\ (3) } & \makecell{Avec var.endo. \\ (4) } & \makecell{Sans var.endo. \\ (5) } & \makecell{Avec var.endo. \\ (6) }\\
\midrule
\endhead

\endfoot
\bottomrule
\insertTableNotes
\endlastfoot
Note au CM2 des pairs $\times \ p$ & 0.267$^{***}$ & 0.204$^{***}$ & 0.285$^{***}$ & 0.216$^{***}$ & 0.213$^{***}$ & 0.197$^{***}$\\
 & (0.039) & (0.044) & (0.032) & (0.032) & (0.039) & (0.041)\\
Note au CM2 des pairs $\times \ p$ $\times \ $Sexe - Garçon & $-$0.089$^{***}$ & $-$0.095$^{***}$ & $-$0.073$^{***}$ & $-$0.083$^{***}$ & $-$0.087$^{**}$ & $-$0.09$^{**}$\\
 & (0.029) & (0.029) & (0.028) & (0.028) & (0.036) & (0.036)\\
Note au CM2 & 0.592$^{***}$ & 0.561$^{***}$ & 0.604$^{***}$ & 0.57$^{***}$ & 0.425$^{***}$ & 0.401$^{***}$\\
 & (0.009) & (0.008) & (0.008) & (0.008) & (0.009) & (0.008)\\
p & 0.075 & 0.073 & 0.177$^{**}$ & 0.167$^{**}$ & 0.231$^{***}$ & 0.242$^{***}$\\
 & (0.097) & (0.096) & (0.072) & (0.068) & (0.081) & (0.083)\\
 &  &  &  &  &  & \\
Contrôles individuels & Oui & Oui & Oui & Oui & Oui & Oui\\
Contrôles chez les pairs & Oui & Oui & Oui & Oui & Oui & Oui\\
Contrôles potentiellement endogènes & Non & Oui & Non & Oui & Non & Oui\\
Observations & 25339 & 25339 & 25340 & 25340 & 25317 & 25317\\
R$^2$ ajusté & 0.377 & 0.386 & 0.436 & 0.447 & 0.218 & 0.223\\*
\end{longtable}
\end{ThreePartTable}
\endgroup{}
\elandscape

\blandscape

\hypertarget{pemodelspcsregmodssmoy}{%
\subsection{Spécification d'effets de pairs linéaires en moyennes, différents selon la catégorie sociale}\label{pemodelspcsregmodssmoy}}

\begingroup\fontsize{8}{10}\selectfont

\begin{ThreePartTable}
\begin{TableNotes}
\item \textit{Sources :} Fichiers DNB (2014 à 2016), fichiers CM2 (2010 à 2012), fichiers CONSTAT (2013 à 2016), calculs de l'auteur.
\item \textit{Notes :} Une colonne correspond à une régression. Les notes au CM2 et au DNB sont normalisées sur l'année scolaire de CM2 et de DNB, respectivement. Écart-types robustes entre parenthèses. Des effets fixes collèges-année sont utilisés. Contrôles potentiellement endogènes : âge, position et taille de classe au DNB. Autres contrôles : sexe, catégorie socio-professionnelle et régime scolaire. CM2 : Cours Moyen 2\textsuperscript{ème} année.
\item Significativité : 10\% * 5\% ** 1\% ***.
\end{TableNotes}
\begin{longtable}[t]{lllllll}
\caption{\label{tab:pemodelspcsregmod}Effets de pairs linéaires en moyenne, hétérogènes selon la catégorie sociale}\\
\toprule
\multicolumn{1}{c}{} & \multicolumn{6}{c}{Variable dépendante : Note} \\
\cmidrule(l{3pt}r{3pt}){2-7}
\multicolumn{1}{c}{} & \multicolumn{2}{c}{Totale (écrits)} & \multicolumn{2}{c}{En français (écrits)} & \multicolumn{2}{c}{En Mathématiques (écrits)} \\
\cmidrule(l{3pt}r{3pt}){2-3} \cmidrule(l{3pt}r{3pt}){4-5} \cmidrule(l{3pt}r{3pt}){6-7}
 & \makecell{Sans var.endo. \\ (1) } & \makecell{Avec var.endo. \\ (2) } & \makecell{Sans var.endo. \\ (3) } & \makecell{Avec var.endo. \\ (4) } & \makecell{Sans var.endo. \\ (5) } & \makecell{Avec var.endo. \\ (6) }\\
\midrule
\endfirsthead
\caption[]{\label{tab:pemodelspcsregmod}Effets de pairs linéaires en moyenne, hétérogènes selon la catégorie sociale (suite)}\\
\toprule
 & \makecell{Sans var.endo. \\ (1) } & \makecell{Avec var.endo. \\ (2) } & \makecell{Sans var.endo. \\ (3) } & \makecell{Avec var.endo. \\ (4) } & \makecell{Sans var.endo. \\ (5) } & \makecell{Avec var.endo. \\ (6) }\\
\midrule
\endhead

\endfoot
\bottomrule
\insertTableNotes
\endlastfoot
Note au CM2 des pairs $\times \ p$ & 0.245$^{***}$ & 0.166$^{***}$ & 0.249$^{***}$ & 0.184$^{***}$ & 0.211$^{***}$ & 0.153$^{***}$\\
 & (0.036) & (0.038) & (0.034) & (0.035) & (0.037) & (0.04)\\
Note au CM2 des pairs $\times \ p$ $\times \ $CSP - Moyenne & 0.043 & 0.063$^{**}$ & $-$0.011 & 0.006 & 0.07$^{**}$ & 0.087$^{**}$\\
 & (0.032) & (0.031) & (0.031) & (0.031) & (0.035) & (0.034)\\
Note au CM2 des pairs $\times \ p$ $\times \ $CSP - Favorisée & 0.105$^{**}$ & 0.13$^{***}$ & 0.031 & 0.054 & 0.141$^{***}$ & 0.16$^{***}$\\
 & (0.047) & (0.046) & (0.049) & (0.048) & (0.052) & (0.051)\\
Note au CM2 des pairs $\times \ p$ $\times \ $CSP - Très favorisée & $-$0.011 & 0.021 & $-$0.112$^{**}$ & $-$0.081$^{*}$ & 0.091 & 0.113$^{**}$\\
 & (0.052) & (0.053) & (0.049) & (0.049) & (0.057) & (0.058)\\
Note au CM2 des pairs $\times \ p$ $\times \ $CSP - Autres & 0.296 & 0.233 & 0.378$^{**}$ & 0.315$^{**}$ & 0.172 & 0.108\\
 & (0.19) & (0.18) & (0.16) & (0.154) & (0.204) & (0.195)\\
Note au CM2 & 0.674$^{***}$ & 0.638$^{***}$ & 0.599$^{***}$ & 0.565$^{***}$ & 0.607$^{***}$ & 0.577$^{***}$\\
 & (0.009) & (0.009) & (0.009) & (0.008) & (0.01) & (0.01)\\
p & 0.148$^{*}$ & 0.141$^{*}$ & 0.235$^{***}$ & 0.232$^{***}$ & 0.054 & 0.053\\
 & (0.082) & (0.079) & (0.072) & (0.07) & (0.085) & (0.083)\\
 &  &  &  &  &  & \\
Contrôles individuels & Oui & Oui & Oui & Oui & Oui & Oui\\
Contrôles chez les pairs & Oui & Oui & Oui & Oui & Oui & Oui\\
Contrôles potentiellement endogènes & Non & Oui & Non & Oui & Non & Oui\\
Observations & 25387 & 25387 & 25347 & 25347 & 25282 & 25282\\
R$^2$ ajusté & 0.506 & 0.519 & 0.435 & 0.446 & 0.396 & 0.407\\*
\end{longtable}
\end{ThreePartTable}
\endgroup{}
\elandscape

\blandscape
\begingroup\fontsize{8}{10}\selectfont

\begin{ThreePartTable}
\begin{TableNotes}
\item \textit{Sources :} Fichiers DNB (2014 à 2016), fichiers CM2 (2010 à 2012), fichiers CONSTAT (2013 à 2016), calculs de l'auteur.
\item \textit{Notes :} Une colonne correspond à une régression. Les notes au CM2 et au DNB sont normalisées sur l'année scolaire de CM2 et de DNB, respectivement. Écart-types robustes entre parenthèses. Des effets fixes collèges-année sont utilisés. Contrôles potentiellement endogènes : âge, position et taille de classe au DNB. Autres contrôles : sexe, catégorie socio-professionnelle et régime scolaire. CM2 : Cours Moyen 2\textsuperscript{ème} année.
\item Significativité : 10\% * 5\% ** 1\% ***.
\end{TableNotes}
\begin{longtable}[t]{lllllll}
\caption{\label{tab:pemodelspcsregmodssmoy}Effets de pairs linéaires en moyenne sur les autres matières, hétérogènes selon la catégorie sociale}\\
\toprule
\multicolumn{1}{c}{} & \multicolumn{6}{c}{Variable dépendante : Note} \\
\cmidrule(l{3pt}r{3pt}){2-7}
\multicolumn{1}{c}{} & \multicolumn{2}{c}{Histoire-Géographie (écrits)} & \multicolumn{2}{c}{Dictée (écrits)} & \multicolumn{2}{c}{Rédaction (écrits)} \\
\cmidrule(l{3pt}r{3pt}){2-3} \cmidrule(l{3pt}r{3pt}){4-5} \cmidrule(l{3pt}r{3pt}){6-7}
 & \makecell{Sans var.endo. \\ (1) } & \makecell{Avec var.endo. \\ (2) } & \makecell{Sans var.endo. \\ (3) } & \makecell{Avec var.endo. \\ (4) } & \makecell{Sans var.endo. \\ (5) } & \makecell{Avec var.endo. \\ (6) }\\
\midrule
\endfirsthead
\caption[]{\label{tab:pemodelspcsregmodssmoy}Effets de pairs linéaires en moyenne sur les autres matières, hétérogènes selon la catégorie sociale (suite)}\\
\toprule
 & \makecell{Sans var.endo. \\ (1) } & \makecell{Avec var.endo. \\ (2) } & \makecell{Sans var.endo. \\ (3) } & \makecell{Avec var.endo. \\ (4) } & \makecell{Sans var.endo. \\ (5) } & \makecell{Avec var.endo. \\ (6) }\\
\midrule
\endhead

\endfoot
\bottomrule
\insertTableNotes
\endlastfoot
Note au CM2 des pairs $\times \ p$ & 0.201$^{***}$ & 0.129$^{***}$ & 0.243$^{***}$ & 0.161$^{***}$ & 0.194$^{***}$ & 0.176$^{***}$\\
 & (0.04) & (0.044) & (0.032) & (0.033) & (0.039) & (0.038)\\
Note au CM2 des pairs $\times \ p$ $\times \ $CSP - Moyenne & 0.047 & 0.062$^{*}$ & 0.013 & 0.033 & $-$0.045 & $-$0.036\\
 & (0.037) & (0.036) & (0.031) & (0.031) & (0.038) & (0.037)\\
Note au CM2 des pairs $\times \ p$ $\times \ $CSP - Favorisée & 0.109$^{**}$ & 0.129$^{**}$ & 0.075 & 0.1$^{**}$ & $-$0.046 & $-$0.034\\
 & (0.054) & (0.053) & (0.05) & (0.049) & (0.059) & (0.059)\\
Note au CM2 des pairs $\times \ p$ $\times \ $CSP - Très favorisée & $-$0.001 & 0.024 & $-$0.073 & $-$0.038 & $-$0.146$^{***}$ & $-$0.13$^{**}$\\
 & (0.068) & (0.068) & (0.047) & (0.048) & (0.055) & (0.055)\\
Note au CM2 des pairs $\times \ p$ $\times \ $CSP - Autres & 0.256 & 0.21 & 0.381$^{**}$ & 0.317$^{**}$ & 0.27 & 0.227\\
 & (0.21) & (0.202) & (0.167) & (0.161) & (0.182) & (0.179)\\
Note au CM2 & 0.592$^{***}$ & 0.561$^{***}$ & 0.604$^{***}$ & 0.57$^{***}$ & 0.425$^{***}$ & 0.402$^{***}$\\
 & (0.009) & (0.008) & (0.008) & (0.008) & (0.009) & (0.008)\\
p & 0.076 & 0.073 & 0.183$^{**}$ & 0.172$^{**}$ & 0.241$^{***}$ & 0.254$^{***}$\\
 & (0.097) & (0.096) & (0.072) & (0.068) & (0.081) & (0.083)\\
 &  &  &  &  &  & \\
Contrôles individuels & Oui & Oui & Oui & Oui & Oui & Oui\\
Contrôles chez les pairs & Oui & Oui & Oui & Oui & Oui & Oui\\
Contrôles potentiellement endogènes & Non & Oui & Non & Oui & Non & Oui\\
Observations & 25339 & 25339 & 25340 & 25340 & 25317 & 25317\\
R$^2$ ajusté & 0.377 & 0.386 & 0.436 & 0.447 & 0.218 & 0.223\\*
\end{longtable}
\end{ThreePartTable}
\endgroup{}
\elandscape

\hypertarget{pemodelsh5corrssmoy}{%
\subsection{Spécification d'effets de pairs hétérogènes}\label{pemodelsh5corrssmoy}}

\begin{figure}[H]

{\centering \includegraphics[width=1\linewidth]{000_files/figure-latex/pemodelsh5corrssmoygraph-1} 

}

\caption{Effets de pairs hétérogènes sur les autres matières}\label{fig:pemodelsh5corrssmoygraph}
\end{figure}

\newpage

\hypertarget{pemodelsh5corrsexemodssmoy}{%
\subsection{Spécification d'effets de pairs hétérogènes, différents selon le sexe}\label{pemodelsh5corrsexemodssmoy}}

\begin{figure}[H]

{\centering \includegraphics[width=1\linewidth]{000_files/figure-latex/pemodelsh5corrsexemodgraph-1} 

}

\caption{Effets de pairs hétérogènes, différents selon le sexe}\label{fig:pemodelsh5corrsexemodgraph}
\end{figure}

\begin{figure}[H]

{\centering \includegraphics[width=1\linewidth]{000_files/figure-latex/pemodelsh5corrsexemodssmoygraph-1} 

}

\caption{Effets de pairs hétérogènes sur les autres matières, différents selon le sexe}\label{fig:pemodelsh5corrsexemodssmoygraph}
\end{figure}

\newpage

\hypertarget{pemodelsh5corrpcsregmodssmoy}{%
\subsection{Spécification d'effets de pairs hétérogènes, différents selon la catégorie sociale}\label{pemodelsh5corrpcsregmodssmoy}}

\begin{figure}[H]

{\centering \includegraphics[width=1\linewidth]{000_files/figure-latex/pemodelsh5corrpcsregmodgraph-1} 

}

\caption{Effets de pairs hétérogènes, différents selon la catégorie sociale}\label{fig:pemodelsh5corrpcsregmodgraph}
\end{figure}

\begin{figure}[H]

{\centering \includegraphics[width=1\linewidth]{000_files/figure-latex/pemodelsh5corrpcsregmodssmoygraph-1} 

}

\caption{Effets de pairs hétérogènes sur les autres matières, différents selon la catégorie sociale}\label{fig:pemodelsh5corrpcsregmodssmoygraph}
\end{figure}

\hypertarget{pemodelshnolinq5corrssmoy}{%
\subsection{Spécification d'effets de pairs hétérogènes et non linéaires}\label{pemodelshnolinq5corrssmoy}}

\begin{figure}[H]

{\centering \includegraphics[width=1\linewidth]{000_files/figure-latex/pemodelshnolinq5corrsDexossmoygraph-1} 

}

\caption{Effets de pairs hétérogènes et non linéaires sur les autres matières, sans contrôles potentiellement endogènes}\label{fig:pemodelshnolinq5corrsDexossmoygraph}
\end{figure}

\hypertarget{pemodelshnolinq5corrsexemodssmoy}{%
\subsection{Spécification d'effets de pairs hétérogènes et non linéaires, différents selon le sexe}\label{pemodelshnolinq5corrsexemodssmoy}}

\begin{figure}[H]

{\centering \includegraphics[width=1\linewidth]{000_files/figure-latex/pemodelshnolinq5corrsDexosexemodgraph-1} 

}

\caption{Effets de pairs hétérogènes et non linéaires, différents selon le sexe, sans contrôles potentiellement endogènes}\label{fig:pemodelshnolinq5corrsDexosexemodgraph}
\end{figure}

\begin{figure}[H]

{\centering \includegraphics[width=1\linewidth]{000_files/figure-latex/pemodelshnolinq5corrsDexosexemodssmoygraph-1} 

}

\caption{Effets de pairs hétérogènes et non linéaires sur les autres matières, différents selon le sexe, sans contrôles potentiellement endogènes}\label{fig:pemodelshnolinq5corrsDexosexemodssmoygraph}
\end{figure}

\hypertarget{pemodelshnolinq5corrpcsregmodssmoy}{%
\subsection{Spécification d'effets de pairs hétérogènes et non linéaires, différents selon la catégorie sociale}\label{pemodelshnolinq5corrpcsregmodssmoy}}

\begin{figure}[H]

{\centering \includegraphics[width=1\linewidth]{000_files/figure-latex/pemodelshnolinq5corrsDexopcsregmodgraph-1} 

}

\caption{Effets de pairs hétérogènes et non linéaires, différents selon la catégorie sociale, sans contrôles potentiellement endogènes}\label{fig:pemodelshnolinq5corrsDexopcsregmodgraph}
\end{figure}

\begin{figure}[H]

{\centering \includegraphics[width=1\linewidth]{000_files/figure-latex/pemodelshnolinq5corrsDexopcsregmodssmoygraph-1} 

}

\caption{Effets de pairs hétérogènes et non linéaires sur les autres matières, différents selon la catégorie sociale, sans contrôles potentiellement endogènes}\label{fig:pemodelshnolinq5corrsDexopcsregmodssmoygraph}
\end{figure}

\setcounter{table}{0}
\setcounter{figure}{0}
\blandscape

\hypertarget{pemodelshbssmoy0}{%
\section{Robustesse des résultats d'estimation des effets de pairs sur les autres matières par rapport au découpage du niveau au CM2}\label{pemodelshbssmoy0}}

\hypertarget{pemodelshbcorrssmoy}{%
\subsection{Spécification d'effets de pairs hétérogènes}\label{pemodelshbcorrssmoy}}

\begingroup\fontsize{8}{10}\selectfont

\begin{ThreePartTable}
\begin{TableNotes}
\item \textit{Sources :} Fichiers DNB (2014 à 2016), fichiers CM2 (2010 à 2012), fichiers CONSTAT (2013 à 2016), calculs de l'auteur.
\item \textit{Notes :} Une colonne correspond à une régression. Les notes au CM2 et au DNB sont normalisées sur l'année scolaire de CM2 et de DNB, respectivement. Q1 à Q5 : quintiles de note au CM2 calculés sur les retrouvés par année de DNB. Écart-types robustes entre parenthèses. Des effets fixes collèges-année sont utilisés. Contrôles potentiellement endogènes : âge, position et taille de classe au DNB. Autres contrôles : découpage alternatif de la note au CM2, sexe, catégorie socio-professionnelle et régime scolaire. CM2 : Cours Moyen 2\textsuperscript{ème} année.
\item Significativité : 10\% * 5\% ** 1\% ***.
\end{TableNotes}
\begin{longtable}[t]{lllllll}
\caption{\label{tab:pemodelshbcorrssmoy}Effets de pairs hétérogènes - découpage alternatif du niveau au CM2}\\
\toprule
\multicolumn{1}{c}{} & \multicolumn{6}{c}{Variable dépendante : Note en} \\
\cmidrule(l{3pt}r{3pt}){2-7}
\multicolumn{1}{c}{} & \multicolumn{2}{c}{Histoire-Géographie (écrits)} & \multicolumn{2}{c}{Dictée (écrits)} & \multicolumn{2}{c}{Rédaction (écrits)} \\
\cmidrule(l{3pt}r{3pt}){2-3} \cmidrule(l{3pt}r{3pt}){4-5} \cmidrule(l{3pt}r{3pt}){6-7}
 & \makecell{Sans var.endo \\ (1) } & \makecell{Avec var.endo \\ (2) } & \makecell{Sans var.endo \\ (3) } & \makecell{Avec var.endo \\ (4) } & \makecell{Sans var.endo \\ (5) } & \makecell{Avec var.endo \\ (6) }\\
\midrule
\endfirsthead
\caption[]{\label{tab:pemodelshbcorrssmoy}Effets de pairs hétérogènes - découpage alternatif du niveau au CM2 (suite)}\\
\toprule
 & \makecell{Sans var.endo \\ (1) } & \makecell{Avec var.endo \\ (2) } & \makecell{Sans var.endo \\ (3) } & \makecell{Avec var.endo \\ (4) } & \makecell{Sans var.endo \\ (5) } & \makecell{Avec var.endo \\ (6) }\\
\midrule
\endhead

\endfoot
\bottomrule
\insertTableNotes
\endlastfoot
Note au CM2 des pairs $\times \ p \ \times$ Q1 & $-$0.021 & $-$0.124$^{**}$ & 0.124$^{***}$ & 0.016 & 0.049 & 0.017\\
 & (0.057) & (0.056) & (0.041) & (0.041) & (0.063) & (0.059)\\
Note au CM2 des pairs $\times \ p \ \times$ (Q2 à Q4) & 0.291$^{***}$ & 0.213$^{***}$ & 0.301$^{***}$ & 0.224$^{***}$ & 0.233$^{***}$ & 0.212$^{***}$\\
 & (0.037) & (0.043) & (0.032) & (0.033) & (0.036) & (0.038)\\
Note au CM2 des pairs $\times \ p \ \times$ Q5 & 0.363$^{***}$ & 0.318$^{***}$ & 0.246$^{***}$ & 0.209$^{***}$ & 0.12$^{***}$ & 0.122$^{***}$\\
 & (0.055) & (0.06) & (0.041) & (0.042) & (0.044) & (0.045)\\
p & 0.048 & 0.032 & 0.167$^{**}$ & 0.149$^{**}$ & 0.225$^{***}$ & 0.232$^{***}$\\
 & (0.098) & (0.095) & (0.072) & (0.067) & (0.08) & (0.082)\\
 &  &  &  &  &  & \\
Contrôles individuels & Oui & Oui & Oui & Oui & Oui & Oui\\
Contrôles chez les pairs & Oui & Oui & Oui & Oui & Oui & Oui\\
Contrôles potentiellement endogènes & Non & Oui & Non & Oui & Non & Oui\\
Observations & 25339 & 25339 & 25340 & 25340 & 25317 & 25317\\
R$^2$ ajusté & 0.379 & 0.388 & 0.436 & 0.447 & 0.219 & 0.224\\*
\end{longtable}
\end{ThreePartTable}
\endgroup{}
\elandscape

\hypertarget{pemodelshnolinqbcorr}{%
\subsection{Spécification d'effets de pairs hétérogènes et non linéaires}\label{pemodelshnolinqbcorr}}

\begin{figure}[H]

{\centering \includegraphics[width=1\linewidth]{000_files/figure-latex/pemodelshnolinqbcorrsDexossmoy-1} 

}

\caption{Effets de pairs hétérogènes et non linéaires sur les autres matières, sans contrôles potentiellement endogènes - découpage alternatif de la note au CM2}\label{fig:pemodelshnolinqbcorrsDexossmoy}
\end{figure}

\setcounter{table}{0}
\setcounter{figure}{0}

\newpage
\blandscape

\hypertarget{pemodelsnpssmoy0}{%
\section{Résultats d'estimation des effets de pairs séparément pour les nouveaux pairs et les anciens pairs sur les autres matières}\label{pemodelsnpssmoy0}}

\hypertarget{pemodelsnppsfessmoy}{%
\subsection{Spécification d'effets de pairs linéaires en moyennes}\label{pemodelsnppsfessmoy}}

\begingroup\fontsize{8}{10}\selectfont

\begin{ThreePartTable}
\begin{TableNotes}
\item \textit{Sources :} Fichiers DNB (2014 à 2016), fichiers CM2 (2010 à 2012), fichiers CONSTAT (2013 à 2016), calculs de l'auteur.
\item \textit{Notes :} Une colonne correspond à une régression. Les notes au CM2 et au DNB sont normalisées sur l'année scolaire de CM2 et de DNB, respectivement. Q1 à Q5 : quintiles de note au CM2 calculés sur les retrouvés par année de DNB. Écart-types robustes entre parenthèses. Des effets fixes collèges-année sont utilisés. Contrôles potentiellement endogènes : âge, position et taille de classe au DNB. Autres contrôles : note au CM2 (panel A)/quintile de note au CM2 (panel B), sexe, catégorie socio-professionnelle et régime scolaire. CM2 : Cours Moyen 2\textsuperscript{ème} année.
\item Significativité : 10\% * 5\% ** 1\% ***.
\end{TableNotes}
\begin{longtable}[t]{lllllll}
\caption{\label{tab:pemodelsnppsfeh5nppsfessmoy}Effets de pairs linéaires en moyenne, séparément pour les nouveaux pairs et les anciens pairs du CM2}\\
\toprule
\multicolumn{1}{c}{} & \multicolumn{6}{c}{Variable dépendante : Note en} \\
\cmidrule(l{3pt}r{3pt}){2-7}
\multicolumn{1}{c}{} & \multicolumn{2}{c}{Histoire-Géographie (écrits)} & \multicolumn{2}{c}{Dictée (écrits)} & \multicolumn{2}{c}{Rédaction (écrits)} \\
\cmidrule(l{3pt}r{3pt}){2-3} \cmidrule(l{3pt}r{3pt}){4-5} \cmidrule(l{3pt}r{3pt}){6-7}
 & \makecell{Sans var.endo. \\ (1) } & \makecell{Avec var.endo. \\ (2) } & \makecell{Sans var.endo. \\ (3) } & \makecell{Avec var.endo. \\ (4) } & \makecell{Sans var.endo. \\ (5) } & \makecell{Avec var.endo. \\ (6) }\\
\midrule
\endfirsthead
\caption[]{\label{tab:pemodelsnppsfeh5nppsfessmoy}Effets de pairs linéaires en moyenne, séparément pour les nouveaux pairs et les anciens pairs du CM2 (suite)}\\
\toprule
 & \makecell{Sans var.endo. \\ (1) } & \makecell{Avec var.endo. \\ (2) } & \makecell{Sans var.endo. \\ (3) } & \makecell{Avec var.endo. \\ (4) } & \makecell{Sans var.endo. \\ (5) } & \makecell{Avec var.endo. \\ (6) }\\
\midrule
\endhead

\endfoot
\bottomrule
\insertTableNotes
\endlastfoot
Note au CM2 (nouveaux pairs) $\times \ p$ & 0.199$^{***}$ & 0.155$^{***}$ & 0.229$^{***}$ & 0.183$^{***}$ & 0.157$^{***}$ & 0.15$^{***}$\\
 & (0.023) & (0.025) & (0.021) & (0.023) & (0.026) & (0.029)\\
Note au CM2 anciens pairs $\times \ p$ & $-$0.028$^{**}$ & $-$0.028$^{***}$ & $-$0.046$^{***}$ & $-$0.046$^{***}$ & $-$0.038$^{***}$ & $-$0.036$^{***}$\\
 & (0.011) & (0.011) & (0.01) & (0.01) & (0.012) & (0.012)\\
p & 0.073 & 0.074 & 0.195$^{***}$ & 0.192$^{***}$ & 0.242$^{***}$ & 0.252$^{***}$\\
 & (0.057) & (0.057) & (0.053) & (0.053) & (0.065) & (0.066)\\
 &  &  &  &  &  & \\
Contrôles individuels & Oui & Oui & Oui & Oui & Oui & Oui\\
Contrôles chez les pairs & Oui & Oui & Oui & Oui & Oui & Oui\\
Contrôles potentiellement endogènes & Non & Oui & Non & Oui & Non & Oui\\
Observations & 25339 & 25339 & 25340 & 25340 & 25317 & 25317\\
R$^2$ ajusté & 0.399 & 0.403 & 0.465 & 0.472 & 0.231 & 0.235\\*
\end{longtable}
\end{ThreePartTable}
\endgroup{}
\elandscape

\hypertarget{pemodelsh5corrnppsfessmoy}{%
\subsection{Spécification d'effets de pairs hétérogènes}\label{pemodelsh5corrnppsfessmoy}}

\begin{figure}[H]

{\centering \includegraphics[width=1\linewidth]{000_files/figure-latex/pemodelsh5corrnppsfesDexossmoygraph-1} 

}

\caption{Effets de pairs hétérogènes sur les autres matières, séparément pour les nouveaux pairs et les anciens pairs du CM2, sans contrôles potentiellement endogènes}\label{fig:pemodelsh5corrnppsfesDexossmoygraph}
\end{figure}

\hypertarget{pemodelshnolinq5corrnpssmoy}{%
\subsection{Spécification d'effets de pairs hétérogènes et non linéaires}\label{pemodelshnolinq5corrnpssmoy}}

\begin{figure}[H]

{\centering \includegraphics[width=1\linewidth]{000_files/figure-latex/pemodelshnolinq5corrnppsfesDexossmoygraph-1} 

}

\caption{Effets de pairs hétérogènes et non linéaires sur les autres matières, séparément pour les nouveaux pairs et les anciens pairs du CM2, sans contrôles potentiellement endogènes}\label{fig:pemodelshnolinq5corrnppsfesDexossmoygraph}
\end{figure}

\setcounter{table}{0}
\setcounter{figure}{0}
\blandscape

\hypertarget{pepcmlmodelsssmoy0}{%
\section{Résultats d'estimation des effets de pairs endogènes au DNB sur les autres matières}\label{pepcmlmodelsssmoy0}}

\hypertarget{pepcmlmodelsssmoy}{%
\subsection{Sur tout l'échantillon}\label{pepcmlmodelsssmoy}}

\begingroup\fontsize{8}{10}\selectfont

\begin{ThreePartTable}
\begin{TableNotes}
\item \textit{Sources :} Fichiers DNB (2014 à 2016), fichiers CONSTAT (2013 à 2016), calculs de l'auteur.
\item \textit{Notes :} Une colonne correspond à une régression. La note contemporaine des pairs est de même nature que la variable dépendante (exemple : Note totale dans la colonne (1) et note en français aux écrits dans la colonne (3)). Les valeurs manquantes sur les variables dépendantes sont exclues avant toute estimation. Écart-types entre parenthèses. Des effets fixes de classe sont utilisés. Contrôles potentiellement endogènes : âge et position au DNB. Autres contrôles : sexe, catégorie socio-professionnelle (CSP) et régime scolaire.
\item Significativité : 10\% * 5\% ** 1\% ***.
\end{TableNotes}
\begin{longtable}[t]{lllllll}
\caption{\label{tab:pepcmlmodelsssmoy}Effets de pairs endogènes sur les autres matières au DNB}\\
\toprule
\multicolumn{1}{c}{} & \multicolumn{6}{c}{Variable dépendante : Note en} \\
\cmidrule(l{3pt}r{3pt}){2-7}
\multicolumn{1}{c}{} & \multicolumn{2}{c}{Histoire-Géographie (écrits)} & \multicolumn{2}{c}{Dictée (écrits)} & \multicolumn{2}{c}{Rédaction (écrits)} \\
\cmidrule(l{3pt}r{3pt}){2-3} \cmidrule(l{3pt}r{3pt}){4-5} \cmidrule(l{3pt}r{3pt}){6-7}
 & \makecell{Sans var.endo. \\ (1) } & \makecell{Avec var.endo. \\ (2) } & \makecell{Sans var.endo. \\ (3) } & \makecell{Avec var.endo. \\ (4) } & \makecell{Sans var.endo. \\ (5) } & \makecell{Avec var.endo. \\ (6) }\\
\midrule
\endfirsthead
\caption[]{\label{tab:pepcmlmodelsssmoy}Effets de pairs endogènes sur les autres matières au DNB (suite)}\\
\toprule
 & \makecell{Sans var.endo. \\ (1) } & \makecell{Avec var.endo. \\ (2) } & \makecell{Sans var.endo. \\ (3) } & \makecell{Avec var.endo. \\ (4) } & \makecell{Sans var.endo. \\ (5) } & \makecell{Avec var.endo. \\ (6) }\\
\midrule
\endhead

\endfoot
\bottomrule
\insertTableNotes
\endlastfoot
\addlinespace[0.3em]
\multicolumn{7}{l}{\textbf{}}\\
\hspace{1em}Note contemporaine des pairs & 2.699$^{***}$ & 4.471$^{***}$ & 0.847$^{**}$ & 1.061$^{***}$ & 0.157 & 0.315\\
\hspace{1em} & (0.452) & (0.59) & (0.377) & (0.389) & (0.263) & (0.277)\\
\addlinespace[0.3em]
\multicolumn{7}{l}{\textbf{}}\\
\hspace{1em}Âge aux examens & - & 0.204$^{***}$ & - & $-$0.045 & - & $-$0.042\\
\hspace{1em} & - & (0.066) & - & (0.05) & - & (0.053)\\
\hspace{1em}Position - À l'heure & - & 1.25$^{***}$ & - & 0.782$^{***}$ & - & 0.571$^{***}$\\
\hspace{1em} & - & (0.092) & - & (0.062) & - & (0.065)\\
\hspace{1em}Position - En avance & - & 1.145$^{***}$ & - & 0.409$^{*}$ & - & 0.218\\
\hspace{1em} & - & (0.274) & - & (0.227) & - & (0.241)\\
\hspace{1em}Sexe - Garçon & $-$0.093$^{**}$ & $-$0.137$^{***}$ & $-$0.42$^{***}$ & $-$0.439$^{***}$ & $-$0.338$^{***}$ & $-$0.355$^{***}$\\
\hspace{1em} & (0.042) & (0.044) & (0.038) & (0.038) & (0.04) & (0.04)\\
\hspace{1em}CSP - Moyenne & 0.419$^{***}$ & 0.305$^{***}$ & 0.327$^{***}$ & 0.261$^{***}$ & 0.288$^{***}$ & 0.249$^{***}$\\
\hspace{1em} & (0.051) & (0.053) & (0.044) & (0.045) & (0.048) & (0.049)\\
\hspace{1em}CSP - Favorisée & 0.465$^{***}$ & 0.369$^{***}$ & 0.407$^{***}$ & 0.37$^{***}$ & 0.341$^{***}$ & 0.333$^{***}$\\
\hspace{1em} & (0.057) & (0.062) & (0.052) & (0.053) & (0.057) & (0.06)\\
\hspace{1em}CSP - Très favorisée & 0.493$^{***}$ & 0.365$^{***}$ & 0.371$^{***}$ & 0.306$^{***}$ & 0.287$^{***}$ & 0.246$^{**}$\\
\hspace{1em} & (0.104) & (0.109) & (0.094) & (0.093) & (0.098) & (0.098)\\
\hspace{1em}CSP - Autre & 0.7$^{***}$ & 0.853$^{***}$ & 0.195 & 0.167 & 0.185 & 0.186\\
\hspace{1em} & (0.213) & (0.223) & (0.305) & (0.297) & (0.316) & (0.314)\\
\addlinespace[0.3em]
\multicolumn{7}{l}{\textbf{Moyenne chez les pairs}}\\
\hspace{1em}Âge aux examens & - & 4.821$^{***}$ & - & $-$0.486 & - & $-$0.948\\
\hspace{1em} & - & (1.424) & - & (1.074) & - & (1.144)\\
\hspace{1em}Position - À l'heure & - & 14.737$^{***}$ & - & 5.171$^{***}$ & - & 4.031$^{***}$\\
\hspace{1em} & - & (1.808) & - & (1.277) & - & (1.353)\\
\hspace{1em}Position - En avance & - & $-$3.918 & - & $-$16.393$^{***}$ & - & $-$13.064$^{**}$\\
\hspace{1em} & - & (6.326) & - & (5.368) & - & (5.681)\\
\hspace{1em}Sexe - Garçon & 0.443 & $-$1.175 & $-$0.41 & $-$1.52$^{*}$ & 0.369 & $-$0.477\\
\hspace{1em} & (0.921) & (0.969) & (0.831) & (0.828) & (0.866) & (0.877)\\
\hspace{1em}CSP - Moyenne & 3.47$^{***}$ & 1.377 & 1.842$^{*}$ & 1.13 & 2.607$^{**}$ & 2.256$^{**}$\\
\hspace{1em} & (1.08) & (1.154) & (0.945) & (0.959) & (1.021) & (1.058)\\
\hspace{1em}CSP - Favorisée & 0.672 & $-$1.087 & 0.934 & 1.083 & 1.839 & 2.326$^{*}$\\
\hspace{1em} & (1.232) & (1.362) & (1.123) & (1.157) & (1.234) & (1.308)\\
\hspace{1em}CSP - Très favorisée & $-$5.529$^{**}$ & $-$7.521$^{***}$ & $-$5.492$^{**}$ & $-$5.198$^{**}$ & $-$3.502 & $-$3.247\\
\hspace{1em} & (2.453) & (2.595) & (2.227) & (2.194) & (2.313) & (2.319)\\
\hspace{1em}CSP - Autre & 16.951$^{***}$ & 18.283$^{***}$ & 4.409 & 1.801 & 5.403 & 4.112\\
\hspace{1em} & (4.384) & (4.587) & (6.497) & (6.345) & (6.743) & (6.7)\\
 &  &  &  &  &  & \\
Observations & 41462 & 41462 & 41478 & 41478 & 41444 & 41444\\*
\end{longtable}
\end{ThreePartTable}
\endgroup{}
\elandscape

\blandscape

\hypertarget{pepcmlmodelsssmoystatutreseau}{%
\subsection{Selon le statut et le réseau d'éducation prioritaire}\label{pepcmlmodelsssmoystatutreseau}}

\begingroup\fontsize{8}{10}\selectfont

\begin{ThreePartTable}
\begin{TableNotes}
\item \textit{Sources :} Fichiers DNB (2014 à 2016), fichiers CONSTAT (2013 à 2016), calculs de l'auteur.
\item \textit{Notes :} Une colonne correspond à une régression. La note contemporaine des pairs est de même nature que la variable dépendante (exemple : Note totale dans la colonne (1) et note en français aux écrits dans la colonne (3)). Les valeurs manquantes sur les variables dépendantes sont exclues avant toute estimation. Écart-types entre parenthèses. Des effets fixes de classe sont utilisés. Contrôles potentiellement endogènes : âge et position au DNB. Autres contrôles : sexe, catégorie socio-professionnelle (CSP) et régime scolaire. (H)EP : (Hors-)Éducation Prioritaire.
\item Significativité : 10\% * 5\% ** 1\% ***.
\end{TableNotes}
\begin{longtable}[t]{lllllll}
\caption{\label{tab:pepcmlmodelsssmoystatutreseau}Effets de pairs endogènes sur les autres matières au DNB, par type de collège}\\
\toprule
\multicolumn{1}{c}{} & \multicolumn{6}{c}{Variable dépendante : Note en} \\
\cmidrule(l{3pt}r{3pt}){2-7}
\multicolumn{1}{c}{} & \multicolumn{2}{c}{Histoire-Géographie (écrits)} & \multicolumn{2}{c}{Dictée (écrits)} & \multicolumn{2}{c}{Rédaction (écrits)} \\
\cmidrule(l{3pt}r{3pt}){2-3} \cmidrule(l{3pt}r{3pt}){4-5} \cmidrule(l{3pt}r{3pt}){6-7}
 & \makecell{Sans var.endo. \\ (1) } & \makecell{Avec var.endo. \\ (2) } & \makecell{Sans var.endo. \\ (3) } & \makecell{Avec var.endo. \\ (4) } & \makecell{Sans var.endo. \\ (5) } & \makecell{Avec var.endo. \\ (6) }\\
\midrule
\endfirsthead
\caption[]{\label{tab:pepcmlmodelsssmoystatutreseau}Effets de pairs endogènes sur les autres matières au DNB, par type de collège (suite)}\\
\toprule
 & \makecell{Sans var.endo. \\ (1) } & \makecell{Avec var.endo. \\ (2) } & \makecell{Sans var.endo. \\ (3) } & \makecell{Avec var.endo. \\ (4) } & \makecell{Sans var.endo. \\ (5) } & \makecell{Avec var.endo. \\ (6) }\\
\midrule
\endhead

\endfoot
\bottomrule
\insertTableNotes
\endlastfoot
\addlinespace[0.3em]
\multicolumn{7}{l}{\textbf{Panel A : Collèges privés}}\\
\hline
\hspace{1em}Note contemporaine des pairs & 10.298$^{**}$ & 8.153$^{*}$ & 1.444 & 1.116 & 2.102 & 1.787\\
\hspace{1em} & (5.19) & (4.552) & (2.373) & (2.264) & (2.671) & (2.57)\\
\hspace{1em} &  &  &  &  &  \vphantom{5} & \\
\hspace{1em}Contrôles individuels & Oui & Oui & Oui & Oui & Oui & \vphantom{2} Oui\\
\hspace{1em}Contrôles chez les pairs & Oui & Oui & Oui & Oui & Oui & \vphantom{2} Oui\\
\hspace{1em}Contrôles potentiellement endogènes & Non & Oui & Non & Oui & Non & \vphantom{2} Oui\\
\hspace{1em}Observations & 3414 & 3414 & 3414 & 3414 & 3415 & 3415\\
 &  &  &  &  &  \vphantom{4} & \\
\addlinespace[0.3em]
\multicolumn{7}{l}{\textbf{Panel B : Collèges publics HEP}}\\
\hline
\hspace{1em}Note contemporaine des pairs & 3.316$^{***}$ & 6.629$^{***}$ & 3.061$^{***}$ & 3.228$^{***}$ & 0.374 & 0.923\\
\hspace{1em} & (0.737) & (1.19) & (0.886) & (0.925) & (0.438) & (0.598)\\
\hspace{1em} &  &  &  &  &  \vphantom{3} & \\
\hspace{1em}Contrôles individuels & Oui & Oui & Oui & Oui & Oui & \vphantom{1} Oui\\
\hspace{1em}Contrôles chez les pairs & Oui & Oui & Oui & Oui & Oui & \vphantom{1} Oui\\
\hspace{1em}Contrôles potentiellement endogènes & Non & Oui & Non & Oui & Non & \vphantom{1} Oui\\
\hspace{1em}Observations & 18118 & 18118 & 18133 & 18133 & 18119 & 18119\\
 &  &  &  &  &  \vphantom{2} & \\
\addlinespace[0.3em]
\multicolumn{7}{l}{\textbf{Panel C : Collèges publics EP}}\\
\hline
\hspace{1em}Note contemporaine des pairs & 5.024$^{***}$ & 8.509$^{***}$ & 3.147$^{***}$ & 3.536$^{***}$ & 1.428$^{***}$ & 1.938$^{***}$\\
\hspace{1em} & (0.891) & (1.326) & (0.769) & (0.819) & (0.524) & (0.609)\\
\hspace{1em} &  &  &  &  &  \vphantom{1} & \\
\hspace{1em}Contrôles individuels & Oui & Oui & Oui & Oui & Oui & Oui\\
\hspace{1em}Contrôles chez les pairs & Oui & Oui & Oui & Oui & Oui & Oui\\
\hspace{1em}Contrôles potentiellement endogènes & Non & Oui & Non & Oui & Non & Oui\\
\hspace{1em}Observations & 19930 & 19930 & 19931 & 19931 & 19910 & 19910\\
 &  &  &  &  &  & \\*
\end{longtable}
\end{ThreePartTable}
\endgroup{}
\elandscape

\setcounter{table}{0}
\setcounter{figure}{0}

\hypertarget{robustesse-des-coefficients-deffets-de-pairs-endoguxe8nes-en-fonction-des-variables-explicatives}{%
\section{Robustesse des coefficients d'effets de pairs endogènes en fonction des variables explicatives}\label{robustesse-des-coefficients-deffets-de-pairs-endoguxe8nes-en-fonction-des-variables-explicatives}}

\hypertarget{pepcmlmodelscm2robcov}{%
\subsection{Au CM2}\label{pepcmlmodelscm2robcov}}

\begingroup\fontsize{8}{10}\selectfont

\begin{ThreePartTable}
\begin{TableNotes}
\item \textit{Sources :} Fichiers CM2 (2010 à 2012), calculs de l'auteur.
\item \textit{Notes :} Chaque coefficient provient d'une régression différente. Méthode d'estimation : maximum de vraisemblance conditionnelle (Boucher et al., 2014). Contrôles 1 : sexe et catégorie socio-professionnelles. Contrôles 2 : Contrôles 1, âge aux évaluations nationales du CM2. Contrôles 3 : Contrôles 2, position (en retard, à l'heure, en avance) au CM2. Contrôles individuels et chez les pairs utilisés. (H)EP : (Hors-)Éducation Prioritaire.
\item Significativité : 10 \% * 5\% ** 1\% ****.
\end{TableNotes}
\begin{longtable}[t]{llll}
\caption{\label{tab:pepcmlmodelscm2robcov}Robustesse des effets de pairs endogènes au CM2 aux variables de contrôles (maximum de vraisemblance)}\\
\toprule
\multicolumn{1}{c}{ } & \multicolumn{3}{c}{Variable dépendante : Note} \\
\cmidrule(l{3pt}r{3pt}){2-4}
  & \makecell{Totale \\ (1) } & \makecell{En français \\ (2) } & \makecell{En mathématiques \\ (3) }\\
\midrule
\endfirsthead
\caption[]{\label{tab:pepcmlmodelscm2robcov}Robustesse des effets de pairs endogènes au CM2 aux variables de contrôles (maximum de vraisemblance) (suite)}\\
\toprule
  & \makecell{Totale \\ (1) } & \makecell{En français \\ (2) } & \makecell{En mathématiques \\ (3) }\\
\midrule
\endhead

\endfoot
\bottomrule
\insertTableNotes
\endlastfoot
\addlinespace[0.3em]
\multicolumn{4}{l}{\textbf{Tout l'échantillon}}\\
\hline
\hspace{1em}Contrôles 1 & 1.466$^{***}$ & 1.288$^{***}$ & 1.306$^{***}$\\
\hspace{1em}Contrôles 2 & 1.423$^{***}$ & 1.248$^{***}$ & 1.261$^{***}$\\
\hspace{1em}Contrôles 3 & 1.355$^{***}$ & 1.202$^{***}$ & 1.169$^{***}$\\
\addlinespace[0.3em]
\multicolumn{4}{l}{\textbf{Écoles privées}}\\
\hline
\hspace{1em}Contrôles 1 & 2.268 & 1.23 & 3.235$^{**}$\\
\hspace{1em}Contrôles 2 & 2.598$^{*}$ & 1.597 & 3.489$^{**}$\\
\hspace{1em}Contrôles 3 & 2.774$^{*}$ & 1.824 & 3.724$^{**}$\\
\addlinespace[0.3em]
\multicolumn{4}{l}{\textbf{Écoles publiques HEP}}\\
\hline
\hspace{1em}Contrôles 1 & 1.423$^{***}$ & 1.351$^{***}$ & 1.107$^{***}$\\
\hspace{1em}Contrôles 2 & 1.365$^{***}$ & 1.284$^{***}$ & 1.061$^{***}$\\
\hspace{1em}Contrôles 3 & 1.311$^{***}$ & 1.288$^{***}$ & 0.949$^{***}$\\
\addlinespace[0.3em]
\multicolumn{4}{l}{\textbf{Écoles publiques EP}}\\
\hline
\hspace{1em}Contrôles 1 & 1.788$^{***}$ & 1.572$^{***}$ & 1.659$^{***}$\\
\hspace{1em}Contrôles 2 & 1.677$^{***}$ & 1.472$^{***}$ & 1.568$^{***}$\\
\hspace{1em}Contrôles 3 & 1.605$^{***}$ & 1.388$^{***}$ & 1.528$^{***}$\\*
\end{longtable}
\end{ThreePartTable}
\endgroup{}

\newpage

\hypertarget{pepcmlmodelsdnbrobcov}{%
\subsection{Au DNB}\label{pepcmlmodelsdnbrobcov}}

\begingroup\fontsize{8}{10}\selectfont

\begin{ThreePartTable}
\begin{TableNotes}
\item \textit{Sources :} Fichiers DNB (2014 à 2015), fichiers CONSTAT (2013 à 2016), calculs de l'auteur.
\item \textit{Notes :} Chaque coefficient provient d'une régression différente. Méthode d'estimation : maximum de vraisemblance conditionnelle (Boucher et al., 2014). Contrôles 1 : sexe et catégorie socio-professionnelles. Contrôles 2 : Contrôles 1, régime scolaire. Contrôles 3 : Contrôles 2, âge aux épreuves du DNB. Contrôles 4 : Contrôles 3, position (en retard, à l'heure, en avance) en 3ème. Contrôles individuels et chez les pairs utilisés. (H)EP : (Hors-)Éducation Prioritaire.
\item Significativité : 10 \% * 5\% ** 1\% ****.
\end{TableNotes}
\begin{longtable}[t]{lllllll}
\caption{\label{tab:pepcmlmodelsrobcov}Robustesse des effets de pairs endogènes au DNB aux variables de contrôles (maximum de vraisemblance)}\\
\toprule
\multicolumn{1}{c}{ } & \multicolumn{6}{c}{Variable dépendante : Note aux écrits de} \\
\cmidrule(l{3pt}r{3pt}){2-7}
  & \makecell{Toutes matières \\ (1) } & \makecell{Français \\ (2) } & \makecell{Mathématiques \\ (3) } & \makecell{Histoire-et-géographie \\ (4) } & \makecell{Dictée \\ (5) } & \makecell{Rédaction \\ (6) }\\
\midrule
\endfirsthead
\caption[]{\label{tab:pepcmlmodelsrobcov}Robustesse des effets de pairs endogènes au DNB aux variables de contrôles (maximum de vraisemblance) (suite)}\\
\toprule
  & \makecell{Toutes matières \\ (1) } & \makecell{Français \\ (2) } & \makecell{Mathématiques \\ (3) } & \makecell{Histoire-et-géographie \\ (4) } & \makecell{Dictée \\ (5) } & \makecell{Rédaction \\ (6) }\\
\midrule
\endhead

\endfoot
\bottomrule
\insertTableNotes
\endlastfoot
\addlinespace[0.3em]
\multicolumn{7}{l}{\textbf{Tout l'échantillon}}\\
\hline
\hspace{1em}Contrôles 1 & 2.771$^{***}$ & 0.562$^{*}$ & 2.986$^{***}$ & 2.57$^{***}$ & 0.846$^{**}$ & 0.164\\
\hspace{1em}Contrôles 2 & 2.825$^{***}$ & 0.553$^{*}$ & 2.982$^{***}$ & 2.699$^{***}$ & 0.847$^{**}$ & 0.157\\
\hspace{1em}Contrôles 3 & 2.954$^{***}$ & 0.68$^{**}$ & 2.928$^{***}$ & 2.727$^{***}$ & 1.003$^{***}$ & 0.228\\
\hspace{1em}Contrôles 4 & 3.354$^{***}$ & 0.803$^{**}$ & 2.854$^{***}$ & 4.471$^{***}$ & 1.061$^{***}$ & 0.315\\
\addlinespace[0.3em]
\multicolumn{7}{l}{\textbf{Collèges privés}}\\
\hline
\hspace{1em}Contrôles 1 & 7.197$^{*}$ & 1.5 & 1.224 & 9.776$^{**}$ & 1.331 & 2.189\\
\hspace{1em}Contrôles 2 & 7.231$^{*}$ & 1.353 & 1.302 & 10.298$^{**}$ & 1.444 & 2.102\\
\hspace{1em}Contrôles 3 & 6.463 & 1.32 & 1.089 & 8.489$^{*}$ & 1.586 & 1.948\\
\hspace{1em}Contrôles 4 & 6.575 & 0.808 & 0.992 & 8.153$^{*}$ & 1.116 & 1.787\\
\addlinespace[0.3em]
\multicolumn{7}{l}{\textbf{Collèges publics HEP}}\\
\hline
\hspace{1em}Contrôles 1 & 3.983$^{***}$ & 1.689$^{***}$ & 1.515$^{***}$ & 3.358$^{***}$ & 2.974$^{***}$ & 0.384\\
\hspace{1em}Contrôles 2 & 3.913$^{***}$ & 1.671$^{***}$ & 1.576$^{***}$ & 3.316$^{***}$ & 3.061$^{***}$ & 0.374\\
\hspace{1em}Contrôles 3 & 4.067$^{***}$ & 2.55$^{***}$ & 1.585$^{***}$ & 5.622$^{***}$ & 3.279$^{***}$ & 0.906\\
\hspace{1em}Contrôles 4 & 3.7$^{***}$ & 2.385$^{***}$ & 1.413$^{***}$ & 6.629$^{***}$ & 3.228$^{***}$ & 0.923\\
\addlinespace[0.3em]
\multicolumn{7}{l}{\textbf{Collèges publics EP}}\\
\hline
\hspace{1em}Contrôles 1 & 6.689$^{***}$ & 2.831$^{***}$ & 6.862$^{***}$ & 5.006$^{***}$ & 3.134$^{***}$ & 1.467$^{***}$\\
\hspace{1em}Contrôles 2 & 6.624$^{***}$ & 2.748$^{***}$ & 6.779$^{***}$ & 5.024$^{***}$ & 3.147$^{***}$ & 1.428$^{***}$\\
\hspace{1em}Contrôles 3 & 6.447$^{***}$ & 2.665$^{***}$ & 6.284$^{***}$ & 5.527$^{***}$ & 3.15$^{***}$ & 1.301$^{**}$\\
\hspace{1em}Contrôles 4 & 8.185$^{***}$ & 3.55$^{***}$ & 6.185$^{***}$ & 8.509$^{***}$ & 3.536$^{***}$ & 1.938$^{***}$\\*
\end{longtable}
\end{ThreePartTable}
\endgroup{}

\hypertarget{annexes-au-chapitre-refg20}{%
\chapter*{Annexes au Chapitre \ref{g20}}\label{annexes-au-chapitre-refg20}}
\addcontentsline{toc}{chapter}{Annexes au Chapitre \ref{g20}}

\setcounter{section}{0}

\renewcommand*{\theHchapter}{\thechapter}
\renewcommand*{\thesection}{\Alph{section}}
\renewcommand*{\theHsection}{Appendix.\thechapter\thesection}

\renewcommand*{\thetable}{\Alph{section}.\arabic{table}}
\renewcommand*{\theHtable}{Appendix.\thetable}

\renewcommand*{\thefigure}{\Alph{section}.\arabic{figure}}
\renewcommand*{\theHfigure}{Appendix.\thefigure}

\setcounter{table}{0}
\setcounter{figure}{0}

\hypertarget{g20age26aoutnotescompl}{%
\section{Distribution de l'âge à la rentrée et des notes des néo-bacheliers}\label{g20age26aoutnotescompl}}

\begin{figure}[H]

{\centering \includegraphics[width=1\linewidth]{000_files/figure-latex/g20age26aoutcompl-1} 

}

\caption{Distributions de l'âge à la rentrée des néo-bacheliers}\label{fig:g20age26aoutcompl}
\end{figure}

\begin{figure}[H]

{\centering \includegraphics[width=1\linewidth]{000_files/figure-latex/g20notes-1} 

}

\caption{Distribution des notes des néo-bacheliers}\label{fig:g20notes}
\end{figure}

\newpage

\newpage

\setcounter{table}{0}
\setcounter{figure}{0}

\hypertarget{g20compintinsc}{%
\section{Comparaison entre les intéressés et les inscrits}\label{g20compintinsc}}

\begingroup\fontsize{6}{8}\selectfont

\begin{ThreePartTable}
\begin{TableNotes}
\item \textit{Sources :} Base Parcoursup - L1 AES et Eco-Ge, plateforme '\textit{J'ai 20 en maths}', APOGEE (version extraite en mai 2021), calculs de l'auteur.
\item \textit{Notes :} Moyennes et proportions. Écart-types entre parenthèses. Les proportions de valeurs manquantes ne sont pas montrées dans ce tableau puisqu'elles sont largement négligeables. Les probabilités critiques correspondent à des tests de comparaison de moyennes pour les variables quantitatives et à des tests de comparaison de proportions pour les variables qualitatives. ES : Économique et social, S : Scientifique, AES : Administration Économique et Sociale. Eco-Ge : Économie-Gestion.
\item Significativité : 10\% * 5\% ** 1\% ***.
\end{TableNotes}
\begin{longtable}[t]{llll}
\caption{\label{tab:g20compintinsc}Comparaison entre les étudiants inscrits et non-inscrits (parmi les intéressés)}\\
\toprule
  & \makecell{\makecell{Non-inscrits \\ \ } \\ (1) } & \makecell{\makecell{Inscrits \\ \ } \\ (2) } & \makecell{\makecell{(1) = (2) \\ probabilité critique} \\ (3) }\\
\midrule
\endfirsthead
\caption[]{\label{tab:g20compintinsc}Comparaison entre les étudiants inscrits et non-inscrits (parmi les intéressés) (suite)}\\
\toprule
  & \makecell{\makecell{Non-inscrits \\ \ } \\ (1) } & \makecell{\makecell{Inscrits \\ \ } \\ (2) } & \makecell{\makecell{(1) = (2) \\ probabilité critique} \\ (3) }\\
\midrule
\endhead

\endfoot
\bottomrule
\insertTableNotes
\endlastfoot
\addlinespace[0.3em]
\multicolumn{4}{l}{\textbf{Note au bac}}\\
\hspace{1em}Totale & 11.82 (1.36) & 12.01 (1.62) & 0.48\\
\hspace{1em}En français (écrits) & 9.97 (2.83) & 10.31 (3.34) & 0.59\\
\hspace{1em}En mathématiques & 12.12 (2.98) & 11.32 (3.3) & 0.16\\
\addlinespace[0.3em]
\multicolumn{4}{l}{\textbf{Série au bac}}\\
\hspace{1em}Pro & 0.2 & 0.14 & 0 ***\\
\hspace{1em}Techno & 0.37 & 0.27 & \\
\hspace{1em}ES & 0.24 & 0.42 & \\
\hspace{1em}S & 0.04 & 0.14 & \\
\hspace{1em}Autre & 0.15 & 0.03 & \\
\addlinespace[0.3em]
\multicolumn{4}{l}{\textbf{Mention au bac}}\\
\hspace{1em}Aucune & 0.61 & 0.5 & 0.53\\
\hspace{1em}Assez bien & 0.32 & 0.35 & \\
\hspace{1em}Bien & 0.07 & 0.13 & \\
\hspace{1em}Très bien & 0 & 0.02 & \\
\addlinespace[0.3em]
\multicolumn{4}{l}{\textbf{ }}\\
\hspace{1em}Âge à la rentrée & 19.96 (5.35) & 18.52 (0.86) & 0***\\
\addlinespace[0.3em]
\multicolumn{4}{l}{\textbf{Sexe}}\\
\hspace{1em}Fille & 0.7 & 0.7 & 1\\
\hspace{1em}Homme & 0.3 & 0.3 & \\
\addlinespace[0.3em]
\multicolumn{4}{l}{\textbf{Pays de naissance}}\\
\hspace{1em}À l'étranger & 0.17 & 0.1 & 0.19\\
\hspace{1em}En France & 0.83 & 0.9 & \\
\addlinespace[0.3em]
\multicolumn{4}{l}{\textbf{Nationalité}}\\
\hspace{1em}Hors union européenne & 0.11 & 0.05 & 0.16\\
\hspace{1em}Française & 0.89 & 0.95 & \\
\addlinespace[0.3em]
\multicolumn{4}{l}{\textbf{Statut de boursier}}\\
\hspace{1em}Non boursier & 0.35 & 0.36 & 0.99\\
\hspace{1em}Boursier du secondaire & 0.65 & 0.64 & \\
\addlinespace[0.3em]
\multicolumn{4}{l}{\textbf{Statut de l'établissement d'origine}}\\
\hspace{1em}Public & 0.95 & 0.95 & 1\\
\hspace{1em}Privé & 0.05 & 0.05 & \\
\addlinespace[0.3em]
\multicolumn{4}{l}{\textbf{Type de l'établissement d'origine}}\\
\hspace{1em}Sans postbac & 0.05 & 0.05 & 1\\
\hspace{1em}Avec postbac & 0.95 & 0.95 & \\
\addlinespace[0.3em]
\multicolumn{4}{l}{\textbf{Département de l'établissement d'origine}}\\
\hspace{1em}La Réunion & 0.69 & 0.79 & 0.25\\
\hspace{1em}France hors Réunion & 0.27 & 0.18 & \\
\hspace{1em}Étranger & 0.04 & 0.03 & \\
\addlinespace[0.3em]
\multicolumn{4}{l}{\textbf{Campus}}\\
\hspace{1em}Tampon & 0.37 & 0.4 & 0.86\\
\hspace{1em}Saint-Denis & 0.63 & 0.6 & \\
\addlinespace[0.3em]
\multicolumn{4}{l}{\textbf{Filière}}\\
\hspace{1em}AES & 0.83 & 0.74 & 0.3\\
\hspace{1em}Eco-Ge & 0.17 & 0.26 & \\
\addlinespace[0.3em]
\multicolumn{4}{l}{\textbf{ }}\\
\hspace{1em}Observations & 46 & 260 & \\*
\end{longtable}
\end{ThreePartTable}
\endgroup{}

\setcounter{table}{0}
\setcounter{figure}{0}

\hypertarget{g20compintinscz0z1}{%
\section{Comparaison des néo-bacheliers intéressés entre eux et des néo-bacheliers inscrits entre eux en fonction de l'incitation}\label{g20compintinscz0z1}}

\begingroup\fontsize{6}{8}\selectfont

\begin{ThreePartTable}
\begin{TableNotes}
\item \textit{Sources :} Base Parcoursup - L1 AES et Eco-Ge, plateforme '\textit{J'ai 20 en maths}', APOGEE (version extraite en mai 2021), calculs de l'auteur.
\item \textit{Notes :} Moyennes et proportions. Écart-types entre parenthèses. Les probabilités critiques correspondent à des tests de comparaison de moyennes pour les variables quantitatives et à des tests de comparaison de proportions pour les variables qualitatives. ES : Économique et social, S : Scientifique, AES : Administration Économique et Sociale. Eco-Ge : Économie-Gestion.
\item Significativité : 10\% * 5\% ** 1\% ***.
\end{TableNotes}
\begin{longtable}[t]{llllll}
\caption{\label{tab:g20compintz0z1}Comparaison entre les étudiants non incités et incités (intéressés, néo-bacheliers)}\\
\toprule
\multicolumn{1}{c}{ } & \multicolumn{2}{c}{Non incités} & \multicolumn{2}{c}{Incités} & \multicolumn{1}{c}{ } \\
\cmidrule(l{3pt}r{3pt}){2-3} \cmidrule(l{3pt}r{3pt}){4-5}
  & \makecell{\makecell{Moyenne \\ (Écart-type)} \\ (1) } & \makecell{\makecell{Proportion \\ manquantes} \\ (2) } & \makecell{\makecell{Moyenne \\ (Écart-type)} \\ (3) } & \makecell{\makecell{Proportion \\ manquantes} \\ (4) } & \makecell{\makecell{(1) = (3) \\ probabilité critique} \\ (5) }\\
\midrule
\endfirsthead
\caption[]{\label{tab:g20compintz0z1}Comparaison entre les étudiants non incités et incités (intéressés, néo-bacheliers) (suite)}\\
\toprule
  & \makecell{\makecell{Moyenne \\ (Écart-type)} \\ (1) } & \makecell{\makecell{Proportion \\ manquantes} \\ (2) } & \makecell{\makecell{Moyenne \\ (Écart-type)} \\ (3) } & \makecell{\makecell{Proportion \\ manquantes} \\ (4) } & \makecell{\makecell{(1) = (3) \\ probabilité critique} \\ (5) }\\
\midrule
\endhead

\endfoot
\bottomrule
\insertTableNotes
\endlastfoot
\addlinespace[0.3em]
\multicolumn{6}{l}{\textbf{Note au bac}}\\
\hspace{1em}Totale & 11.99 (1.53) & 0.02 & 11.99 (1.65) & 0.05 & 0.98\\
\hspace{1em}En français (écrits) & 10.04 (3.12) & 0.2 & 10.53 (3.44) & 0.21 & 0.25\\
\hspace{1em}En mathématiques & 11.34 (3.29) & 0.04 & 11.52 (3.24) & 0.1 & 0.65\\
\addlinespace[0.3em]
\multicolumn{6}{l}{\textbf{Série au bac}}\\
\hspace{1em}Pro & 0.15 & 0 & 0.15 & 0 & 0.92\\
\hspace{1em}Techno & 0.27 &  & 0.29 &  & \\
\hspace{1em}ES & 0.4 &  & 0.38 &  & \\
\hspace{1em}S & 0.14 &  & 0.12 &  & \\
\hspace{1em}Autre & 0.04 &  & 0.06 &  & \\
\addlinespace[0.3em]
\multicolumn{6}{l}{\textbf{Mention au bac}}\\
\hspace{1em}Aucune & 0.49 & 0.02 & 0.54 & 0.01 & 0.54\\
\hspace{1em}Assez bien & 0.38 &  & 0.31 &  & \\
\hspace{1em}Bien & 0.12 &  & 0.13 &  & \\
\hspace{1em}Très bien & 0.01 &  & 0.03 &  & \\
\addlinespace[0.3em]
\multicolumn{6}{l}{\textbf{ }}\\
\hspace{1em}Âge à la rentrée & 18.78 (2.9) & 0 & 18.69 (1.2) & 0 & 0.74\\
\addlinespace[0.3em]
\multicolumn{6}{l}{\textbf{Sexe}}\\
\hspace{1em}Fille & 0.7 & 0 & 0.7 & 0 & 1\\
\hspace{1em}Homme & 0.3 &  & 0.3 &  & \\
\addlinespace[0.3em]
\multicolumn{6}{l}{\textbf{Pays de naissance}}\\
\hspace{1em}À l'étranger & 0.12 & 0 & 0.1 & 0 & 0.7\\
\hspace{1em}En France & 0.88 &  & 0.9 &  & \\
\addlinespace[0.3em]
\multicolumn{6}{l}{\textbf{Nationalité}}\\
\hspace{1em}Hors union européenne & 0.06 & 0 & 0.06 & 0 & 0.99\\
\hspace{1em}Française & 0.94 &  & 0.94 &  & \\
\addlinespace[0.3em]
\multicolumn{6}{l}{\textbf{Statut de boursier}}\\
\hspace{1em}Non boursier & 0.35 & 0 & 0.37 & 0 & 0.86\\
\hspace{1em}Boursier du secondaire & 0.65 &  & 0.63 &  & \\
\addlinespace[0.3em]
\multicolumn{6}{l}{\textbf{Statut de l'établissement d'origine}}\\
\hspace{1em}Public & 0.95 & 0.01 & 0.94 & 0.01 & 1\\
\hspace{1em}Privé & 0.05 &  & 0.06 &  & \\
\addlinespace[0.3em]
\multicolumn{6}{l}{\textbf{Type de l'établissement d'origine}}\\
\hspace{1em}Sans postbac & 0.06 & 0.01 & 0.04 & 0.01 & 0.77\\
\hspace{1em}Avec postbac & 0.94 &  & 0.96 &  & \\
\addlinespace[0.3em]
\multicolumn{6}{l}{\textbf{Département de l'établissement d'origine}}\\
\hspace{1em}La Réunion & 0.8 & 0.01 & 0.75 & 0 & 0.58\\
\hspace{1em}France hors Réunion & 0.17 &  & 0.22 &  & \\
\hspace{1em}Étranger & 0.03 &  & 0.03 &  & \\
\addlinespace[0.3em]
\multicolumn{6}{l}{\textbf{Campus}}\\
\hspace{1em}Tampon & 0.4 & 0 & 0.39 & 0 & 1\\
\hspace{1em}Saint-Denis & 0.6 &  & 0.61 &  & \\
\addlinespace[0.3em]
\multicolumn{6}{l}{\textbf{Filière}}\\
\hspace{1em}AES & 0.74 & 0 & 0.77 & 0 & 0.63\\
\hspace{1em}Eco-Ge & 0.26 &  & 0.23 &  & \\
\addlinespace[0.3em]
\multicolumn{6}{l}{\textbf{ }}\\
\hspace{1em}Observations & 162 &  & 144 &  & \\*
\end{longtable}
\end{ThreePartTable}
\endgroup{}

\begingroup\fontsize{5}{7}\selectfont

\begin{ThreePartTable}
\begin{TableNotes}
\item \textit{Sources :} Base Parcoursup - L1 AES et Eco-Ge, plateforme '\textit{J'ai 20 en maths}', APOGEE (version extraite en mai 2021), calculs de l'auteur.
\item \textit{Notes :} Moyennes et proportions. Écart-types entre parenthèses. Les proportions de valeurs manquantes ne sont pas montrées puisqu'il n'y a que la note au bac qui contient 2 valeurs manquantes pour tous les inscrits. Les probabilités critiques correspondent à des tests de comparaison de moyennes pour les variables quantitatives et à des tests de comparaison de proportions pour les variables qualitatives. ES : Économique et social, S : Scientifique, AES : Administration Économique et Sociale. Eco-Ge : Économie-Gestion.
\item Significativité : 10\% * 5\% ** 1\% ***.
\end{TableNotes}
\begin{longtable}[t]{llll}
\caption{\label{tab:g20compinscz0z1}Comparaison entre les étudiants non incités et incités (inscrits)}\\
\toprule
  & \makecell{\makecell{Non incités \\ \ } \\ (1) } & \makecell{\makecell{Incités \\ \ } \\ (2) } & \makecell{\makecell{(1) = (3) \\ probabilité critique} \\ (3) }\\
\midrule
\endfirsthead
\caption[]{\label{tab:g20compinscz0z1}Comparaison entre les étudiants non incités et incités (inscrits) (suite)}\\
\toprule
  & \makecell{\makecell{Non incités \\ \ } \\ (1) } & \makecell{\makecell{Incités \\ \ } \\ (2) } & \makecell{\makecell{(1) = (3) \\ probabilité critique} \\ (3) }\\
\midrule
\endhead

\endfoot
\bottomrule
\insertTableNotes
\endlastfoot
\addlinespace[0.3em]
\multicolumn{4}{l}{\textbf{Note au bac}}\\
\hspace{1em}Totale & 12.03 (1.56) & 11.97 (1.69) & 0.77\\
\hspace{1em}En français (écrits) & 10.03 (3.24) & 10.6 (3.46) & 0.21\\
\hspace{1em}En mathématiques & 11.18 (3.36) & 11.41 (3.16) & 0.59\\
\addlinespace[0.3em]
\multicolumn{4}{l}{\textbf{Série au bac}}\\
\hspace{1em}Pro & 0.16 & 0.12 & 0.47\\
\hspace{1em}Techno & 0.24 & 0.29 & \\
\hspace{1em}ES & 0.41 & 0.42 & \\
\hspace{1em}S & 0.16 & 0.12 & \\
\hspace{1em}Autre & 0.02 & 0.05 & \\
\addlinespace[0.3em]
\multicolumn{4}{l}{\textbf{Mention au bac}}\\
\hspace{1em}Aucune & 0.48 & 0.52 & 0.59\\
\hspace{1em}Assez bien & 0.38 & 0.32 & \\
\hspace{1em}Bien & 0.13 & 0.12 & \\
\hspace{1em}Très bien & 0.01 & 0.03 & \\
\addlinespace[0.3em]
\multicolumn{4}{l}{\textbf{ }}\\
\hspace{1em}Âge à la rentrée & 18.45 (0.7) & 18.6 (1) & 0.16\\
\addlinespace[0.3em]
\multicolumn{4}{l}{\textbf{Sexe}}\\
\hspace{1em}Fille & 0.68 & 0.73 & 0.41\\
\hspace{1em}Homme & 0.32 & 0.27 & \\
\addlinespace[0.3em]
\multicolumn{4}{l}{\textbf{Pays de naissance}}\\
\hspace{1em}À l'étranger & 0.1 & 0.09 & 0.99\\
\hspace{1em}En France & 0.9 & 0.91 & \\
\addlinespace[0.3em]
\multicolumn{4}{l}{\textbf{Nationalité}}\\
\hspace{1em}Hors union européenne & 0.04 & 0.06 & 0.78\\
\hspace{1em}Française & 0.96 & 0.94 & \\
\addlinespace[0.3em]
\multicolumn{4}{l}{\textbf{Statut de boursier}}\\
\hspace{1em}Non boursier & 0.36 & 0.38 & 0.87\\
\hspace{1em}Boursier du secondaire & 0.64 & 0.62 & \\
\addlinespace[0.3em]
\multicolumn{4}{l}{\textbf{Statut de l'établissement d'origine}}\\
\hspace{1em}Public & 0.94 & 0.95 & 1\\
\hspace{1em}Privé & 0.06 & 0.05 & \\
\addlinespace[0.3em]
\multicolumn{4}{l}{\textbf{Type de l'établissement d'origine}}\\
\hspace{1em}Sans postbac & 0.06 & 0.04 & 0.78\\
\hspace{1em}Avec postbac & 0.94 & 0.96 & \\
\addlinespace[0.3em]
\multicolumn{4}{l}{\textbf{Département de l'établissement d'origine}}\\
\hspace{1em}La Réunion & 0.81 & 0.78 & 0.41\\
\hspace{1em}France hors Réunion & 0.16 & 0.21 & \\
\hspace{1em}Étranger & 0.04 & 0.02 & \\
\addlinespace[0.3em]
\multicolumn{4}{l}{\textbf{Campus}}\\
\hspace{1em}Tampon & 0.39 & 0.41 & 0.81\\
\hspace{1em}Saint-Denis & 0.61 & 0.59 & \\
\addlinespace[0.3em]
\multicolumn{4}{l}{\textbf{Filière}}\\
\hspace{1em}AES & 0.72 & 0.77 & 0.49\\
\hspace{1em}Eco-Ge & 0.28 & 0.23 & \\
\addlinespace[0.3em]
\multicolumn{4}{l}{\textbf{Plateforme}}\\
\hspace{1em}Connecté & 0.74 (0.44) & 0.88 (0.33) & 0.01***\\
\hspace{1em}Vidéos (complètes) ou exercices $\geq 1$ & 0.61 (0.49) & 0.73 (0.44) & 0.03**\\
\hspace{1em}Vidéos (complètes) ou exercices & 9.43 (20.63) & 16.87 (34.98) & 0.03**\\
\hspace{1em}Vidéos ou exercices $\geq 1$ & 0.61 (0.49) & 0.74 (0.44) & 0.02**\\
\hspace{1em}Vidéos ou exercices & 13.42 (30.41) & 21.37 (40.48) & 0.07*\\
\hspace{1em}Vidéos (complètes) $\geq 1$ & 0.09 (0.28) & 0.17 (0.38) & 0.03**\\
\hspace{1em}Vidéos (complètes) & 0.3 (1.14) & 1.38 (4.41) & 0.01***\\
\hspace{1em}Vidéos $\geq 1$ & 0.27 (0.45) & 0.42 (0.5) & 0.01**\\
\hspace{1em}Vidéos & 4.29 (11.72) & 5.88 (12.6) & 0.3\\
\hspace{1em}Exercices $\geq 1$ & 0.61 (0.49) & 0.73 (0.44) & 0.03**\\
\hspace{1em}Exercices & 9.13 (20.23) & 15.49 (32.64) & 0.06*\\
\addlinespace[0.3em]
\multicolumn{4}{l}{\textbf{ }}\\
\hspace{1em}Observations & 140 & 120 & \\*
\end{longtable}
\end{ThreePartTable}
\endgroup{}

\newpage

\setcounter{table}{0}
\setcounter{figure}{0}

\hypertarget{g20compinscvenutd0venutd1}{%
\section{Comparaison entre les inscrits ayant une note aux TD ou non}\label{g20compinscvenutd0venutd1}}

\begingroup\fontsize{5}{7}\selectfont

\begin{ThreePartTable}
\begin{TableNotes}
\item \textit{Sources :} Base Parcoursup - L1 AES et Eco-Ge, plateforme '\textit{J'ai 20 en maths}', APOGEE (version extraite en mai 2021), calculs de l'auteur.
\item \textit{Notes :} Moyennes et proportions. Écart-types entre parenthèses. Les proportions de valeurs manquantes ne sont pas montrées dans ce tableau puisqu'elles sont largement négligeables. Les probabilités critiques correspondent à des tests de comparaison de moyennes pour les variables quantitatives et à des tests de comparaison de proportions pour les variables qualitatives. ES : Économique et social, S : Scientifique, AES : Administration Économique et Sociale. Eco-Ge : Économie-Gestion.
\item Significativité : 10\% * 5\% ** 1\% ***.
\end{TableNotes}
\begin{longtable}[t]{llll}
\caption{\label{tab:g20compinscvenutd0venutd1}Comparaison entre les étudiants ayant une note aux TD ou non (parmi les inscrits)}\\
\toprule
  & \makecell{\makecell{N'ayant pas une note aux TD \\ \ } \\ (1) } & \makecell{\makecell{Ayant une note aux TD \\ \ } \\ (2) } & \makecell{\makecell{(1) = (2) \\ probabilité critique} \\ (3) }\\
\midrule
\endfirsthead
\caption[]{\label{tab:g20compinscvenutd0venutd1}Comparaison entre les étudiants ayant une note aux TD ou non (parmi les inscrits) (suite)}\\
\toprule
  & \makecell{\makecell{N'ayant pas une note aux TD \\ \ } \\ (1) } & \makecell{\makecell{Ayant une note aux TD \\ \ } \\ (2) } & \makecell{\makecell{(1) = (2) \\ probabilité critique} \\ (3) }\\
\midrule
\endhead

\endfoot
\bottomrule
\insertTableNotes
\endlastfoot
\addlinespace[0.3em]
\multicolumn{4}{l}{\textbf{ }}\\
\hspace{1em}Incité & 0.5 (0.51) & 0.45 (0.5) & 0.59\\
\addlinespace[0.3em]
\multicolumn{4}{l}{\textbf{Note au bac}}\\
\hspace{1em}Totale & 11.39 (1.22) & 12.11 (1.66) & 0.01***\\
\hspace{1em}En français (écrits) & 8.83 (2.7) & 10.48 (3.38) & 0.02**\\
\hspace{1em}En mathématiques & 10.54 (3.02) & 11.42 (3.3) & 0.13\\
\addlinespace[0.3em]
\multicolumn{4}{l}{\textbf{Série au bac}}\\
\hspace{1em}Pro & 0.38 & 0.1 & 0 ***\\
\hspace{1em}Techno & 0.26 & 0.27 & \\
\hspace{1em}ES & 0.29 & 0.44 & \\
\hspace{1em}S & 0.02 & 0.17 & \\
\hspace{1em}Autre & 0.05 & 0.03 & \\
\addlinespace[0.3em]
\multicolumn{4}{l}{\textbf{Mention au bac}}\\
\hspace{1em}Aucune & 0.69 & 0.46 & 0.05 **\\
\hspace{1em}Assez bien & 0.26 & 0.37 & \\
\hspace{1em}Bien & 0.05 & 0.14 & \\
\hspace{1em}Très bien & 0 & 0.03 & \\
\addlinespace[0.3em]
\multicolumn{4}{l}{\textbf{ }}\\
\hspace{1em}Âge à la rentrée & 18.87 (1.46) & 18.45 (0.67) & 0***\\
\addlinespace[0.3em]
\multicolumn{4}{l}{\textbf{Sexe}}\\
\hspace{1em}Fille & 0.71 & 0.7 & 1\\
\hspace{1em}Homme & 0.29 & 0.3 & \\
\addlinespace[0.3em]
\multicolumn{4}{l}{\textbf{Pays de naissance}}\\
\hspace{1em}À l'étranger & 0.12 & 0.09 & 0.57\\
\hspace{1em}En France & 0.88 & 0.91 & \\
\addlinespace[0.3em]
\multicolumn{4}{l}{\textbf{Nationalité}}\\
\hspace{1em}Hors union européenne & 0.1 & 0.04 & 0.24\\
\hspace{1em}Française & 0.9 & 0.96 & \\
\addlinespace[0.3em]
\multicolumn{4}{l}{\textbf{Statut de boursier}}\\
\hspace{1em}Non boursier & 0.4 & 0.36 & 0.69\\
\hspace{1em}Boursier du secondaire & 0.6 & 0.64 & \\
\addlinespace[0.3em]
\multicolumn{4}{l}{\textbf{Statut de l'établissement d'origine}}\\
\hspace{1em}Public & 0.95 & 0.94 & 1\\
\hspace{1em}Privé & 0.05 & 0.06 & \\
\addlinespace[0.3em]
\multicolumn{4}{l}{\textbf{Type de l'établissement d'origine}}\\
\hspace{1em}Sans postbac & 0.07 & 0.05 & 0.45\\
\hspace{1em}Avec postbac & 0.93 & 0.95 & \\
\addlinespace[0.3em]
\multicolumn{4}{l}{\textbf{Département de l'établissement d'origine}}\\
\hspace{1em}La Réunion & 0.76 & 0.8 & 0.81\\
\hspace{1em}France hors Réunion & 0.21 & 0.17 & \\
\hspace{1em}Étranger & 0.02 & 0.03 & \\
\addlinespace[0.3em]
\multicolumn{4}{l}{\textbf{Campus}}\\
\hspace{1em}Tampon & 0.26 & 0.42 & 0.08 *\\
\hspace{1em}Saint-Denis & 0.74 & 0.58 & \\
\addlinespace[0.3em]
\multicolumn{4}{l}{\textbf{Filière}}\\
\hspace{1em}AES & 0.74 & 0.74 & 1\\
\hspace{1em}Eco-Ge & 0.26 & 0.26 & \\
\addlinespace[0.3em]
\multicolumn{4}{l}{\textbf{Plateforme}}\\
\hspace{1em}Connecté & 0.71 (0.46) & 0.82 (0.39) & 0.13\\
\hspace{1em}Vidéos (complètes) ou exercices $\geq 1$ & 0.57 (0.5) & 0.68 (0.47) & 0.16\\
\hspace{1em}Vidéos (complètes) ou exercices & 7.14 (17.6) & 13.96 (29.9) & 0.15\\
\hspace{1em}Vidéos ou exercices $\geq 1$ & 0.57 (0.5) & 0.69 (0.46) & 0.14\\
\hspace{1em}Vidéos ou exercices & 8.88 (22.56) & 18.67 (37.39) & 0.1\\
\hspace{1em}Vidéos (complètes) $\geq 1$ & 0.1 (0.3) & 0.13 (0.34) & 0.5\\
\hspace{1em}Vidéos (complètes) & 0.98 (3.44) & 0.76 (3.09) & 0.69\\
\hspace{1em}Vidéos $\geq 1$ & 0.17 (0.38) & 0.37 (0.48) & 0.01**\\
\hspace{1em}Vidéos & 2.71 (8.81) & 5.47 (12.65) & 0.18\\
\hspace{1em}Exercices $\geq 1$ & 0.57 (0.5) & 0.68 (0.47) & 0.16\\
\hspace{1em}Exercices & 6.17 (14.54) & 13.2 (28.47) & 0.12\\
\addlinespace[0.3em]
\multicolumn{4}{l}{\textbf{ }}\\
\hspace{1em}Observations & 42 & 218 & \\*
\end{longtable}
\end{ThreePartTable}
\endgroup{}

\setcounter{table}{0}
\setcounter{figure}{0}

\hypertarget{g20compinscvenuctqcm0venuctqcm1}{%
\section{Comparaison entre les inscrits ayant une note aux examens ou non}\label{g20compinscvenuctqcm0venuctqcm1}}

\begingroup\fontsize{5}{7}\selectfont

\begin{ThreePartTable}
\begin{TableNotes}
\item \textit{Sources :} Base Parcoursup - L1 AES et Eco-Ge, plateforme '\textit{J'ai 20 en maths}', APOGEE (version extraite en mai 2021), calculs de l'auteur.
\item \textit{Notes :} Moyennes et proportions. Écart-types entre parenthèses. Les proportions de valeurs manquantes ne sont pas montrées dans ce tableau puisqu'elles sont largement négligeables. Les probabilités critiques correspondent à des tests de comparaison de moyennes pour les variables quantitatives et à des tests de comparaison de proportions pour les variables qualitatives. ES : Économique et social, S : Scientifique, AES : Administration Économique et Sociale. Eco-Ge : Économie-Gestion.
\item Significativité : 10\% * 5\% ** 1\% ***.
\end{TableNotes}
\begin{longtable}[t]{llll}
\caption{\label{tab:g20compinscvenuctqcm0venuctqcm1}Comparaison entre les étudiants ayant une note aux examens ou non (parmi les inscrits)}\\
\toprule
  & \makecell{\makecell{N'ayant pas une note aux examens \\ \ } \\ (1) } & \makecell{\makecell{Ayant une note aux examens \\ \ } \\ (2) } & \makecell{\makecell{(1) = (2) \\ probabilité critique} \\ (3) }\\
\midrule
\endfirsthead
\caption[]{\label{tab:g20compinscvenuctqcm0venuctqcm1}Comparaison entre les étudiants ayant une note aux examens ou non (parmi les inscrits) (suite)}\\
\toprule
  & \makecell{\makecell{N'ayant pas une note aux examens \\ \ } \\ (1) } & \makecell{\makecell{Ayant une note aux examens \\ \ } \\ (2) } & \makecell{\makecell{(1) = (2) \\ probabilité critique} \\ (3) }\\
\midrule
\endhead

\endfoot
\bottomrule
\insertTableNotes
\endlastfoot
\addlinespace[0.3em]
\multicolumn{4}{l}{\textbf{ }}\\
\hspace{1em}Incité & 0.49 (0.51) & 0.46 (0.5) & 0.76\\
\addlinespace[0.3em]
\multicolumn{4}{l}{\textbf{Note au bac}}\\
\hspace{1em}Totale & 11.31 (1.22) & 12.11 (1.65) & 0.01***\\
\hspace{1em}En français (écrits) & 10 (3.18) & 10.33 (3.38) & 0.64\\
\hspace{1em}En mathématiques & 10.62 (2.76) & 11.39 (3.33) & 0.2\\
\addlinespace[0.3em]
\multicolumn{4}{l}{\textbf{Série au bac}}\\
\hspace{1em}Pro & 0.26 & 0.12 & 0.1 *\\
\hspace{1em}Techno & 0.34 & 0.25 & \\
\hspace{1em}ES & 0.31 & 0.43 & \\
\hspace{1em}S & 0.06 & 0.16 & \\
\hspace{1em}Autre & 0.03 & 0.04 & \\
\addlinespace[0.3em]
\multicolumn{4}{l}{\textbf{Mention au bac}}\\
\hspace{1em}Aucune & 0.71 & 0.47 & 0.04 **\\
\hspace{1em}Assez bien & 0.26 & 0.36 & \\
\hspace{1em}Bien & 0.03 & 0.14 & \\
\hspace{1em}Très bien & 0 & 0.03 & \\
\addlinespace[0.3em]
\multicolumn{4}{l}{\textbf{ }}\\
\hspace{1em}Âge à la rentrée & 18.51 (0.63) & 18.52 (0.89) & 0.9\\
\addlinespace[0.3em]
\multicolumn{4}{l}{\textbf{Sexe}}\\
\hspace{1em}Fille & 0.74 & 0.7 & 0.73\\
\hspace{1em}Homme & 0.26 & 0.3 & \\
\addlinespace[0.3em]
\multicolumn{4}{l}{\textbf{Pays de naissance}}\\
\hspace{1em}À l'étranger & 0.17 & 0.08 & 0.19\\
\hspace{1em}En France & 0.83 & 0.92 & \\
\addlinespace[0.3em]
\multicolumn{4}{l}{\textbf{Nationalité}}\\
\hspace{1em}Hors union européenne & 0.17 & 0.03 & 0 ***\\
\hspace{1em}Française & 0.83 & 0.97 & \\
\addlinespace[0.3em]
\multicolumn{4}{l}{\textbf{Statut de boursier}}\\
\hspace{1em}Non boursier & 0.51 & 0.34 & 0.08 *\\
\hspace{1em}Boursier du secondaire & 0.49 & 0.66 & \\
\addlinespace[0.3em]
\multicolumn{4}{l}{\textbf{Statut de l'établissement d'origine}}\\
\hspace{1em}Public & 1 & 0.94 & 0.23\\
\hspace{1em}Privé & 0 & 0.06 & \\
\addlinespace[0.3em]
\multicolumn{4}{l}{\textbf{Type de l'établissement d'origine}}\\
\hspace{1em}Sans postbac & 0 & 0.06 & 0.23\\
\hspace{1em}Avec postbac & 1 & 0.94 & \\
\addlinespace[0.3em]
\multicolumn{4}{l}{\textbf{Département de l'établissement d'origine}}\\
\hspace{1em}La Réunion & 0.86 & 0.78 & 0.66\\
\hspace{1em}France hors Réunion & 0.14 & 0.19 & \\
\hspace{1em}Étranger & 0 & 0.03 & \\
\addlinespace[0.3em]
\multicolumn{4}{l}{\textbf{Campus}}\\
\hspace{1em}Tampon & 0.37 & 0.4 & 0.89\\
\hspace{1em}Saint-Denis & 0.63 & 0.6 & \\
\addlinespace[0.3em]
\multicolumn{4}{l}{\textbf{Filière}}\\
\hspace{1em}AES & 0.83 & 0.73 & 0.3\\
\hspace{1em}Eco-Ge & 0.17 & 0.27 & \\
\addlinespace[0.3em]
\multicolumn{4}{l}{\textbf{Plateforme}}\\
\hspace{1em}Connecté & 0.74 (0.44) & 0.81 (0.39) & 0.37\\
\hspace{1em}Vidéos (complètes) ou exercices $\geq 1$ & 0.57 (0.5) & 0.68 (0.47) & 0.21\\
\hspace{1em}Vidéos (complètes) ou exercices & 8.26 (18.94) & 13.58 (29.53) & 0.3\\
\hspace{1em}Vidéos ou exercices $\geq 1$ & 0.57 (0.5) & 0.68 (0.47) & 0.19\\
\hspace{1em}Vidéos ou exercices & 10.63 (24.86) & 18.09 (36.89) & 0.25\\
\hspace{1em}Vidéos (complètes) $\geq 1$ & 0.06 (0.24) & 0.14 (0.35) & 0.18\\
\hspace{1em}Vidéos (complètes) & 0.86 (3.54) & 0.79 (3.09) & 0.9\\
\hspace{1em}Vidéos $\geq 1$ & 0.26 (0.44) & 0.35 (0.48) & 0.28\\
\hspace{1em}Vidéos & 3.23 (9.74) & 5.3 (12.46) & 0.35\\
\hspace{1em}Exercices $\geq 1$ & 0.57 (0.5) & 0.68 (0.47) & 0.21\\
\hspace{1em}Exercices & 7.4 (15.82) & 12.79 (28.1) & 0.27\\
\addlinespace[0.3em]
\multicolumn{4}{l}{\textbf{ }}\\
\hspace{1em}Observations & 35 & 225 & \\*
\end{longtable}
\end{ThreePartTable}
\endgroup{}

\newpage
\setcounter{table}{0}
\setcounter{figure}{0}

\hypertarget{g20mails}{%
\section{Les 4 mails envoyés aux incités}\label{g20mails}}

\begin{figure}[H]

{\centering \includegraphics[width=1\linewidth]{000_files/figure-latex/g20mails-1} 

}

\caption{Les 4 mails envoyés aux incités}\label{fig:g20mails}
\end{figure}

\newpage
\setcounter{table}{0}
\setcounter{figure}{0}

\hypertarget{g20pemodelsinsc}{%
\section{Régressions de première étape pour les inscrits}\label{g20pemodelsinsc}}

\begingroup\fontsize{8}{10}\selectfont

\begin{ThreePartTable}
\begin{TableNotes}
\item \textit{Sources :} Base Parcoursup - L1 AES et Eco-Ge, plateforme '\textit{J'ai 20 en maths}', APOGEE (version extraite en mai 2021), calculs de l'auteur.
\item \textit{Notes :} Moindres carrés ordinaires. Écart-types robustes entre parenthèses. Une colonne correspond à une régression. ES : Économique et social, S : Scientifique, AES : Administration Économique et Sociale. Eco-Ge : Économie-Gestion.
\item Significativité : 10\% * 5\% ** 1\% ***.
\end{TableNotes}
\begin{longtable}[t]{llllll}
\caption{\label{tab:g20pemodelsinsc}Résultats de première étape (Inscrits)}\\
\toprule
\multicolumn{1}{c}{ } & \multicolumn{5}{c}{Variable dépendante : Note } \\
\cmidrule(l{3pt}r{3pt}){2-6}
  & \makecell{Vidéos complètes et exercices \\ (1) } & \makecell{Vidéos et exercices \\ (2) } & \makecell{Vidéos complètes \\ (3) } & \makecell{Vidéos \\ (4) } & \makecell{Exercices \\ (5) }\\
\midrule
\endfirsthead
\caption[]{\label{tab:g20pemodelsinsc}Résultats de première étape (Inscrits) (suite)}\\
\toprule
  & \makecell{Vidéos complètes et exercices \\ (1) } & \makecell{Vidéos et exercices \\ (2) } & \makecell{Vidéos complètes \\ (3) } & \makecell{Vidéos \\ (4) } & \makecell{Exercices \\ (5) }\\
\midrule
\endhead

\endfoot
\bottomrule
\insertTableNotes
\endlastfoot
Constante & $-$4.467 & $-$8.883 & 0.358 & $-$4.058 & $-$4.825\\
 & (44.553) & (57.484) & (4.711) & (19.087) & (42.581)\\
Incitation - Oui & 7.502$^{**}$ & 8.083$^{*}$ & 1.031$^{***}$ & 1.612 & 6.471$^{*}$\\
 & (3.465) & (4.237) & (0.394) & (1.411) & (3.285)\\
Note au bac & 3.176$^{***}$ & 4.598$^{***}$ & 0.148 & 1.571$^{***}$ & 3.027$^{***}$\\
 & (1.022) & (1.306) & (0.131) & (0.497) & (0.986)\\
Série au bac - Techno & 6.139$^{**}$ & 7.648$^{**}$ & 0.606 & 2.115$^{*}$ & 5.533$^{**}$\\
 & (3.044) & (3.654) & (0.377) & (1.082) & (2.758)\\
Série au bac - ES & 10.793$^{***}$ & 16.203$^{***}$ & 1.026$^{**}$ & 6.436$^{***}$ & 9.767$^{***}$\\
 & (2.962) & (4.076) & (0.411) & (1.64) & (2.786)\\
Série au bac - S & 23.464$^{***}$ & 31.323$^{***}$ & 0.554 & 8.414$^{***}$ & 22.909$^{***}$\\
 & (6.813) & (8.36) & (0.5) & (2.261) & (6.681)\\
Série au bac - Autre & 7.734$^{**}$ & 9.773$^{**}$ & $-$0.072 & 1.967 & 7.807$^{**}$\\
 & (3.601) & (4.6) & (0.431) & (1.613) & (3.329)\\
Campus - Saint-Denis & 3.279 & 2.555 & $-$0.202 & $-$0.926 & 3.481\\
 & (3.177) & (3.944) & (0.408) & (1.417) & (2.984)\\
Filière - Eco-Ge & $-$2.653 & $-$2.872 & 0.33 & 0.112 & $-$2.983\\
 & (3.529) & (4.61) & (0.349) & (1.597) & (3.425)\\
Âge à la rentrée & $-$1.378 & $-$1.71 & $-$0.106 & $-$0.438 & $-$1.272\\
 & (1.792) & (2.3) & (0.197) & (0.764) & (1.687)\\
Sexe - Homme & $-$3.699 & $-$5.276 & $-$0.787$^{***}$ & $-$2.365$^{*}$ & $-$2.912\\
 & (3.725) & (4.514) & (0.296) & (1.233) & (3.613)\\
Pays de naissance - En France & $-$5.683 & $-$10.729 & $-$0.141 & $-$5.187 & $-$5.542\\
 & (6.718) & (10.05) & (0.654) & (4.018) & (6.487)\\
Statut de boursier - Secondaire & $-$4.707 & $-$5.989 & $-$0.159 & $-$1.441 & $-$4.548\\
 & (4.381) & (5.02) & (0.511) & (1.542) & (4.062)\\
Établissement d'origine - Privé & $-$11.432 & $-$12.919 & $-$0.528 & $-$2.016 & $-$10.904\\
 & (8.587) & (13.609) & (0.575) & (6.115) & (8.198)\\
 &  &  &  &  & \\
Observations & 258 & 258 & 258 & 258 & 258\\
R$^2$ ajusté & 0.092 & 0.119 & 0.019 & 0.11 & 0.093\\
F & 4.764 & 3.627 & 6.765 & 1.226 & 3.971\\*
\end{longtable}
\end{ThreePartTable}
\endgroup{}

\newpage

\setcounter{table}{0}
\setcounter{figure}{0}

\hypertarget{g20notions}{%
\section{Notions mathématiques proposées aux étudiants dans le cadre de l'expérimentation}\label{g20notions}}

Cette liste est tirée d'une convention (confidentielle) entre l'Université de La Réunion et la plateforme établie dans le cadre de l'expérimentation.

\begin{enumerate}
\def\labelenumi{\arabic{enumi}.}
\tightlist
\item
  Règles de calcul dans \(\mathbb{R}\), développement, factorisation, fractions, puissances.
\item
  Équations et inéquations du premier degré, intervalles.
\item
  Équations et inéquations du second degré.
\item
  Droites : équations, représentations graphiques.
\item
  Quelques fonctions (simples) : fonctions affines, fonctions polynômes du second degré, fonction racine carrée, fonction inverse (Seulement : domaine de définition, propriétés graphiques, pas de dérivées).
\item
  Dérivées : définition, interprétation géométrique, règles de dérivation.
\item
  Sens de variation d'une fonction, représentation graphique.
\end{enumerate}

\setcounter{table}{0}
\setcounter{figure}{0}

\hypertarget{g20cfztreat}{%
\section{Utilisation de la plateforme par campus, par filière et par incitation}\label{g20cfztreat}}

\begingroup\fontsize{8}{10}\selectfont

\begin{ThreePartTable}
\begin{TableNotes}
\item \textit{Sources :} Base Parcoursup - L1 AES et Eco-Ge, plateforme '\textit{J'ai 20 en maths}', APOGEE (version extraite en mai 2021), calculs de l'auteur.
\item \textit{Notes :} Moyennes et effectifs. AES : Administration Économique et Sociale, Eco-Ge : Économie-Gestion.
\end{TableNotes}
\begin{longtable}[t]{lllrrrrrr}
\caption{\label{tab:g20cfztreat}Utilisation de la plateforme par campus, par filière et par incitation}\\
\toprule
\makecell{Campus \\ \ } & \makecell{Filière \\ \ } & \makecell{Incitation \\ \ } & \makecell{Effectif \\ \ } & \makecell{Vidéos (complètes) \\ et exercices} & \makecell{Vidéos et exercices \\ \ } & \makecell{Vidéos (complètes) \\ \ } & \makecell{Vidéos \\ \ } & \makecell{Exercices \\ \ }\\
\midrule
\endfirsthead
\caption[]{\label{tab:g20cfztreat}Utilisation de la plateforme par campus, par filière et par incitation (suite)}\\
\toprule
\makecell{Campus \\ \ } & \makecell{Filière \\ \ } & \makecell{Incitation \\ \ } & \makecell{Effectif \\ \ } & \makecell{Vidéos (complètes) \\ et exercices} & \makecell{Vidéos et exercices \\ \ } & \makecell{Vidéos (complètes) \\ \ } & \makecell{Vidéos \\ \ } & \makecell{Exercices \\ \ }\\
\midrule
\endhead

\endfoot
\bottomrule
\insertTableNotes
\endlastfoot
Tampon & AES & Non & 38 & 6.50 & 9.79 & 0.21 & 3.50 & 6.29\\
 
 &  & Oui & 35 & 14.14 & 18.34 & 1.57 & 5.77 & 12.57\\
 
 & Eco-Ge & Non & 16 & 18.12 & 29.38 & 0.88 & 12.12 & 17.25\\
 
 &  & Oui & 14 & 10.07 & 12.29 & 1.36 & 3.57 & 8.71\\
 
Saint-Denis & AES & Non & 64 & 9.55 & 12.81 & 0.23 & 3.50 & 9.31\\
 
 &  & Oui & 55 & 20.49 & 25.60 & 1.24 & 6.35 & 19.25\\
 
 & Eco-Ge & Non & 22 & 7.82 & 9.86 & 0.23 & 2.27 & 7.59\\
 
 &  & Oui & 16 & 16.31 & 21.38 & 1.44 & 6.50 & 14.88\\*
\end{longtable}
\end{ThreePartTable}
\endgroup{}

\newpage
\setcounter{table}{0}
\setcounter{figure}{0}

\hypertarget{g20rfmodelsnormpop}{%
\section{Effets de l'incitation sur les notes de mathématiques (intention de traiter) - versions normalisées des notes}\label{g20rfmodelsnormpop}}

\begingroup\fontsize{8}{10}\selectfont

\begin{ThreePartTable}
\begin{TableNotes}
\item \textit{Sources :} Base Parcoursup - L1 AES et Eco-Ge, plateforme '\textit{J'ai 20 en maths}', APOGEE (version extraite en mai 2021), calculs de l'auteur.
\item \textit{Notes :} Écart-types robustes entre parenthèses. 
    Les notes sont normalisées sur la population de tous les inscrits, y compris ceux hors de l'expérimentation. Une colonne correspond à une régression. ES : Économique et social, S : Scientifique, AES : Administration Économique et Sociale. Eco-Ge : Économie-Gestion. TD : Travaux Dirigés. UE : Unité d'Enseignement.
\item Significativité : 10\% * 5\% ** 1\% ***.
\end{TableNotes}
\begin{longtable}[t]{llll}
\caption{\label{tab:g20rfmodelsnormpop}Effets de l'incitation sur les notes de mathématiques (intention de traiter) - versions normalisées des notes}\\
\toprule
\multicolumn{1}{c}{ } & \multicolumn{3}{c}{Variable dépendante : Note } \\
\cmidrule(l{3pt}r{3pt}){2-4}
  & \makecell{Aux TD \\ (1) } & \makecell{Aux examens \\ (2) } & \makecell{À l'UE \\ (3) }\\
\midrule
\endfirsthead
\caption[]{\label{tab:g20rfmodelsnormpop}Effets de l'incitation sur les notes de mathématiques (intention de traiter) - versions normalisées des notes (suite)}\\
\toprule
  & \makecell{Aux TD \\ (1) } & \makecell{Aux examens \\ (2) } & \makecell{À l'UE \\ (3) }\\
\midrule
\endhead

\endfoot
\bottomrule
\insertTableNotes
\endlastfoot
Constante & $-$3.297$^{*}$ & $-$3.9$^{***}$ & $-$3.239$^{**}$\\
 & (1.866) & (1.286) & (1.275)\\
Incitation - Oui & 0.319$^{***}$ & 0.055 & 0.165$^{*}$\\
 & (0.108) & (0.103) & (0.094)\\
Note au bac & 0.226$^{***}$ & 0.26$^{***}$ & 0.263$^{***}$\\
 & (0.04) & (0.039) & (0.033)\\
Série au bac - Techno & 0.622$^{***}$ & 0.412$^{***}$ & 0.529$^{***}$\\
 & (0.157) & (0.11) & (0.105)\\
Série au bac - ES & 1.074$^{***}$ & 1.008$^{***}$ & 1.137$^{***}$\\
 & (0.144) & (0.119) & (0.109)\\
Série au bac - S & 1.703$^{***}$ & 2.064$^{***}$ & 2.134$^{***}$\\
 & (0.172) & (0.205) & (0.186)\\
Série au bac - Autre & 0.982$^{**}$ & 0.46$^{***}$ & 0.675$^{***}$\\
 & (0.444) & (0.137) & (0.211)\\
Campus - Saint-Denis & $-$0.384$^{***}$ & $-$0.066 & $-$0.153\\
 & (0.113) & (0.102) & (0.094)\\
Filière - Eco-Ge & $-$0.212$^{*}$ & $-$0.326$^{***}$ & $-$0.281$^{***}$\\
 & (0.12) & (0.115) & (0.105)\\
Âge à la rentrée & 0.002 & 0.019 & $-$0.027\\
 & (0.085) & (0.055) & (0.058)\\
Sexe - Homme & 0.121 & 0.087 & 0.164\\
 & (0.123) & (0.114) & (0.102)\\
Pays de naissance - En France & $-$0.077 & $-$0.154 & $-$0.023\\
 & (0.2) & (0.153) & (0.155)\\
Statut de boursier - Secondaire & $-$0.079 & $-$0.042 & 0.045\\
 & (0.138) & (0.115) & (0.104)\\
Établissement d'origine - Privé & $-$0.151 & 0.079 & 0.165\\
 & (0.226) & (0.27) & (0.264)\\
 &  &  & \\
Observations & 218 & 223 & 258\\
R$^2$ ajusté & 0.392 & 0.536 & 0.547\\*
\end{longtable}
\end{ThreePartTable}
\endgroup{}

\newpage
\setcounter{table}{0}
\setcounter{figure}{0}

\hypertarget{g20modelsnormpop}{%
\section{Effets de l'utilisation de la plateforme sur les notes de mathématiques - versions normalisées des notes}\label{g20modelsnormpop}}

\begingroup\fontsize{8}{10}\selectfont

\begin{ThreePartTable}
\begin{TableNotes}
\item \textit{Sources :} Base Parcoursup - L1 AES et Eco-Ge, plateforme '\textit{J'ai 20 en maths}', APOGEE (version extraite en mai 2021), calculs de l'auteur.
\item \textit{Notes :} Écart-types robustes entre parenthèses. 
    Les notes sont normalisées sur la population de tous les inscrits, y compris ceux hors de l'expérimentation. Une colonne et un panel donnés correspondent à une régression. Les variables liées à la plateforme sont exprimées en comptages. ES : Économique et social, S : Scientifique, AES : Administration Économique et Sociale. Eco-Ge : Économie-Gestion. TD : Travaux Dirigés. UE : Unité d'Enseignement. MCO : Moindres Carrés Ordinaires. VI : Variables Instrumentales.
\item Significativité : 10\% * 5\% ** 1\% ***.
\end{TableNotes}
\begin{longtable}[t]{lllllll}
\caption{\label{tab:g20modelsnormpop}Effets de l'utilisation de la plateforme sur les notes normalisées de mathématiques}\\
\toprule
\multicolumn{1}{c}{ } & \multicolumn{6}{c}{Variable dépendante : Note } \\
\cmidrule(l{3pt}r{3pt}){2-7}
\multicolumn{1}{c}{ } & \multicolumn{2}{c}{Aux TD} & \multicolumn{2}{c}{Aux examens} & \multicolumn{2}{c}{À l'UE} \\
\cmidrule(l{3pt}r{3pt}){2-3} \cmidrule(l{3pt}r{3pt}){4-5} \cmidrule(l{3pt}r{3pt}){6-7}
  & \makecell{MCO \\ (1) } & \makecell{VI \\ (2) } & \makecell{MCO \\ (3) } & \makecell{VI \\ (4) } & \makecell{MCO \\ (5) } & \makecell{VI \\ (6) }\\
\midrule
\endfirsthead
\caption[]{\label{tab:g20modelsnormpop}Effets de l'utilisation de la plateforme sur les notes normalisées de mathématiques (suite)}\\
\toprule
  & \makecell{MCO \\ (1) } & \makecell{VI \\ (2) } & \makecell{MCO \\ (3) } & \makecell{VI \\ (4) } & \makecell{MCO \\ (5) } & \makecell{VI \\ (6) }\\
\midrule
\endhead

\endfoot
\bottomrule
\insertTableNotes
\endlastfoot
\addlinespace[0.3em]
\multicolumn{7}{l}{\textbf{Panel A : Vidéos (complètes) et exercices}}\\
\hline
\hspace{1em}Vidéos (complètes) et exercices & 0.008$^{***}$ & 0.045$^{*}$ & 0.002 & 0.008 & 0.006$^{***}$ & 0.022\\
\hspace{1em} & (0.003) & (0.026) & (0.002) & (0.014) & (0.002) & (0.014)\\
\hspace{1em} &  &  &  &  &  \vphantom{8} & \\
\hspace{1em}Contrôles & Oui & Oui & Oui & Oui & Oui & \vphantom{4} Oui\\
\hspace{1em}Observations & 218 & 218 & 223 & 223 & 258 & \vphantom{4} 258\\
\hspace{1em} &  &  &  &  &  \vphantom{7} & \\
\addlinespace[0.3em]
\multicolumn{7}{l}{\textbf{Panel B : Vidéos et exercices}}\\
\hline
\hspace{1em}Vidéos et exercices & 0.007$^{***}$ & 0.043 & 0.003 & 0.007 & 0.005$^{***}$ & 0.02\\
\hspace{1em} & (0.002) & (0.03) & (0.002) & (0.013) & (0.002) & (0.014)\\
\hspace{1em} &  &  &  &  &  \vphantom{6} & \\
\hspace{1em}Contrôles & Oui & Oui & Oui & Oui & Oui & \vphantom{3} Oui\\
\hspace{1em}Observations & 218 & 218 & 223 & 223 & 258 & \vphantom{3} 258\\
\hspace{1em} &  &  &  &  &  \vphantom{5} & \\
\addlinespace[0.3em]
\multicolumn{7}{l}{\textbf{Panel C : Vidéos (complètes)}}\\
\hline
\hspace{1em}Vidéos (complètes) & 0.065$^{**}$ & 0.387$^{*}$ & 0.03$^{**}$ & 0.059 & 0.027 & 0.16\\
\hspace{1em} & (0.027) & (0.21) & (0.015) & (0.111) & (0.02) & (0.106)\\
\hspace{1em} &  &  &  &  &  \vphantom{4} & \\
\hspace{1em}Contrôles & Oui & Oui & Oui & Oui & Oui & \vphantom{2} Oui\\
\hspace{1em}Observations & 218 & 218 & 223 & 223 & 258 & \vphantom{2} 258\\
\hspace{1em} &  &  &  &  &  \vphantom{3} & \\
\addlinespace[0.3em]
\multicolumn{7}{l}{\textbf{Panel D : Vidéos}}\\
\hline
\hspace{1em}Vidéos & 0.014$^{***}$ & 0.305 & 0.015$^{**}$ & 0.036 & 0.013$^{***}$ & 0.102\\
\hspace{1em} & (0.004) & (0.467) & (0.006) & (0.066) & (0.005) & (0.097)\\
\hspace{1em} &  &  &  &  &  \vphantom{2} & \\
\hspace{1em}Contrôles & Oui & Oui & Oui & Oui & Oui & \vphantom{1} Oui\\
\hspace{1em}Observations & 218 & 218 & 223 & 223 & 258 & \vphantom{1} 258\\
\hspace{1em} &  &  &  &  &  \vphantom{1} & \\
\addlinespace[0.3em]
\multicolumn{7}{l}{\textbf{Panel E : Exercices}}\\
\hline
\hspace{1em}Exercices & 0.008$^{***}$ & 0.051 & 0.002 & 0.009 & 0.006$^{***}$ & 0.026\\
\hspace{1em} & (0.003) & (0.032) & (0.002) & (0.017) & (0.002) & (0.017)\\
\hspace{1em} &  &  &  &  &  & \\
\hspace{1em}Contrôles & Oui & Oui & Oui & Oui & Oui & Oui\\
\hspace{1em}Observations & 218 & 218 & 223 & 223 & 258 & 258\\*
\end{longtable}
\end{ThreePartTable}
\endgroup{}

\newpage

\setcounter{table}{0}
\setcounter{figure}{0}

\hypertarget{g20modelsautresnotesnormpop}{%
\section{Effets de l'utilisation de la plateforme sur les versions normalisées des autres notes}\label{g20modelsautresnotesnormpop}}

\begingroup\fontsize{8}{10}\selectfont

\begin{ThreePartTable}
\begin{TableNotes}
\item \textit{Sources :} Base Parcoursup - L1 AES et Eco-Ge, plateforme '\textit{J'ai 20 en maths}', APOGEE (version extraite en mai 2021), calculs de l'auteur.
\item \textit{Notes :} Écart-types robustes entre parenthèses. 
    Les notes sont normalisées sur la population de tous les inscrits, y compris ceux hors de l'expérimentation. Une colonne et un panel donnés correspondent à une régression. Les variables liées à la plateforme sont exprimées en comptages. ES : Économique et social, S : Scientifique, AES : Administration Économique et Sociale. Eco-Ge : Économie-Gestion. TD : Travaux Dirigés. UE : Unité d'Enseignement. MCO : Moindres Carrés Ordinaires. VI : Variables Instrumentales.
\item Significativité : 10\% * 5\% ** 1\% ***.
\end{TableNotes}
\begin{longtable}[t]{lllllllll}
\caption{\label{tab:g20modelsautresnotesnormpop}Effets de l'utilisation de la plateforme sur les versions normalisées des autres notes}\\
\toprule
\multicolumn{1}{c}{ } & \multicolumn{8}{c}{Variable dépendante : Note } \\
\cmidrule(l{3pt}r{3pt}){2-9}
\multicolumn{1}{c}{ } & \multicolumn{4}{c}{En gestion} & \multicolumn{4}{c}{En économie} \\
\cmidrule(l{3pt}r{3pt}){2-5} \cmidrule(l{3pt}r{3pt}){6-9}
\multicolumn{1}{c}{ } & \multicolumn{8}{c}{Ayant une note } \\
\cmidrule(l{3pt}r{3pt}){2-9}
\multicolumn{1}{c}{ } & \multicolumn{2}{c}{Aux TD} & \multicolumn{2}{c}{Aux examens} & \multicolumn{2}{c}{Aux TD} & \multicolumn{2}{c}{Aux examens} \\
\cmidrule(l{3pt}r{3pt}){2-3} \cmidrule(l{3pt}r{3pt}){4-5} \cmidrule(l{3pt}r{3pt}){6-7} \cmidrule(l{3pt}r{3pt}){8-9}
  & \makecell{MCO \\ (1) } & \makecell{VI \\ (2) } & \makecell{MCO \\ (3) } & \makecell{VI \\ (4) } & \makecell{MCO \\ (5) } & \makecell{VI \\ (6) } & \makecell{MCO \\ (7) } & \makecell{VI \\ (8) }\\
\midrule
\endfirsthead
\caption[]{\label{tab:g20modelsautresnotesnormpop}Effets de l'utilisation de la plateforme sur les versions normalisées des autres notes (suite)}\\
\toprule
  & \makecell{MCO \\ (1) } & \makecell{VI \\ (2) } & \makecell{MCO \\ (3) } & \makecell{VI \\ (4) } & \makecell{MCO \\ (5) } & \makecell{VI \\ (6) } & \makecell{MCO \\ (7) } & \makecell{VI \\ (8) }\\
\midrule
\endhead

\endfoot
\bottomrule
\insertTableNotes
\endlastfoot
\addlinespace[0.3em]
\multicolumn{9}{l}{\textbf{Panel A : Vidéos (complètes) et exercices}}\\
\hline
\hspace{1em}Vidéos (complètes) et exercices & 0.005$^{***}$ & 0.028 & 0.004$^{***}$ & 0.014 & 0.002 & $-$0.002 & 0.003 & $-$0.007\\
\hspace{1em} & (0.001) & (0.019) & (0.001) & (0.015) & (0.002) & (0.02) & (0.002) & (0.019)\\
\hspace{1em} &  &  &  &  &  &  &  \vphantom{8} & \\
\hspace{1em}Contrôles & Oui & Oui & Oui & Oui & Oui & Oui & Oui & \vphantom{4} Oui\\
\hspace{1em}Observations & 203 & 203 & 210 & 210 & 205 & 205 & 214 & \vphantom{4} 214\\
\hspace{1em} &  &  &  &  &  &  &  \vphantom{7} & \\
\addlinespace[0.3em]
\multicolumn{9}{l}{\textbf{Panel B : Vidéos et exercices}}\\
\hline
\hspace{1em}Vidéos et exercices & 0.004$^{***}$ & 0.027 & 0.003$^{***}$ & 0.013 & 0.001 & $-$0.002 & 0.002 & $-$0.007\\
\hspace{1em} & (0.001) & (0.021) & (0.001) & (0.014) & (0.002) & (0.02) & (0.002) & (0.018)\\
\hspace{1em} &  &  &  &  &  &  &  \vphantom{6} & \\
\hspace{1em}Contrôles & Oui & Oui & Oui & Oui & Oui & Oui & Oui & \vphantom{3} Oui\\
\hspace{1em}Observations & 203 & 203 & 210 & 210 & 205 & 205 & 214 & \vphantom{3} 214\\
\hspace{1em} &  &  &  &  &  &  &  \vphantom{5} & \\
\addlinespace[0.3em]
\multicolumn{9}{l}{\textbf{Panel C : Vidéos (complètes)}}\\
\hline
\hspace{1em}Vidéos (complètes) & 0.023 & 0.231 & 0.018 & 0.106 & $-$0.005 & $-$0.016 & $-$0.003 & $-$0.054\\
\hspace{1em} & (0.015) & (0.154) & (0.014) & (0.112) & (0.03) & (0.163) & (0.029) & (0.141)\\
\hspace{1em} &  &  &  &  &  &  &  \vphantom{4} & \\
\hspace{1em}Contrôles & Oui & Oui & Oui & Oui & Oui & Oui & Oui & \vphantom{2} Oui\\
\hspace{1em}Observations & 203 & 203 & 210 & 210 & 205 & 205 & 214 & \vphantom{2} 214\\
\hspace{1em} &  &  &  &  &  &  &  \vphantom{3} & \\
\addlinespace[0.3em]
\multicolumn{9}{l}{\textbf{Panel D : Vidéos}}\\
\hline
\hspace{1em}Vidéos & 0.006 & 0.185 & 0.007$^{*}$ & 0.067 & $-$0.003 & $-$0.013 & 0 & $-$0.034\\
\hspace{1em} & (0.004) & (0.288) & (0.004) & (0.093) & (0.007) & (0.141) & (0.006) & (0.094)\\
\hspace{1em} &  &  &  &  &  &  &  \vphantom{2} & \\
\hspace{1em}Contrôles & Oui & Oui & Oui & Oui & Oui & Oui & Oui & \vphantom{1} Oui\\
\hspace{1em}Observations & 203 & 203 & 210 & 210 & 205 & 205 & 214 & \vphantom{1} 214\\
\hspace{1em} &  &  &  &  &  &  &  \vphantom{1} & \\
\addlinespace[0.3em]
\multicolumn{9}{l}{\textbf{Panel E : Exercices}}\\
\hline
\hspace{1em}Exercices & 0.005$^{***}$ & 0.032 & 0.005$^{***}$ & 0.016 & 0.003 & $-$0.002 & 0.003 & $-$0.008\\
\hspace{1em} & (0.001) & (0.023) & (0.001) & (0.017) & (0.002) & (0.023) & (0.002) & (0.022)\\
\hspace{1em} &  &  &  &  &  &  &  & \\
\hspace{1em}Contrôles & Oui & Oui & Oui & Oui & Oui & Oui & Oui & Oui\\
\hspace{1em}Observations & 203 & 203 & 210 & 210 & 205 & 205 & 214 & 214\\*
\end{longtable}
\end{ThreePartTable}
\endgroup{}

\newpage

\setcounter{table}{0}
\setcounter{figure}{0}

\hypertarget{g20ovbmodels}{%
\section{Relation entre l'utilisation de la plateforme et les observables dans un monde sans incitation}\label{g20ovbmodels}}

\begingroup\fontsize{8}{10}\selectfont

\begin{ThreePartTable}
\begin{TableNotes}
\item \textit{Sources :} Base Parcoursup - L1 AES et Eco-Ge, plateforme '\textit{J'ai 20 en maths}', APOGEE (version extraite en mai 2021), calculs de l'auteur.
\item \textit{Notes :} Écart-types robustes entre parenthèses. Une colonne correspond à une régression. Les régressions sont effectuées sur les non incitées. ES : Économique et social, S : Scientifique, AES : Administration Économique et Sociale. Eco-Ge : Économie-Gestion.
\item Significativité : 10\% * 5\% ** 1\% ***.
\end{TableNotes}
\begin{longtable}[t]{llllll}
\caption{\label{tab:g20ovbmodels}Relation entre l'utilisation de la plateforme et les observables dans un monde sans incitation}\\
\toprule
\multicolumn{1}{c}{ } & \multicolumn{5}{c}{Variable dépendante : Nombre } \\
\cmidrule(l{3pt}r{3pt}){2-6}
  & \makecell{\makecell{Vidéos (complètes) \\ et exercices} \\ (1) } & \makecell{\makecell{Vidéos et exercices \\ \ } \\ (2) } & \makecell{\makecell{Vidéos (complètes) \\ \ } \\ (3) } & \makecell{\makecell{Vidéos \\ \ } \\ (4) } & \makecell{\makecell{Exerices \\ \ } \\ (5) }\\
\midrule
\endfirsthead
\caption[]{\label{tab:g20ovbmodels}Relation entre l'utilisation de la plateforme et les observables dans un monde sans incitation (suite)}\\
\toprule
  & \makecell{\makecell{Vidéos (complètes) \\ et exercices} \\ (1) } & \makecell{\makecell{Vidéos et exercices \\ \ } \\ (2) } & \makecell{\makecell{Vidéos (complètes) \\ \ } \\ (3) } & \makecell{\makecell{Vidéos \\ \ } \\ (4) } & \makecell{\makecell{Exerices \\ \ } \\ (5) }\\
\midrule
\endhead

\endfoot
\bottomrule
\insertTableNotes
\endlastfoot
Constante & 1.655 & 8.597 & $-$4.902$^{*}$ & 2.04 & 6.557\\
 & (47.951) & (70.114) & (2.927) & (26.666) & (47.247)\\
Note au bac & 3.351$^{***}$ & 4.913$^{***}$ & 0.27$^{**}$ & 1.831$^{***}$ & 3.081$^{***}$\\
 & (1.108) & (1.567) & (0.111) & (0.655) & (1.067)\\
Série au bac - Techno & 1.107 & 1.519 & 0.226 & 0.638 & 0.881\\
 & (2.164) & (3.198) & (0.144) & (1.207) & (2.112)\\
Série au bac - ES & 9.192$^{***}$ & 14.511$^{***}$ & 0.413$^{**}$ & 5.733$^{***}$ & 8.779$^{***}$\\
 & (3.225) & (4.966) & (0.179) & (2.021) & (3.171)\\
Série au bac - S & 22.319$^{***}$ & 30.242$^{***}$ & 0.152 & 8.076$^{***}$ & 22.166$^{***}$\\
 & (5.7) & (7.639) & (0.225) & (2.643) & (5.675)\\
Série au bac - Autre & 0.824 & $-$1.389 & $-$0.251 & $-$2.464 & 1.075\\
 & (5.344) & (7.587) & (0.408) & (2.922) & (5.078)\\
Campus - Saint-Denis & 0.172 & $-$2.32 & $-$0.054 & $-$2.547 & 0.226\\
 & (2.741) & (4.22) & (0.184) & (1.841) & (2.685)\\
Filière - Eco-Ge & $-$2.757 & $-$2.552 & 0.143 & 0.347 & $-$2.9\\
 & (3.879) & (5.873) & (0.26) & (2.46) & (3.818)\\
Âge à la rentrée & $-$1.417 & $-$2.339 & 0.097 & $-$0.825 & $-$1.514\\
 & (1.891) & (2.794) & (0.095) & (1.073) & (1.87)\\
Sexe - Homme & 0.608 & $-$0.245 & $-$0.139 & $-$0.992 & 0.747\\
 & (3.279) & (4.58) & (0.227) & (1.736) & (3.215)\\
Pays de naissance - En France & $-$12.177 & $-$18.244 & 0.108 & $-$5.958 & $-$12.286\\
 & (10.189) & (15.627) & (0.125) & (5.698) & (10.166)\\
Statut de boursier - Secondaire & $-$3.254 & $-$4.642 & $-$0.205 & $-$1.594 & $-$3.048\\
 & (3.246) & (4.506) & (0.213) & (1.737) & (3.188)\\
Établissement d'origine - Privé & $-$8.553 & $-$10.366 & $-$0.103 & $-$1.915 & $-$8.45\\
 & (11.476) & (18.084) & (0.553) & (7.26) & (11.044)\\
 &  &  &  &  & \\
Observations & 140 & 140 & 140 & 140 & 140\\
R$^2$ ajusté & 0.174 & 0.167 & 0.106 & 0.124 & 0.169\\*
\end{longtable}
\end{ThreePartTable}
\endgroup{}

\newpage  
\setcounter{table}{0}
\setcounter{figure}{0}

\hypertarget{g20compliers}{%
\section{Caractérisation des compliers des vidéos et exercices}\label{g20compliers}}

\begingroup\fontsize{8}{10}\selectfont

\begin{ThreePartTable}
\begin{TableNotes}
\item \textit{Sources :} Base Parcoursup - L1 AES et Eco-Ge, plateforme '\textit{J'ai 20 en maths}', APOGEE (version extraite en mai 2021), calculs de l'auteur
\item \textit{Notes :} Le traitement est binarisé (au moins un) par souci de simplification. Rapport de parts de compliers = différence de moyenne du traitement en fonction de l'incitation divisée par différence de moyenne du traitement en fonction de l'incitation sur l'échantillon considéré (ayant une note aux TD ou ayant une note aux examens) - Angrist et Pischke (2008, p. 171). UE : Unité d'enseignement, ES : Économique et social, S : Scientifique, AES : Administration Économique et Sociale, Eco-Ge : Économie-Gestion. TD : Travaux Dirigés.
\end{TableNotes}
\begin{longtable}[t]{llrrrr}
\caption{\label{tab:g20compliersvideosviewstot}Caractérisation des compliers - Vidéos}\\
\toprule
\multicolumn{2}{c}{ } & \multicolumn{2}{c}{Ayant une note aux TD} & \multicolumn{2}{c}{Ayant une note aux examens} \\
\cmidrule(l{3pt}r{3pt}){3-4} \cmidrule(l{3pt}r{3pt}){5-6}
Variable & Modalité & Effectif & \makecell{Rapport de parts \\ de compliers} & Effectif & \makecell{Rapport de parts \\ de compliers}\\
\midrule
\endfirsthead
\caption[]{\label{tab:g20compliersvideosviewstot}Caractérisation des compliers - Vidéos (suite)}\\
\toprule
Variable & Modalité & Effectif & \makecell{Rapport de parts \\ de compliers} & Effectif & \makecell{Rapport de parts \\ de compliers}\\
\midrule
\endhead

\endfoot
\bottomrule
\insertTableNotes
\endlastfoot
\addlinespace[0.3em]
\multicolumn{6}{l}{\textbf{ }}\\
Série au bac & Pro & 21 & 0.29 & 28 & -0.40\\
 & Techno & 58 & 1.78 & 57 & 1.39\\
 & ES & 96 & 1.01 & 97 & 1.22\\
 & S & 36 & 0.23 & 35 & 0.46\\
 & Autre & 7 & 5.22 & 8 & 6.38\\
\addlinespace[0.3em]
\multicolumn{6}{l}{\textbf{ }}\\
Filière & AES & 161 & 0.75 & 164 & 0.74\\
 & Eco-Ge & 57 & 1.81 & 61 & 1.78\\
\addlinespace[0.3em]
\multicolumn{6}{l}{\textbf{ }}\\
Quintile de la note au bac & Q1 & 45 & 2.41 & 42 & 2.19\\
 & Q2 & 36 & -0.48 & 40 & 0.58\\
 & Q3 & 43 & -0.02 & 45 & 0.57\\
 & Q4 & 48 & 2.07 & 50 & 1.47\\
 & Q5 & 46 & 0.97 & 46 & 0.59\\
\addlinespace[0.3em]
\multicolumn{6}{l}{\textbf{ }}\\
Quintile de l'âge à la rentrée & Q1 & 44 & 0.68 & 46 & 0.84\\
 & Q2 & 44 & 0.72 & 45 & 0.67\\
 & Q3 & 45 & -0.16 & 44 & 1.02\\
 & Q4 & 45 & 2.82 & 45 & 1.95\\
 & Q5 & 40 & 1.98 & 45 & 1.88\\
\addlinespace[0.3em]
\multicolumn{6}{l}{\textbf{ }}\\
Sexe & Fille & 153 & 1.12 & 157 & 1.36\\
 & Homme & 65 & 0.78 & 68 & 0.19\\
\addlinespace[0.3em]
\multicolumn{6}{l}{\textbf{ }}\\
Campus & Tampon & 92 & 0.10 & 90 & 0.03\\
 & Saint-Denis & 126 & 1.66 & 135 & 1.65\\
\addlinespace[0.3em]
\multicolumn{6}{l}{\textbf{ }}\\
Statut de boursier & Non boursier & 78 & 2.22 & 77 & 2.78\\
 & Boursier du secondaire & 140 & 0.21 & 148 & 0.08\\
\addlinespace[0.3em]
\multicolumn{6}{l}{\textbf{ }}\\
Pays de naissance & À l'étranger & 20 & 3.25 & 19 & 2.51\\
 & En France & 198 & 0.77 & 206 & 0.84\\
\addlinespace[0.3em]
\multicolumn{6}{l}{\textbf{ }}\\
Statut d'établissement d'origine & Public & 206 & 0.76 & 211 & 0.83\\
 & Privé & 12 & 5.43 & 14 & 3.59\\*
\end{longtable}
\end{ThreePartTable}
\endgroup{}

\newpage
\begingroup\fontsize{8}{10}\selectfont

\begin{ThreePartTable}
\begin{TableNotes}
\item \textit{Sources :} Base Parcoursup - L1 AES et Eco-Ge, plateforme '\textit{J'ai 20 en maths}', APOGEE (version extraite en mai 2021), calculs de l'auteur
\item \textit{Notes :} Le traitement est binarisé (au moins un) par souci de simplification. Rapport de parts de compliers = différence de moyenne du traitement en fonction de l'incitation divisée par différence de moyenne du traitement en fonction de l'incitation sur l'échantillon considéré (ayant une note aux TD ou ayant une note aux examens) - Angrist et Pischke (2008, p. 171). UE : Unité d'enseignement, ES : Économique et social, S : Scientifique, AES : Administration Économique et Sociale, Eco-Ge : Économie-Gestion. TD : Travaux Dirigés.
\end{TableNotes}
\begin{longtable}[t]{llrrrr}
\caption{\label{tab:g20complierssheetsviewstot}Caractérisation des compliers - Exercices}\\
\toprule
\multicolumn{2}{c}{ } & \multicolumn{2}{c}{Ayant une note aux TD} & \multicolumn{2}{c}{Ayant une note aux examens} \\
\cmidrule(l{3pt}r{3pt}){3-4} \cmidrule(l{3pt}r{3pt}){5-6}
Variable & Modalité & Effectif & \makecell{Rapport de parts \\ de compliers} & Effectif & \makecell{Rapport de parts \\ de compliers}\\
\midrule
\endfirsthead
\caption[]{\label{tab:g20complierssheetsviewstot}Caractérisation des compliers - Exercices (suite)}\\
\toprule
Variable & Modalité & Effectif & \makecell{Rapport de parts \\ de compliers} & Effectif & \makecell{Rapport de parts \\ de compliers}\\
\midrule
\endhead

\endfoot
\bottomrule
\insertTableNotes
\endlastfoot
\addlinespace[0.3em]
\multicolumn{6}{l}{\textbf{ }}\\
Série au bac & Pro & 21 & 2.27 & 28 & 2.05\\
 & Techno & 58 & 1.87 & 57 & 1.38\\
 & ES & 96 & 0.77 & 97 & 0.97\\
 & S & 36 & -1.05 & 35 & -0.84\\
 & Autre & 7 & 4.27 & 8 & 3.51\\
\addlinespace[0.3em]
\multicolumn{6}{l}{\textbf{ }}\\
Filière & AES & 161 & 0.80 & 164 & 0.85\\
 & Eco-Ge & 57 & 1.58 & 61 & 1.39\\
\addlinespace[0.3em]
\multicolumn{6}{l}{\textbf{ }}\\
Quintile de la note au bac & Q1 & 45 & 0.99 & 42 & 1.14\\
 & Q2 & 36 & 0.47 & 40 & 1.24\\
 & Q3 & 43 & 2.07 & 45 & 1.82\\
 & Q4 & 48 & 1.22 & 50 & 0.99\\
 & Q5 & 46 & 0.36 & 46 & 0.03\\
\addlinespace[0.3em]
\multicolumn{6}{l}{\textbf{ }}\\
Quintile de l'âge à la rentrée & Q1 & 44 & -0.11 & 46 & -0.29\\
 & Q2 & 44 & -0.21 & 45 & -0.07\\
 & Q3 & 45 & 0.47 & 44 & 1.10\\
 & Q4 & 45 & 2.83 & 45 & 2.47\\
 & Q5 & 40 & 2.95 & 45 & 2.63\\
\addlinespace[0.3em]
\multicolumn{6}{l}{\textbf{ }}\\
Sexe & Fille & 153 & 0.74 & 157 & 1.12\\
 & Homme & 65 & 1.32 & 68 & 0.58\\
\addlinespace[0.3em]
\multicolumn{6}{l}{\textbf{ }}\\
Campus & Tampon & 92 & 0.69 & 90 & 0.65\\
 & Saint-Denis & 126 & 1.23 & 135 & 1.23\\
\addlinespace[0.3em]
\multicolumn{6}{l}{\textbf{ }}\\
Statut de boursier & Non boursier & 78 & 2.71 & 77 & 2.67\\
 & Boursier du secondaire & 140 & 0.04 & 148 & 0.13\\
\addlinespace[0.3em]
\multicolumn{6}{l}{\textbf{ }}\\
Pays de naissance & À l'étranger & 20 & 5.43 & 19 & 4.68\\
 & En France & 198 & 0.55 & 206 & 0.66\\
\addlinespace[0.3em]
\multicolumn{6}{l}{\textbf{ }}\\
Statut d'établissement d'origine & Public & 206 & 0.60 & 211 & 0.73\\
 & Privé & 12 & 7.46 & 14 & 4.97\\*
\end{longtable}
\end{ThreePartTable}
\endgroup{}

\blandscape
\setcounter{table}{0}
\setcounter{figure}{0}

\hypertarget{g20rfhetermodels}{%
\section{Effets potentiellement hétérogènes de l'incitation sur les notes de mathématiques (intention de traiter) - version normalisée des notes}\label{g20rfhetermodels}}

\begingroup\fontsize{7}{9}\selectfont

\begin{ThreePartTable}
\begin{TableNotes}
\item \textit{Sources :} Base Parcoursup - L1 AES et Eco-Ge, plateforme '\textit{J'ai 20 en maths}', APOGEE (version extraite en mai 2021), calculs de l'auteur.
\item \textit{Notes :} Écart-types robustes entre parenthèses. 
    Les variables dépendantes sont des notes normalisées sur la population de tous les inscrits, y compris ceux hors de l'expérimentation. Une colonne et un panel donnés correspondent à une régression. ES : Économique et social, S : Scientifique, AES : Administration Économique et Sociale. Eco-Ge : Économie-Gestion. TD : Travaux Dirigés. UE : Unité d'Enseignement.
\item Significativité : 10\% * 5\% ** 1\% ***.
\end{TableNotes}
\begin{longtable}[t]{llll}
\caption{\label{tab:g20rfhetermodelsnormpop}Effets potentiellement hétérogènes de l'incitation sur les notes de mathématiques (intention de traiter) - version normalisée des notes}\\
\toprule
\multicolumn{1}{c}{ } & \multicolumn{3}{c}{Variable dépendante : Note } \\
\cmidrule(l{3pt}r{3pt}){2-4}
  & \makecell{Aux TD \\ (1) } & \makecell{Aux examens \\ (2) } & \makecell{À l'UE \\ (3) }\\
\midrule
\endfirsthead
\caption[]{\label{tab:g20rfhetermodelsnormpop}Effets potentiellement hétérogènes de l'incitation sur les notes de mathématiques (intention de traiter) - version normalisée des notes (suite)}\\
\toprule
  & \makecell{Aux TD \\ (1) } & \makecell{Aux examens \\ (2) } & \makecell{À l'UE \\ (3) }\\
\midrule
\endhead

\endfoot
\bottomrule
\insertTableNotes
\endlastfoot
\addlinespace[0.3em]
\multicolumn{4}{l}{\textbf{Panel A : Hétérogénéité en fonction de la mention au bac}}\\
\hline
\hspace{1em}Incitation - Oui & 0.428$^{**}$ & 0.035 & 0.212$^{*}$\\
\hspace{1em} & (0.174) & (0.117) & (0.122)\\
\hspace{1em}Mention au bac - Assez bien & 0.282$^{*}$ & 0.386$^{***}$ & 0.411$^{***}$\\
\hspace{1em} & (0.157) & (0.135) & (0.131)\\
\hspace{1em}Mention au bac - Bien & 0.75$^{***}$ & 0.919$^{***}$ & 0.998$^{***}$\\
\hspace{1em} & (0.256) & (0.217) & (0.206)\\
\hspace{1em}Mention au bac - Très bien & 1.983$^{***}$ & 1.5$^{*}$ & 1.694$^{**}$\\
\hspace{1em} & (0.294) & (0.89) & (0.749)\\
\hspace{1em}Incitation - Oui $\times$ Mention au bac - Assez bien & $-$0.345 & $-$0.046 & $-$0.186\\
\hspace{1em} & (0.233) & (0.221) & (0.205)\\
\hspace{1em}Incitation - Oui $\times$ Mention au bac - Bien & $-$0.243 & 0.067 & $-$0.136\\
\hspace{1em} & (0.354) & (0.406) & (0.339)\\
\hspace{1em}Incitation - Oui $\times$ Mention au bac - Très bien & 0.082 & 0.702 & 0.617\\
\hspace{1em} & (0.403) & (0.917) & (0.772)\\
\hspace{1em} &  &  \vphantom{10} & \\
\hspace{1em}Contrôles & Oui & Oui & \vphantom{5} Oui\\
\hspace{1em}Observations & 218 & 225 & \vphantom{5} 260\\
\hspace{1em}R$^2$ ajusté & 0.398 & 0.536 & 0.548\\
\hspace{1em} &  &  \vphantom{9} & \\
\addlinespace[0.3em]
\multicolumn{4}{l}{\textbf{Panel B : Hétérogénéité en fonction de la série au bac}}\\
\hline
\hspace{1em}Incitation - Oui & 0.015 & $-$0.088 & $-$0.106\\
\hspace{1em} & (0.212) & (0.136) & (0.128)\\
\hspace{1em}Série au bac - Techno & 0.361$^{**}$ & 0.264$^{**}$ & 0.235$^{**}$\\
\hspace{1em} & (0.182) & (0.129) & (0.107)\\
\hspace{1em}Série au bac - ES & 1.026$^{***}$ & 0.91$^{***}$ & 1.046$^{***}$\\
\hspace{1em} & (0.169) & (0.143) & (0.128)\\
\hspace{1em}Série au bac - S & 1.484$^{***}$ & 1.813$^{***}$ & 1.833$^{***}$\\
\hspace{1em} & (0.232) & (0.228) & (0.226)\\
\hspace{1em}Série au bac - Autre & 1.221 & 0.308 & 0.449\\
\hspace{1em} & (1.006) & (0.293) & (0.44)\\
\hspace{1em}Incitation - Oui $\times$ Série au bac - Techno & 0.402 & 0.078 & 0.406$^{**}$\\
\hspace{1em} & (0.316) & (0.178) & (0.196)\\
\hspace{1em}Incitation - Oui $\times$ Série au bac - ES & 0.077 & 0.075 & 0.086\\
\hspace{1em} & (0.259) & (0.23) & (0.209)\\
\hspace{1em}Incitation - Oui $\times$ Série au bac - S & 0.643$^{**}$ & 0.479 & 0.63$^{*}$\\
\hspace{1em} & (0.303) & (0.4) & (0.341)\\
\hspace{1em}Incitation - Oui $\times$ Série au bac - Autre & $-$0.239 & 0.147 & 0.288\\
\hspace{1em} & (1.106) & (0.331) & (0.508)\\
\hspace{1em} &  &  \vphantom{8} & \\
\hspace{1em}Contrôles & Oui & Oui & \vphantom{4} Oui\\
\hspace{1em}Observations & 218 & 225 & \vphantom{4} 260\\
\hspace{1em}R$^2$ ajusté & 0.402 & 0.536 & 0.552\\
\hspace{1em} &  &  \vphantom{7} & \\
\addlinespace[0.3em]
\multicolumn{4}{l}{\textbf{Panel C : Hétérogénéité en fonction du quintile de l'âge à la rentrée}}\\
\hline
\hspace{1em}Incitation - Oui & 0.143 & 0.277 & 0.287\\
\hspace{1em} & (0.215) & (0.202) & (0.194)\\
\hspace{1em}Quintile de l'âge à la rentrée - Q2 & 0.197 & 0.167 & 0.137\\
\hspace{1em} & (0.233) & (0.188) & (0.186)\\
\hspace{1em}Quintile de l'âge à la rentrée - Q3 & $-$0.047 & 0.077 & $-$0.115\\
\hspace{1em} & (0.251) & (0.209) & (0.196)\\
\hspace{1em}Quintile de l'âge à la rentrée - Q4 & $-$0.044 & 0.166 & 0.006\\
\hspace{1em} & (0.272) & (0.199) & (0.176)\\
\hspace{1em}Quintile de l'âge à la rentrée - Q5 & 0.209 & 0.199 & 0.103\\
\hspace{1em} & (0.423) & (0.251) & (0.231)\\
\hspace{1em}Incitation - Oui $\times$ Quintile de l'âge à la rentrée - Q2 & 0.094 & $-$0.446 & $-$0.286\\
\hspace{1em} & (0.342) & (0.29) & (0.284)\\
\hspace{1em}Incitation - Oui $\times$ Quintile de l'âge à la rentrée - Q3 & 0.138 & $-$0.355 & $-$0.136\\
\hspace{1em} & (0.306) & (0.351) & (0.304)\\
\hspace{1em}Incitation - Oui $\times$ Quintile de l'âge à la rentrée - Q4 & 0.321 & $-$0.145 & $-$0.013\\
\hspace{1em} & (0.321) & (0.312) & (0.276)\\
\hspace{1em}Incitation - Oui $\times$ Quintile de l'âge à la rentrée - Q5 & $-$0.065 & $-$0.206 & $-$0.316\\
\hspace{1em} & (0.323) & (0.262) & (0.269)\\
\hspace{1em} &  &  \vphantom{6} & \\
\hspace{1em}Contrôles & Oui & Oui & \vphantom{3} Oui\\
\hspace{1em}Observations & 218 & 225 & \vphantom{3} 260\\
\hspace{1em}R$^2$ ajusté & 0.388 & 0.531 & 0.542\\
\hspace{1em} &  &  \vphantom{5} & \\
\addlinespace[0.3em]
\multicolumn{4}{l}{\textbf{Panel D : Hétérogénéité en fonction du sexe}}\\
\hline
\hspace{1em}Incitation - Oui & 0.317$^{**}$ & $-$0.049 & 0.135\\
\hspace{1em} & (0.126) & (0.117) & (0.109)\\
\hspace{1em}Sexe - Homme & 0.206 & $-$0.049 & 0.172\\
\hspace{1em} & (0.16) & (0.125) & (0.119)\\
\hspace{1em}Incitation - Oui $\times$ Sexe - Homme & $-$0.169 & 0.311 & 0.021\\
\hspace{1em} & (0.235) & (0.238) & (0.214)\\
\hspace{1em} &  &  \vphantom{4} & \\
\hspace{1em}Contrôles & Oui & Oui & \vphantom{2} Oui\\
\hspace{1em}Observations & 218 & 225 & \vphantom{2} 260\\
\hspace{1em}R$^2$ ajusté & 0.4 & 0.542 & 0.548\\
\hspace{1em} &  &  \vphantom{3} & \\
\addlinespace[0.3em]
\multicolumn{4}{l}{\textbf{Panel E : Hétérogénéité en fonction du pays de naissance}}\\
\hline
\hspace{1em}Incitation - Oui & 0.549$^{*}$ & 0.054 & 0.13\\
\hspace{1em} & (0.316) & (0.332) & (0.289)\\
\hspace{1em}Pays de naissance - En France & 0.008 & $-$0.191 & $-$0.03\\
\hspace{1em} & (0.247) & (0.156) & (0.169)\\
\hspace{1em}Incitation - Oui $\times$ Pays de naissance - En France & $-$0.311 & $-$0.01 & 0.012\\
\hspace{1em} & (0.337) & (0.352) & (0.303)\\
\hspace{1em} &  &  \vphantom{2} & \\
\hspace{1em}Contrôles & Oui & Oui & \vphantom{1} Oui\\
\hspace{1em}Observations & 218 & 225 & \vphantom{1} 260\\
\hspace{1em}R$^2$ ajusté & 0.4 & 0.538 & 0.548\\
\hspace{1em} &  &  \vphantom{1} & \\
\addlinespace[0.3em]
\multicolumn{4}{l}{\textbf{Panel F : Hétérogénéité en fonction du statut de boursier}}\\
\hline
\hspace{1em}Incitation - Oui & 0.39$^{*}$ & 0.167 & 0.28$^{*}$\\
\hspace{1em} & (0.2) & (0.191) & (0.165)\\
\hspace{1em}Statut de boursier - Secondaire & $-$0.012 & 0.025 & 0.124\\
\hspace{1em} & (0.175) & (0.143) & (0.133)\\
\hspace{1em}Incitation - Oui $\times$ Statut de boursier - Secondaire & $-$0.191 & $-$0.187 & $-$0.219\\
\hspace{1em} & (0.239) & (0.22) & (0.199)\\
\hspace{1em} &  &  & \\
\hspace{1em}Contrôles & Oui & Oui & Oui\\
\hspace{1em}Observations & 218 & 225 & 260\\
\hspace{1em}R$^2$ ajusté & 0.4 & 0.54 & 0.55\\*
\end{longtable}
\end{ThreePartTable}
\endgroup{}
\elandscape

\setcounter{table}{0}
\setcounter{figure}{0}

\hypertarget{g20modelsrobcovsgraphnormpop}{%
\section{Robustesse des résultats principaux - versions normalisées des notes}\label{g20modelsrobcovsgraphnormpop}}

\begin{figure}[H]

{\centering \includegraphics[width=1\linewidth]{000_files/figure-latex/g20modelsrobcovsgraphnormpop-1} 

}

\caption{Robustesse des résultats principaux - Vidéos visionnées entièrement}\label{fig:g20modelsrobcovsgraphnormpop}
\end{figure}

\newpage

\setcounter{table}{0}
\setcounter{figure}{0}

\hypertarget{g20modelsrob2}{%
\section{Robustesse des résultats par rapport à l'absence des pics de 20 aux notes de TD et à l'exclusion des étudiants non connectés}\label{g20modelsrob2}}

\begingroup\fontsize{8}{10}\selectfont

\begin{ThreePartTable}
\begin{TableNotes}
\item \textit{Sources :} Base Parcoursup - L1 AES et Eco-Ge, plateforme '\textit{J'ai 20 en maths}', APOGEE (version extraite en mai 2021), calculs de l'auteur.
\item \textit{Notes :} Écart-types robustes entre parenthèses. 
    Les variables dépendantes sont des notes sur 20. Une colonne et un panel donnés correspondent à une régression. Les vidéos visionnées entièrement sont des comptages. TD : Travaux Dirigés. UE : Unité d'Enseignement.
\item Significativité : 10\% * 5\% ** 1\% ***.
\end{TableNotes}
\begin{longtable}[t]{lllllll}
\caption{\label{tab:g20modelsrob2}Robustesse des résultats par rapport à deux éléments}\\
\toprule
\multicolumn{1}{c}{ } & \multicolumn{6}{c}{Variable dépendante : Note } \\
\cmidrule(l{3pt}r{3pt}){2-7}
  & \makecell{Aux TD \\ (1) } & \makecell{Aux TD $\geq 10$ \\ (2) } & \makecell{Aux examens \\ (3) } & \makecell{Aux examens $\geq10$ \\ (4) } & \makecell{À l'UE \\ (5) } & \makecell{À l'UE $\geq 10$ \\ (6) }\\
\midrule
\endfirsthead
\caption[]{\label{tab:g20modelsrob2}Robustesse des résultats par rapport à deux éléments (suite)}\\
\toprule
  & \makecell{Aux TD \\ (1) } & \makecell{Aux TD $\geq 10$ \\ (2) } & \makecell{Aux examens \\ (3) } & \makecell{Aux examens $\geq10$ \\ (4) } & \makecell{À l'UE \\ (5) } & \makecell{À l'UE $\geq 10$ \\ (6) }\\
\midrule
\endhead

\endfoot
\bottomrule
\insertTableNotes
\endlastfoot
\addlinespace[0.3em]
\multicolumn{7}{l}{\textbf{Panel A : Sans les étudiants ayant eu 20 sur 20 aux TD}}\\
\hline
\hspace{1em}Vidéos (complètes) & 2.313$^{*}$ & 0.076 & 0.081 & 0.046 & 1.28$^{*}$ & 0.06\\
 
\hspace{1em} & (1.264) & (0.069) & (0.494) & (0.047) & (0.739) & (0.055)\\
 
\hspace{1em} &  &  &  &  &  \vphantom{2} & \\
 
\hspace{1em}Contrôles & Oui & Oui & Oui & Oui & Oui & \vphantom{1} Oui\\
 
\hspace{1em}Observations & 198 & 198 & 187 & 187 & 198 & 198\\
 
\hspace{1em} &  &  &  &  &  \vphantom{1} & \\
 
\addlinespace[0.3em]
\multicolumn{7}{l}{\textbf{Panel B : Sans les étudiants non connectés}}\\
\hline
\hspace{1em}Vidéos (complètes) & 1.818 & 0.052 & $-$0.251 & 0.043 & 0.322 & $-$0.006\\
 
\hspace{1em} & (1.237) & (0.07) & (0.568) & (0.051) & (0.456) & (0.039)\\
 
\hspace{1em} &  &  &  &  &  & \\
 
\hspace{1em}Contrôles & Oui & Oui & Oui & Oui & Oui & Oui\\
 
\hspace{1em}Observations & 178 & 178 & 181 & 181 & 207 & 207\\*
\end{longtable}
\end{ThreePartTable}
\endgroup{}

\newpage

\hypertarget{bibliographie}{%
\section*{Bibliographie}\label{bibliographie}}
\addcontentsline{toc}{section}{Bibliographie}

\hypertarget{refs}{}
\begin{CSLReferences}{1}{2}
\leavevmode\vadjust pre{\hypertarget{ref-ABD:eal:14}{}}%
Abdulkadiroğlu, A., Angrist, J., \& Pathak, P. (2014). The elite illusion: Achievement effects at Boston and New York exam schools. \emph{Econometrica}, \emph{82}(1), 137‑196.

\leavevmode\vadjust pre{\hypertarget{ref-ACA:12}{}}%
Académie de La Réunion. (2012). Repères statistiques, édition 2012. In \emph{Repères statistiques}.

\leavevmode\vadjust pre{\hypertarget{ref-AJI:15}{}}%
Ajir, S. (2015). Des ménages toujours plus petits. \emph{INSEE Flash Réunion}, \emph{40}.

\leavevmode\vadjust pre{\hypertarget{ref-ALE:eal:13}{}}%
Alet, E., Bonnal, L., \& Favard, P. (2013). Repetition: Medicine for a short-run remission. \emph{Annals of Economics and Statistics/Annales d'Économie et de Statistiques}, \emph{111/112}, 227‑250.

\leavevmode\vadjust pre{\hypertarget{ref-ALI:12TRUE}{}}%
Aliprantis, D. (2012). Redshirting, compulsory schooling laws, and educational attainment. \emph{Journal of Educational and Behavioral Statistics}, \emph{37}(2), 316‑338.

\leavevmode\vadjust pre{\hypertarget{ref-AMM:PIS:09}{}}%
Ammermueller, A., \& Pischke, J.-S. (2009). Peer effects in European primary schools: Evidence from the progress in international reading literacy study. \emph{Journal of Labor Economics}, \emph{27}(3), 315‑348.

\leavevmode\vadjust pre{\hypertarget{ref-ANG:IMB:95}{}}%
Angrist, J. D., \& Imbens, G. W. (1995). Two-stage least squares estimation of average causal effects in models with variable treatment intensity. \emph{Journal of the American statistical Association}, \emph{90}(430), 431‑442.

\leavevmode\vadjust pre{\hypertarget{ref-ANG:eal:96}{}}%
Angrist, J. D., Imbens, G. W., \& Rubin, D. B. (1996). Identification of causal effects using instrumental variables. \emph{Journal of the American statistical Association}, \emph{91}(434), 444‑455.

\leavevmode\vadjust pre{\hypertarget{ref-ANG:LAN:04}{}}%
Angrist, J. D., \& Lang, K. (2004). Does school integration generate peer effects? Evidence from Boston's Metco Program. \emph{American Economic Review}, \emph{94}(5), 1613‑1634.

\leavevmode\vadjust pre{\hypertarget{ref-ANG:PIS:08}{}}%
Angrist, J. D., \& Pischke, J.-S. (2009). \emph{Mostly harmless econometrics: An empiricist's companion}. Princeton university press.

\leavevmode\vadjust pre{\hypertarget{ref-RAP:GON:20}{}}%
Arruda Raposo, I. P. de, \& Gonçalves, M. B. C. (2020). Peer effects and educational achievement: evidence of causal effects using age at school entry as exogenous variation for Peer quality. \emph{EconomiA}, \emph{21}(1), 18‑37.

\leavevmode\vadjust pre{\hypertarget{ref-ASA:CHA:08}{}}%
Asadullah, M. N. (2008). \emph{Social interactions and student achievement in a developing country: An instrumental variables approach} (Vol. 4508). World Bank Publications.

\leavevmode\vadjust pre{\hypertarget{ref-ATT:COH:18}{}}%
Attar, I., \& Cohen-Zada, D. (2018). The effect of school entrance age on educational outcomes: Evidence using multiple cutoff dates and exact date of birth. \emph{Journal of Economic Behavior \& Organization}, \emph{153}, 38‑57.

\leavevmode\vadjust pre{\hypertarget{ref-AVV:eal:14}{}}%
Avvisati, F., Gurgand, M., Guyon, N., \& Maurin, E. (2014). Getting parents involved: A field experiment in deprived schools. \emph{Review of Economic Studies}, \emph{81}(1), 57‑83.

\leavevmode\vadjust pre{\hypertarget{ref-BAH:08}{}}%
Bahr, P. R. (2008). Does mathematics remediation work?: A comparative analysis of academic attainment among community college students. \emph{Research in Higher Education}, \emph{49}(5), 420‑450.

\leavevmode\vadjust pre{\hypertarget{ref-BAK:eal:19}{}}%
Baktavatsalou, R., Chaussy, C., Delvoye, S., Legros, F., Parvedy, J.-E., \& Payet, F. (2019). Projections du nombre d'élèves à La Réunion. Une baisse modérée du nombre d'élèves scolarisés à l'horizon 2030. \emph{INSEE Analyses}, \emph{42}.

\leavevmode\vadjust pre{\hypertarget{ref-BAN:eal:07}{}}%
Banerjee, A. V., Cole, S., Duflo, E., \& Linden, L. (2007). Remedying education: Evidence from two randomized experiments in India. \emph{The Quarterly Journal of Economics}, \emph{122}(3), 1235‑1264.

\leavevmode\vadjust pre{\hypertarget{ref-BAR:eal:09}{}}%
Barrow, L., Markman, L., \& Rouse, C. E. (2009). Technology's edge: The educational benefits of computer-aided instruction. \emph{American Economic Journal: Economic Policy}, \emph{1}(1), 52‑74.

\leavevmode\vadjust pre{\hypertarget{ref-BAR:LAN:09}{}}%
Barua, R., \& Lang, K. (2009). \emph{School entry, educational attainment and quarter of birth: A cautionary tale of LATE} (Working paper nᵒ 15236). National Bureau of Economic Research.

\leavevmode\vadjust pre{\hypertarget{ref-BEC:eal:21}{}}%
Bechichi, N., Grenet, J., \& Thebault, G. (2021). \emph{S{é}gr{é}gation {à} l'entr{é}e des {é}tudes sup{é}rieures en France et en r{é}gion parisienne: quels effets du passage {à} Parcoursup?} (Document de travail nᵒ 2021-03). INSEE.

\leavevmode\vadjust pre{\hypertarget{ref-BEC:09}{}}%
Becker, G. S. (2009). \emph{Human capital: A theoretical and empirical analysis, with special reference to education}. University of Chicago press.

\leavevmode\vadjust pre{\hypertarget{ref-BED:DHU:06}{}}%
Bedard, K., \& Dhuey, E. (2006). The persistence of early childhood maturity: International evidence of long-run age effects. \emph{The Quarterly Journal of Economics}, \emph{121}(4), 1437‑1472.

\leavevmode\vadjust pre{\hypertarget{ref-BEL:eal:17}{}}%
Bellity, E., Gilles, F., \& L'Horty, Y. (2017). Does practicing literacy skills improve academic performance in first-year university students : Results from a randomized experiment. \emph{Unpublished Manuscript}, 2017‑2002.

\leavevmode\vadjust pre{\hypertarget{ref-BER:eal:19}{}}%
Bernigole, V., Bret, A., Chabanon, L., Roussel, L., \& Verlet, I. (2019). PISA 2018 : culture mathématique, culture scientifique et vie de l'élève. \emph{Note d'information de la DEPP}, \emph{19.50}.

\leavevmode\vadjust pre{\hypertarget{ref-BLA:eal:11}{}}%
Black, S. E., Devereux, P. J., \& Salvanes, K. G. (2011). Too young to leave the nest? The effects of school starting age. \emph{The Review of Economics and Statistics}, \emph{93}(2), 455‑467.

\leavevmode\vadjust pre{\hypertarget{ref-BOE:HEL:09}{}}%
Boesen, J., \& Helenius, O. (2009). Am{é}liorer l'enseignement des math{é}matiques. Le cas de la Su{è}de. \emph{Revue internationale d'{é}ducation de S{è}vres}, \emph{51}, 91‑101.

\leavevmode\vadjust pre{\hypertarget{ref-BOO:eal:17}{}}%
Booij, A. S., Leuven, E., \& Oosterbeek, H. (2017). Ability peer effects in university: Evidence from a randomized experiment. \emph{The review of economic studies}, \emph{84}(2), 547‑578.

\leavevmode\vadjust pre{\hypertarget{ref-BOO:CAC:01}{}}%
Boozer, M. A., \& Cacciola, S. E. (2001). \emph{Inside the Black Box of Project Star: Estimation of Peer Effects using Experimental Data} (Center discussion paper nᵒ 832). Yale University; Citeseer.

\leavevmode\vadjust pre{\hypertarget{ref-BOU:eal:14}{}}%
Boucher, V., Bramoullé, Y., Djebbari, H., \& Fortin, B. (2014). Do peers affect student achievement? evidence from canada using group size variation. \emph{Journal of applied econometrics}, \emph{29}(1), 91‑109.

\leavevmode\vadjust pre{\hypertarget{ref-BOU:eal:16}{}}%
Boudesseul, G., Caro, P., Grelet, Y., Minassian, L., Monso, O., \& Vivent, C. (2016). \emph{Atlas des risques sociaux d'{é}chec scolaire : l'exemple du d{é}crochage, France m{é}tropolitaine et Dom.} (Atlas académique des risques sociaux d\textquotesingle échec scolaire nᵒ 2016.2). Ministère de l'Éducation Nationale, de l'Enseignement Supérieur et de la Recherche.

\leavevmode\vadjust pre{\hypertarget{ref-BOU:MAI:18}{}}%
Boutchenik, B., \& Maillard, S. (2018). \emph{Peer effects with peer and student heterogeneity: An assessment for french baccalaureat}. Contribution aux 13èmes Journées de Méthodologie Statistique de l'Insee.

\leavevmode\vadjust pre{\hypertarget{ref-BOZ:eal:20}{}}%
Bozzoli, C. G., Gonzalez-Gadea, M. L., Hermida, M. J., Navarro, L., Olego, T., \& Klingberg, T. (2020). \emph{Digital, mathematical and cognitive training: Evidence from a randomized trial}. PsyArXiv.

\leavevmode\vadjust pre{\hypertarget{ref-BRA:eal:09}{}}%
Bramoullé, Y., Djebbari, H., \& Fortin, B. (2009). Identification of peer effects through social networks. \emph{Journal of econometrics}, \emph{150}(1), 41‑55.

\leavevmode\vadjust pre{\hypertarget{ref-BRA:eal:20}{}}%
Bramoullé, Y., Djebbari, H., \& Fortin, B. (2020). Peer effects in networks: A survey. \emph{Annual Review of Economics}, \emph{12}, 603‑629.

\leavevmode\vadjust pre{\hypertarget{ref-BRE:eal:21}{}}%
Breton, D., Temporal, F., Marie, C.-V., \& Antoine, R. (2021). Enjeux d{é}mographiques des d{é}partements et r{é}gions d'outre-mer. \emph{Regards}, \emph{59}(1), 25‑39.

\leavevmode\vadjust pre{\hypertarget{ref-BRI:eal:79}{}}%
Bridge, R. G., \& others. (1979). \emph{The Determinants of Educational Outcomes: The Impact of Families, Peers, Teachers, and Schools.} Ballinger Publishing Company.

\leavevmode\vadjust pre{\hypertarget{ref-BRI:RUD:11}{}}%
Brière, L., \& Rudolf, M. (2011). Comparaison entre pays des coûts de l'éducation : des sources de financement aux dépenses. \emph{Éducation \& formations}, \emph{80}, 31‑40.

\leavevmode\vadjust pre{\hypertarget{ref-BRI:VIG:17}{}}%
Britton, J., \& Vignoles, A. (2017). Education production functions. In \emph{Handbook of contemporary education economics} (p. 246‑271). Edward Elgar Publishing.

\leavevmode\vadjust pre{\hypertarget{ref-BRO:10}{}}%
Brodaty, T. (2010). Les effets de Pairs dans l'{É}ducation: une Revue de Litt{é}rature. \emph{Revue d'{é}conomie politique}, \emph{120}(5), 739‑757.

\leavevmode\vadjust pre{\hypertarget{ref-BRO:GUR:16}{}}%
Brodaty, T., \& Gurgand, M. (2016). Good peers or good teachers? Evidence from a French University. \emph{Economics of Education Review}, \emph{54}, 62‑78.

\leavevmode\vadjust pre{\hypertarget{ref-BRU:eal:10}{}}%
Brunello, G., De Paola, M., \& Scoppa, V. (2010). Peer effects in higher education: Does the field of study matter? \emph{Economic Inquiry}, \emph{48}(3), 621‑634.

\leavevmode\vadjust pre{\hypertarget{ref-BRU:eal:15}{}}%
Brunello, G., Fort, M., Schneeweis, N., \& Winter-Ebmer, R. (2016). The causal effect of education on health: What is the role of health behaviors? \emph{Health economics}, \emph{25}(3), 314‑336.

\leavevmode\vadjust pre{\hypertarget{ref-BUC:HUN:13}{}}%
Buckles, K. S., \& Hungerman, D. M. (2013). Season of birth and later outcomes: Old questions, new answers. \emph{Review of Economics and Statistics}, \emph{95}(3), 711‑724.

\leavevmode\vadjust pre{\hypertarget{ref-BUR:SAS:13}{}}%
Burke, M. A., \& Sass, T. R. (2013). Classroom peer effects and student achievement. \emph{Journal of Labor Economics}, \emph{31}(1), 51‑82.

\leavevmode\vadjust pre{\hypertarget{ref-BUR:eal:12}{}}%
Burns, M. K., Kanive, R., \& DeGrande, M. (2012). Effect of a computer-delivered math fact intervention as a supplemental intervention for math in third and fourth grades. \emph{Remedial and Special Education}, \emph{33}(3), 184‑191.

\leavevmode\vadjust pre{\hypertarget{ref-CAR:01}{}}%
Card, D. (2001). Estimating the return to schooling: Progress on some persistent econometric problems. \emph{Econometrica}, \emph{69}(5), 1127‑1160.

\leavevmode\vadjust pre{\hypertarget{ref-CAR:eal:15}{}}%
Carlsson, M., Dahl, G. B., Öckert, B., \& Rooth, D.-O. (2015). The effect of schooling on cognitive skills. \emph{Review of Economics and Statistics}, \emph{97}(3), 533‑547.

\leavevmode\vadjust pre{\hypertarget{ref-CAR:18}{}}%
Caro, P. (2018). \emph{In{é}galit{é}s scolaires d'origine territoriale en France m{é}tropolitaine et d'Outre-mer}. CNESCO (Conseil national d'évaluation du système scolaire).

\leavevmode\vadjust pre{\hypertarget{ref-CAR:eal:09}{}}%
Carrell, S. E., Fullerton, R. L., \& West, J. E. (2009). Does your cohort matter? Measuring peer effects in college achievement. \emph{Journal of Labor Economics}, \emph{27}(3), 439‑464.

\leavevmode\vadjust pre{\hypertarget{ref-CAR:eal:13}{}}%
Carrell, S. E., Sacerdote, B. I., \& West, J. E. (2013). From natural variation to optimal policy? The importance of endogenous peer group formation. \emph{Econometrica}, \emph{81}(3), 855‑882.

\leavevmode\vadjust pre{\hypertarget{ref-CAR:12}{}}%
Carron, A. (2012). \emph{Parcours scolaire des élèves de Section d'Enseignement Général et Professionnel Adapté à l'île de La Réunion : analyse et processus} {[}Thèse de doctorat{]}. La Réunion.

\leavevmode\vadjust pre{\hypertarget{ref-CAS:LEW:06}{}}%
Cascio, E., \& Lewis, E. (2006). Schooling and the armed forces qualifying test evidence from school-entry laws. \emph{Journal of Human resources}, \emph{41}(2), 294‑318.

\leavevmode\vadjust pre{\hypertarget{ref-CAS:08}{}}%
Cascio, E., \& others. (2008). How and why does age at kindergarten entry matter? \emph{FRBSF Economic Letter}, \emph{2008-24}.

\leavevmode\vadjust pre{\hypertarget{ref-CAS:SCH:16}{}}%
Cascio, E., \& Schanzenbach, D. (2016). First in the class? Age and the education production function. \emph{Education Finance and Policy}, \emph{11}(3), 225‑250.

\leavevmode\vadjust pre{\hypertarget{ref-CAY:19}{}}%
Cayouette-Remblière, J., Moulin, L., \& Dutreuilh, C. (2019). How inequalities in academic performance evolve in lower secondary school in France: A longitudinal follow-up of students. \emph{Population}, \emph{74}(4), 507‑540.

\leavevmode\vadjust pre{\hypertarget{ref-CHA:eal:19}{}}%
Chabanon, L., Durand de Monestrol, H., \& Verlet, I. (2019). PISA 2018 : stabilit{é} des r{é}sultats en compr{é}hension de l'{é}crit. \emph{Note d'information de la DEPP}, \emph{19.49}.

\leavevmode\vadjust pre{\hypertarget{ref-CHE:PAR:94}{}}%
Chevillon, M., \& Parain, C. (1994). Parcours scolaires et milieu social {à} la R{é}union. \emph{Éducation et formations}, \emph{38}, 51‑63.

\leavevmode\vadjust pre{\hypertarget{ref-CLE:eal:11}{}}%
Clements, D. H., Sarama, J., Spitler, M. E., Lange, A. A., \& Wolfe, C. B. (2011). Mathematics learned by young children in an intervention based on learning trajectories: A large-scale cluster randomized trial. \emph{Journal for Research in Mathematics Education}, \emph{42}(2), 127‑166.

\leavevmode\vadjust pre{\hypertarget{ref-COL:68}{}}%
Coleman, J. S. (1968). Equality of educational opportunity. \emph{Integrated Education}, \emph{6}(5), 19‑28.

\leavevmode\vadjust pre{\hypertarget{ref-COL:LEC:17}{}}%
Colmant, M., \& Le Cam, M. (2017). PIRLS 2016 : évaluation internationale des élèves de CM1 en compréhension de l'écrit. Évolution des performances sur quinze ans. \emph{Note d'information de la DEPP}, \emph{17.24}.

\leavevmode\vadjust pre{\hypertarget{ref-COL:LEC:20}{}}%
Colmant, M., \& Le Cam, M. (2020). TIMSS 2019 - Évaluation internationale des élèves de CM1 en mathématiques et en sciences : les résultats de la France toujours en retrait. \emph{Note d'information de la DEPP}, \emph{20.46}.

\leavevmode\vadjust pre{\hypertarget{ref-COU:20}{}}%
Cour des comptes. (2020). \emph{Le système éducatif dans les académies ultramarines}. Communication à la commission des finances du Sénat.

\leavevmode\vadjust pre{\hypertarget{ref-CRA:eal:14}{}}%
Crawford, C., Dearden, L., \& Greaves, E. (2014). The drivers of month-of-birth differences in children's cognitive and non-cognitive skills. \emph{Journal of the Royal Statistical Society: Series A (Statistics in Society)}, \emph{177}(4), 829‑860.

\leavevmode\vadjust pre{\hypertarget{ref-CRA:eal:07}{}}%
Crawford, C., Dearden, L., \& Meghir, C. (2007). \emph{When you are born matters: The impact of date of birth on child cognitive outcomes in England}. Institute for Fiscal Studies.

\leavevmode\vadjust pre{\hypertarget{ref-CUN:HEC:07}{}}%
Cunha, F., \& Heckman, J. (2007). The technology of skill formation. \emph{American Economic Review}, \emph{97}(2), 31‑47.

\leavevmode\vadjust pre{\hypertarget{ref-DAL:21}{}}%
Daleau-Gauvin, L. (2021). \emph{La Co-alphab{é}tisation cr{é}ole-fran{ç}ais comme facteur de r{é}ussite scolaire {à} la R{é}union} {[}Thèse de doctorat{]}. Universit{é} de la R{é}union.

\leavevmode\vadjust pre{\hypertarget{ref-DAL:eal:11}{}}%
Dalous, J., Jeljoul, M., \& Rudolf, M. (2011). La d{é}pense par {é}l{è}ve ou {é}tudiant en France et dans l'OCDE. \emph{Note d'information de la DEPP}, \emph{12.29}.

\leavevmode\vadjust pre{\hypertarget{ref-DAT:06}{}}%
Datar, A. (2006). Does delaying kindergarten entrance give children a head start? \emph{Economics of Education Review}, \emph{25}(1), 43‑62.

\leavevmode\vadjust pre{\hypertarget{ref-DAV:04}{}}%
Davezies, L. (2004). Influence des caract{é}ristiques du groupe des pairs sur la scolarit{é} {é}l{é}mentaire. \emph{{É}ducation et formations}, \emph{72}, 171.

\leavevmode\vadjust pre{\hypertarget{ref-DAV:FLA:08}{}}%
Davidson, R., \& Flachaire, E. (2008). The wild bootstrap, tamed at last. \emph{Journal of Econometrics}, \emph{146}(1), 162‑169.

\leavevmode\vadjust pre{\hypertarget{ref-CHA:eal:21}{}}%
De Chaisemartin, C., Daviot, Q., Gurgand, M., \& Kern, S. (2021). Lutter contre les in{é}galit{é}s d{è}s la petite enfance : {é}valuation {à} grande {é}chelle du programme Parler Bambin. \emph{Notes de l'IPP}, \emph{72}.

\leavevmode\vadjust pre{\hypertarget{ref-DEG:98}{}}%
De Guzmán, M., Hodgson, B. R., Robert, A., \& Villani, V. (1998). Difficulties in the passage from secondary to tertiary education. \emph{Proceedings of the international Congress of Mathematicians}, \emph{3}, 747‑762.

\leavevmode\vadjust pre{\hypertarget{ref-DEE:20}{}}%
Dee, T. S. (2020). Education and civic engagement. In \emph{The economics of education} (p. 103‑108). Elsevier.

\leavevmode\vadjust pre{\hypertarget{ref-DEP:21GEO}{}}%
DEPP. (2021a). Géographie de l'École 2021. In \emph{Géographie de l'école}.

\leavevmode\vadjust pre{\hypertarget{ref-DEP:21}{}}%
DEPP. (2021b). Repères et références statistiques 2021. In \emph{Repères et références statistiques}.

\leavevmode\vadjust pre{\hypertarget{ref-DHU:eal:17}{}}%
Dhuey, E., Figlio, D., Karbownik, K., \& Roth, J. (2017). \emph{School starting age and cognitive development} (Working paper nᵒ 23660). National Bureau of Economic Research.

\leavevmode\vadjust pre{\hypertarget{ref-DOB:FER:10}{}}%
Dobkin, C., \& Ferreira, F. (2010). Do school entry laws affect educational attainment and labor market outcomes? \emph{Economics of education review}, \emph{29}(1), 40‑54.

\leavevmode\vadjust pre{\hypertarget{ref-DOI:01}{}}%
Doisneau, L. (2001). \emph{Bilan d{é}mographique 2000: une ann{é}e de naissances et de mariages}. INSEE.

\leavevmode\vadjust pre{\hypertarget{ref-DUP:eal:08}{}}%
Du Preez, J., Steyn, T., \& Owen, R. (2008). Mathematical preparedness for tertiary mathematics--a need for focused intervention in the first year? \emph{Perspectives in Education}, \emph{26}(1).

\leavevmode\vadjust pre{\hypertarget{ref-DUF:eal:11}{}}%
Duflo, E., Dupas, P., \& Kremer, M. (2011). Peer effects, teacher incentives, and the impact of tracking: Evidence from a randomized evaluation in Kenya. \emph{American Economic Review}, \emph{101}(5), 1739‑1774.

\leavevmode\vadjust pre{\hypertarget{ref-DUF:eal:07}{}}%
Duflo, E., Glennerster, R., \& Kremer, M. (2007). Using randomization in development economics research: A toolkit. \emph{Handbook of development economics}, \emph{4}, 3895‑3962.

\leavevmode\vadjust pre{\hypertarget{ref-ELD:LUB:09}{}}%
Elder, T. E., \& Lubotsky, D. H. (2009). Kindergarten entrance age and children's achievement impacts of state policies, family background, and peers. \emph{Journal of human resources}, \emph{44}(3), 641‑683.

\leavevmode\vadjust pre{\hypertarget{ref-ENG:eal:15}{}}%
Engelbrecht, J., \& Harding, A. (2015). Interventions to improve teaching and learning in first year mathematics courses. \emph{International Journal of Mathematical Education in Science and Technology}, \emph{46}(7), 1046‑1060.

\leavevmode\vadjust pre{\hypertarget{ref-ENG:00}{}}%
Engeneering Council. (2000). Measuring the mathematics problem. \emph{Note par pays - France}.

\leavevmode\vadjust pre{\hypertarget{ref-EPP:ROM:11}{}}%
Epple, D., \& Romano, R. E. (2011). Peer effects in education: A survey of the theory and evidence. In \emph{Handbook of social economics} (Vol. 1, p. 1053‑1163). Elsevier.

\leavevmode\vadjust pre{\hypertarget{ref-FAUL:eal:14}{}}%
Faulkner, F., Hannigan, A., \& Fitzmaurice, O. (2014). The role of prior mathematical experience in predicting mathematics performance in higher education. \emph{International Journal of Mathematical Education in Science and Technology}, \emph{45}(5), 648‑667.

\leavevmode\vadjust pre{\hypertarget{ref-FIO:STE:21}{}}%
Fiorini, M., \& Stevens, K. (2021). Scrutinizing the Monotonicity Assumption in IV and fuzzy RD designs. \emph{Oxford Bulletin of Economics and Statistics}, \emph{83}(6), 1475‑1526.

\leavevmode\vadjust pre{\hypertarget{ref-FLE:12}{}}%
Fleury, N. (2012). {Â}ge d'entr{é}e {à} l'{é}cole, accumulation de capital humain et parcours scolaire. \emph{Revue {é}conomique}, \emph{63}(3), 475‑490.

\leavevmode\vadjust pre{\hypertarget{ref-FOU:eal:17}{}}%
Fougère, D., Kiefer, N., Monso, O., \& Pirus, C. (2017). La concentration des enfants {é}trangers dans les classes de coll{è}ges. \emph{Éducation et formations}, \emph{139}(95).

\leavevmode\vadjust pre{\hypertarget{ref-FRE:OCK:05}{}}%
Fredriksson, P., \& Ockert, B. (2005). \emph{Is early learning really more productive? The effect of school starting age on school and labor market performance} (Discussion paper nᵒ 1659). IZA.

\leavevmode\vadjust pre{\hypertarget{ref-FRU:14}{}}%
Fruehwirth, J. C. (2014). Can achievement peer effect estimates inform policy? a view from inside the black box. \emph{Review of Economics and Statistics}, \emph{96}(3), 514‑523.

\leavevmode\vadjust pre{\hypertarget{ref-GAR:84}{}}%
Garen, J. (1984). The returns to schooling: A selectivity bias approach with a continuous choice variable. \emph{Econometrica (pre-1986)}, \emph{52}(5), 1199.

\leavevmode\vadjust pre{\hypertarget{ref-GEL:IMB:19}{}}%
Gelman, A., \& Imbens, G. (2019). Why high-order polynomials should not be used in regression discontinuity designs. \emph{Journal of Business \& Economics Statistics}, \emph{37}(3), 447‑456.

\leavevmode\vadjust pre{\hypertarget{ref-GIB:TEL:16}{}}%
Gibbons, S., \& Telhaj, S. (2016). Peer effects: Evidence from secondary school transition in England. \emph{Oxford Bulletin of Economics and Statistics}, \emph{78}(4), 548‑575.

\leavevmode\vadjust pre{\hypertarget{ref-GIV:20}{}}%
Givord, P. (2020). \emph{How a student's month of birth is linked to performance at school: New evidence from PISA} (Working paper nᵒ 221). OECD.

\leavevmode\vadjust pre{\hypertarget{ref-GOU:MAU:07}{}}%
Goux, D., \& Maurin, E. (2007). Close neighbours matter: Neighbourhood effects on early performance at school. \emph{The Economic Journal}, \emph{117}(523), 1193‑1215.

\leavevmode\vadjust pre{\hypertarget{ref-GRE:09}{}}%
Grenet, J. (2009). Academic performance, educational trajectories and the persistence of date of birth effects. Evidence from France. \emph{Unpublished manuscript}.

\leavevmode\vadjust pre{\hypertarget{ref-GRE:10}{}}%
Grenet, J. (2010). La date de naissance influence-t-elle les trajectoires scolaires et professionnelles? \emph{Revue {é}conomique}, \emph{61}(3), 589‑598.

\leavevmode\vadjust pre{\hypertarget{ref-GRI:RAS:14}{}}%
Griffith, A. L., \& Rask, K. N. (2014). Peer effects in higher education: A look at heterogeneous impacts. \emph{Economics of Education Review}, \emph{39}, 65‑77.

\leavevmode\vadjust pre{\hypertarget{ref-GUE:VAN:22}{}}%
Gueudet, G., \& Vandebrouck, F. (2022). \emph{Transition secondaire-sup{é}rieur: Ce que nous apprend la recherche en didactique des math{é}matiques}.

\leavevmode\vadjust pre{\hypertarget{ref-HAE:GHY:17}{}}%
Haelermans, C., \& Ghysels, J. (2017). The effect of individualized digital practice at home on math skills--Evidence from a two-stage experiment on whether and why it works. \emph{Computers \& Education}, \emph{113}, 119‑134.

\leavevmode\vadjust pre{\hypertarget{ref-HAH:eal:01}{}}%
Hahn, J., Todd, P., \& Van der Klaauw, W. (2001). Identification and estimation of treatment effects with a regression-discontinuity design. \emph{Econometrica}, \emph{69}(1), 201‑209.

\leavevmode\vadjust pre{\hypertarget{ref-HAM:KOL:12}{}}%
Hámori, S., \& Köllő, J. (2012). Whose children gain from starting school later?--evidence from Hungary. \emph{Educational Research and Evaluation}, \emph{18}(5), 459‑488.

\leavevmode\vadjust pre{\hypertarget{ref-HAN:eal:03}{}}%
Hanushek, E. A., Kain, J. F., Markman, J. M., \& Rivkin, S. G. (2003). Does peer ability affect student achievement? \emph{Journal of applied econometrics}, \emph{18}(5), 527‑544.

\leavevmode\vadjust pre{\hypertarget{ref-HAN:eal:15}{}}%
Hanushek, E. A., Schwerdt, G., Wiederhold, S., \& Woessmann, L. (2015). Returns to skills around the world: Evidence from PIAAC. \emph{European Economic Review}, \emph{73}, 103‑130.

\leavevmode\vadjust pre{\hypertarget{ref-HAN:eal:17}{}}%
Hanushek, E. A., Schwerdt, G., Wiederhold, S., \& Woessmann, L. (2017). Coping with change: International differences in the returns to skills. \emph{Economics Letters}, \emph{153}, 15‑19.

\leavevmode\vadjust pre{\hypertarget{ref-HAN:WOE:15}{}}%
Hanushek, E. A., \& Woessmann, L. (2015). \emph{The knowledge capital of nations: Education and the economics of growth}. MIT press.

\leavevmode\vadjust pre{\hypertarget{ref-HAN:WOE:20}{}}%
Hanushek, E. A., \& Woessmann, L. (2020). Education, knowledge capital, and economic growth. \emph{The economics of education}, 171‑182.

\leavevmode\vadjust pre{\hypertarget{ref-HEC:12}{}}%
Heckman, J. J. (2012). Invest in early childhood development: Reduce deficits, strengthen the economy. \emph{The Heckman Equation}, \emph{7}, 1‑2.

\leavevmode\vadjust pre{\hypertarget{ref-HEC:eal:06}{}}%
Heckman, J. J., Lochner, L. J., \& Todd, P. E. (2006). Earnings functions, rates of return and treatment effects: The Mincer equation and beyond. \emph{Handbook of the Economics of Education}, \emph{1}, 307‑458.

\leavevmode\vadjust pre{\hypertarget{ref-HEN:10}{}}%
Henningsen, A. (2010). Estimating censored regression models in R using the censReg Package. \emph{R package vignettes}, \emph{5}, 12.

\leavevmode\vadjust pre{\hypertarget{ref-HOL:88}{}}%
Holland, P. W. (1988). Causal inference, path analysis and recursive structural equations models. \emph{ETS Research Report Series}, \emph{1988}(1), i‑50.

\leavevmode\vadjust pre{\hypertarget{ref-HOW:eal:19}{}}%
Howard, E., Meehan, M., \& Parnell, A. (2019). Quantifying participation in, and the effectiveness of, remediating assessment in a university mathematics module. \emph{Assessment \& Evaluation in Higher Education}, \emph{44}(1), 97‑110.

\leavevmode\vadjust pre{\hypertarget{ref-HOX:00}{}}%
Hoxby, C. (2000). \emph{Peer effects in the classroom: Learning from gender and race variation} (Working paper nᵒ 7867). National Bureau of Economic Research.

\leavevmode\vadjust pre{\hypertarget{ref-HOX:WEI:05}{}}%
Hoxby, C., \& Weingarth, G. (2005). \emph{Taking race out of the equation: School reassignment and the structure of peer effects}. Working paper.

\leavevmode\vadjust pre{\hypertarget{ref-HYD:eal:90}{}}%
Hyde, J. S., Fennema, E., \& Lamon, S. J. (1990). Gender differences in mathematics performance: a meta-analysis. \emph{Psychological bulletin}, \emph{107}(2), 139.

\leavevmode\vadjust pre{\hypertarget{ref-IDE:20}{}}%
IDEOM. (2020). \emph{Rapport annuel économique La Réunion 2020}. Institut d'Émission des Départements d'Outre-mer.

\leavevmode\vadjust pre{\hypertarget{ref-IMB:ANG:94}{}}%
Imbens, G. W., \& Angrist, J. D. (1994). Identification and Estimation of Local Average Treatment Effects. \emph{Econometrica}, \emph{62}(2), 467‑475. \url{http://www.jstor.org/stable/2951620}

\leavevmode\vadjust pre{\hypertarget{ref-IMB:LEM:08}{}}%
Imbens, G. W., \& Lemieux, T. (2008). Regression discontinuity designs: A guide to practice. \emph{Journal of econometrics}, \emph{142}(2), 615‑635.

\leavevmode\vadjust pre{\hypertarget{ref-IMB:eal:12}{}}%
Imberman, S. A., Kugler, A. D., \& Sacerdote, B. I. (2012). Katrina's children: Evidence on the structure of peer effects from hurricane evacuees. \emph{American Economic Review}, \emph{102}(5), 2048‑2082.

\leavevmode\vadjust pre{\hypertarget{ref-INS:20}{}}%
INSEE. (2020). \emph{Tableaux de l'économie française, édition 2020}.

\leavevmode\vadjust pre{\hypertarget{ref-INS:21}{}}%
INSEE. (2021). La France et ses territoires, édition 2021. In \emph{INSEE Références}.

\leavevmode\vadjust pre{\hypertarget{ref-IZA:DIC:20}{}}%
Izaguirre, A., \& Di Capua, L. (2020). Exploring peer effects in education in Latin America and the Caribbean. \emph{Research in Economics}, \emph{74}(1), 73‑86.

\leavevmode\vadjust pre{\hypertarget{ref-JAC:eal:16}{}}%
Jaciw, A. P., Hegseth, W. M., Lin, L., Toby, M., Newman, D., Ma, B., \& Zacamy, J. (2016). Assessing impacts of Math in Focus, a ``Singapore Maths? program. \emph{Journal of Research on Educational Effectiveness}, \emph{9}(4), 473‑502.

\leavevmode\vadjust pre{\hypertarget{ref-JAC:PRE:16}{}}%
Jacobs, M., \& Pretorius, E. (2016). First-year seminar intervention: Enhancing firstyear mathematics performance at the University of Johannesburg. \emph{Journal of Student Affairs in Africa}, \emph{4}(1), 77‑86.

\leavevmode\vadjust pre{\hypertarget{ref-KAI:17}{}}%
Kaila, M. (2017). \emph{The Effects of Relative School Starting Age on Educational Outcomes in Finland} (Working paper nᵒ 84). VATT Institute for Economic Research.

\leavevmode\vadjust pre{\hypertarget{ref-KAJ:LOV:05}{}}%
Kajander, A., \& Lovric, M. (2005). Transition from secondary to tertiary mathematics: McMaster University experience. \emph{International Journal of Mathematical Education in Science and Technology}, \emph{36}(2-3), 149‑160.

\leavevmode\vadjust pre{\hypertarget{ref-KAR:AKM:19}{}}%
Karademir, A., \& Akman, B. (2019). Effect of Inquiry-Based Mathematics Activities on Preschoolers' Math Skills. \emph{International Journal of Progressive Education}, \emph{15}(5), 198‑215.

\leavevmode\vadjust pre{\hypertarget{ref-KAW:11}{}}%
Kawaguchi, D. (2011). Actual age at school entry, educational outcomes, and earnings. \emph{Journal of the Japanese and International Economies}, \emph{25}(2), 64‑80.

\leavevmode\vadjust pre{\hypertarget{ref-KIM:21}{}}%
Kim, T. (2021). Age culture, school-entry cutoff, and the choices of birth month and school-entry timing in South Korea. \emph{Journal of Demographic Economics}, \emph{87}(1), 33‑65.

\leavevmode\vadjust pre{\hypertarget{ref-KUC:11}{}}%
Kucharčíková, A. (2011). Human capital -- definitions and approaches. \emph{Human Resources Management \& Ergonomics}, \emph{5}(2), 60‑70.

\leavevmode\vadjust pre{\hypertarget{ref-LAN:eal:18}{}}%
Landaud, F., Ly, S.-T., \& Maurin, E. (2018). Competitive Schools and the Gender Gap in the Choice of Field of Study. \emph{Journal of Human Resources}, 0617‑8864R.

\leavevmode\vadjust pre{\hypertarget{ref-LAN:eal:84}{}}%
Langer, P., Kalk, J. M., \& Searls, D. T. (1984). Age of admission and trends in achievement: A comparison of Blacks and Caucasians. \emph{American Educational Research Journal}, \emph{21}(1), 61‑78.

\leavevmode\vadjust pre{\hypertarget{ref-LAV:eal:12}{}}%
Lavy, V., Silva, O., \& Weinhardt, F. (2012). The good, the bad, and the average: Evidence on ability peer effects in schools. \emph{Journal of Labor Economics}, \emph{30}(2), 367‑414.

\leavevmode\vadjust pre{\hypertarget{ref-LEC:PAC:19}{}}%
Le Cam, M., \& Pac, S. (2019). ICILS 2018 : évaluation internationale des élèves de quatrième en littératie numérique et pensée informatique. \emph{Note d'information de la DEPP}, \emph{19.40}.

\leavevmode\vadjust pre{\hypertarget{ref-LEM:eal:17}{}}%
Le Mener, M., Meuret, D., Morlaix, S., \& Garnier, L. (2017). The Increased Impact of Social Background on Educational Performance in France: Where Has This Come From? \emph{Revue francaise de sociologie}, \emph{58}(2), 207‑231.

\leavevmode\vadjust pre{\hypertarget{ref-LEE:07}{}}%
Lee, L. (2007). Identification and estimation of econometric models with group interactions, contextual factors and fixed effects. \emph{Journal of Econometrics}, \emph{140}(2), 333‑374.

\leavevmode\vadjust pre{\hypertarget{ref-LEO:AL:14}{}}%
Leong, K. E., \& Alexander, N. (2014). College students attitude and mathematics achievement using web based homework. \emph{Eurasia Journal of Mathematics, Science and Technology Education}, \emph{10}(6), 609‑615.

\leavevmode\vadjust pre{\hypertarget{ref-LEU:eal:04}{}}%
Leuven, E., Lindahl, M., Oosterbeek, H., \& Webbink, D. (2004). \emph{{New evidence on the effect of time in school on early achievement}} (HEW nᵒ 0410001). University Library of Munich, Germany. \url{https://ideas.repec.org/p/wpa/wuwphe/0410001.html}

\leavevmode\vadjust pre{\hypertarget{ref-LEU:RON:11}{}}%
Leuven, E., \& Rønning, M. (2016). Classroom Grade Composition and Pupil Achievement. \emph{The Economic Journal}, \emph{126}(593), 1164‑1192. https://doi.org/\url{https://doi.org/10.1111/ecoj.12177}

\leavevmode\vadjust pre{\hypertarget{ref-LOC:04}{}}%
Lochner, L. (2004). Education, work, and crime: A human capital approach. \emph{International Economic Review}, \emph{45}(3), 811‑843.

\leavevmode\vadjust pre{\hypertarget{ref-LOC:20}{}}%
Lochner, L. (2020). Education and crime. In \emph{The economics of education} (p. 109‑117). Elsevier.

\leavevmode\vadjust pre{\hypertarget{ref-LOG:eal:16}{}}%
Logue, A. W., Watanabe-Rose, M., \& Douglas, D. (2016). Should students assessed as needing remedial mathematics take college-level quantitative courses instead? A randomized controlled trial. \emph{Educational Evaluation and Policy Analysis}, \emph{38}(3), 578‑598.

\leavevmode\vadjust pre{\hypertarget{ref-LUC:MBI:14}{}}%
Lucas, A. M., \& Mbiti, I. M. (2014). Effects of school quality on student achievement: Discontinuity evidence from kenya. \emph{American Economic Journal: Applied Economics}, \emph{6}(3), 234‑263.

\leavevmode\vadjust pre{\hypertarget{ref-LY:RIE:14}{}}%
Ly, S. T., \& Riegert, A. (2014). \emph{Persistent classmates: How familiarity with peers protects from disruptive school transitions}.

\leavevmode\vadjust pre{\hypertarget{ref-MAN:93}{}}%
Manski, C. F. (1993). Identification of endogenous social effects: The reflection problem. \emph{The review of economic studies}, \emph{60}(3), 531‑542.

\leavevmode\vadjust pre{\hypertarget{ref-MAR:BRE:15}{}}%
Marie, C.-V., \& Breton, D. (2015). Les modèles familiaux dans les Dom: entre bouleversements et permanence. Ce que nous apprend l'enquête Migrations, famille et vieillissement. \emph{Revue des politiques sociales et familiales}, \emph{119}(1), 55‑64.

\leavevmode\vadjust pre{\hypertarget{ref-MAT:eal:16}{}}%
Matta, R., Ribas, R. P., Sampaio, B., \& Sampaio, G. R. (2016). The effect of age at school entry on college admission and earnings: a regression-discontinuity approach. \emph{IZA Journal of Labor Economics}, \emph{5}(1), 9.

\leavevmode\vadjust pre{\hypertarget{ref-MAY:KNU:97}{}}%
Mayer, S. E., \& Knutson, D. (1997). \emph{Does age at enrollment in first grade affect children's cognitive test scores}. Northwestern University/University of Chicago Joint Center for Poverty Research.

\leavevmode\vadjust pre{\hypertarget{ref-MCC:08}{}}%
McCrary, J. (2008). Manipulation of the running variable in the regression discontinuity design: A density test. \emph{Journal of econometrics}, \emph{142}(2), 698‑714.

\leavevmode\vadjust pre{\hypertarget{ref-MCE:SHA:08}{}}%
McEwan, P. J., \& Shapiro, J. S. (2008). The benefits of delayed primary school enrollment discontinuity estimates using exact birth dates. \emph{Journal of human Resources}, \emph{43}(1), 1‑29.

\leavevmode\vadjust pre{\hypertarget{ref-MEN:eal:18}{}}%
Mendolia, S., Paloyo, A. R., \& Walker, I. (2018). Heterogeneous effects of high school peers on educational outcomes. \emph{Oxford Economic Papers}, \emph{70}(3), 613‑634.

\leavevmode\vadjust pre{\hypertarget{ref-MET:eal:17}{}}%
Métayer, C., Monso, O., Boudesseul, G., Caro, P., Grelet, Y., Minassian, L., \& Vivent, C. (2017). Les in{é}galit{é}s territoriales de risques sociaux d'{é}chec scolaire. In \emph{Géographie de l'école}. MENESR-DEPP.

\leavevmode\vadjust pre{\hypertarget{ref-MO:eal:14}{}}%
Mo, D., Zhang, L., Luo, R., Qu, Q., Huang, W., Wang, J., Qiao, Y., Boswell, M., \& Rozelle, S. (2014). Integrating computer-assisted learning into a regular curriculum: Evidence from a randomised experiment in rural schools in Shaanxi. \emph{Journal of development effectiveness}, \emph{6}(3), 300‑323.

\leavevmode\vadjust pre{\hypertarget{ref-MON:eal:19}{}}%
Monso, O., Fougere, D., Givord, P., \& Pirus, C. (2019). \emph{Les camarades influencent-ils la r{é}ussite et le parcours des {é}l{è}ves?} (Série études - Document de travail nᵒ 2019-E02). DEPP.

\leavevmode\vadjust pre{\hypertarget{ref-MON:10}{}}%
Monteil, C. (2010). Le cr{é}ole encore tr{è}s largement majoritaire {à} La R{é}union. \emph{Revue {É}conomique de l'INSEE}, \emph{137}.

\leavevmode\vadjust pre{\hypertarget{ref-MOR:eal:20}{}}%
Moreno-Guerrero, A.-J., Aznar-Díaz, I., Cáceres-Reche, P., \& Alonso-García, S. (2020). E-learning in the teaching of mathematics: An educational experience in adult high school. \emph{Mathematics}, \emph{8}(5), 840.

\leavevmode\vadjust pre{\hypertarget{ref-MSO:BAN:18}{}}%
Msomi, A., \& Bansilal, S. (2018). The experiences of first-year students in mathematics in using an e-learning platform at a university of technology. \emph{South African Journal of Higher Education}, \emph{32}(5), 124‑129.

\leavevmode\vadjust pre{\hypertarget{ref-MUR:eal:91}{}}%
Murphy, K. M., Shleifer, A., \& Vishny, R. W. (1991). The allocation of talent: Implications for growth. \emph{The quarterly journal of economics}, \emph{106}(2), 503‑530.

\leavevmode\vadjust pre{\hypertarget{ref-NAM:14}{}}%
Nam, K. (2014). Until when does the effect of age on academic achievement persist? Evidence from Korean data. \emph{Economics of Education Review}, \emph{40}, 106‑122.

\leavevmode\vadjust pre{\hypertarget{ref-OCDR:00}{}}%
OCDE. (2000). Regards sur l'éducation - édition 2000. In \emph{Regards sur l'éducation}.

\leavevmode\vadjust pre{\hypertarget{ref-OCDR:10}{}}%
OCDE. (2010). Regards sur l'éducation - édition 2010. In \emph{Regards sur l'éducation}.

\leavevmode\vadjust pre{\hypertarget{ref-OCD:19}{}}%
OCDE. (2019). Résultats du PISA 2018. \emph{Note par pays - France}.

\leavevmode\vadjust pre{\hypertarget{ref-OCDR:20}{}}%
OCDE. (2020). Regards sur l'éducation - édition 2020. In \emph{Regards sur l'éducation}.

\leavevmode\vadjust pre{\hypertarget{ref-OEC:13}{}}%
OECD. (2013). \emph{PISA 2012 results: Excellence through equity: Giving every student the chance to succeed (Volume II)}. OECD Publishing Paris, France.

\leavevmode\vadjust pre{\hypertarget{ref-PAL:11}{}}%
Palheta, U. (2011). Le coll{è}ge divise. Appartenance de classe, trajectoires scolaires et enseignement professionnel. \emph{Sociologie}, \emph{2}(4), 363‑386.

\leavevmode\vadjust pre{\hypertarget{ref-PAL:20}{}}%
Paloyo, A. R. (2020). Peer effects in education: recent empirical evidence. In \emph{The Economics of Education} (p. 291‑305). Elsevier.

\leavevmode\vadjust pre{\hypertarget{ref-PEL:12}{}}%
Pellet, S. (2012). {É}cole publique, {é}cole priv{é}e: une comparaison. \emph{Regards crois{é}s sur l'{é}conomie}, \emph{2}, 184‑188.

\leavevmode\vadjust pre{\hypertarget{ref-PEN:17}{}}%
Peña, P. A. (2017). Creating winners and losers: Date of birth, relative age in school, and outcomes in childhood and adulthood. \emph{Economics of Education Review}, \emph{56}, 152‑176.

\leavevmode\vadjust pre{\hypertarget{ref-PIK:VAL:06}{}}%
Piketty, T., Valdenaire, M., \& others. (2006). \emph{L'impact de la taille des classes sur la r{é}ussite scolaire dans les {é}coles, coll{è}ges et lyc{é}es fran{ç}ais: estimations {à} partir du panel primaire 1997 et du panel secondaire 1995}. Direction de l'{é}valuation et de la prospective.

\leavevmode\vadjust pre{\hypertarget{ref-PIS:18}{}}%
Pischke, J. (2018). \emph{Weak instruments}. Retrieved from The London School of Economics: http://econ.lse.ac.uk/staff/spischke/ec533/Weak\%20IV.pdf.

\leavevmode\vadjust pre{\hypertarget{ref-PLA:DAV:20}{}}%
Plank, D. N., \& Davis, T. E. (2020). The economic role of the state in education. In \emph{The Economics of Education} (p. 445‑454). Elsevier.

\leavevmode\vadjust pre{\hypertarget{ref-PRA:DES:14}{}}%
Praet, M., \& Desoete, A. (2014). Enhancing young children's arithmetic skills through non-intensive, computerised kindergarten interventions: A randomised controlled study. \emph{Teaching and Teacher Education}, \emph{39}, 56‑65.

\leavevmode\vadjust pre{\hypertarget{ref-PSA:PAT:18}{}}%
Psacharopoulos, G., \& Patrinos, H. A. (2018). Returns to investment in education: a decennial review of the global literature. \emph{Education Economics}, \emph{26}(5), 445‑458.

\leavevmode\vadjust pre{\hypertarget{ref-PUH:WEB:05}{}}%
Puhani, P., \& Weber, A. (2005). \emph{Does the Early Bird Catch the Worm? Instrumental Variable Coefficients of Educational Effects of Age of School Enty in Germany} (IZA Discussion Paper nᵒ 1827).

\leavevmode\vadjust pre{\hypertarget{ref-PUT:eal:20}{}}%
Putz, L.-M., Hofbauer, F., \& Treiblmaier, H. (2020). Can gamification help to improve education? Findings from a longitudinal study. \emph{Computers in Human Behavior}, \emph{110}, 106392.

\leavevmode\vadjust pre{\hypertarget{ref-RAG:08}{}}%
Ragoucy, C. (2008). Les indicateurs de L'OCDE sur les depenses d'éducation en 2005: quelques tendances sur la sitiation comparee de la France. \emph{Education \& formations}, \emph{78}, 45‑61.

\leavevmode\vadjust pre{\hypertarget{ref-ROS:eal:16}{}}%
Roschelle, J., Feng, M., Murphy, R. F., \& Mason, C. A. (2016). Online mathematics homework increases student achievement. \emph{AERA open}, \emph{2}(4), 1‑12.

\leavevmode\vadjust pre{\hypertarget{ref-BAR:ROU:09}{}}%
Rouse, C. E., \& Barrow, L. (2009). School vouchers and student achievement: Recent evidence and remaining questions. \emph{Annu. Rev. Econ.}, \emph{1}(1), 17‑42.

\leavevmode\vadjust pre{\hypertarget{ref-SAC:01}{}}%
Sacerdote, B. (2001). Peer effects with random assignment: Results for Dartmouth roommates. \emph{The Quarterly journal of economics}, \emph{116}(2), 681‑704.

\leavevmode\vadjust pre{\hypertarget{ref-SAC:11}{}}%
Sacerdote, B. (2011). Peer effects in education: How might they work, how big are they and how much do we know thus far? In \emph{Handbook of the Economics of Education} (Vol. 3, p. 249‑277). Elsevier.

\leavevmode\vadjust pre{\hypertarget{ref-SAC:14}{}}%
Sacerdote, B. (2014). Experimental and quasi-experimental analysis of peer effects: two steps forward? \emph{Annual Review of Economics}, \emph{6}(1), 253‑272.

\leavevmode\vadjust pre{\hypertarget{ref-SAL:LEC:20}{}}%
Salles, F., \& Le Cam, M. (2020). TIMSS 2019 - Mathématiques au niveau de la classe de quatrième : des résultats inquiétants en France. \emph{Note d'information de la DEPP}, \emph{20.46}.

\leavevmode\vadjust pre{\hypertarget{ref-SAN:STO:05}{}}%
Sandgren, S., \& Strøm, B. (2005). Peer effects in primary school: Evidence from age variation. \emph{Annual Conference EALE}.

\leavevmode\vadjust pre{\hypertarget{ref-SHA:82}{}}%
Schafer, A. (1982). The ethics of the randomized clinical trial. \emph{New England Journal of Medicine}, \emph{307}(12), 719‑724.

\leavevmode\vadjust pre{\hypertarget{ref-SHI:15}{}}%
Shigeoka, H. (2015). \emph{School entry cutoff date and the timing of births} (Working paper nᵒ 21402). National Bureau of Economic Research.

\leavevmode\vadjust pre{\hypertarget{ref-SIM:15}{}}%
Si Moussa, A. (2015). {É}valuations nationales en France et contexte ultrap{é}riph{é}rique: un enjeu significatif pour l'{é}cole {à} La R{é}union? \emph{Mesure et {é}valuation en {é}ducation}, \emph{38}(1), 31‑56.

\leavevmode\vadjust pre{\hypertarget{ref-SIA:REE:03}{}}%
Sianesi, B., \& Reenen, J. V. (2003). The returns to education: Macroeconomics. \emph{Journal of economic surveys}, \emph{17}(2), 157‑200.

\leavevmode\vadjust pre{\hypertarget{ref-SIL:09}{}}%
Silles, M. A. (2009). The causal effect of education on health: Evidence from the United Kingdom. \emph{Economics of Education review}, \emph{28}(1), 122‑128.

\leavevmode\vadjust pre{\hypertarget{ref-SMI:09}{}}%
Smith, J. (2009). Can regression discontinuity help answer an age-old question in education? The effect of age on elementary and secondary school achievement. \emph{The BE Journal of Economic Analysis \& Policy}, \emph{9}(1).

\leavevmode\vadjust pre{\hypertarget{ref-SOJ:13}{}}%
Sojourner, A. (2013). Identification of peer effects with missing peer data: Evidence from Project STAR. \emph{The Economic Journal}, \emph{123}(569), 574‑605.

\leavevmode\vadjust pre{\hypertarget{ref-SPO:14}{}}%
Sposato, D. (2014). \emph{The impact of parents' background and students' age on educational outcomes in italian primary school: evidence from INVALSI data} {[}Thèse de doctorat{]}.

\leavevmode\vadjust pre{\hypertarget{ref-STA:STO:97}{}}%
Staiger, D., \& Stock, J. H. (1997). Instrumental Variables Regression with Weak Instruments. \emph{Econometrica}, \emph{65}(3), 557‑586. \url{http://www.jstor.org/stable/2171753}

\leavevmode\vadjust pre{\hypertarget{ref-STI:02}{}}%
Stipek, D. (2002). At what age should children enter kindergarten? A question for policy makers and parents. \emph{Social Policy Report}, \emph{16}(2), 1‑20.

\leavevmode\vadjust pre{\hypertarget{ref-STO:04}{}}%
Strøm, B. (2004). \emph{Student achievement and birthday effects}. Unpublished manuscript.

\leavevmode\vadjust pre{\hypertarget{ref-TAH:eal:14}{}}%
Tahir, M., Haji, D. H. N. B. P., \& Ali, O. (2014). Trade openness and economic growth: a review of the literature. \emph{Asian Social Science}, \emph{10}(9), 137.

\leavevmode\vadjust pre{\hypertarget{ref-TIE:MAH:08}{}}%
Tienken, C. H., \& Maher, J. A. (2008). The influence of computer-assisted instruction on eighth grade mathematics achievement. \emph{RMLE Online}, \emph{32}(3), 1‑13.

\leavevmode\vadjust pre{\hypertarget{ref-TIE:WIL:07}{}}%
Tienken, C. H., \& Wilson, M. J. (2007). The Impact of Computer Assisted Instruction on Seventh-Grade Students' Mathematics Achievement. \emph{Planning and Changing}, \emph{38}, 181‑190.

\leavevmode\vadjust pre{\hypertarget{ref-TOD:WOL:03}{}}%
Todd, P. E., \& Wolpin, K. I. (2003). On the specification and estimation of the production function for cognitive achievement. \emph{The Economic Journal}, \emph{113}(485), F3‑F33.

\leavevmode\vadjust pre{\hypertarget{ref-TOP:eal:11}{}}%
Topping, K., Miller, D., Murray, P., Henderson, S., Fortuna, C., \& Conlin, N. (2011). Outcomes in a randomised controlled trial of mathematics tutoring. \emph{Educational Research}, \emph{53}(1), 51‑63.

\leavevmode\vadjust pre{\hypertarget{ref-VAL:21}{}}%
Valat, E. (2021). Les in{é}galit{é}s d'{é}ducation entre les natifs des Drom et de m{é}tropole: le r{ô}le d{é}terminant du capital humain transmis par les parents. \emph{Population}, \emph{76}(1), 115‑153.

\leavevmode\vadjust pre{\hypertarget{ref-VEN:JAE:13}{}}%
Venezia, A., \& Jaeger, L. (2013). Transitions from high school to college. \emph{The future of children}, 117‑136.

\leavevmode\vadjust pre{\hypertarget{ref-VID:NEC:06}{}}%
Vigdor, J., \& Nechyba, T. (2007). Peer effects in North Carolina public schools. \emph{Schools and the equal opportunity problem}, 73‑101.

\leavevmode\vadjust pre{\hypertarget{ref-VIL:eal:18}{}}%
Villani, C., Torossian, C., \& Dias, T. (2018). \emph{21 mesures pour l'enseignement des math{é}matiques}.

\leavevmode\vadjust pre{\hypertarget{ref-WAT:20}{}}%
Watts, T. W. (2020). Academic achievement and economic attainment: Reexamining associations between test scores and long-run earnings. \emph{AERA Open}, \emph{6}(2), 1‑16.

\leavevmode\vadjust pre{\hypertarget{ref-WEI:15}{}}%
Weiss, Y. (2015). Gary Becker on human capital. \emph{Journal of Demographic Economics}, \emph{81}(1), 27‑31.

\leavevmode\vadjust pre{\hypertarget{ref-WOO:02}{}}%
Wooldridge, J. M. (2002). Econometric analysis of cross section and panel data MIT press. \emph{Cambridge, MA}, \emph{108}(2), 245‑254.

\leavevmode\vadjust pre{\hypertarget{ref-WOO:15}{}}%
Wooldridge, J. M. (2015). Control function methods in applied econometrics. \emph{Journal of Human Resources}, \emph{50}(2), 420‑445.

\leavevmode\vadjust pre{\hypertarget{ref-WOE:03}{}}%
Wößmann, L. (2003). Specifying human capital. \emph{Journal of economic surveys}, \emph{17}(3), 239‑270.

\leavevmode\vadjust pre{\hypertarget{ref-XIE:eal:20}{}}%
Xie, C., Cheung, A. C., Lau, W. W., \& Slavin, R. E. (2020). The effects of computer-assisted instruction on mathematics achievement in mainland China: A meta-analysis. \emph{International Journal of Educational Research}, \emph{102}, 101565.

\leavevmode\vadjust pre{\hypertarget{ref-YEU:NGU:16}{}}%
Yeung, R., \& Nguyen-Hoang, P. (2016). Endogenous peer effects: Fact or fiction? \emph{The Journal of Educational Research}, \emph{109}(1), 37‑49.

\leavevmode\vadjust pre{\hypertarget{ref-ZAB:08}{}}%
Zabel, J. E. (2008). The impact of peer effects on student outcomes in New York City public schools. \emph{Education Finance and Policy}, \emph{3}(2), 197‑249.

\end{CSLReferences}

\end{document}
